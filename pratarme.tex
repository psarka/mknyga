\thispagestyle{empty}
\addcontentsline{toc}{section}{Apie knygą}
\section*{Apie knygą}

Matematikos knyga - tai knyga skirta matematika besidomintiems moksleiviams
ir moksleivėms. Jos turinys yra gerokai nutolęs nuo sutinkamo mokykloje ir,
pagal matematikos olimpiadų tradiciją, orientuotas į keturias matematikos
sritis: skaičių teoriją, algebrą, kombinatoriką ir geometriją.
Turinys pateiktas naudojant įprastą matematinę kalbą,
tad prie teoremų, įrodymų ir matematinių pažymėjimų nepratusiems gali
prireikti šiek tiek daugiau atkaklumo ir mokytojo(-os) pagalbos.\\

\noindent Prie knygos kūrimo prisidėjo keletas žmonių, kuriuos norėtume
paminėti:\\
\noindent \textbf{Benas Bačanskas} ir \textbf{Gintautas Miliauskas}-- radę ir ištaisę krūvą
klaidų.\\
\noindent \textbf{Gabrielė Bakšytė} -- pirmojo leidimo redaktorė. \\
\noindent \textbf{Žymantas Darbėnas} -- vienas iš knygos idėjos autorių ir
įkvepėjų.\\
\noindent \textbf{Albertas Zinevičius} -- padėjęs rašyti ir apipavidalinti
knygą.\\

\noindent Taip pat norėtume paminėti Nacionalinę moksleivių akademiją, kuri
nekartą įkvėpė ir paskatino tęsti knygos kūrimą.
\begin{flushright}Ačiū jums!\end{flushright}

\begin{center}*\end{center}

\noindent Kaip jau pastebėjote iš pirmųjų puslapių, ši knyga yra išleista pagal
\emph{Creative Commons} licenziją. Tai reiškia, kad kartu su knyga jūs
gaunate gerokai daugiau laisvės, nei įprastai, ir mes tikimės, kad ta laisve
jūs drąsiai naudositės. \\

\noindent Kartu tai šiek tiek paaiškina, kodėl išleidžiama nebaigta knyga. Rašyti po
skyrių ar skyrelį yra daug paprasčiau ir efektyviau, nei iš karto griebtis
sunkiai įkandamo užmojo. Atviras formatas neriboja nei autorių skaičiaus,
nei rašymo trukmės, tad jei žvilgtelėjus į turinį jums galvoje rikiuojasi
trūkstamo skyrelio tekstas, galbūt metas sėsti prie klaviatūros?
