\thispagestyle{empty}
\addcontentsline{toc}{section}{Apie knygą}
\section*{Apie knygą}

Matematikos knyga - tai knyga skirta matematika besidomintiems moksleiviams
ir moksleivėms. Jos turinys yra gerokai nutolęs nuo sutinkamo mokykloje ir,
pagal matematikos olimpiadų tradiciją, orientuotas į keturias matematikos
sritis: skaičių teoriją, algebrą, kombinatoriką ir geometriją (pastarosios
kol kas visai nėra). Turinys pateiktas naudojant įprastą matematinę kalbą,
tad prie teoremų, įrodymų ir matematinių pažymėjimų nepratusiems gali
prireikti šiek tiek daugiau atkaklumo ir mokytojo(-os) pagalbos.\\

\noindent Prie knygos atsiradimo prisidėjo keletas žmonių, kuriuos norėtume
paminėti:\\

\noindent Žymantas Darbėnas - vienas iš pirmųjų idėjos autorių,
drąsiai sakęs, kad knygą parašyti įmanoma, ir pradėjęs rašyti pirmuosius
tekstus. \\

\noindent Albertas Zinevičius - vieno iš autorių dėka susidūręs su
absoliučiai visomis iškilusiomis knygos rašymo ir apipavidalinimo
problemomis ir padėjęs jas išspręsti. \\

\noindent Gabrielė Bakšytė - fantastiškai iš niekur nieko atsiradusi ir mus
išgelbėjusi teksto redaktorė. \\

\noindent Taip pat norėtume paminėti Nacionalinę Moksleivių Akademiją, kuri iki šiol
visuomet buvo šalia ir kantriai palaikė ilgo ir banguoto proceso metu.
Idėjai užgimus Paulius ir Žymantas buvo jos dėstytojai, o vėliau prisijungę
Paulius, Pijus, Lukas ir Gabrielė - jos moksleiviai. 
\begin{flushright}Ačiū jums!\end{flushright}

\begin{center}*\end{center}

\noindent Kaip jau pastebėjote iš pirmųjų puslapių, ši knyga yra išleista pagal
\emph{Creative Commons} licenziją. Tai reiškia, kad kartu su knyga jūs
gaunate gerokai daugiau laisvės, nei įprastai, ir mes tikimės, kad ta laisve
jūs drąsiai naudositės. \\

\noindent Kartu tai šiek tiek paaiškina, kodėl išleidžiama nebaigta knyga. Rašyti po
skyrių ar skyrelį yra daug paprasčiau ir efektyviau, nei iš karto griebtis
sunkiai įkandamo užmojo. Atviras formatas neriboja nei autorių skaičiaus,
nei rašymo trukmės, tad jei žvilgtelėjus į turinį jums galvoje rikiuojasi
trūkstamo skyrelio tekstas, galbūt metas sėsti prie klaviatūros?
