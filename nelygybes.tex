\section{Nelygybės}

Šiame skyrelyje daugiausia to, ką veiksime su nelygybėmis, sudarys bandymai
jas įrodyti. Nelygybių įrodinėjimas bus pagrindinė veikla, o jų įrodymas
bus aukščiausia siekiamybė ir didžiausia vertybė. Skaitytojui,
susipažinusiam tik su mokykliniu nelygybių kursu, tai gali atrodyti ne tik
naujai, bet keistai ar baisiai. Nuo ko pradėti, norint įrodyti? Nelygybių
įrodinėjimo filosofija remiasi vos keliais paprastais principais.

Bandydami įrodyti, naudosimės nelygybėmis-teoremomis, su kuriomis
susipažinsime šiame skyriuje ir žinosime, kad jos tikrai tikrai galioja.
Šios teoremos - tarsi laiptai, kuriais lipame iš kairės nelygybės pusės į
dešinę. Jei turime įrodyti $A\geq B$, o pagal teoremą $T$, turime
$A\geq B$, tai mes įrodėme nelygybę ,,vienu šuoliu'', kas nebuvo labai
įdomu. Tik nuo sprendėjo priklauso, kiek ir kokių ,,šuolių'' reikės atlikti
norint pasiekti rezultatą. Kadangi dažniausiai tenka lipti daugiau nei
vienu laipteliu, reikėtų sužinoti, kaip tai daroma.

Tarkime, norime įrodyti $A\geq C$. Tegu, remiantis teorema $X$,
tikrai tikrai galioja $A\geq B$. Jei pasistengę gausime, kad, anot
teoremos $Y$, $B\geq C$, tai tada $A\geq C$, ką ir reikėjo įrodyti. Bet
jeigu netyčia pagal teoremą $Z$ tikrai galioja $B\leq C$, tai reikš
ne tai, kad įrodoma nelygybė yra neteisinga, bet kad teoremos $X$
,,laiptelis'' buvo per ,,status''. Pagalvokite: jei iš taško $A$ stipriai
nusileidžiate į tašką $B$, bet pamatote, kad $C$ - aukščiau už $B$,
niekaip negalėsite pasakyti kuris iš $A$ ir $C$ yra aukščiau, nes nelygybės
gali nurodyti tik, ar kažkas yra daugiau/mažiau už kažką, bet ne kiek
stipriai. Kitaip tariant, žinome tik tiek, kad jei mes tik leidomės, tai
esame žemiau, o jeigu tik kilome - tai aukščiau, na o jeigu kaip liftu
važinėjomes tai aukštyn tai žemyn, tai jau niekas nebesupaisys, kokiame
aukštyje esame. Tai yra pagrindinis nelygybių įrodinėjimo principas, tačiau
yra kelios plačiai naudojamos jo formos.

Nelygybę visada galime ekvivalenčiai pertvarkyti (ekvivalenčiai reiškia,
kad jei atlikome tam tikrus pertvarkymus ir iš nelygybės $X$ gavome
nelygybę $Y$, tai atlikdami logiškus atvirkščius pertvarkymus, iš $Y$
galime vėl gauti $X$) ir tada naudoti/įrodinėti pertvarkytąją. Tie
pertvarkymai gali būti labai įvairūs: prie abiejų nelygybės pusių galime
pridėti po konstantą, padauginti iš jos, pakelti laipsniu, logaritmuoti ir
antilogaritmuoti. Visada reikia būti atsargiems: kai kada ne visi šie
veiksmai yra galimi. Taip pat prisiminkite, kad nelygybę dauginant iš
neigiamos konstantos ar keliant neigiamu laipsniu, nelygybės ženklas
apsiverčia, t.y.: iš $\geq$ virsta į $\leq$, o iš $>$ į $<$ ir atvirkščiai.

Dvi teisingas nelygybes galime visada sudėti, o jei jos abi teigiamos, ir
sudauginti. Taigi, jei turime $A\geq C$ ir $B\geq D$, tai įrodėme $A+B\geq
C+D$, o jei $A$, $B$, $C$ ir $D$ teigiami, tai ir $A\cdot B\geq C\cdot D$.
Pastebėkime, kad nei dalinti, nei atimti nelygybių vienos iš kitos
negalime.  Netikintiems: imkime dvi teisingas nelygybes $8\geq4$ ir
$8\geq3$. Nei atėmę, nei padalinę teisingos nelygybės negausime.

Lygybės atvejis yra viena subtiliausių negriežtų nelygybių dalių. Naudojant
teoremas privalu stebėti, ar vis dar įmanoma pasiekti lygybę. Jei pasidaro
neįmanoma, tai uždavinio išspręsti greičiausiai nepavyks. Kaip sužinoti
lygybės atvejį? Dažniausiai reikia tiesiog atspėti, kas neretai yra gana
paprasta. Atsižvelgimas į lygybės atvejį leis sutaupyti laiko ir aklai
nenaudoti žūčiai pasmerktų strategijų. Lygybės atvejis naudojamas dar ir
ekstremumų ieškojimui.

Ekstremumas - mažiausia arba didžiausia funkcijos reikšmė duotame
intervale. Profesionalai ekstremumų ieškojimui naudoja išvestines ir Lagranžo
daugiklius. Skaitytojus raginame susipažinti su šia įstabios galios
technika, jei to padaryti dar nespėjote. Šiame skyriuje ekstremumų ieškojimui
naudosime alternatyvų būdą - nelygybes. Ne paslaptis, kad remiantis
klasikinėmis nelygybėmis, ciklinių ar simetrinių reiškinių nuo kelių
kintamųjų ekstemumų ieškojimas yra daug paprastesnis ir, dažnai, greitesnis.
Reiškinio minimumo ar maksimumo ieškojimas nelygybėmis remiasi dviem
elementariais žingsniais:

\begin{enumerate}
\item Randame reiškinio maksimalią ar minimalią ribą, tai yra, už ką jis
  yra tikrai ne didesnis ar ne mažesnis. Pavyzdžiui, jei gautume, kad
  funkcija $F\geq C$, kur $C$ - kokia tai konstanta, tai su jokiais
  funkcijos parametrais negalime gauti $F$ reikšmės, mažesnės už $C$. Gali
  pasirodyti, kad tai reikštų, jog $C$ yra vienas funkcijos ekstremumų -
  minimumas, bet taip nebūtinai yra, todėl privaloma žengti antrąjį
  žingsnį.
\item Radę galimą reikšmę, privalu patikrinti, ar ji pasiekiama.
  Ji bus įgyjama lygybės atveju, taigi, iš pritaikytų nelygybių lygybės
  atvejų turime atsekti, kokios turi būti kintamųjų reikšmės.
\end{enumerate}

Jei antrojo žingsnio išpildyti nepavyksta, tai reiškia, kad pirmasis
žingsnis atliktas neteisingai. Dažniausia klaida - panaudotų nelygybių
lygybės atvejų praradimas, kai šie neegzistuoja arba netenkina reiškinio
apibrėžimo srities.
Na, o jeigu pavyko atlikti abu veiksmus, jūs sėkmingai radote funkcijos
ekstremumą. Tokios užduoties atsakymas formuluojamas įvardijant ne tik
rastą reikšmę, bet ir parametrų, su kuriais tai pasiekiama, reikšmes.

Nelygybės yra itin plati ir labai įvairi matematikos šaka. Šiame skyriuje
supažindinsime su pagrindinėmis sprendimo technikomis, triukais. Neįmanoma
mintinai išmokti visų nelygybių, tačiau galima išmokti suprasti pagrindines
tendencijas ir greičiau surasti idėją, padėsiančią atlikti užduotį.
Idėjoms įgyvendinti reikalingi įrankiai. Jais ir taps įvairios
nelygybės-teoremos, metodai, pavyzdžių, uždavinių rezultatai. Tai padės
išspręsti didžiąją dalį uždavinių, kurie pasirodys ne tik skyreliuose
,,Uždaviniai'', bet ir olimpiadose. Nesitikime, kad skaitytojas pajėgs pats
išspręsti visus pateiktus uždavinius, juk kai kurie jų - tikri algebros
briliantai, tačiau pastangos nenueis perniek. Būkite drąsūs!

\newpage
\subsection{Pirmieji žingsniai}

Beveik visos klasikinės nelygybės remiasi faktu, kad realaus skaičiaus
kvadratas yra nemažesnis už nulį. Tačiau suvesti bet kokią nelygybę į
kvadratų, padaugintų iš teigiamų skaičių, sumą dažniausiai būna mažų
mažiausiai šlykštu. Todėl gausybė talentingų pasaulio matematikų per
amžius sunkiai dirbo, kurdami vis įspūdingesnius ir galingesnius įrankius,
kuriems paklūsta net pačios sudėtingiausios problemos. Šių įrankių veikimo
principai reikalauja dėmesio, o jų supratimas leis juos naudoti itin
efektyviai ir sumaniai. Šiame skyrelyje ir pradėsime nuo pačių pamatų:
nagrinėsime, ką galime pasiekti iš tokio nekaltai atrodančio fakto kaip:

\begin{thm}
  Jei $x\in\R$, tai $x^2\geq0$. Lygybė galios tada ir tik tada, kai $x=0$.
\end{thm}

Kai kurios teoremos bus įrodytos, bet tik ne ši. Žinoma, tai labai svarbi
nelygybė ir sunku įsivaizduoti nelygybę, kuri ja nesiremtų, bet įrodymas
yra toks paprastas, kad žymiai daugiau prasmės yra švaistyti popierių ir
laiką šnekant apie jos akivaizdumą negu iš tikrųjų ją įrodyti. Įrodymas
remiasi tokiais gerai žinomais teiginiais kaip ,,Mano draugo draugas yra
mano draugas'' ir  ,,Mano priešo priešas yra mano draugas''. Pravartu
žinoti, kad ,,Mano daugo priešas yra mano priešas'' ir ,,Mano priešo
draugas yra mano priešas'', nors  paskutinieji du nelygybės įrodyti ir
nepadeda.

Kadangi pavyzdžiai kalba geriau už bet kokią nelygybių sprendimo ir įrodymo
teoriją, tai ir judėkime prie jų.

\subsubsection{Pavyzdžiai}

\begin{pavnr}
  Jei $0\leq a,b\in\R$, tai: $$a+b\geq2\sqrt{ab}.$$ Lygybė galios tada ir tik
  tada, kai $a=b$.
\end{pavnr}

\begin{proof}[Įrodymas]
  Pertvarkykime nelygybę į (matematikų kalba šnekant, nelygybė yra
  ekvivalenti) $(\sqrt{a}-\sqrt{b})^2\geq0$. Iš akivaizdžios ankstesnės
  teoremos seka, kad gauta nelygybė yra teisinga. Štai kaip
  ,,vienu šuoliu'' išsprendėme pirmajį nelygybių skyriaus uždavinį.
\end{proof}

Dauguma šio skyrelio uždavinių, kaip ir pirmasis, bus paremti reiškinių
pertvarkymais į kvadratų sumą. Be to, nepamiršime naudotis gautais
uždavinių ir pavyzdžių rezultatais, kurie žymiai supaprastins sprendimus.

\begin{pavnr}
  Raskite $S=2a^2+9c^2+5b^2+2ab-8bc-8ac-2a+4c+2$ minimumą, kai $a,b,c\in\R$.
\end{pavnr}

\begin{sprendimas}
  Pertvarkome reiškinį: $S=(a+b-2c)^2+(2c-a+1)^2+(2b-c)^2+1\geq1$. Spėjamas
  minimumas yra 1, belieka patikrinti, ar jis pasiekiamas. Tai atliekame
  spręsdami lygčių sistemą:

  $$\left\{\begin{array}{lr}
  a+b-2c=0;\\
  2c-a+1=0;\\
  2b-c=0.\end{array}\right.
  \Rightarrow
  \left\{\begin{array}{lr}
  a=-3;\\
  b=-1;\\
  c=-2.\end{array}\right. $$
  Vadinasi, minimali $S$ reikšmė lygi $1$, ir ji gaunama, kai $a=-3$,
  $b=-1$, $c=-2$.
\end{sprendimas}

\begin{pastaba}
  Galima ir kitaip sugrupuoti duoto reškinio narius, tačiau tuomet gauta
  lygčių sistema neturės visų reikiamų sprendinių, arba šie bus netinkami.
\end{pastaba}

\begin{pavnr}
  Įrodykite, kad su teigiamais realiaisiais skaičiais $x$ ir $y$ galioja
  nelygybė $$\frac{1}{x}+\frac{1}{y}\geq \frac{4}{x+y}.$$
\end{pavnr}

\begin{proof}[Įrodymas]
  Padauginame nelygybę iš $xy(x+y)$. Gausime nelygybę $xy+y^2+x^2+xy\geq
  4xy$, kuri yra ekvivalenti $(x-y)^2\geq0$, kas yra akivaizdu.
\end{proof}

\begin{pastaba}
  Beveik visos miniatiūrinės dviejų kintamųjų nelygybės gali būti lengvai
  ,,nulaužtos'' naudojant ,,brutalios jėgos'' taktiką, ką mes ir padarėme
  paskutiniame nagrinėtame pavyzdyje. Žinoma, tokia taktika gali ,,nulaužti'' ir
  daug masyvesnes kelių kintamųjų nelygybes, tačiau taip spręsti nėra taip
  malonu ir greita, kaip ieškant teisingo kokios nors teoremos pritaikymo būdo.
  Tai pamatysime kitame pavyzdyje:
\end{pastaba}

\begin{pavnr}
  Tegu $a,b,c$ bus teigiami realieji skaičiai. Įrodykite, kad
  $$\frac{ab}{c}+\frac{ac}{b}+\frac{bc}{a}\geq a+b+c.$$
\end{pavnr}

\begin{proof}[Įrodymas]
  Pagal nelygybę $a+b \geq 2\sqrt{ab}$, gauname
  $$\frac{ab}{c}+\frac{ac}{b}\geq 2a.$$ Taip pat
  $$\frac{ab}{c}+\frac{bc}{a}\geq 2b,$$ ir $$\frac{ac}{b}+\frac{bc}{a}\geq
  2c.$$ Sudėję šias nelygybes gausime norimą rezultatą.
\end{proof}

Dažnai tenka įrodinėti griozdiškas nelygybes, kur daugybę kartų tenka
perrašinėti ilgus ir vienus į kitus panašius reiškinius. Matematikai,
būdami nepataisomi tinginiai yra sugalvoję keletą žymėjimų, kurie
sumažina sugadinamo popieriaus kiekį ir padeda sistemingai pateikti
reikiamą informaciją. Susipažinkime su ciklinėmis ir simetrinėmis sumomis
bei sandaugomis:

\begin{api} Tegu $A_0=\{a_{1},a_{2},...,a_{n}\}.$ Tuomet
  \begin{align*}
    &\sum_{cyc}{f(A_0)}=f(a_{1},a_{2},...,a_{n})+f(a_{2},a_{3},...,a_{n},a_{1})+\\
    &+f(a_{3},a_{4},...,a_{n},a_{1},a_{2})+...+f(a_{n},a_{1},...,a_{n-1}).
  \end{align*}
\end{api}

Taigi, ciklinė suma - tai suma, kur sumuojamos funkcijos argumentai
yra perstumiami per vieną poziciją $n$ kartų. Pavyzdžiui:

$$\sum_{cyc}{\frac{a^2+a}{b}}=\frac{a^2+a}{b}+\frac{b^2+b}{c}+\frac{c^2+c}{a};$$
$$\sum_{cyc}{\frac{0\cdot
a+b^4}{c^3-d}}=\frac{b^4}{c^3-d}+\frac{c^4}{d^3-a}+\frac{d^4}{a^3-b}+\frac{a^4}{b^3-c}.$$

\begin{api}
  $$\sum_{sym}{f(A_{0})=f(A_{1})+f(A_{2})+...+f(A_{n!})}.$$ Čia $A_{i}$ -
  visos aibės $A_{0}$ narių perstatos. Kadangi perstatų yra $n!$, tai
  tiek dėmenų ir gausime.
\end{api}

Kitaip sakant, simetrinė suma - tai funkcijų suma, kur funkcijų argumentai
yra visos jų perstatos. Pavyzdžiui:
$$\sum_{sym}{\frac{a^2+b}{c^3}}=\frac{a^2+b}{c^3}+\frac{b^2+a}{c^3}+
\frac{c^2+b}{a^3}+\frac{b^2+c}{a^3}+\frac{a^2+c}{b^3}+\frac{c^2+a}{b^3}.$$

Analogiškai apibrėžiamos ir sandaugos
$$\prod_{cyc}{f(A_0)}\hspace{1cm}\mbox{bei}\hspace{1cm}
\prod_{sym}{f(A_0)}.$$

\begin{pavnr}[L.M.]
  Tegu $a,b,c$ bus tokie realieji skaičiai, kad $abc=1/2$. Įrodykite, kad
  $$\sum_{cyc}{\frac{1}{a^2+b^2}}\leq a+b+c.$$
\end{pavnr}

\begin{proof}[Įrodymas]
  Pagal nelygybę $a^2+b^2\geq 2ab$ gauname $(a^2+b^2)^{-1}\leq (2ab)^{-1}$.
  Taigi $$\sum_{cyc}{\frac{1}{a^2+b^2}}\leq
  \sum_{cyc}{\frac{c}{2abc}}=\frac{a+b+c}{2abc}=a+b+c.$$
\end{proof}

\begin{pavnr}
  Tegu $a,b,c$ - tokie teigiami skaičiai, kad $a^2+b^2+c^2=3$. Įrodykite
  nelygybę $a^3(b+c)+b^3(a+c)+c^3(a+b)\leq 6$.
\end{pavnr}

\begin{proof}[Įrodymas]
  Kadangi $a^2+b^2+c^2=3$, duotoji nelygybė ekvivalenti
  \begin{align*}
    & a^3(b+c)+b^3(a+c)+c^3(a+b)\leq \frac{2}{3}(a^2+b^2+c^2)^2 \\
    & \Leftrightarrow 2\sum_{cyc}a^4+4\sum_{cyc}a^2b^2 \geq 3\sum_{cyc}ab(a^2+b^2) \\
    & \Leftrightarrow \sum_{cyc}a^4+b^4+4a^2b^2-3ab(a^2+b^2)\geq 0 \\
    & \Leftrightarrow \sum_{cyc}a^4-4a^3b+6a^2b^2-4ab^3+b^4 +ab(a^2+b^2)-2ab \cdot ab \geq 0 \\
    & \Leftrightarrow \sum_{cyc}(a-b)^4+\sum_{cyc}ab(a-b)^2 \geq 0,
  \end{align*}
  kas yra akivaizdu.
\end{proof}

\subsubsection{Uždaviniai}

\begin{enumerate}
  \item Įrodykite, kad jei $x,y$ yra teigiami realieji skaičiai, tai
    galioja $x^3+y^3\geq xy(x+y)$.
    %Iš $a^2+b^2\geq2ab$:
    %$x^3+y^3=(x+y)(x^2-xy+y^2)\geq(xy)(2xy-xy)=xy(x+y)$.
  \item Įrodykite, kad visiems realiesiems $a,b,c$ galioja nelygybė
    $a^2+b^2+c^2\geq ab+ac+bc$.
    %Nelygybė ekvivalenti
    %$\frac{1}{2}(a-b)^2+\frac{1}{2}(b-c)^2+\frac{1}{2}(c-a)^2\geq 0$, kas
    %yra akivaizdu.
  \item Įrodykite, kad visiems realiesiems $a,b,c,d$ galioja nelygybė
    $a^2+b^2+c^2+d^2\geq ab+ac+ad$. Kada galios lygybė?
    %Nelygybė ekvivalenti
    %$\frac{a^2}{4}+\left(\frac{a^2}{4}-b\right)^2+\left(\frac{a^2}{4}-c\right)^2+\left(\frac{a^2}{4}-d\right)^2\geq0$,
    %kas yra akivaizdu. Lygybė galios, kai $a=b=c=d=0$.
  \item Įrodykite, kad visiems realiesiems teigiamiems $a,b,c$ galioja
    nelygybė $a^3+b^3+c^3 \geq 3abc$.
    %Nelygybė ekvivalenti
    %$a^3+b^3+c^3-3abc=(a+b+c)(a^2+b^2+c^2-ab-bc-ac)\geq0.$ Iš uždavinio
    %nr. 2 rezultato seka, kad ji yra teisinga.
  \item Duoti realieji $a,b,x,y$, kur $x,y>0$. Įrodykite, kad galioja
    $\frac{a^2}{x}+\frac{b^2}{y}\geq\frac{(a+b)^2}{x+y}$. Kada galios
    lygybė? Kaip galėtume praplėsti (apibendrinti) šią nelygybę?
    %Padauginame nelygybę iš $ab(a+b)$. Gausime
    %$a^2xy+a^2y^2+b^2yx+b^2x^2\geq a^2xy+b^2xy+2abxy \Leftrightarrow
    %(ay-bx)^2\geq0$, kas yra akivaizdu. Lygybė galios, kai
    %$\frac{a}{x}=\frac{b}{y}$. Pagal matematinės indukcijos principą,
    %nelygybę galime praplėsti:\begin{align*}
    %\frac{a_1^2}{b_1}+\frac{a_2^2}{b_2}+\frac{a_3^2}{b_3}+...+\frac{a_n^2}{b_n}&\geq
    %\frac{(a_1+a_2)^2}{b_1+b_2}+\frac{a_3^2}{b_3}+...+\frac{a_n^2}{b_n}\\
    %&\geq\frac{(a_1+a_2+a_3)^2}{b_1+b_2+b_3}+...+\frac{a_n^2}{b_n}\geq...\\
    %&\geq\frac{(a_1+a_2+a_3+...+a_n)^2}{b_1+b_2+b_3+...+b_n}.\end{align*}
    %Lygybė galios, kai
    %$\frac{a_1}{b_1}=\frac{a_2}{b_2}=...=\frac{a_n}{b_n}.$
  \item Įrodykite, kad visiems teigiamiems realiesiems $x,y$ galioja
    nelygybė $\frac{1}{x}+\frac{1}{y}\geq 2\sqrt{\frac{2}{x^2+y^2}}$.
    %Nelygybę keliame kvadratu ir dauginame iš $x^2y^2(x^2+y^2)$. Gausime
    %$y^4+x^2y^2+2xy(x^2+y^2)+x^4+x^2y^2\geq 8x^2y^2.$ Pastebėkime, kad
    %sudėję akivaizdžias nelygybes $x^4+y^4\geq2x^2y^2$ ir
    %$2xy(x^2+y^2)\geq4x^2y^2$, gausime tai, ką reikėjo įrodyti.
  \item Duoti tokie teigiami realieji $a,b,c$, kad $ab+bc+ac=1$. Įrodykite,
    kad $10a^2+10b^2+c^2\geq4$.
    %Nelygybė ekvivalenti $10a^2+10b^2+c^2\geq4ab+4ac+4bc.$ Belieka tik
    %pasukti galvą, kaip sukonstruoti nelygybę iš akivaizdžių kitų:
    %\begin{align*}
    %&8a^2+\frac{1}{2}c^2\geq 4ac;\\ &8b^2+\frac{1}{2}c^2\geq 4bc;\\
    %&2a^2+2b^2\geq4ab. \end{align*}
  \item Tegu $a,b,c$ - teigiami realieji skaičiai, tokie, kad
    $a^2+b^2+c^2=1.$ Raskite minimumą
    $$S=\frac{a^2b^2}{c^2}+\frac{b^2c^2}{a^2}+\frac{c^2a^2}{b^2}.$$
    %Naudosime uždavinio nr. 2 rezultatą: $S\geq a^2+b^2+c^2=1.$ Minimumas
    %$S=1$ pasiekiamas, kai $a=b=c=\frac{1}{\sqrt{3}}$.
  \item Raskite minimalią reiškinio $\Omega=5a^2+6b^2+5c^2+2ac-4a+4c$
    reikšmę, kai $a,b,c$ - realieji skaičiai.
    %$\Omega=(2a-1)^2+(a+c)^2+(2c+1)^2+6b^2-2\geq -2$. Minimumas yra $-2$,
    %pasiekiamas, kai $a=\frac{1}{2}, b=0, c=-\frac{1}{2}$.
  \item Įrodykite, kad jei $x$ ir $y$ yra realieji iš intervalo $(0,1)$,
    tai galioja nelygybė
    $$\frac{1}{1-x^2}+\frac{1}{1-y^2}\geq\frac{2}{1-xy}.$$
    %Naudojame $a+b\geq2\sqrt{ab}$:
    %$\frac{1}{1-x^2}+\frac{1}{1-y^2}\geq\frac{2}{\sqrt{(1-x^2)(1-y^2)}}$.
    %Pastebėkime, kad
    %$(1-x^2)(1-y^2)=1-x^2-y^2+x^2y^2\leq1-2xy+x^2y^2=(1-xy)^2.$ Tai ir
    %užbaigia įrodymą.
  \item Tegu $a,b,c$ - tokie teigiami realieji, kurie tenkina
    $\frac{1}{a}+\frac{1}{b}+\frac{1}{c}\geq a+b+c$. Įrodykite, kad
    $a+b+c\geq3abc$.
    %$\frac{1}{a}+\frac{1}{b}+\frac{1}{c}\geq
    %a+b+c\Leftrightarrow\frac{3(ab+bc+ac)}{a+b+c}\geq3abc.$ Tuomet,
    %belieka įrodyti $a+b+c\geq\frac{3(ab+bc+ac)}{a+b+c}\Leftrightarrow
    %a^2+b^2+c^2\geq ab+bc+ac$, o remiantis užd. nr. 2, tai yra įrodyta.
  \item \text{[LitKo 2006 \textit{(sąlyga mažumėlę modifikuota)}]} Tegu
    $$E=5(x^2+y^2+z^2)+6(xy+yz+zx)-4(13x+15y+16z)+\Psi.$$ Raskite minimalią
    E reikšmę, kai $x,y,z$ yra realieji skaičiai, o $\Psi$ - jūsų
    mėgstamiausias realusis skaičius.
    %Tegu
    %$E=(x+y-a)^2+(x+z-b)^2+(y+z-c)^2+(x-d)^2+(y-e)^2+(z-f)^2+2(x+y+z-k)^2+C
    %\geq C$. Kvadratus parinkome tokius, kad viską sudauginus koeficientai
    %prie kvadratų ir narių $xy$, $xz$, $yz$ atitiktų originalią $E$
    %išraišką, nepriklausomai nuo $a,b,c,d,e,f,k$. Tuomet
    %\begin{equation*}\left\{ \begin{array}{ll} -2xa-2xb-2xd-4xk=-52x, & \\
    %-2ya-2yc-2ye-4yk=-60y, & \\ -2zb-2zc-2zf-4zk=-64z & \end{array}
    %\right. \Rightarrow \left\{ \begin{array}{ll} a+b+d+2k=26, &  \\
    %a+c+e+2k=30, &  \mbox{(1)} \\ b+c+f+2k=32, & \end{array} \right.
    %\end{equation*} Kad $E$ minimumas būtų $C$, visi kvadratai turi būti
    %lygūs 0: \begin{equation*}\left\{ \begin{array}{ll} x+y-a=0, &  \\
    %x+z-b=0, &  \\ y+z-c=0, &  \\ x-d=0, &  \\ y-e=0, &  \\ z-f=0, &  \\
    %x+y+z-k=0, & \end{array} \right. \Rightarrow \left\{ \begin{array}{ll}
    %d+e=a, &  \\ d+f=b, &  \\ e+f=c, &  \\ d+e+f=k, & \end{array}
    %\right.\tag{2}\end{equation*} Iš (1) ir (2) sudarę bendrą sistemą ir
    %ją išsprendę gausime $a=4$, $b=5$, $c=7$, $d=1$, $e=3$, $f=4$, $k=8$,
    %o $a^2+b^2+c^+d^2+e^2+f^2+k^2=244$. Taigi,
    %$E=(x+y-4)^2+(x+z-5)^2+(y+z-7)^2+(x-1)^2+(y-3)^2+(z-4)^2+2(x+y+z-8)^2-244+\Psi\geq
    %\Psi-244.$ Vadinasi, $E$ minimumas yra $\Psi-244$, o jis pasiekiamas,
    %kai $x=1$, $y=3$, $z=4$.
  \item \text{[LitMo 1987]} Įrodykite, kad teigiamiems realiesiems $a,b,c$
    galioja $$\frac{a^3}{a^2+ab+b^2} + \frac{b^3}{b^2+bc+c^2} +
    \frac{c^3}{c^2+ac+a^2}\geq\frac{a+b+c}{3}.$$
    %\begin{eqnarray*} \Leftrightarrow
    %\sum_{cyc}{\frac{a^3}{a^2+ab+b^2}-\frac{a}{3}}&\geq& 0. \hspace{1cm}
    %\mbox{Naudojame uždavinio nr.1 rezultatą:}\\
    %\sum_{cyc}{\frac{a^3}{a^2+ab+b^2}-\frac{a}{3}}&=&\sum_{cyc}{\frac{3a^3-a^3-a^2b-ab^2}{3(a^2+ab+b^2)}}\\
    %&\geq&\sum_{cyc}{\frac{2a^3-a^3-b^3}{3(a^2+ab+b^2)}}\\ &=&
    %\sum_{cyc}{\frac{a-b}{3}}=0. \end{eqnarray*}
  \item \text{[USAMO 1998]} Įrodykite, kad teigiamiems realiesiems $a,b,c$
    galioja $$\frac{1}{a^3+b^3+abc} + \frac{1}{b^3+c^3+abc} +
    \frac{1}{c^3+a^3+abc}\leq\frac{1}{abc}.$$
    %Naudojame uždavinio nr. 1 rezultatą: \begin{eqnarray*} \mbox{KAIRĖ
    %PUSĖ}&\leq&\frac{1}{ab(a+b)+abc}+\frac{1}{bc(b+c)+abc}+\frac{1}{ac(a+c)+abc}\\
    %&=&\frac{c}{abc(a+b+c)}+\frac{a}{abc(a+b+c)}+\frac{b}{abc(a+b+c)}\\
    %&=&\frac{a+b+c}{abc(a+b+c)}\\ &=&\frac{1}{abc}. \end{eqnarray*}
  \item \text{[IMO 1996 Shortlist]} Tegu $a,b,c$ bus tokie teigiami
    realieji, kad $abc=1$. Įrodykite, kad
    $$\frac{ab}{a^5+b^5+ab}+\frac{bc}{b^5+c^5+bc}+\frac{ca}{c^5+a^5+ca}\leq1.$$
    %\textit{Lema.} Jei $x,y$ - teigiami realieji, tai $x^5+y^5\geq
    %x^2y^2(x+y)$.\\ \noindent\textit{Lemos įrodymas.}
    %\begin{align*}
    % x^5+y^5 &=(x+y)(x^4-x^3y+x^2y^2-xy^3+y^4)\\
    %         &=(x+y)((x-y)^2(x^2+xy+y^2)+x^2y^2) \\
    %         &\geq x^2y^2(x+y).
    %\end{align*}
    %
    %Naudodami sąlygą $abc=1$, nelygybę pertvarkome:
    %\begin{eqnarray*} \mbox{KAIRĖ
    %PUSĖ}&=&\sum_{cyc}{\frac{a^2b^2c}{a^5+b^5+a^2b^2c}}\\
    %&\leq&\sum_{cyc}{\frac{a^2b^2c}{a^2b^2(a+b)+a^2b^2c}}\\
    %&=&\sum_{cyc}{\frac{c}{a+b+c}}\\ &=&1. \end{eqnarray*}
  \item \text{[IMO 2005]} Duota, kad $a,b,c$ - realieji, tokie, kad $abc
    \geq 1$. Įrodykite, kad galioja nelygybė $$\frac{a^5-a^2}{a^5+b^2+c^2}
    + \frac{b^5-b^2}{b^5+a^2+c^2} + \frac{c^5-c^2}{c^5+a^2+b^2}\geq 0 .$$
    %\textit{Lema 1.} $b^3c+bc^3\leq b^4+c^4$.\\ \noindent\textit{Lemos 1
    %įrodymas.} $\Leftrightarrow b^3(b-c)+c^3(c-b)\geq0 \Leftrightarrow
    %(b-c)^2(b^2+bc+c^2)\geq0$. Jei $bc\geq0$, nelygybė akivaizdi, o jei
    %$bc<0$, tenka įrodinėti $b^2+bc+c^2\geq0$: nelygybė ekvivalenti
    %$(b+c)^2\geq bc$, kas yra akivaizdu.\hfill{$\square$} \\
    %\textit{Lema 2.} $a^2bc\leq \frac{1}{2}a^2b^2+\frac{1}{2}a^2c^2$.\\
    %\noindent\textit{Lemos 2 įrodymas.} $\Leftrightarrow (ab-ac)^2\geq0$,
    %kas yra akivaizdu.\hfill{$\square$} \\ Naudodami sąlygą $abc\geq1$,
    %nelygybę pertvarkome: \begin{align*} \mbox{KAIRĖ PUSĖ}&\geq
    %\sum_{cyc}{\frac{a^5-a^2\cdot
    %abc}{a^5+abc(b^2+c^2)}}=\sum_{cyc}{\frac{a^4-a^2bc}{a^4+b^3c+bc^3}}\\
    %&\geq\sum_{cyc}{\frac{a^4-\frac{1}{2}a^2b^2-\frac{1}{2}a^2c^2}{a^4+b^3c+bc^3}}
    %\tag{Lema 2}\\
    %&\geq\frac{a^4-a^2b^2-a^2c^2+b^4-b^2c^2+c^4}{a^4+b^4+c^4} \tag{Lema
    %1}\\
    %&=\frac{1}{2}\cdot\frac{(a^2-b^2)^2+(b^2-c^2)^2+(c^2-a^2)^2}{a^4+b^4+c^4}\\
    %&\geq 0.\end{align*}
  \item \text{[Vascile Cartoaje]} Įrodykite, kad realiesiems $a,b,c$
    galioja $$(a^2+b^2+c^2)^2\geq3(a^3b+b^3c+c^3a).$$
    %Pastebime, kad galioja tapatybė: $$
    %(a^2+b^2+c^2)^2-3(a^3b+b^3c+c^3a)=\frac{1}{2}\sum_{cyc}{(a^2-2ab+bc-c^2+ca)^2}\geq
    %0.$$
\end{enumerate}

\newpage
\subsection{Vidurkių nelygybės}

\begin{api}
  Duoti teigiami realieji $x_{1},x_{2},...,x_{n}$. Laipsnio $r$ vidurkis yra
  žymimas $M_{r}(x)$ ir apibrėžiamas
  $$M_{r}(x)=\left(\frac{x_{1}^r+x_{2}^r+...+x_{n}^r}{n}\right)^{1/r}.$$
\end{api}

\begin{itemize}
  \item $M_{1}(x_{1},x_{2},...,x_{n})$ yra žymimas $A(x_{1},x_{2},...,x_{n})$ ir vadinamas aritmetiniu vidurkiu.
  \item $M_{2}(x_{1},x_{2},...,x_{n})$ yra žymimas $S(x_{1},x_{2},...,x_{n})$ ir vadinamas kvadratiniu vidurkiu.
  \item $M_{-1}(x_{1},x_{2},...,x_{n})$ yra žymimas $H(x_{1},x_{2},...,x_{n})$ ir vadinamas harmoniniu vidurkiu.
  \item Nors iš pateiktos išraiškos sunku apibrėžti $M_{0}$, yra žinoma,
    kad kai $r\rightarrow0$, tai $M_{r}(x)\rightarrow
    G(x_{1},x_{2},...,x_{n})=(x_{1}x_{2}...x_{n})^{1/n}$, kas yra vadinama
    geometriniu vidurkiu.
\end{itemize}

\begin{thm}[Bendroji vidurkių nelygybė]
  Jei $x=(x_{1},x_{2},...,x_{n})$ yra realiųjų teigiamų skaičių aibė, tai su
  $r\geq s$ galios nelygybė $M_{s}(x)\geq M_{r}(x)$. Lygybė bus pasiekiama
  tada ir tik tada, kai $x_{1}=x_{2}=...=x_{n}$.
\end{thm}

Nelygybė yra įrodoma su H\"{o}lder'io nelygybe, su kuria skaitytoją
supažindinsime gerokai vėliau.

Dažniausiai naudojamos vidurkių nelygybės yra atskiri bendrosios teoremos
atvejai.

\begin{thm}[AM-GM nelygybė]
  Jei $x_{1},x_{2},...,x_{n}$ yra teigiami sveikieji, tai galioja
  $$\frac{x_{1}+x_{2}+...+x_{n}}{n}\geq \sqrt[n]{x_{1}x_{2}...x_{n}}.$$
\end{thm}

\begin{proof}[Įrodymas]
  Yra beveik 40 šios nelygybės įrodymo būdų. Čia pateiksime vieno didžiausio
  visų laikų matematikos korifėjaus prancūzo Augustin-Louis Cauchy įrodymą.

  Kai $n=1$ ir $n=2$, nelygybė teisinga. Įrodysime, kad jei nelygybė teisinga
  su $n$, tai ji teisinga su $2n$:
  \begin{eqnarray*}
    \frac{x_{1}+x_{2}+...+x_{2n}}{2n} &=&
    \frac{1}{2}\left(\frac{x_{1}+x_{2}+...+x_{n}}{n} +
    \frac{x_{n+1}+x_{n+2}+...+x_{2n}}{n}\right) \\
    & \geq & \frac{1}{2}\left(\sqrt[n]{x_{1}x_{2}...x_{n}}+\sqrt[n]{x_{n+1}x_{n+2}...x_{2n}}\right) \\
    & \geq & \frac{1}{2}\cdot 2\sqrt{\sqrt[n]{x_{1}x_{2}...x_{2n}}} \\
    & = & \sqrt[2n]{x_{1}x_{2}...x_{2n}}
  \end{eqnarray*}
  Taigi, nelygybė yra teisinga, kai $n$ - dvejeto laipsnis. Jei $n$ nėra
  dvejeto laipsnis, tai būtinai rasime tokį $m>n$, kuris yra dvejeto laipsnis.
  Tegu tada $\alpha$ - tų $n$ skaičių aritmetinis vidurkis. Tuomet
  \begin{eqnarray*}\alpha&=&\frac{x_{1}+x_{2}+...+x_{n}}{n}\\
    &=&\frac{x_{1}+x_{2}+...+x_{n}+(m-n)\alpha}{m}\\&\geq&
    \sqrt[m]{x_{1}\cdot x_{2}\cdot ...\cdot
    x_{n}\cdot\alpha^{(m-n)}}\\&\Rightarrow& \alpha\geq\sqrt[n]{a_1\cdot
    a_2\cdot ... \cdot a_n}.
  \end{eqnarray*}
  Tai užbaigia įrodymą. Taigi,
  nelygybė yra teisinga visiems $n$, o lygybės atvejis galios tada ir tik
  tada, kai $x_{1}=x_{2}=...=x_{n}$.
\end{proof}

\begin{pastaba}
  Kitos plačiai naudojamos šios nelygybės formos:
  \begin{itemize}
    \item $x_{1}+x_{2}+...+x_{n}\geq n\sqrt[n]{x_{1}x_{2}...x_{n}};$
    \item $ x_{1}x_{2}...x_{n}\leq\left(\frac{x_{1}+x_{2}+...+x_{n}}{n}\right)^n$.
  \end{itemize}
\end{pastaba}

\begin{thm}[SM-AM nelygybė]
  Jei $x_{1},x_{2},...,x_{n}$ yra teigiami realieji, tai galioja
  $$\sqrt{\frac{x_{1}^2+x_{2}^2+...+x_{n}^2}{n}}\geq
  \frac{x_{1}+x_{2}+...+x_{n}}{n}.$$ Lygybė galios tada ir tik tada, kai
  $x_{1}=x_{2}=...=x_{n}$.
\end{thm}

\begin{proof}[Įrodymas]
  Kaip lemą naudosime ankstesnio skyrelio penkto uždavinio rezultatą. \\
  \textit{Lema.} Jei $x_{1},x_{2},...,x_{n}$ ir $y_{1},y_{2},...,y_{n}$ -
  teigiami realieji, tai galioja nelygybė
  $$\frac{x_{1}^2}{y_{1}}+\frac{x_{2}^2}{y_{2}}+...+\frac{x_{n}^2}{y_{n}}
  \geq \frac{(x_{1}+x_{2}+...+x_{n})^2}{y_{1}+y_{2}+...+y_{n}}.$$

  Jei taikysime lemą su $y_{1}=y_{2}=...=y_{n}=n$, tai ir gausime norimą
  nelygybę. Lygybės atvejis bus tada ir tik tada, kai
  $x_{1}=x_{2}=...=x_{n}$.
\end{proof}

Iš šių dviejų teoremų seka trečioji.

\begin{thm}[SM-GM nelygybė]
  Jei $x_{1},x_{2},...,x_{n}$ yra teigiami realieji, tai galioja
  $$\sqrt{\frac{x_{1}^2+x_{2}^2+...+x_{n}^2}{n}}\geq
  \sqrt[n]{x_{1}x_{2}...x_{n}}.$$ Lygybė galios tada ir tik tada, kai
  $x_{1}=x_{2}=...=x_{n}$.
\end{thm}

Dažnai naujai išvestų nelygybių teisingumą reikia tikrinti, juk nenorime
bandyti įrodyti neteisingų. Yra žinoma Muirhead'o nelygybė,
apibendrinanti AM-GM nelygybę, kuri yra dažniausiai taikoma iš vidurkių
nelygybių. Šiai naujajai nelygybei įvesime keletą apibrėžimų ir žymėjimų.

\begin{api} Žymėsime $$T[a_1,a_2,...,a_n]=\sum_{sym}{f(x_1,x_2,...,x_n)},$$
  kai $f(x_1,x_2,...,x_n)=x_1^{a_1}\cdot x_2^{a_2}\cdot...\cdot x_n^{a_n}$, o
  $a_1,a_2,...,a_n$ ir $x_1,x_2,...,x_n$ - teigiami realieji
  skaičiai.
\end{api}

\begin{api} Seka $A=\{a_1,a_2,...,a_n\}$ mažoruoja (angl. \textit{majorize})
  seką $B=\{b_1,b_2,...,b_n\}$ (žymėsime $A \succ B$), jeigu tenkinamos trys
  sąlygos:
  \begin{itemize}
    \item $a_1+a_2+a_3+..+a_n=b_1+b_2+b_3+...+b_n;$
    \item $a_1\geq a_2\geq a_3 \geq ... \geq a_n\geq 0$ ir $b_1\geq b_2\geq
      b_3\geq...\geq b_n\geq0;$
    \item $a_1+a_2+...+a_i\geq b_1+b_2+...+b_i$, su visais $0<i<n$.
  \end{itemize}
\end{api}

Jau galime formuluoti teoremą:

\begin{thm}[Muirhead]
  Jei $A=\{a_1,a_2,...,a_n\}$ ir $B=\{b_1,b_2,...,b_n\}$ yra teigiamų
  realiųjų skaičių sekos, ir $A\succ B$, tai galioja nelygybė
  $$T[A]\geq T[B].$$ Lygybė galios tada, kai sekos $A$ ir $B$ yra identiškos.
\end{thm}

\begin{pastaba} Nors Muirhead'o nelygybė turi normalų teoremos statusą, ji
  nėra pripažįstama kaip dalis oficialaus olimpiados uždavinio sprendimo.
  Ji dažniausiai naudojama nustatyti, ar naujai gautą nelygybę galime įrodyti
  tinkamai pritaikę AM-GM nelygybę. Pats AM-GM nelygybės taikymas yra grynai
  techninė problema, kuri atskirais atvejais yra lengvai išsprendžiama.
\end{pastaba}

Iliustruokime naujas žinias keliais pavyzdžiais.

\noindent \textbf{Pavyzdys.} Tarkime, sprendėme sprendėme kokį labai įdomų
uždavinį ir gavome, kad lieka įrodyti
\begin{eqnarray*}a^6b^2c+a^6c^2b+b^6c^2a+b^6a^2c+c^6a^2b+c^6b^2a\geq\\
  a^5b^4+a^5c^4+b^5c^4+b^5a^4+c^5a^4+c^5b^4.
\end{eqnarray*}
Vienintelė mintis, kuri šauna į galvą, pamačius tokią nelygybę yra: ,,Blogai.''
Pastebėkime, kad kairė nelygybės pusė yra, taip sakant, $T[6,2,1]$, o dešinė -
$T[4,5,0]$. Šios dvi laipsnių sekos nemažoruoja. Vadinasi, Muirhead'o nelygybės
taikyti negalima. Tai reikš, kad AM-GM nelygybė yra per silpna, o tai byloja, kad
problema yra pakankamai sudėtinga, jei ši nelygybė yra apskritai teisinga. Jei
gautume, kad su kuriuo nors kintamųjų rinkiniu nelygybė yra neteisinga, teks
sugrįžti prie pradinės nelygybės.

\noindent \textbf{Pavyzdys.} Jei turėtume panašią į anktesnio pavyzdžio,
bet vos kitokią nelygybę
\begin{eqnarray*}a^6b^2c+a^6c^2b+b^6c^2a+b^6a^2c+c^6a^2b+c^6b^2a\geq
  \\a^5b^3c+a^5c^3b+b^5c^3a+b^5a^3c+c^5a^3b+c^5b^3a,
\end{eqnarray*}
tai matytume, kad kairės pusės seka $T[6,2,1]$ mažoruoja dešinės $T[5,3,1]$
pusės seką ir nelygybė yra teisinga. Pilnam įrodymui trūksta tik tinkamos
AM-GM nelygybės formos. Štai kaip galime ją konstruoti: matome, kad
dešinėje mažiausias laipsnis yra 1, kaip ir kairėje. Taigi, norėdami gauti
narį, pavyzdžiui, $b^5a^3c$, iš kairės pusės galime naudoti tik narius
$a^6b^2c$ ir $b^6a^2c$, nes visi kiti prie $c$ duos laipsnį, didesnį už 1.
Tebūnie prie šių dalių esantys koeficientai atitinkamai $k$ ir $l$. Pagal
AM-GM: $$ka^6b^2c+lb^6a^2c\geq(k+l)\sqrt[k+l]{a^{6k+2l}b^{6l+2k}c^{k+l}}.$$
Tuomet, kadangi turime gauti narį $b^5a^3c$, spręsime lygčių sistemą:
  $$\left\{
  \begin{array}{lr}
    6k+2l=3(k+l)\\
    6l+2k=5(k+l)
  \end{array}
  \right.
  \Rightarrow l=3k.$$ Ir tikrai, kai $l=1$, o $k=3$, pagal AM-GM bus:
  $$a^6b^2c+3b^6a^2c\geq4\sqrt[4]{b^{20}a^{12}c^4}=4b^5a^3c.$$ Pritaikę
  nelygybę simetrinėms sumoms ir gausime tai, ką reikėjo įrodyti:
  $$4\sum_{sym}{a^6b^2c}=\sum_{sym}{a^6b^2c+3b^6a^2c}\geq4\sum_{sym}{b^5a^3c}.$$

Kai kurie skyrelyje pateikti uždaviniai yra tiesioginės Muirhead'o
nelygybės išvados. Tikimės, kad skaitytojui bus drąsiau juos spręsti
žinant, kad nelygybės tikrai galioja.

Vidurkių nelygybės yra neatsiejamos nuo homogeniškumo sąvokos, tad pats
laikas su ja susipažinti. Kad geriau suvoktume, kaip atpažinti homogenišką
nelygybę, susipažinsime su funkcijos laipsnio sąvoka.

Tegu $f(a_{1},a_{2},a_{3},...,a_{n})$ bus tiesiog funkcija nuo kintamųjų
$a_{1},a_{2},a_{3},...,a_{n}$.

Jei turima funkcija yra vienanaris, t.y.:
$f(a_{1},a_{2},a_{3},...,a_{n})=a_{1}^{\alpha_{1}}a_{2}^{\alpha_{2}}a_{3}^{\alpha_{3}}...a_{n}^{\alpha_{n}}$,
tai vienanario laipsnis bus $\text{deg } f(a_{1},a_{2},a_{3},...,a_{n})=\alpha_{1}+\alpha_{2}+\alpha_{3}+...+\alpha_{n}$.
\begin{pav} 3-io laipsnio vienanariai yra
    $a^3,b^2c,\frac{a^4}{d},\frac{a^7}{bc^3}$.
\end{pav}

Sudėdami ar atimdami vienanarius, gausime vis naujas funkcijas, kurių laipsnius
galėsime nustatyti pasinaudodami keliomis taisyklėmis. Tegu $f(A_{i}), g(A_{i}), h(A_{i})$ -
funkcijos, kur $A_{i}$ - kokia nors realiųjų kintamųjų aibė.

\begin{itemize}
\item Jei turime $f(A_{1})\neq 0$ ir $f(A_{1})=g(A_{2}) \pm h(A_{3})$, o
	$\text{deg } g(A_{2}) = \text{deg } h(A_{2})$, tai $\text{deg }f(A_{1})=\text{deg }g(A_{2})=\text{ deg }h(A_{3})$.
\item Jei $f(A_{1})=g(A_{2})\cdot h(A_{3})$, tai $\text{deg }f(A_{1}) = \text{deg } g(A_{2})+\text{deg }h(A_{3})$.
\end{itemize}
\begin{pav} Tokias taisykles ir jų derinius taikydami galėsime skaičiuoti
kai kurių funkcijų laipsnius: $\frac{a^3}{b+a}$	laipsnis bus 2, $\frac{\sqrt[3]{a^2-c^2}}{\sqrt[5]{a^7-b^7+c^7}}$
laipsnis bus $\frac{2}{3}-\frac{7}{5}=-\frac{11}{15}$. \textit{Dėmesio!}
Tokios funkcijos kaip $f(a,b)=\frac{a^3}{a^2-b}$ laipsnio skaičiuoti negalime.
\end{pav}

Funkciją (nelygybę) galėsime vadinti homogenine, jei ją galima pertvarkyti
į pavidalą $h(A)=\sum{f_{i}(A_{j})}$ ir visų funkcijų $f_{i}(A_{j})$
laipsniai lygūs.
\\
\\ \indent Homogeniškumo oficialus apibrėžimas:
\begin{api}
Jei $h(A)$yra funkcija nuo kintamųjų aibės $A=\{a_1,a_2,\ldots,a_n\}$, tai
$h$ yra homogeninė funkcija tada ir tik tada, kai
$h(ta_{1},ta_{2},ta_{3},...,ta_{n})=t^n h(a_{1},a_{2},a_{3},...,a_{n})$,
kur $t$ - bet koks teigiamas skaičius. \end{api}

Jei yra duota nehomogeninė nelygybė, tačiau taip pat yra duota ir papildoma
sąlyga, dažnai naudinga nelygybę pertvarkyti taip, kad ji taptų
homogenine, na o tada, nelygybė turės būti teisinga net tada, kai
kintamieji netenkins duotos papildomos sąlygos. Tą jau esame atlikę kelis
kartus, net ir nežinodami homogeniškumo sąvokos.

Pažymėtina, kad visos vidurkių nelygybės galioja tik su teigiamais
realiaisiais skaičiais, o lygybės atvejis pasiekiamas, kai visi kintamieji
lygūs. Nors tai nėra labai sudėtingas dalykas, dažnai svarbu atkreipti
dėmesį, kad jis būtų išlaikomas, ypač sprendžiant nehomogenines nelygybes
ar ieškant funkcijų ekstremumų.

Nors daugiausiai naudosime AM-GM nelygybę, kartais praverčia ir kitos
vidurkių nelygybės. Pavyzdžiai iliustruos, kad nelygybę galime taikyti tiek
pereinant nuo aritmetinio vidurkio prie geometrinio, tiek atvirkščiai.
Pirmuosiuose nagrinėsime, kas vyksta, kai apibrėžimo sritis yra ribota arba
yra konkreti sąlyga, neleidžianti tiesiogiai taikyti nelygybės. Toliau
pateikiami pavyzdžiai atspindi neretai pasitaikančius atvejus, kada
tiesioginis, ,,aklas'' AM-GM nelygybės taikymas neduoda jokios naudos, o
reikia sugalvoti kaip pertvarkyti duotą nelygybę, ar prisidėti ir atsiimti
papildomų reiškinių, kad taikoma AM-GM nelygybė padėtų pasiekti norimą
rezultatą. Pateiksime ir keletą nehomogeninių nelygybių, kurioms išspręsti
reikės itin daug fantazijos.

\subsubsection{Pavyzdžiai}

\begin{pavnr}
  Duotas realus $a \geq 3$. Raskite $S=a+\frac{1}{a}$ minimumą.
\end{pavnr}

\noindent \textit{Dažna klaida.} Pagal AM-GM nelygybę, $S=a+\frac{1}{a}\geq
2\sqrt{a\cdot\frac{1}{a}}=2 \Rightarrow$ Min $S=2$.

\noindent \textit{Paaiškinimas.} Jei $S$ minimumas yra 2, tai tada
$a=\frac{a}{1}=1$, kas prieštarauja duotai sąlygai, kad $a\geq3$.

\noindent \textit{Sprendimo ieškojimas.} Pastebime, kad $S > a \geq 3$.
Spėjame, kad minimumas bus pasiekiamas, kai $a=3$. Tuomet
$\frac{1}{a}=\frac{1}{3}=\frac{a}{9}$.

\begin{sprendimas}
  $S=a+\frac{1}{a}=\frac{1}{a}+ \frac{a}{9}+\frac{8a}{9}\geq
  2\sqrt{\frac{1}{a}\cdot\frac{a}{9}}+\frac{8 \cdot 3}{9}=\frac{10}{3} $.
  Minimumas bus pasiekiamas, kai $\frac{1}{a}=\frac{a}{9} \Rightarrow a=3. $
\end{sprendimas}

\begin{pastaba}
  Teisingo sprendimo paslaptis šiame uždavinyje, kaip ir kituose panašiuose
  šio skyrelio uždaviniuose, yra teisingo lygybės atvejo atspėjimas.
\end{pastaba}

\begin{pavnr}[Macedonia 1999]
  Realieji teigiami $a,b,c$ tenkina $a^2+b^2+c^2=1$. Raskite minimumą
  $T=a+c+b+\frac{1}{abc}$.
\end{pavnr}

\noindent \textit{Dažna klaida.} Pagal AM-GM nelygybę
$T\geq4\sqrt[4]{a\cdot b\cdot c\cdot \frac{1}{abc}}=4.$ $\Rightarrow$ $T$
minimumas yra 4.

\noindent \textit{Paaiškinimas.} Jei $T$ minimumas yra 4, tai tada
$a=b=c=\frac{1}{abc}=1$, kas prieštarauja duotai sąlygai.

\noindent \textit{Sprendimo ieškojimas.} Kadangi $T$ yra simetrinė,
minimumas greičiausiai bus pasiekiamas, kai $a=b=c=\frac{1}{\sqrt{3}}.$

\begin{sprendimas}
  \begin{eqnarray*}T&=&a+b+c+\frac{1}{9abc}+\frac{8}{9abc}\\
    &\geq & 4 \sqrt[4]{a\cdot b\cdot c\cdot \frac{1}{9abc}}+\frac{8}{9abc}
    \hspace{1.5 cm} \mbox{(AM-GM nelygybė)}\\ &\geq
    &\frac{4}{\sqrt{3}}+\frac{8}{9\left(\sqrt{\frac{a^2+b^2+c^2}{3}}\right)^3}
    \hspace{1.5 cm} \mbox{(SM-GM nelygybė)}\\
    &=&\frac{4}{\sqrt{3}}+\frac{8}{\sqrt{3}}=4\sqrt{3}.
  \end{eqnarray*}
\end{sprendimas}

\begin{pavnr}
  Duota $a,b,c$ - teigiami realieji skaičiai, tokie, kad
  $a+b+c\leq\frac{3}{2}$. Raskite minimumą
  $$S=\sqrt{a^2+\frac{1}{b^2}}+\sqrt{b^2+\frac{1}{c^2}}+\sqrt{c^2+\frac{1}{a^2}}.$$
\end{pavnr}

\noindent\textit{Dažna klaida.} Pagal AM-GM nelygybę
$S\geq3\sqrt[3]{\sqrt{a^2+\frac{1}{b^2}}\cdot\sqrt{b^2+\frac{1}{c^2}}\cdot\sqrt{c^2+\frac{1}{a^2}}}$\\
$\geq3\sqrt[6]{\left(2\sqrt{a^2\cdot\frac{1}{b^2}}\right)\left(2\sqrt{b^2\cdot\frac{1}{c^2}}\right)\left(2\sqrt{c^2\cdot\frac{1}{a^2}}\right)}=3\sqrt[6]{8}=3\sqrt{2}\Rightarrow$
Min $S=3\sqrt{2}.$

\noindent \textit{Paaiškinimas.} Jei $S$ minimumas yra $3\sqrt{2}$, tai
tada $a=b=c=\frac{a}{1}=\frac{b}{1}=\frac{c}{1}=1$, kas prieštarauja duotai
sąlygai.

\noindent \textit{Sprendimo ieškojimas.} Kadangi $S$ yra ciklinė $a,b,c$
išraiška, labai tikėtina, kad minimumas bus pasiekiamas, kai
$a=b=c=\frac{1}{2}.$ Tuomet
$a^2=b^2=c^2=\frac{1}{4}=\frac{1}{16a^2}=\frac{1}{16b^2}=\frac{1}{16c^2}.$

\begin{sprendimas}Visur taikome AM-GM nelygybę:
  \begin{align*}
    S&=\sum_{cyc}{\sqrt{a^2+\frac{1}{16b^2}+...+\frac{1}{16b^2}}}
    \geq\sum_{cyc}{\sqrt{17\sqrt[17]{\frac{a^2}{16^{16}b^{32}}}}}
    \geq\sqrt{17}\cdot\sum_{cyc}{\sqrt[17]{\frac{a}{16^{8}b^{16}}}}\\
    &\geq\sqrt{17}\left(3\sqrt[3]{\prod_{cyc}{\sqrt[17]{\frac{a}{16^8b^{16}}}}}\right)=
    3\sqrt{17}\cdot\sqrt[17]{\frac{1}{16^8a^5b^5c^5}}=\frac{3\sqrt{17}}{2\sqrt[17]{\left(2a\cdot2b\cdot2c\right)^{5}}}\\
    &\geq \frac{3\sqrt{17}}{2\sqrt[17]{\left(\frac{2a+2b+2c}{3}\right)^{15}}}\geq \frac{3\sqrt{17}}{2}.
  \end{align*}
  Minimumas yra $\frac{3\sqrt{17}}{2}$, pasiekiamas, kai $a=b=c=1/2.$
\end{sprendimas}

\begin{pavnr}
  Tegu $a,b,c$ - teigiami realieji, tokie, kad $a+b+c=3$. Raskite
  $S=\sqrt[3]{a(b+2c)}+\sqrt[3]{b(c+2a)}+\sqrt[3]{c(a+2b)}$ maksimumą.
\end{pavnr}

\begin{sprendimas}
  Taikome AM-GM, tačiau priešinga puse:
  \begin{eqnarray*}S&=&\sum_{cyc}{\sqrt[3]{a(b+2c)}}=\sum_{cyc}{\frac{1}{\sqrt[3]{9}}\cdot\sqrt[3]{3a(b+2c)\cdot3}}\\
    &\leq&\frac{1}{\sqrt[3]{9}}\sum_{cyc}{\frac{3a+(b+2c)+3}{3}}=\frac{1}{\sqrt[3]{9}}\cdot\frac{6(a+b+c)+9}{3}=3\sqrt[3]{3}.
  \end{eqnarray*}
  Vadinasi, maksimumas yra $3\sqrt[3]{3}$ ir pasiekiamas, kai $a=b=c=1$.
\end{sprendimas}

\begin{pavnr}
  Įrodykite, kad kai $n$ - natūralusis skaičius didesnis už 1, galioja
  $$\sqrt[n]{1+\frac{\sqrt[n]{n}}{n}}+\sqrt[n]{1-\frac{\sqrt[n]{n}}{n}}<2.$$
\end{pavnr}

\begin{proof}[Įrodymas.] Taikome AM-GM. Akivaizdu, kad lygybės atvejis
  negalios, tad nelygybė bus griežta.
  $$+\left\{
  \begin{array}{ll}
    \sqrt[n]{1+\frac{\sqrt[n]{n}}{n}}<\frac{1}{n}\cdot\left[\left(1+\frac{\sqrt[n]{n}}{n}\right)+n-1\right]=1+\frac{\sqrt[n]{n}}{n^2}\\
    \sqrt[n]{1-\frac{\sqrt[n]{n}}{n}}<\frac{1}{n}\cdot\left[\left(1-\frac{\sqrt[n]{n}}{n}\right)+n-1\right]=1-\frac{\sqrt[n]{n}}{n^2}
  \end{array}
  \right.$$
  Sudėję gausime tai, ką reikėjo įrodyti.
\end{proof}

\begin{pavnr}
  Įrodykite, kad teigiami realieji $a,b,c$ tenkina nelygybę
  $$\frac{a^3}{b^2}+\frac{b^3}{c^2}+\frac{c^3}{a^2}\geq a+b+c.$$
\end{pavnr}

\begin{proof}[Sprendimas]
  Pagal AM-GM nelygybę:
  \begin{eqnarray*}
    \frac{a^3}{b^2}+b+b & \geq & 3\sqrt[3]{\frac{a^3}{b^2}\cdot b\cdot b}=3a, \\
    \frac{b^3}{c^2}+c+c & \geq & 3\sqrt[3]{\frac{b^3}{c^2}\cdot c\cdot c}=3b, \\
    \frac{c^3}{a^2}+a+a & \geq & 3\sqrt[3]{\frac{c^3}{a^2}\cdot a\cdot a}=3c.
  \end{eqnarray*}
  Sudėję šias nelygybes gausime tai, ką ir reikėjo įrodyti.
\end{proof}

\begin{pavnr}
  Įrodykite, kad teigiamiems realiesiems $a,b,c$ galioja
  $$\frac{a^5}{b^3}+\frac{b^5}{c^3}+\frac{c^5}{a^3}\geq\frac{a^4}{b^2}+\frac{b^4}{c^2}+\frac{c^4}{a^2}.$$
\end{pavnr}

\begin{proof}[Įrodymas]
  Pagal AM-GM nelygybę:
  \begin{align*}
     \frac{a^5}{b^3}+\frac{a^5}{b^3}+\frac{a^5}{b^3}+\frac{a^5}{b^3}+b^2 & \geq 5\sqrt[5]{\left(\frac{a^5}{b^3}\right)^4\cdot
b^2}=5\cdot\frac{a^4}{b^2};\\
     \frac{b^5}{c^3}+\frac{b^5}{c^3}+\frac{b^5}{c^3}+\frac{b^5}{c^3}+c^2 & \geq 5\sqrt[5]{\left(\frac{b^5}{c^3}\right)^4\cdot
c^2}=5\cdot\frac{b^4}{c^2};\\
     \frac{c^5}{a^3}+\frac{c^5}{a^3}+\frac{c^5}{a^3}+\frac{c^5}{a^3}+a^2 & \geq 5\sqrt[5]{\left(\frac{c^5}{a^3}\right)^4\cdot
a^2}=5\cdot\frac{c^4}{a^2}.
  \end{align*}

  Sudėję gausime
  \begin{equation}
    4\left(\frac{a^5}{b^3}+\frac{b^5}{c^3}+\frac{c^5}{a^3}\right)+a^2+b^2+c^2\geq5\left(\frac{a^4}{b^2}+\frac{b^4}{c^2}+\frac{c^4}{a^2}\right).\tag{1}
  \end{equation}
  Taip pat:
  \begin{eqnarray*}
    \frac{a^4}{b^2}+b^2 & \geq & 2\sqrt{\frac{a^4}{b^2}\cdot b^2}=2a^2;\\
    \frac{b^4}{c^2}+c^2 & \geq & 2\sqrt{\frac{b^4}{c^2}\cdot c^2}=2b^2;\\
    \frac{c^4}{a^2}+a^2 & \geq & 2\sqrt{\frac{c^4}{a^2}\cdot a^2}=2c^2.
  \end{eqnarray*}
  Sudėję gausime
  \begin{equation}
    \frac{a^4}{b^2}+\frac{b^4}{c^2}+\frac{c^4}{a^2}\geq a^2+b^2+c^2.\tag{2}
  \end{equation}
  Sudėję nelygybes $(1)$ ir $(2)$ gausime tai, ką ir reikėjo įrodyti.
\end{proof}

\begin{pavnr}[Nesbitt'o nelygybė]
  Įrodykite, kad teigiamiems realiesiems $a,b,c$ galioja
  $$\frac{a}{b+c}+\frac{b}{a+c}+\frac{c}{a+b}\geq\frac{3}{2}.$$
\end{pavnr}

\begin{pastaba} Matematikos profesionalai dažnai rungiasi, kuris žino
  daugiau šios nelygybės įrodymo būdų. Kvalifikacinis raundas - 4.
  Kol kas pateiksime tik vieną. Nesbitt'o nelygybė yra dalinis Shapiro
  nelygybės atvejis.
\end{pastaba}

\begin{proof}[Įrodymas]
  Tegu $S=\frac{a}{b+c}+\frac{b}{a+c}+\frac{c}{a+b}$,
  $A=\frac{b}{b+c}+\frac{c}{a+c}+\frac{a}{a+b}$,
  $B=\frac{c}{b+c}+\frac{a}{a+c}+\frac{b}{a+b}$.
  Tada pagal AM-GM
  \begin{eqnarray*}
    A+S&=&\frac{a+b}{b+c}+\frac{b+c}{a+c}+\frac{c+a}{a+b}\geq3\sqrt[3]{\frac{a+b}{b+c}\cdot\frac{b+c}{a+c}\cdot\frac{c+a}{a+b}}=3;\\
    B+S&=&\frac{a+c}{b+c}+\frac{b+a}{a+c}+\frac{c+b}{a+b}\geq3\sqrt[3]{\frac{a+c}{b+c}\cdot\frac{b+a}{a+c}\cdot\frac{c+b}{a+b}}=3; \hspace{1cm}\mbox{be
to,}\\
    A+B&=&3.
  \end{eqnarray*}
  $\Rightarrow S=\frac{A+S+B+S-A-B}{2}\geq\frac{3+3-3}{2}=\frac{3}{2}.$
\end{proof}

\begin{pavnr}
  Įrodykite, kad tokiems realiesiems teigiamiems $a,b,c$, kur $a+b+c=3$,
  galioja
  $$\frac{a^3}{(a+b)(a+c)}+\frac{b^3}{(b+c)(b+a)}+\frac{c^3}{(c+a)(c+b)}\geq\frac{3}{4}.$$
\end{pavnr}

\begin{proof}[Įrodymas]
  Duotą nelygybę verčiame homogenine naudodami duotą sąlygą ir pertvarkome:
  $$\Leftrightarrow
  \frac{1}{a+b+c}\cdot\sum_{cyc}{\left(\frac{a^3}{(a+b)(a+c)}+\frac{a+b}{8}+\frac{a+c}{8}-\frac{a+b}{8}-\frac{a+c}{8}\right)}\geq\frac{1}{4}.$$
  Sprendžiame naudodami AM-GM nelygybę:
  \begin{eqnarray*}
    \mbox{KAIRĖ
    PUSĖ}&\geq&\frac{1}{a+b+c}\cdot\sum_{cyc}{\left(3\sqrt[3]{\frac{a^3}{(a+b)(a+c)}\cdot\frac{a+b}{8}\cdot\frac{a+c}{8}}\right)}-\frac{1}{2}\\
    &=&\frac{1}{a+b+c}\cdot\sum_{cyc}{\frac{3a}{4}}-\frac{1}{2}=\frac{3(a+b+c)}{4(a+b+c)}
    - \frac{1}{2}=\frac{1}{4}.
  \end{eqnarray*}
\end{proof}

\begin{pavnr}
  Įrodykite, kad jei $a,b,c$ - tokie teigiami realieji skaičiai, kad
  $a+b+c=3abc$, tai $\frac{1}{a^3}+\frac{1}{b^3}+\frac{1}{c^3}\geq3.$
\end{pavnr}

\begin{proof}[Įrodymas]
  $a+b+c=3abc\Rightarrow\frac{1}{ab}+\frac{1}{bc}+\frac{1}{ca}=3.$ Pagal AM-GM gausime
  $$2\left(\frac{1}{a^3}+\frac{1}{b^3}+\frac{1}{c^3}\right)+3=\sum_{cyc}{\left(\frac{1}{a^3}+\frac{1}{b^3}+1\right)}\geq3\left(\frac{1}{ab}+\frac{1}{bc}+\frac{1}{ca}\right)=9.$$
\end{proof}

\begin{pavnr}
  Duoti $a,b,c$ yra teigiami realieji skaičiai. Įrodykite, kad
  $$\frac{a}{b}+\sqrt{\frac{b}{c}}+\sqrt[3]{\frac{c}{a}}>\frac{5}{2}.$$
\end{pavnr}

\begin{proof}[Įrodymas]
  Pertvarkome ir naudojame AM-GM nelygybę:
  \begin{align*}
    S & = \frac{a}{b} + \frac{1}{2} \cdot \sqrt{\frac{b}{c}} + \frac{1}{2}
    \cdot \sqrt{\frac{b}{c}} + \frac{1}{3} \sqrt[3]{\frac{c}{a}} +
    \frac{1}{3} \sqrt[3]{\frac{c}{a}} + \frac{1}{3} \sqrt[3]{\frac{c}{a}}\\
      & > 6\sqrt[6]{\frac{a}{b}\cdot\left(\frac{1}{2}\sqrt{\frac{b}{c}}\right)^2\left(\frac{1}{3}\sqrt[3]{\frac{c}{a}}\right)^3}\\
      & = \frac{6}{\sqrt[6]{108}}>\frac{5}{2}.
  \end{align*}
\end{proof}

\subsubsection{Uždaviniai}

\begin{enumerate}
  \item Tegu $a,b$ - teigiami realieji, tokie, kad $a+b\leq1$. Raskite
    $S=ab+\frac{1}{ab}$ minimumą.
    %Naudosime AM-GM nelygybę:
    %$$S=ab+\frac{1}{16ab}+\frac{15}{16ab}\geq2\sqrt{ab\cdot\frac{1}{16ab}}+\frac{15}{16\left(\frac{a+b}{2}\right)^2}
    %\geq\frac{1}{2}+\frac{15}{16\cdot\frac{1}{4}}=4\frac{1}{4}.$$
    %Minimumas yra $4\frac{1}{4}$, pasiekiamas, kai $a=b=\frac{1}{2}$.
  \item Tegu $a,b,c$ - teigiami realieji, tokie, kad
    $a+b+c\leq\frac{3}{2}$. Raskite
    $S=a+b+c+\frac{1}{a}+\frac{1}{b}+\frac{1}{c}$ minimumą.
    %Naudosime AM-GM nelygybę: \begin{eqnarray*}
    %S&=&a+\frac{1}{4a}+b+\frac{1}{4b}+c+\frac{1}{4c}+\frac{3}{4}\left(\frac{1}{a}+\frac{1}{b}+\frac{1}{c}\right)\\
    %&\geq&2\sqrt{a\cdot\frac{1}{4a}}+2\sqrt{b\cdot\frac{1}{4b}}+
    %2\sqrt{c\cdot\frac{1}{4c}}+ \frac{3}{4}\cdot3\sqrt[3]{\frac{1}{abc}}\\
    %&\geq&3+\frac{9}{4}\cdot\frac{1}{\frac{a+b+c}{3}}\\
    %&\geq&3+\frac{9}{4}\cdot\frac{1}{\frac{1}{2}}=7\frac{1}{2}.
    %\end{eqnarray*} $S$ minimumas yra $7\frac{1}{2}$, ir jis pasiekiamas,
    %kai $a=b=c=\frac{1}{2}$.
  \item Tegu $a,b,c$ - teigiami realieji, tokie, kad $a+b+c=1$. Raskite
    maksimalią $S=\sqrt[3]{a+b}+\sqrt[3]{b+c}+\sqrt[3]{a+c}$ reikšmę.
    %Naudosime AM-GM nelygybę: \begin{eqnarray*}
    %S&=&\sum_{cyc}{\sqrt[3]{\frac{9}{4}}\cdot\sqrt[3]{(a+b)\cdot\frac{2}{3}\cdot\frac{2}{3}}}\\
    %&\leq&\sum_{cyc}{\sqrt[3]{\frac{9}{4}}\cdot\frac{a+b+\frac{2}{3}+\frac{2}{3}}{3}}\\
    %&=&\sqrt[3]{\frac{9}{4}}\cdot\frac{2(a+b+c)+4}{3}\\
    %&=&\sqrt[3]{\frac{9}{4}}\cdot\frac{6}{3}=\sqrt[3]{18}.\end{eqnarray*}
    %Maksimumas yra $\sqrt[3]{18}$, o jis pasiekiamas, kai
    %$a=b=c=\frac{1}{3}$.
  \item Tegu $a,b,c$ - teigiami realieji, tokie, kad
    $a\geq 2, b\geq 6, c\geq 12$. Raskite didžiausią galimą reikšmę, kurią
    įgyja
    $$\Gamma=\frac{bc\sqrt{a-2}+ac\sqrt[3]{b-6}+ab\sqrt[4]{c-12}}{abc}.$$
    %Galėsime naudoti AM-GM nelygybę, nes $a-2\geq0; b-6\geq0; c-12\geq0$:
    %$$+\left\{ \begin{array}{ll}
    %bc\sqrt{a-2}=\frac{bc}{\sqrt{2}}\sqrt{(a-2)\cdot2}\leq\frac{bc}{\sqrt{2}}\cdot\frac{(a-2)+2}{2}=
    %\frac{abc}{2\sqrt{2}}, & \\ ca\sqrt[3]{b-6}=
    %\frac{ca}{\sqrt[3]{9}}\sqrt[3]{(b-6)\cdot3\cdot3}\leq\frac{ca}{\sqrt[3]{9}}\cdot\frac{(b-6)+3+3}{3}=
    %\frac{abc}{2\sqrt[3]{9}},&\\ ab\sqrt[4]{c-12}=
    %\frac{ab}{\sqrt[4]{64}}\sqrt[4]{(c-12)\cdot4\cdot4\cdot4}\leq\frac{ab}{\sqrt[4]{64}}\cdot\frac{(c-12)+4+4+4}{4}=
    %\frac{abc}{8\sqrt{2}},& \end{array} \right.$$
    %$\Rightarrow\Gamma\leq\frac{1}{abc}\cdot\left(\frac{abc}{2\sqrt{2}}
    %+\frac{abc}{2\sqrt[3]{9}}+ \frac{abc}{8\sqrt{2}}\right)=
    %\frac{5}{8\sqrt{2}}+\frac{1}{2\sqrt[3]{9}}.$\\ $\Gamma$ įgauna
    %maksimalią reikšmę, kai $a=4$, $b=9$, $c=16$. Ji lygi
    %$\frac{5}{8\sqrt{2}}+\frac{1}{2\sqrt[3]{9}}.$
  \item Įrodykite, kad natūraliesiems $n$ galioja
    $$I=\sqrt{\frac{2+1}{2}}+\sqrt[3]{\frac{3+1}{3}}+...+\sqrt[n]{\frac{n+1}{n}}<n.$$
    %Taikydami AM-GM nelygybę prarandame jos lygybės atvejį, tačiau jis
    %mums ir nereikalingas. \\ $$\mbox{Turime }\sqrt[k]{\frac{k+1}{k}}=
    %\sqrt[k]{\frac{k+1}{k}\cdot\underbrace{1\cdot1\cdot\ldots\cdot1}_{k-1}}<\frac{1}{k}\left(\frac{k+1}{k}+(k-1)\right)=
    %1+\frac{1}{k^2}.$$\\ $\mbox{Tuomet
    %}I<n-1+\frac{1}{2^2}+\frac{1}{3^2}+\ldots+ \frac{1}{n^2}<
    %n-1+\frac{1}{1\cdot2}+\frac{1}{2\cdot3}+\ldots+\frac{1}{(n-1)\cdot
    %n}=n-1+\left(\frac{1}{1}-\frac{1}{2}\right)+
    %\left(\frac{1}{2}-\frac{1}{3}\right)+
    %\ldots+\left(\frac{1}{n-1}-\frac{1}{n}\right)=n-1+\left(1-\frac{1}{n}\right)<n.$
  \item Įrodykite, kad teigiamiems realiesiems $a,b,c$ galioja
    $$\frac{a^3}{b^2}+\frac{b^3}{c^2}+\frac{c^3}{a^2}\geq
    \frac{a^2}{b}+\frac{b^2}{c}+\frac{c^2}{a}.$$
    %Pagal AM-GM: $$+\left\{\begin{array}{ll}
    %7\cdot\frac{a^3}{b^2}+2\cdot\frac{b^2}{c}+
    %\frac{c^2}{a}\geq10\sqrt[10]{\frac{a^{21}b^4c^2}{ab^{14}c^2}}=
    %10\frac{a^2}{b},&\\
    %7\cdot\frac{b^3}{c^2}+2\cdot\frac{c^2}{a}+\frac{a^2}{b}\geq10\sqrt[10]{\frac{b^{21}c^4a^2}{bc^{14}a^2}}=
    %10\frac{b^2}{c},&\\
    %7\cdot\frac{c^3}{a^2}+2\cdot\frac{a^2}{b}+\frac{b^2}{c}\geq10\sqrt[10]{\frac{c^{21}a^4b^2}{ca^{14}b^2}}=
    %10\frac{c^2}{a},&\end{array}\right.$$ Sudedame ir gauname tai, ką
    %reikėjo įrodyti. \begin{pastaba} Šį
    %uždavinį galima daug paprasčiau įrodyti, naudojant nesunkiai įrodomą
    %lemą: Su realiaisiais teigiamais $a,b,c$ galioja
    %$\frac{a^2}{b}+\frac{b^2}{c}+\frac{c^2}{a}\geq a+b+c.$ \end{pastaba}
  \item \text{[Mircea Lascu, \textit{Gazeta Matematic\u{a}}]} Tegu $a, b,
    c$ tokie teigiami realieji skaičiai, kad $abc=1$. Įrodykite nelygybę
    $$\frac{b+c}{\sqrt{a}}+\frac{c+a}{\sqrt{b}}+\frac{a+b}{\sqrt{c}}\geq\sqrt{a}+\sqrt{b}+\sqrt{c}+3.$$
    %Pagal AM-GM nelygybę, galioja šios nelygybės:
    %$$+\left\{\begin{array}{ll}
    %\frac{b+c}{\sqrt{a}}+2\sqrt{a}=\frac{b}{\sqrt{a}}+\sqrt{a}+\frac{c}{\sqrt{a}}+\sqrt{a}\geq2\sqrt{b}+2\sqrt{c},&\\
    %\frac{c+a}{\sqrt{b}}+2\sqrt{b}=\frac{c}{\sqrt{b}}+\sqrt{b}+\frac{a}{\sqrt{b}}+\sqrt{b}\geq2\sqrt{c}+2\sqrt{a},&\\
    %\frac{a+b}{\sqrt{c}}+2\sqrt{c}=\frac{a}{\sqrt{c}}+\sqrt{c}+\frac{b}{\sqrt{c}}+\sqrt{c}\geq2\sqrt{a}+2\sqrt{b},&\\
    %\sqrt{a}+\sqrt{b}+\sqrt{c}\geq3\sqrt[3]{\sqrt{abc}}=3.&\end{array}\right.$$
    %Viską sudėję gausime norimą rezultatą.
  \item Įrodykite, kad teigiamiems realiesiems $a,b,c$ galioja
    $$\frac{a^2}{b^2}+\frac{b^2}{c^2}+\frac{c^2}{a^2}\leq\frac{a^3}{b^3}+\frac{b^3}{c^3}+\frac{c^3}{a^3}.$$
    %Pagal AM-GM: $$+\left\{\begin{array}{ll}
    %\frac{a^3}{b^3}+\frac{a^3}{b^3}+1\geq3\sqrt[3]{\frac{a^6}{b^6}}=3\cdot\frac{a^2}{b^2},&\\
    %\frac{b^3}{c^3}+\frac{b^3}{c^3}+1\geq3\sqrt[3]{\frac{b^6}{c^6}}=3\cdot\frac{b^2}{c^2},&\\
    %\frac{c^3}{a^3}+\frac{c^3}{a^3}+1\geq3\sqrt[3]{\frac{c^6}{a^6}}=3\cdot\frac{c^2}{a^2},&\end{array}\right.$$
    %\begin{eqnarray*}
    %\Rightarrow2\left(\frac{a^3}{b^3}+\frac{b^3}{c^3}+\frac{c^3}{a^3}\right)+
    %3&\geq&2\left(\frac{a^2}{b^2}+\frac{b^2}{c^2}+\frac{c^2}{a^2}\right)+
    %\left(\frac{a^2}{b^2}+\frac{b^2}{c^2}+\frac{c^2}{a^2}\right)\\
    %&\geq&2\left(\frac{a^2}{b^2}+\frac{b^2}{c^2}+\frac{c^2}{a^2}\right)+
    %3\sqrt[3]{\frac{a^2}{b^2}\cdot\frac{b^2}{c^2}\cdot\frac{c^2}{a^2}}\\
    %&=&2\left(\frac{a^2}{b^2}+\frac{b^2}{c^2}+\frac{c^2}{a^2}\right)+3.
    %\end{eqnarray*}
  \item Įrodykite, kad teigiamiems realiesiems $a,b,c$ galioja
    $$\frac{a^2}{b^5}+\frac{b^2}{c^5}+\frac{c^2}{a^5}\geq\frac{1}{a^3}+\frac{1}{b^3}+\frac{1}{c^3}.$$
    %Pagal AM-GM: $$+\left\{\begin{array}{ll}
    %3\cdot\frac{a^2}{b^5}+2\cdot\frac{1}{a^3}\geq5\sqrt[5]{\frac{a^6}{b^{15}a^6}}=5\cdot\frac{1}{b^3},&\\
    %3\cdot\frac{b^2}{c^5}+2\cdot\frac{1}{b^3}\geq5\sqrt[5]{\frac{b^6}{c^{15}b^6}}=5\cdot\frac{1}{c^3},&\\
    %3\cdot\frac{c^2}{a^5}+2\cdot\frac{1}{c^3}\geq5\sqrt[5]{\frac{c^6}{a^{15}c^6}}=5\cdot\frac{1}{a^3}.&
    %\end{array} \right. $$ Sudėję gausime tai, ką reikėjo įrodyti.
  \item Tegu realieji teigiami $a,b,c$ tenkina $a+b+c=1$. Įrodykite, kad
    jiems galioja $(1+a)(1+b)(1+c)\geq8(1-a)(1-b)(1-c).$
    %Naudosime AM-GM nelygybę: \begin{eqnarray*} \mbox{KAIRĖ
    %PUSĖ}&=&(a+b+a+c)(a+b+b+c)(a+c+b+c)\\
    %&\geq&2\sqrt{(a+b)(a+c)}\cdot2\sqrt{(a+b)(b+c)}\cdot2\sqrt{(a+c)(b+c)}\\
    %&=&8(a+b)(a+c)(b+c)\\ &=&8(1-a)(1-b)(1-c). \end{eqnarray*}
  \item \text{[APMO 1998]} Įrodykite, kad teigiamiems realiesiems $a,b,c$
    galioja
    $$\left(1+\frac{a}{b}\right)\left(1+\frac{b}{c}\right)\left(1+\frac{c}{a}\right)\geq2+\frac{2(a+b+c)}{\sqrt[3]{abc}}.$$
    %Duota nelygybė ekvivalenti $\frac{a}{b}+\frac{b}{c}+\frac{c}{a}+\frac{b}{a}+\frac{c}{b}+\frac{a}{c}\geq\frac{2(a+b+c)}{\sqrt[3]{abc}}$.
    %Pagal AM-GM nelygybę: $$+\left\{\begin{array}{ll}
    %\frac{a}{b}+\frac{a}{b}+\frac{b}{c}\geq3\sqrt[3]{\frac{a}{b}\cdot\frac{a}{b}\cdot\frac{b}{c}}=\frac{3a}{\sqrt[3]{abc}},&\\
    %\frac{b}{c}+\frac{b}{c}+\frac{c}{a}\geq3\sqrt[3]{\frac{b}{c}\cdot\frac{b}{c}\cdot\frac{c}{a}}=\frac{3b}{\sqrt[3]{abc}},&\\
    %\frac{c}{a}+\frac{c}{a}+\frac{a}{b}\geq3\sqrt[3]{\frac{c}{a}\cdot\frac{c}{a}\cdot\frac{a}{b}}=\frac{3c}{\sqrt[3]{abc}},&
    %\end{array}\right.$$ \begin{equation*}
    %\Rightarrow\frac{a}{b}+\frac{b}{c}+\frac{c}{a}\geq\frac{a+b+c}{\sqrt[3]{abc}}.\tag{1}\end{equation*}
    %Taip pat: $$+\left\{\begin{array}{ll}
    %\frac{b}{a}+\frac{b}{a}+\frac{a}{c}\geq3\sqrt[3]{\frac{b}{a}\cdot\frac{b}{a}\cdot\frac{a}{c}}=\frac{3b}{\sqrt[3]{abc}},&\\
    %\frac{c}{b}+\frac{c}{b}+\frac{b}{a}\geq3\sqrt[3]{\frac{c}{b}\cdot\frac{c}{b}\cdot\frac{b}{a}}=\frac{3c}{\sqrt[3]{abc}},&\\
    %\frac{a}{c}+\frac{a}{c}+\frac{c}{b}\geq3\sqrt[3]{\frac{a}{c}\cdot\frac{a}{c}\cdot\frac{c}{b}}=\frac{3a}{\sqrt[3]{abc}},&
    %\end{array}\right.$$ \begin{equation*}
    %\Rightarrow\frac{b}{a}+\frac{c}{b}+\frac{a}{c}\geq\frac{a+b+c}{\sqrt[3]{abc}}.\tag{2}\end{equation*}
    %Sudėję (1) ir (2) gausime tai, ką reikėjo įrodyti.
  \item Duoti teigiami realieji $a,b,c,d$. Raskite minimalią reiškinio
    reikšmę:
    $$S=\left(1+\frac{2a}{3b}\right)\left(1+\frac{2b}{3c}\right)\left(1+\frac{2c}{3d}\right)\left(1+\frac{2d}{3a}\right).$$
    %Pagal AM-GM: $$\left\{\begin{array}{ll}
    %1+\frac{2a}{3b}=\frac{1}{3}+\frac{1}{3}+\frac{1}{3}+\frac{a}{3b}+\frac{a}{3b}\geq5\sqrt[5]{\left(\frac{1}{3}\right)^3\cdot\left(\frac{a}{3b}\right)^2}=\frac{5}{3}\left(\frac{a}{b}\right)^{\frac{2}{5}},&\\
    %1+\frac{2b}{3c}=\frac{1}{3}+\frac{1}{3}+\frac{1}{3}+\frac{b}{3c}+\frac{b}{3c}\geq5\sqrt[5]{\left(\frac{1}{3}\right)^3\cdot\left(\frac{b}{3c}\right)^2}=\frac{5}{3}\left(\frac{b}{c}\right)^{\frac{2}{5}},&\\
    %1+\frac{2c}{3d}=\frac{1}{3}+\frac{1}{3}+\frac{1}{3}+\frac{c}{3d}+\frac{c}{3d}\geq5\sqrt[5]{\left(\frac{1}{3}\right)^3\cdot\left(\frac{c}{3d}\right)^2}=\frac{5}{3}\left(\frac{c}{d}\right)^{\frac{2}{5}},&\\
    %1+\frac{2d}{3a}=\frac{1}{3}+\frac{1}{3}+\frac{1}{3}+\frac{d}{3a}+\frac{d}{3a}\geq5\sqrt[5]{\left(\frac{1}{3}\right)^3\cdot\left(\frac{d}{3a}\right)^2}=\frac{5}{3}\left(\frac{d}{a}\right)^{\frac{2}{5}},&
    %\end{array}\right.$$ $\Rightarrow
    %S=\left(1+\frac{2a}{3b}\right)\left(1+\frac{2b}{3c}\right)\left(1+\frac{2c}{3d}\right)\left(1+\frac{2d}{3a}\right)\geq\frac{625}{81}\cdot\left(\frac{a}{b}\cdot\frac{b}{c}\cdot\frac{c}{d}\cdot\frac{d}{a}\right)^{\frac{2}{5}}=\frac{625}{81}.$
    %$S$ Minimumas yra $\frac{625}{81}$. Jis pasiekiamas, kai $a=b=c=d>0.$
  \item Duoti teigiami realieji $a,b,c$ tokie, kad $a+b+c=3$. Įrodykite,
    kad
    $$\frac{a^3}{b(2c+a)}+\frac{b^3}{c(2a+b)}+\frac{c^3}{a(2b+c)}\geq1.$$
    %Naudodami sąlygą, verčiame nelygybę homogenine:
    %$\frac{a^3}{b(2c+a)}+\frac{b^3}{c(2a+b)}+\frac{c^3}{a(2b+c)}\geq
    %\frac{a+b+c}{3}.$ Pagal AM-GM: $$+\left\{\begin{array}{ll}
    %\frac{9a^3}{b(2c+a)}+3b+(2c+a)\geq3\sqrt[3]{\frac{9a^3}{b(2c+a)}\cdot3b(2c+a)}=9a,&\\
    %\frac{9b^3}{c(2a+b)}+3c+(2a+b)\geq3\sqrt[3]{\frac{9b^3}{b(2a+b)}\cdot3c(2a+b)}=9b,&\\
    %\frac{9c^3}{a(2b+c)}+3a+(2b+c)\geq3\sqrt[3]{\frac{9c^3}{b(2b+c)}\cdot3a(2b+c)}=9c,&
    %\end{array}\right.$$ Sudęję ir sutvarkę nelygybę ir gausime tai, ką
    %reikėjo įrodyti.
  \item Duoti teigiami realieji $a,b,c$ tokie, kad $ab+bc+ac=1$. Įrodykite,
    kad
    $$\frac{1}{a(a+b)}+\frac{1}{b(b+c)}+\frac{1}{c(c+a)}\geq\frac{9}{2}.$$
    %Nelygybę galime paversti homogenine, naudodami duotą sąlygą:
    %\begin{align*}
    %&\frac{c(a+b)+ab}{a(a+b)}+\frac{a(b+c)+bc}{b(b+c)}+\frac{b(c+a)+ca}{c(c+a)}\geq\frac{9}{2}\\
    %\Leftrightarrow&\frac{a}{b}+\frac{b}{c}+\frac{c}{a}+\frac{b}{a+b}+\frac{c}{b+c}+\frac{a}{c+a}\geq\frac{9}{2}\\
    %\Leftrightarrow&\frac{a+b}{b}+\frac{b+c}{c}+\frac{c+a}{a}+\frac{b}{a+b}+\frac{c}{b+c}+\frac{a}{c+a}\geq\frac{15}{2}
    %\tag{1} \end{align*} Naudosime AM-GM nelygybę:
    %\begin{eqnarray*}\mbox{KAIRĖ PUSĖ(1)}
    %&=&\frac{a+b}{4b}+\frac{b+c}{4c}+\frac{c+a}{4a}+\frac{b}{a+b}+\frac{c}{b+c}+\frac{a}{c+a}\\
    %&&\hspace{1cm}+\frac{3}{4}\left(\frac{a+b}{b}+\frac{b+c}{c}+\frac{c+a}{a}\right)\\
    %&\geq&6\sqrt[6]{\frac{a+b}{4b}\cdot\frac{b+c}{4c}\cdot\frac{c+a}{4a}\cdot\frac{b}{a+b}\cdot\frac{c}{b+c}\cdot\frac{a}{c+a}}\\
    %&&\hspace{1cm}+\frac{3}{4}\left(3\sqrt[3]{\frac{a}{b}\cdot\frac{b}{c}\cdot\frac{c}{a}}+3\right)\\
    %&=&\frac{15}{2}. \end{eqnarray*}
  \item \text{[Romania Junior Balkan TST 2008]} Duoti teigiami realieji
    skaičiai tenkina $ab+bc+ac=3$. Įrodykite, kad jiems galioja nelygybė
    $$\frac{1}{1+a^2(b+c)}+\frac{1}{1+b^2(a+c)}+\frac{1}{1+c^2(a+b)}\leq\frac{1}{abc}.$$
    %Pagal AM-GM nelygybę: $$3=ab+bc+ac\geq3\sqrt[3]{a^2b^2c^2}\Rightarrow
    %abc\leq1.$$ Pagal duotą sąlygą ir turimą rezultatą:
    %\begin{eqnarray*}\text{KAIRĖ
    %PUSĖ}&=&\sum_{cyc}{\frac{1}{1+a(3-bc)}}=\sum_{cyc}{\frac{1}{1+3a-abc}}\\&\leq&\sum_{cyc}{\frac{1}{3a}}=\frac{ab+ac+bc}{3abc}=\frac{1}{abc}.\end{eqnarray*}
  \item \text{[France Pre-MO 2005]} Įrodykite, kad jei $a,b,c$  - tokie
    teigiami realieji, kad $a^2+b^2+c^2=3$, tai galioja
    $$\frac{ab}{c}+\frac{bc}{a}+\frac{ca}{b}\geq3.$$
    %Nelygybę keliame kvadratu ir sutvarkome:
    %$\Leftrightarrow\frac{a^2b^2}{c^2}+\frac{b^2c^2}{a^2}+\frac{c^2a^2}{b^2}\geq
    %a^2+b^2+c^2.$ Pagal AM-GM: $$+\left\{\begin{array}{ll}
    %\frac{a^2b^2}{c^2}+\frac{b^2c^2}{a^2}\geq2\sqrt{\frac{a^2b^2}{c^2}\cdot\frac{b^2c^2}{a^2}}=2b^2,&\\
    %\frac{b^2c^2}{a^2}+\frac{c^2a^2}{b^2}\geq2\sqrt{\frac{b^2c^2}{a^2}\cdot\frac{c^2a^2}{b^2}}=2c^2,&\\
    %\frac{c^2a^2}{b^2}+\frac{a^2b^2}{c^2}\geq2\sqrt{\frac{c^2a^2}{b^2}\cdot\frac{a^2b^2}{c^2}}=2a^2,&\end{array}\right.
    %$$
    %$\Rightarrow\frac{a^2b^2}{c^2}+\frac{b^2c^2}{a^2}+\frac{c^2a^2}{b^2}\geq
    %a^2+b^2+c^2.$
  \item \text{[Walther Janous, \textit{Crux Mathematicorum}]} Įrodykite,
    kad su teigiamais realiaisiais $x,y,z$ galioja nelygybė
    $$\frac{x}{x+\sqrt{(x+y)(x+z)}}+\frac{y}{y+\sqrt{(y+z)(x+y)}}+\frac{z}{z+\sqrt{(x+z)(y+z)}}\leq1.$$
    %Pagal AM-GM nelygybę: $$(x+y)(x+z)=xy+(x^2+zy)+xz\geq
    %xy+2x\sqrt{yz}+xz=(\sqrt{xy}+\sqrt{xz})^2.$$ Taigi,
    %$$\sum_{cyc}{\frac{x}{x+\sqrt{(x+y)(x+z)}}}\leq\sum_{cyc}{\frac{x}{x+\sqrt{xy}+\sqrt{xz}}}=
    %\sum_{cyc}{\frac{\sqrt{x}}{\sqrt{x}+\sqrt{y}+\sqrt{z}}}=1.$$
  \item \text{[Russia 2002]} Tegu $x,y,z$ - teigiami realieji skaičiai,
    kurie tenkina $x+y+z=3$. Įrodykite nelygybę
    $$\sqrt{x}+\sqrt{y}+\sqrt{z}\geq xy+xz+yz.$$
    %Padauginę iš 2 ir prie abiejų nelygybės pusių pridėję $x^2+y^2+z^2$,
    %gausime $$x^2+2\sqrt{x}+y^2+2\sqrt{y}+z^2+2\sqrt{z}\geq3.$$ Iš AM-GM
    %nelygybės:
    %$$\sum_{cyc}{x^2+\sqrt{x}+\sqrt{x}}\geq\sum_{cyc}{3\sqrt[3]{x^3}}=9.$$
    %Tą ir reikėjo įrodyti.
  \item \text{[IMO 1998 Shortlist]} Tegu $a,b,c$ bus tokie teigiami
    realieji skaičiai, kad $abc=1$. Įrodykite, kad
    $$\frac{a^3}{(1+b)(1+c)}+\frac{b^3}{(1+c)(1+a)}+\frac{c^3}{(1+a)(1+b)}\geq\frac{3}{4}.$$
    %Pagal AM-GM: $$+\left\{\begin{array}{ll}
    %\frac{a^3}{(1+b)(1+c)}+\frac{1+b}{8}+\frac{1+c}{8}\geq3\sqrt[3]{\frac{a^3}{(1+b)(1+c)}\cdot\frac{1+b}{8}\cdot\frac{1+c}{8}}=\frac{3a}{4},&\\
    %\frac{b^3}{(1+c)(1+a)}+\frac{1+c}{8}+\frac{1+a}{8}\geq3\sqrt[3]{\frac{b^3}{(1+c)(1+a)}\cdot\frac{1+c}{8}\cdot\frac{1+a}{8}}=\frac{3b}{4},&\\
    %\frac{c^3}{(1+a)(1+b)}+\frac{1+a}{8}+\frac{1+b}{8}\geq3\sqrt[3]{\frac{c^3}{(1+a)(1+b)}\cdot\frac{1+a}{8}\cdot\frac{1+b}{8}}=\frac{3c}{4},&
    %\end{array}\right.$$
    %$\Rightarrow\frac{a^3}{(1+b)(1+c)}+\frac{b^3}{(1+c)(1+a)}+\frac{c^3}{(1+a)(1+b)}\geq\frac{a+b+c}{2}-\frac{3}{4}\geq\frac{3\sqrt[3]{abc}}{2}-\frac{3}{4}=\frac{3}{4}.$
  \item Duoti teigiami realieji $a,b,c$ tokie, kad
    $a\sqrt{\frac{b}{c}}+b\sqrt{\frac{c}{a}}+c\sqrt{\frac{a}{b}}=3$.
    Įrodykite, kad
    $$\frac{a^6}{b^3}+\frac{b^6}{c^3}+\frac{c^6}{a^3}\geq3.$$
    %Naudodami AM-GM nelygybę gauname:\\
    %$\left(\frac{a^6}{b^3}+\frac{b^6}{c^3}+4\right)+\left(\frac{b^6}{c^3}+\frac{c^6}{a^3}+4\right)+\left(\frac{c^6}{a^3}+\frac{a^6}{b^3}+4\right)\geq6\left(\sqrt[6]{\frac{a^6b^3}{c^3}}+\sqrt[6]{\frac{b^6c^3}{a^3}}+\sqrt[6]{\frac{c^6a^3}{b^3}}\right)=18$\\
    %$\Leftrightarrow2\left(\frac{a^6}{b^3}+\frac{b^6}{c^3}+\frac{c^6}{a^3}\right)+12\geq18\Leftrightarrow\frac{a^6}{b^3}+\frac{b^6}{c^3}+\frac{c^6}{a^3}\geq3.$
  \item \text{[IMO 1990 Shortlist]} Realieji $a,b,c,d$ tenkina
    $ab+bc+cd+da=1$. Įrodykite, kad jie tenkins ir
    $$\frac{a^3}{b+c+d}+\frac{b^3}{c+d+a}+\frac{c^3}{d+a+b}+\frac{d^3}{a+b+c}\geq\frac{1}{3}.$$
    %Pastebėkime, kad \\
    %$(a-b+c-d)^2\geq0\Leftrightarrow(a+b+c+d)^2\geq4(ab+bc+cd+da)=4\Leftrightarrow
    %a+b+c+d\geq2.$ Pagal AM-GM: $$+\left\{\begin{array}{ll}
    %\frac{36a^3}{b+c+d}+2(b+c+d)+6a+3\geq4\sqrt[4]{\frac{36a^3}{b+c+d}\cdot2(b+c+d)\cdot6a\cdot3}=24a,&\\
    %\frac{36b^3}{c+d+a}+2(c+d+a)+6b+3\geq4\sqrt[4]{\frac{36b^3}{c+d+a}\cdot2(c+d+a)\cdot6b\cdot3}=24b,&\\
    %\frac{36c^3}{d+a+b}+2(d+a+b)+6c+3\geq4\sqrt[4]{\frac{36c^3}{d+a+b}\cdot2(d+a+b)\cdot6c\cdot3}=24c,&\\
    %\frac{36d^3}{a+b+c}+2(a+b+c)+6d+3\geq4\sqrt[4]{\frac{36d^3}{a+b+c}\cdot2(a+b+c)\cdot6d\cdot3}=24d,&
    %\end{array}\right.$$ $$\Rightarrow\mbox{KAIRĖ
    %PUSĖ}\geq{\frac{a+b+c+d}{3}}-\frac{1}{3}\geq\frac{2}{3}-\frac{1}{3}=\frac{1}{3}.$$
  \item \text{[Tran Phuong]} Įrodykite, kad su visais teigiamais
    realiaisiais $a,b,c$ galioja
    $$\frac{bc}{a^2}+\frac{ca}{b^2}+\frac{ab}{c^2}+abc\leq\frac{a^7}{b^2c^2}+\frac{b^7}{c^2a^2}+\frac{c^7}{a^2b^2}+\frac{1}{a^2b^2c^2}.$$
    %Pagal AM-GM: $$+\left\{\begin{array}{ll}
    %\frac{bc}{a^2}=\sqrt[3]{\frac{b^7}{a^2c^2}\cdot\frac{c^7}{a^2b^2}\cdot\frac{1}{a^2b^2c^2}}\leq\frac{1}{3}\left(\frac{b^7}{a^2c^2}+\frac{c^7}{a^2b^2}+\frac{1}{a^2b^2c^2}\right),&\\
    %\frac{ca}{b^2}=\sqrt[3]{\frac{c^7}{b^2a^2}\cdot\frac{a^7}{b^2c^2}\cdot\frac{1}{a^2b^2c^2}}\leq\frac{1}{3}\left(\frac{c^7}{b^2a^2}+\frac{a^7}{b^2c^2}+\frac{1}{a^2b^2c^2}\right),&\\
    %\frac{ab}{c^2}=\sqrt[3]{\frac{a^7}{c^2b^2}\cdot\frac{b^7}{c^2a^2}\cdot\frac{1}{a^2b^2c^2}}\leq\frac{1}{3}\left(\frac{a^7}{c^2b^2}+\frac{b^7}{c^2a^2}+\frac{1}{a^2b^2c^2}\right),&\\
    %abc=\sqrt[3]{\frac{b^7}{a^2c^2}\cdot\frac{c^7}{a^2b^2}\cdot\frac{a^7}{b^2c^2}}\leq\frac{1}{3}\left(\frac{b^7}{a^2c^2}+\frac{c^7}{a^2b^2}+\frac{a^7}{b^2c^2}\right),&
    %\end{array}\right.$$ Sudėję gausime tai, ką reikėjo įrodyti.
\end{enumerate}

\newpage
\subsection{Cauchy-Schwarz nelygybė}

Cauchy-Schwarz nelygybė yra viena dažniausiai taikomų ir labiausiai naudingų olimpiadų uždavinių sprendimuose.

\begin{thm}[Cauchy-Schwarz nelygybė]
  Tegu $(a_1,a_2,a_3,\ldots,a_n)$ ir $(b_1,b_2,b_3,\ldots,b_n)$ bus
  realiųjų skaičių sekos. Tuomet galios nelygybė
  $$(a_1^2+a_2^2+a_3^2+\ldots+a_n^2)(b_1^2+b_2^2+b_3^2+\ldots+b_n^2)\geq(a_1b_1+a_2b_2+a_3b_3+\ldots+a_nb_n)^2.$$
  Lygybė galios tada ir tik tada, kai
  $\frac{a_1}{b_1}=\frac{a_2}{b_2}=\ldots=\frac{a_n}{b_n}.$
\end{thm}

Pateiksime keletą populiariausių nelygybės įrodymų.

\begin{proof}[Pirmas įrodymas]
  Pasinaudosime Lagrange (Lagranžo) tapatybe, kuri padeda nelygybę įrodyti
  iškart:
  $$(a_1^2+a_2^2+a_3^2+\ldots+a_n^2)(b_1^2+b_2^2+b_3^2+\ldots+b_n^2)-(a_1b_1+a_2b_2+a_3b_3+\ldots+a_nb_n)^2$$
  $$=\sum_{1\leq i< j\leq n}{(a_ib_j-a_jb_i)^2}.$$
\end{proof}

\begin{proof}[Antras įrodymas]
  Tegu $(a_1,a_2,\ldots,a_n)$ ir $(b_1,b_2,\ldots,b_n)$ bus realiųjų
  skaičių sekos. Imkime funkciją
  $$f(x)=(a_1x-b_1)^2+(a_2x-b_2)^2+\ldots+(a_nx-b_n)^2.$$
  Pastebėkime, kad $f(x)\geq0$, vadinasi $f(x)$ diskriminantas $D\leq0$. Kita
  vertus,
  $$f(x)=(a_1^2+a_2^2+\ldots+a_n^2)x^2-2(a_1b_1+a_2b_2+\ldots+a_nb_n)x+(b_1^2+b_2^2+\ldots+b_n^2).$$
  Tada
  \begin{align*}
    D&=4(a_1b_1+a_2b_2+\ldots+a_nb_n)^2-4(a_1^2+a_2^2+\ldots+a_n^2)(b_1^2+b_2^2+\ldots+b_n^2)\leq0\\
    &\Leftrightarrow(a_1^2+a_2^2+\ldots+a_n^2)(b_1^2+b_2^2+\ldots+b_n^2)\geq(a_1b_1+a_2b_2+\ldots+a_nb_n)^2.
  \end{align*}
\end{proof}

\begin{proof}[Trečias įrodymas]
  Pagal nelygybę $x^2+y^2\geq2xy$:
  \begin{eqnarray*}&&\frac{a_i^2}{a_1^2+a_2^2+a_3^2+\ldots+a_n^2}+\frac{b_i^2}{b_1^2+b_2^2+b_3^2+\ldots+b_n^2}\\
    &&\geq\frac{2a_ib_i}{\sqrt{(a_1^2+a_2^2+a_3^2+\ldots+a_n^2)(b_1^2+b_2^2+b_3^2+\ldots+b_n^2)}}.
  \end{eqnarray*}
  Sudėję visus dėmenis su visais $i$, kai $1\geq i \geq n$, gausime tai,
  ką ir reikėjo įrodyti.
\end{proof}

\begin{proof}[Ketvirtas įrodymas]
  Prisiminkime nelygybių skyrelio \textbf{Pirmieji žingsniai} uždavinį nr. 5.
  Gaunama nelygybė yra vadinama Cauchy-Schwarz (CS) nelygybės Engel forma.

  \begin{thm}[CS - Engel forma]
    Jei $(a_1,a_2,\ldots,a_n)$ ir $(b_1,b_2,\ldots,b_n)$ yra realiųjų skaičių sekos, kai visi $b_i>0$, tai galioja
$$\frac{a_1^2}{b_1}+\frac{a_2^2}{b_2}+\ldots+\frac{a_n^2}{b_n}\geq\frac{(a_1+a_2+\ldots+a_n)^2}{b_1+b_2+\ldots+b_n}.$$
    Lygybė galios tada ir tik tada, kai $\frac{a_1}{b_1}=\frac{a_2}{b_2}=\ldots=\frac{a_n}{b_n}.$
  \end{thm}

  Iš esmės tai ir yra Cauchy-Schwarz nelygybė, tiksliau, kitokia jos forma:
  belieka visiems $i$ įstatyti $a_i\rightarrow a_ib_i$ ir $b_i\rightarrow
  b_i^2$ ir gausime standartinę išraišką, kuri bus teisinga su visais
  realiaisiais $(a_1,a_2,\ldots,a_n)$ ir $(b_1,b_2,\ldots,b_n)$.
\end{proof}

Kitos Cauchy-Schwarz nelygybės formos:
\begin{itemize}
  \item $a_1^2+a_2^2+\ldots+a_n^2\geq\frac{(a_1b_1+a_2b_2+\ldots+a_nb_n)^2}{b_1^2+b_2^2+\ldots+b_n^2};$
  \item $\sqrt{(a_1^2+a_2^2+...+a_n^2)(b_1^2+b_2^2+...+b_n^2)}\geq a_1b_1+a_2b_2+\ldots+a_nb_n;$
\end{itemize}

Kai $a_1,a_2,\ldots,a_n$ ir $b_1,b_2,\ldots,b_n$ teigiami:
\begin{itemize}
  \item $(a_1+a_2+...+a_n)(b_1+b_2+...+b_n)\geq(\sqrt{a_1b_1}+\sqrt{a_2b_2}+...+\sqrt{a_nb_n})^2; $
  \item $a_1+a_2+...+a_n\geq\frac{(\sqrt{a_1b_1}+\sqrt{a_2b_2}+...+\sqrt{a_nb_n})^2}{b_1+b_2+...+b_n};$
  \item $\sqrt{(a_1+a_2+...+a_n)(b_1+b_2+...+b_n)}\geq \sqrt{a_1b_1}+\sqrt{a_2b_2}+\ldots+\sqrt{a_nb_n}.$
\end{itemize}

Sunku net pasakyti, ar naudingesnė Engel forma, ar pati Cauchy-Schwarz
nelygybė. Dažniausiai, jas taikant gaunamas tas pats rezultatas. Svarbu
atkreipti dėmesį, kad skiriasi Cauchy-Schwarz ir Engel formos nelygybių
apibrėžimo sritys: pirmoji galioja su visais realiaisiais, o antroji
reikalauja, kad trupmenų vardikliai būtų teigiami. Nepaisant šių skirtumų,
šios dvi nelygybės yra vadinamos vienu vardu - Cauchy-Schwarz nelygybe.

Sprendžiant iš lygybės atvejo, jei AM-GM nelygybė sumažina reiškinį iki
lygių kintamųjų, kuomet Cauchy-Schwarz  lygybės atvejis pasiekiamas tada,
kai kintamieji yra proporcingi, galime sakyti, kad Cauchy-Schwarz nelygybė
yra lankstesnė ir bendresnė.

Daugelį ankstesnių pavyzdžių ir uždavinių galima padaryti ir naudojant
Cauhcy-Schwarz nelygybę. Skaitytoją raginame pačiam pabandyti tai atlikti.
Mes žengsime prie pavydžių, kuriuose matysis, kaip įvairiai galime
pritaikyti Cauchy-Schwarz nelygybę, gaudami neįtikėtinus rezultatus.

\subsubsection{Pavyzdžiai}

\begin{pavnr}[Baltic Way 2008]
  Įrodykite, kad jei realieji $a,b,c$ tenkina $a^2+b^2+c^2=3$, tai galioja
  $$\frac{a^2}{2+b+c^2}+\frac{b^2}{2+c+a^2}+\frac{c^2}{2+a+b^2}\geq\frac{(a+b+c)^2}{12}.$$
  Kada galios lygybė?
\end{pavnr}

\begin{proof}[Įrodymas]
  Pastebėkime, kad $2+b>0$, nes $b^2\leq3$. Taip pat bus $2+a>0$ ir $2+c>0$.
  Tuomet pagal Cauchy-Schwarz nelygybę:
  $$\mbox{KAIRĖ PUSĖ}\geq\frac{(a+b+c)^2}{6+a^2+b^2+c^2+a+b+c}.$$ Taigi,
  belieka įrodyti $a+b+c\leq3\Leftrightarrow
  2a+2b+2c\leq3+a^2+b^2+c^2\Leftrightarrow (a-1)^2+(b-1)^2+(c-1)^2\geq0$, kas
  yra akivaizdu. Lygybė galios, kai $a=b=c=1$.
\end{proof}

\begin{pavnr}
  Duoti teigiami realieji $a\geq b\geq c\geq d$ tenkina $a+b+c+d=1$. Raskite
  mažiausią reiškinio $Z=4a^2+3b^2+2c^2+d^2$ reikšmę.
\end{pavnr}

\begin{sprendimas}
  Pastebėkime, kad $a\geq\frac{1}{4}$, $a+b\geq\frac{1}{2}$,
  $a+b+c\geq\frac{3}{4}$, $a+b+c+d=1$. Sudėję gausime
  $4a+3b+2c+d\geq\frac{10}{4}$. Pagal Cauchy-Schwarz nelygybę:
  $$Z=4a^2+3b^2+2c^2+d^2\geq\frac{(4a+3b+2c+d)^2}{10}\geq\frac{5}{8}.$$
  Minimumas bus $\frac{5}{8}$. Jis pasiekiamas, kai $a=b=c=d=\frac{1}{4}$.
\end{sprendimas}

\begin{pavnr}[Pham Kim Hung]
  Įrodykite, kad teigiami realieji $a,b,c$ tenkina
  $$\frac{a^2-bc}{2a^2+b^2+c^2}+\frac{b^2-ac}{2b^2+a^2+c^2}+\frac{c^2-ab}{2c^2+a^2+b^2}\geq0.$$
\end{pavnr}

\begin{proof}[Įrodymas]
  Jei nelygybę padauginsime iš -2 ir prie kairės pusės trupmenų pridėsime
  po 1, o dešinėje pridėsime 3, tai gausime ekvivalenčią nelygybę:
  \begin{equation*}\frac{(b+c)^2}{2a^2+b^2+c^2}+\frac{(a+c)^2}{2b^2+a^2+c^2}+\frac{(a+b)^2}{2c^2+a^2+b^2}\leq3. \tag{1}
  \end{equation*}
  Pagal Cauchy-Schwarz nelygybę:
  \begin{eqnarray*} \mbox{KAIRĖ PUSĖ (1)}&\leq&\sum_{cyc}{\frac{b^2}{b^2+a^2}}+\sum_{cyc}{\frac{c^2}{c^2+a^2}} \\
    &=&\sum_{cyc}{\frac{b^2+a^2}{b^2+a^2}}=3
  \end{eqnarray*}
\end{proof}

\begin{pavnr}[Nesbitt'o nelygybė]
  Jei $a,b,c$ - teigiami realieji skaičiai, tai galioja nelygybė
  $$\frac{a}{b+c}+\frac{b}{a+c}+\frac{c}{a+b}\geq\frac{3}{2}.$$
\end{pavnr}

\begin{proof}[Įrodymas]
  Naudosime Cauchy-Schwarz nelygybę:
  $$\mbox{KAIRĖ
  PUSĖ}=\frac{a^2}{ab+ac}+\frac{b^2}{ab+bc}+\frac{c^2}{ac+bc}\geq\frac{(a+b+c)^2}{2(ab+bc+ac)}\geq\frac{3(ab+bc+ac)}{2(ab+bc+ac)}=\frac{3}{2}.$$
\end{proof}

\begin{pavnr}[Iran 1998]
  Skaičiai $x,y,z$ tokie, kad $x\geq1$, $y\geq1$, $z\geq1$, ir
  $\frac{1}{x}+\frac{1}{y}+\frac{1}{z}=2$. Įrodykite, kad galios nelygybė
  $$\sqrt{x+y+z}\geq\sqrt{x-1}+\sqrt{y-1}+\sqrt{z-1}.$$
\end{pavnr}

\begin{proof}[Įrodymas]
  Pertvarkykime duotą sąlygą: $\frac{1}{x}+\frac{1}{y}+\frac{1}{z}=2
  \Leftrightarrow\frac{x-1}{x}+\frac{y-1}{y}+\frac{z-1}{z}=1$.
  Taikysime Cauchy-Schwarz nelygybę:
  $$x+y+z=(x+y+z)(\frac{x-1}{x}+\frac{y-1}{y}+\frac{z-1}{z})\geq(\sqrt{x-1}+\sqrt{y-1}+\sqrt{z-1})^2,$$
  ką ir reikėjo įrodyti.
\end{proof}

\subsubsection{Uždaviniai}
\begin{enumerate}
  \item Dešimt teigiamų realiųjų skaičių tenkina $a_1+a_2+...+a_{10}=1$ ir
    $a_1\geq a_2+a_3\geq a_4+a_5+a_6\geq a_7+a_8+a_9+a_{10}$. Raskite
    reiškinio $Z=a_1^2+a_2^2+...+a_{10}^2$ mažiausią pasiekiamą reikšmę.
    %Pažymime $a_1=\alpha$, $a_2+a_3=\beta$, $a_4+a_5+a_6=\gamma$ ir
    %$a_7+a_8+a_9+a_{10}=\delta$. Tuomet $\alpha+\beta+\gamma+\delta=1$ ir
    %$\alpha\geq\beta\geq\gamma\geq\delta$. Pagal Cauchy-Schwarz nelygybę:
    %\\$Z=a_1^2+a_2^2+...+a_{10}^2\geq\alpha^2+\frac{\beta^2}{2}+\frac{\gamma^2}{3}+\frac{\delta^2}{4}\Leftrightarrow12Z\geq12\alpha^2+6\beta^2+4\gamma^2+3\delta^2.$
    %Pastebime, kad $\alpha\geq\frac{1}{4}$, $\alpha+\beta\geq\frac{1}{2}$,
    %$\alpha+\beta+\gamma\geq\frac{3}{4}$, be to
    %$\alpha+\beta+\gamma+\delta=1$. Teisingai padauginę ir sudėję gausime
    %$12\alpha+6\beta+4\gamma+3\delta\geq\frac{25}{4}.$ Na o pagal
    %Cauchy-Schwarz nelygybę:
    %$$12Z\geq\frac{(12\alpha+6\beta+4\gamma+3\delta)^2}{25}\geq\frac{25^2}{4^2\cdot25}=\frac{25}{16}.$$
    %Taigi $Z$ minimumas yra $\frac{25}{192}$, o jis pasiekiamas, kai
    %$a_1=\frac{1}{4}$, $a_2=a_3=\frac{1}{8}$, $a_4=a_5=a_6=\frac{1}{12}$
    %ir $a_7=a_8=a_9=a_{10}=\frac{1}{16}$.
  \item Teigiami realieji $a,b,c,d$ tenkina nelygybes $a\leq1$, $a+b\leq5$,
    $a+b+c\leq14$ ir $a+b+c+d\leq30$. Įrodykite, kad galioja nelygybė
    $\sqrt{a}+\sqrt{b}+\sqrt{c}+\sqrt{d}\leq10$.
    %Pažymėkime $\mbox{Ž} = \sqrt{a} + \sqrt{b} + \sqrt{c} + \sqrt{d}.$
    %Pagal Cauchy-Schwarz nelygybės Engel formą:
    %$$\frac{\mbox{Ž}^2}{10}=\frac{(\sqrt{a}+\sqrt{b}+\sqrt{c}+\sqrt{d})^2}{10}\leq
    %a+\frac{b}{2}+\frac{c}{3}+\frac{d}{4}$$
    %\begin{eqnarray*}\Leftrightarrow12\cdot\frac{\mbox{Ž}^2}{10}&\leq&12a+6b+4c+3d\\
    %&=&3(a+b+c+d)+(a+b+c)+2(a+b)+6a\\
    %&\leq&3\cdot30+14+2\cdot5+6\cdot1=120
    %\end{eqnarray*}$$\Leftrightarrow\mbox{Ž}\leq10.$$
  \item \text{[IMO 1995]} Teigiami realieji $a,b,c$ yra tokie, kad $abc=1$.
    Įrodykite, kad teisinga nelygybė
    $$\frac{1}{a^3(b+c)}+\frac{1}{b^3(a+c)}+\frac{1}{c^3(a+b)}\geq\frac{3}{2}.$$
    %Nelygybę transformuojame naudodami duotą sąlygą ir tada sprendžiame
    %naudodami Cauchy-Schwarz nelygybę: \begin{align*} \mbox{KAIRĖ
    %PUSĖ}&=\frac{b^2c^2}{a(b+c)}+\frac{a^2c^2}{b(a+c)}+\frac{a^2b^2}{c(a+b)}\\
    %&\geq\frac{(ab+bc+ac)^2}{2(ab+bc+ac)}=\frac{ab+bc+ac}{2}\\
    %&\geq\frac{3\sqrt[3]{a^2b^2c^2}}{3} \tag{AM-GM}\\
    %&=\frac{3}{2}.\end{align*}
  \item Įrodykite, kad teigiamiems realiesiems $x_{1},x_{2},...,x_{n}$
    galioja nelygybė
    $$\sqrt{x_{1}(3x_2+x_3)}+\sqrt{x_{2}(3x_3+x_4)}+...+\sqrt{x_n(3x_{1}+x_2)}\leq2(x_{1}+x_2+...+x_n).$$
    %Pagal Cauchy-Schwarz nelygybę: \begin{eqnarray*} \text{KAIRĖ PUSĖ}
    %&=&\sum_{cyc}{\sqrt{x_n\left(3x_1+x_2\right)}}\\
    %&\leq&\sqrt{\left(\sum_{cyc}{x_n}\right)\left(\sum_{cyc}{3x_1+x_2}\right)}\\
    %&=&\sqrt{4\left(\sum_{cyc}{x_n}\right)^2}=2(x_1+x_2+...+x_n).\end{eqnarray*}
  \item \text{[Darij Grinberg]} Įrodykite, kad teigiamiems realiesiems
    $a,b,c$ galioja
    $$\frac{a}{(b+c)^2}+\frac{b}{(a+c)^2}+\frac{c}{(a+b)^2}\geq\frac{9}{4(a+b+c)}.$$
    %Pertvarkę taikome Cauchy-Schwarz nelygybę:
    %$$(a+b+c)\left(\frac{a}{(b+c)^2}+\frac{b}{(a+c)^2}+\frac{c}{(a+b)^2}\right)\geq\left(\frac{a}{b+c}+\frac{b}{a+c}+\frac{c}{a+b}\right)^2\geq\frac{9}{4}.$$
    %Paskutinė nelygybė remiasi Nesbitt'o nelygybe, o tai ir užbaigia
    %įrodymą.
  \item Tegu $a,b,x,y,z$ bus teigiami realieji skaičiai. Parodykite, kad
    $$\frac{x}{ay+bz}+\frac{y}{az+bx}+\frac{z}{ax+by}\geq\frac{3}{a+b}.$$
    %Pagal Cauchy-Schwarz nelygybę:
    %\begin{eqnarray*}\frac{x}{ay+bz}+\frac{y}{az+bx}+\frac{z}{ax+by}&\geq&\frac{(x+y+z)^2}{x(ay+bz)+y(az+bx)+z(ax+by)}\\
    %&=&\frac{(x+y+z)^2}{(xy+yz+xz)(a+b)}\\&\geq&\frac{3}{a+b}.\end{eqnarray*}
    %Paskutinė nelygybė teisinga pagal $$(x+y+z)^2\geq3(xy+yz+xz).$$
  \item Įrodykite, kad teigiamiems realiesiems skaičiams, tokiems, kad
    $a+b+c=3$, galioja nelygybė
    $$\frac{a}{1+b^2c}+\frac{b}{1+c^2a}+\frac{c}{1+a^2b}\geq\frac{3}{2}.$$
    %Nelygybę pertvarkome, tada taikome Cauchy-Schwarz nelygybę, tada vėl
    %pertvarkome: $$\text{KAIRĖ
    %PUSĖ}=\sum_{cyc}{\frac{a^2}{a+ab^2c}}\geq\frac{(a+b+c)^2}{a+b+c+abc(a+b+c)}=\frac{a+b+c}{abc+1}.$$
    %Belieka įrodyti $$2(a+b+c)\geq3abc+3,$$ kas pagal duotą sąlygą yra
    %ekvivalentu $$(a+b+c)^3\geq27abc,$$ kas seka iš AM-GM nelygybės.
  \item Parodykite, kad teigiamiems realiesiems $a_1,a_2,\ldots,a_n$ ir
    $b_1,b_2,\ldots,b_n$ galioja nelygybė
    $$\sqrt{a_1^2+b_1^2}+\ldots+\sqrt{a_n^2+b_n^2}\geq\sqrt{(a_1+a_2+\ldots+a_n)^2+(b_1+b_2+\ldots+b_n)^2}.$$
    %Įrodymas remiasi matematine indukcija. Akivaizdu, kad jei nelygybė
    %teisinga su $n=k$, tai teisinga ir su $n=k+1$. Taigi, belieka įrodyti
    %kai $n=2$:
    %$$\sqrt{a_1^2+b_1^2}+\sqrt{a_2^2+b_2^2}\geq\sqrt{(a_1+a_2)^2+(b_1+b_2)^2}.$$
    %Atskliaudus ir sutvarkius:
    %$$\Leftrightarrow(a_1^2+a_2^2)(b_1^2+b_2^2)\geq(a_1b_1+a_2b_2)^2,$$
    %kas yra tiesiog Cauchy-Schwarz nelygybė.
    %Šią nelygybę taip pat galima įrodyti naudojantis Pitagoro teoremą, čia
    %įrodymo nepateiksime, bet galite pabandyti jį patys atrasti.
  \item \text{[JBMO 2002 Shortlist]} Įrodykite, kad jei teigiami realieji
    skaičiai tenkina $abc=2$, tai galios nelygybė $$a^3+b^3+c^3\geq
    a\sqrt{b+c}+b\sqrt{a+c}+c\sqrt{a+b}.$$
    %\textit{Lemma.} $3(a^3+b^3+c^3)\geq(a+b+c)(a^2+b^2+c^2)$.\\
    %\noindent\textit{Lemos įrodymas.} Naudojame AM-GM nelygybę:
    %$3(a^3+b^3+c^3)=\sum\limits_{cyc}{a^3}+\sum\limits_{sym}{\frac{a^3+a^3+b^3}{3}}\geq\sum\limits_{cyc}{a^3}+\sum\limits_{sym}{\frac{3a^2b}{3}}=(a+b+c)(a^2+b^2+c^2).$\hfill{$\square$}
    %\\Pagal Cauchy-Schwarz nelygybę ir lemą:
    %\begin{eqnarray*}(\text{DEŠINĖ
    %PUSĖ})^2&\leq&(a^2+b^2+c^2)((b+c)+(a+c)+(a+b))\\
    %&=&2(a^2+b^2+c^2)(a+b+c)\\ &\leq&6(a^3+b^3+c^3)=6(\text{KAIRĖ
    %PUSĖ}).\end{eqnarray*} Kita vertus, pagal AM-GM:
    %\begin{eqnarray*}\text{DEŠINĖ
    %PUSĖ}&\geq&3\sqrt[3]{abc\sqrt{(b+c)(a+c)(a+b)}}\\&\geq&3{\sqrt[3]{abc\sqrt{8abc}}}\\&=&3\sqrt[3]{2\cdot\sqrt{8\cdot2}}=6.\end{eqnarray*}
    %Gauname, kad:\\ $6(\text{DEŠINĖ PUSĖ})\leq(\text{DEŠINĖ
    %PUSĖ})^2\leq6(\text{KAIRĖ PUSĖ})\Rightarrow$\\ $\text{KAIRĖ
    %PUSĖ}\geq\text{DEŠINĖ PUSĖ}$, ką ir reikėjo įrodyti.
  \item \text{[Walther Janous, \textit{Crux Mathematicorum}]} Tegu $x,y$ ir
    $z$ bus teigiami realieji. Įrodykite, kad galios
    $$\frac{x}{x+\sqrt{(x+y)(x+z)}}+\frac{y}{y+\sqrt{(y+x)(y+z)}}+\frac{z}{z+\sqrt{(z+x)(z+y)}}\leq1.$$
    %Pagal Cauchy-Schwarz nelygybę:
    %$$(x+y)(z+x)\geq(\sqrt{xy}+\sqrt{xz})^2.$$ Taip sumažinę visų trupmenų
    %vardiklius gausime:
    %$$\sum_{cyc}{\frac{x}{x+\sqrt{(x+y)(x+z)}}}\leq\sum_{cyc}{\frac{x}{x+\sqrt{xy}+\sqrt{xz}}}=\sum_{cyc}{\frac{\sqrt{x}}{\sqrt{x}+\sqrt{y}+\sqrt{z}}}=1.$$
  \item Įrodykite, kad teigiamiems realiesiems skaičiams $a,b,c,d,e,f$
    galioja nelygybė
    $$\frac{a}{b+c}+\frac{b}{c+d}+\frac{c}{d+e}+\frac{d}{e+f}+\frac{e}{f+a}+\frac{f}{a+b}\geq3.$$
    %Pagal Cauchy-Schwarz nelygybę: \\$\mbox{KAIRĖ PUSĖ} =
    %\sum\limits_{cyc}\frac{a^2}{ab+ac} \geq
    %\frac{(a+b+c+d+e+f)^2}{ab+ac+bc+bd+cd+ce+de+df+ef+ea+fa+fb}.$
    %\\Pavadinkime gautą vardiklį $V$. Tada:
    %$$2V=(a+b+c+d+e+f)^2-(a+d)^2-(b+e)^2-(c+f)^2.$$ Tačiau vėl iš
    %Cauchy-Schwarz nelygybės:
    %$$\left(1+1+1\right)\left((a+d)^2+(b+e)^2+(c+f)^2\right)\geq(a+b+c+d+e+f)^2.$$
    %Taigi, $V\leq\frac{1}{3}\cdot(a+b+c+d+e+f)^2$, kas užbaigia įrodymą.
  \item \text{[Ukraine 2001]} Įrodykite, kad teigiamiems realiesiems
    $a,b,c,x,y,z$, kai $x+y+z=1$, galioja nelygybė
    $$ax+by+cz+2\sqrt{(xy+xz+yz)(ab+bc+ac)}\leq a+b+c.$$
    %Cauchy-Schwarz nelygybę naudosime dukart. Pirmiausia,
    %$$ax+by+cz\leq\sqrt{a^2+b^2+c^2}\cdot\sqrt{x^2+y^2+z^2}.$$ Taigi,
    %\begin{eqnarray*}\text{KAIRĖ
    %PUSĖ}&\leq&\sqrt{\sum_{cyc}{a^2}\cdot}\sqrt{\sum_{cyc}{x^2}}+\sqrt{2\sum_{cyc}{ab}}\cdot\sqrt{2\sum_{cyc}{xy}}\\
    %&\leq&\sqrt{\sum_{cyc}{x^2}+2\sum_{cyc}{xy}}\cdot\sqrt{\sum_{cyc}{a^2}+2\sum_{cyc}{ab}}\\
    %&=&(a+b+c)(x+y+z)\\ &=&a+b+c.\end{eqnarray*}
  \item \text{[Japan TST 2004]} Tegu $a,b,c$ - tokie teigiami realieji
    skaičiai, kurių suma lygi 1. Įrodykite, kad galios nelygybė
    $$\frac{1+a}{1-a}+\frac{1+b}{1-b}+\frac{1+c}{1-c}\leq\frac{2a}{b}+\frac{2b}{c}+\frac{2c}{a}.$$
    %Pirmiausia pertvarkome:
    %\begin{eqnarray*}&\Leftrightarrow&\sum_{cyc}\frac{a+b+c}{b+c}+\sum_{cyc}\frac{a}{b+c}\leq\sum_{cyc}\frac{2a}{b}\Leftrightarrow3+2\sum_{cyc}\frac{a}{b+c}\leq\sum_{cyc}\frac{2a}{b}\\
    %&\Leftrightarrow&\sum_{cyc}{\frac{a}{b}-\frac{a}{b+c}}\geq\frac{3}{2}\Leftrightarrow\frac{ac}{b(b+c)}+\frac{ab}{c(a+c)}+\frac{bc}{a(a+b)}\geq\frac{3}{2}\\
    %&\Leftrightarrow&\frac{a^2c^2}{abc(b+c)}+\frac{a^2b^2}{abc(a+c)}+\frac{b^2c^2}{abc(a+b)}\geq\frac{3}{2}.\end{eqnarray*}
    %Paskutinei nelygybei pritaikę Cauchy-Schwarz nelygybę gausime:
    %$$\frac{a^2c^2}{abc(b+c)}+\frac{a^2b^2}{abc(a+c)}+\frac{b^2c^2}{abc(a+b)}\geq\frac{(ab+bc+ac)^2}{2abc(a+b+c)}\geq\frac{3}{2}.$$
    %Paskutinei nelygybei įrodyti naudojome gerai žinomą faktą, kad
    %realiesiems $x,y,z$ galioja $(x+y+z)^2\geq3(xy+xz+yz)$.
  \item \text{[Iran TST 2009]} Duoti teigiami realieji $a,b,c$, kurių suma
    lygi 3. Įrodykite, kad
    $$\frac{1}{2+a^2+b^2}+\frac{1}{2+c^2+a^2}+\frac{1}{2+b^2+c^2}\leq\frac{3}{4}.$$
    %Padauginę nelygybę iš -2 ir prie abiejų pusių pridėję po 3, gausime
    %ekvivalenčią nelygybę
    %\begin{equation*}\frac{a^2+b^2}{2+a^2+b^2}+\frac{a^2+c^2}{2+a^2+c^2}+\frac{c^2+b^2}{2+c^2+b^2}\geq\frac{3}{2}.\tag{1}\end{equation*}
    %Naudosimes Cauchy-Schwarz nelygybe: \begin{eqnarray*}\text{KAIRĖ PUSĖ
    %(1)}&\geq&\frac{\left(\sqrt{a^2+b^2}+\sqrt{a^2+c^2}+\sqrt{b^2+c^2}\right)^2}{6+2(a^2+b^2+c^2)}\\&=&\frac{2(a^2+b^2+c^2)+2\sum\limits_{cyc}{\sqrt{(a^2+b^2)(a^2+c^2)}}}{6+2(a^2+b^2+c^2)}\\
    %&\geq&\frac{2(a^2+b^2+c^2)+2\sum\limits_{cyc}{(a^2+bc)}}{6+2(a^2+b^2+c^2)}\\
    %&=&\frac{(a+b+c)^2+3(a^2+b^2+c^2)}{6+2(a^2+b^2+c^2)}\\&=&\frac{3(3+a^2+b^2+c^2)}{2(3+a^2+b^2+c^2)}=\frac{3}{2}.\end{eqnarray*}
  \item \text{[Komal Magazine]} Įrodykite, kad teigiamiems realiesiems
    $a,b,c$ galioja $$(a^2+2)(b^2+2)(c^2+2)\geq3(a+b+c)^2.$$
    %Visur taikysime Cauchy-Schwarz nelygybę. Pastebėkime, kad
    %\begin{eqnarray*}
    %(a^2+2)(b^2+2)&=&(a^2+1)(1+b^2)+a^2+b^2+3\\&\geq&(a+b)^2+\frac{(a+b)^2}{2}+3\\&=&\frac{3}{2}((a+b)^2+2).\end{eqnarray*}
    %Tuomet
    %\begin{eqnarray*}(a^2+2)(b^2+2)(c^2+2)&\geq&\frac{3}{2}((a+b)^2+2)(2+c^2)\\&\geq&\frac{3}{2}(\sqrt{2}(a+b)+\sqrt2c)^2\\&=&3(a+b+c)^2.\end{eqnarray*}
\end{enumerate}

\newpage
\subsection{Specialios technikos}

Šiame skyrelyje susipažinsime su keliomis populiariomis gudrybėmis, kurios
gali labai pagelbėti uždavinių sprendime. Sprendimų
,,varikliukais'' liks mums jau gerai žinomos nelygybės, tokios kaip AM-GM
ir Cauchy-Schwarz. Pagrindinė gudrybė - keitiniai. Jei skaitytojas abejoja
jų galingumu, tegu pabando pateiktas nelygybes išspęsti alternatyviu būdu.
Dalis pavyzdžių ir uždavinių yra susiję su geometrija, tačiau algebrinėse
nelygybėse užtenka ir elementarių žinių.

\subsubsection{Homogenizacija ir Normalizacija}

Homogenizacija - tai nehomogeninės nelygybės vertimas homogenine,
dažniausiai tam naudojant duotą papildomą sąlyga. Iki šiol mes nieko
nebijodami drąsiai homogenizuodavome nelygybes ir bėdų nematėme, tačiau
neretai taip primityviai homogenizuoti nehomogeninę nelygybę yra bjauroka
ir visiškai nenaudinga. Todėl šiame skyrelyje susipažinsime su specialiais
homogenizuojančiais keitiniais, kurie duos gerokai daugiau naudos.

Šių keitinių esmė yra išnaudoti papildomą sąlygą taip, kad visi kintamieji
taptų nulinio laipsnio, o ir tuomet visa nelygybė taps nulinio laipsnio.
Kiekvienai duotai sąlygai galime sugalvoti atitinkamų keitinių.
\begin{itemize}
  \item Duota $abc=k^3$. Geras keitinys būtų $a=\frac{kx}{y}$,
    $b=\frac{ky}{z}$, $c=\frac{kz}{x}$. Visada galime sugalvoti
    įspūdingesnį: $a=\frac{kx^3y}{z^4}$ ir t.t.
  \item Duota $a+b+c=k$. Bene vienintelis naudingas keitinys būtų
    $a=\frac{xk}{x+y+z}$, $b=\frac{yk}{x+y+z}$, $c=\frac{zk}{x+y+z}$,
    tačiau neribokime savo fantazijos: $a=\frac{kx(x+2y)}{(x+y+z)^2}$,
    $b=\frac{ky(y+2z)}{(x+y+z)^2}$, $c=\frac{kz(z+2x)}{(x+y+z)^2}$ ir pan..
  \item Duota $ab+bc+ac=k$. Kintamuosius galime keisti poromis:
    $bc=\frac{xk}{x+y+z}$, $ac=\frac{yk}{x+y+z}$, $ab=\frac{zk}{x+y+z}$.
\end{itemize}

Žinoma, kai turime daugiau kintamųjų, reikės sugalvoti analogiškų keitinių,
tačiau nereiktų persistengti - dažnai tokie keitiniai tik ,,subjauroja''
nelygybę ir ji tampa tik dar labiau komplikuota.

\begin{pavnr}
  Tegu $a,b,c$ - tokie teigiami skaičiai, kad $abc=1$. Įrodykite nelygybę
  $$\frac{1}{a^2+a+1}+\frac{1}{b^2+b+1}+\frac{1}{c^2+c+1}\geq1.$$
\end{pavnr}

\begin{sprendimas}
  Pakeiskime $a=\frac{yz}{x^2}$, $b=\frac{xz}{y^2}$, $c=\frac{xy}{z^2}$. Nelygybė tampa:
  \begin{eqnarray*}&&\sum_{cyc}{\frac{1}{\frac{y^2z^2}{x^4}+\frac{yz}{x^2}+1}}\geq1\\
    &\Leftrightarrow&\sum_{cyc}{\frac{x^4}{y^2z^2+x^2yz+x^4}}\geq1.
  \end{eqnarray*}
  O pagal Cauchy-Schwarz ir AM-GM nelygybes:
  \begin{eqnarray*}\sum_{cyc}{\frac{x^4}{y^2z^2+x^2yz+x^4}}&\geq&\frac{(x^2+y^2+z^2)^2}{x^4+y^4+z^4+\sum\limits_{cyc}{x^2y^2}+\sum\limits_{cyc}{yzx^2}}\\
    &\geq&\frac{(x^2+y^2+z^2)^2}{x^4+y^4+z^4+\sum\limits_{cyc}{x^2y^2}+\sum\limits_{cyc}{\frac{1}{2}(x^2y^2+x^2z^2)}}\\
    &=&\frac{(x^2+y^2+z^2)^2}{x^4+y^4+z^4+2\sum\limits_{cyc}{x^2y^2}}=1.
  \end{eqnarray*}
\end{sprendimas}

Normalizacija yra tarsi priešingas dalykas homogenizacijai. Tegu
$N(a_1,a_2,...,a_n)\geq0$ - homogeninė nelygybė. Pagal homogeniškumo
apibrėžimą, pakeitę $a_i=tx_i$ visiems $i$, kur $t$ - teigiamas skaičius
gausime, $N(a_1,a_2,...,a_n)=t^n N(x_1,x_2,...,x_n)$. Vadinasi, liks
įrodyti $N(x_1,x_2,...,x_n)\geq0$, kur visi $x_i$ yra proporcingai norimai
stipriai padidėję/sumažėję. Tai reiškia, kad naujos kintamųjų aibės savybės
(suma, sandauga, kvadratų suma, ir pan.) yra pasikeitę. Niekas nedraudžia
juos mažinti tiek, kad jų suma, sandauga ar dar kokia aibės savybė būtų
lygi konkrečiam, mūsų pasirinktam dydžiui.

Pavyzdžiui, jei norime įrodyti homogeninę nelygybė nuo trijų
teigiamų kintamųjų $f(a,b,c)\geq0$, nemažindami bendrumo galime tarti, kad
$ab+bc+ac=3$. Tuomet, naudodami AM-GM ir kitas nelygybes galime nustatyti
kitų kintamųjų aibės savybių ribas:
$3=ab+bc+ac\geq3\sqrt[3]{a^2b^2c^2}\Rightarrow abc\leq1$, $3=ab+bc+ac\leq
a^2+b^2+c^2$, $9=3(ab+bc+ac)\leq(a+b+c)^2\Rightarrow a+b+c\geq3$.

\begin{pavnr}[Nesbitt'o nelygybė]
  Įrodykite, kad teigiamiems skaičiams galioja $$\frac{a}{b+c}+\frac{b}{a+c}+\frac{c}{a+b}\geq\frac{3}{2}.$$
\end{pavnr}

\begin{sprendimas}
  Nelygybė yra homogeninė. Nemažindami bendrumo tariame, kad $a+b+c=1$.
  Žinome, kad $$ab+bc+ac\leq\frac{1}{3}(a+b+c)^2=\frac{1}{3}.$$
  Tuomet
  $$\frac{3}{2}=3-\frac{9}{2}\cdot\frac{1}{3}\leq3-\frac{9}{2}(ab+bc+ac).$$
  Reiškia, liks įrodyti
  $$\frac{a}{b+c}+\frac{b}{a+c}+\frac{c}{a+b}\geq3-\frac{9}{2}(ab+bc+ac)$$
  arba $$\sum_{cyc}{\frac{a}{b+c}+\frac{9a(b+c)}{4}}\geq3.$$ Na o pagal
  AM-GM nelygybę:
  $$\sum_{cyc}{\frac{a}{b+c}+\frac{9a(b+c)}{4}}\geq\sum_{cyc}{2\sqrt{\frac{a\cdot9a(b+c)}{4(b+c)}}}=3\sum_{cyc}a=3.$$
\end{sprendimas}

\begin{pavnr}
  Įrodykite, kad teigiamiems realiesiems $a,b,c$ galioja nelygybė
  $$\sqrt{\frac{ab+bc+ac}{3}}\leq\sqrt[3]{\frac{(a+b)(b+c)(a+c)}{8}}.$$
\end{pavnr}

\begin{sprendimas}
  Nelygybė homogeninė, tad neprarasdami bendrumo tariame, kad $ab+bc+ac=3$.
  Tada $\text{KAIRĖ PUSĖ}=1$. Be to, pagal AM-GM nelygybę galime nesunkiai
  rasti, kad $a+b+c\geq3$ ir $abc\leq1$. Žinodami tapatybę, nelygybę
  pertvarkome: $$(a+b)(b+c)(a+c)=(a+b+c)(ab+bc+ac)-abc=3(a+b+c)-abc\geq8.$$
  Tuomet $\text{DEŠINĖ PUSĖ}\geq1=\text{KAIRĖ PUSĖ}$, ką ir reikėjo
  įrodyti.
\end{sprendimas}

\subsubsection{Algebriniai ir trigonometriniai keitiniai}

Visi kiti nei anksčiau aprašyti algebriniai keitiniai yra grynas fantazijos
reikalas. Būdami itin paprasti, jie dažnai labai stipriai palengvina darbą.

\begin{pavnr}[Nguyen Van Thach]
  Tegu $a,b,c$ - teigiami realieji skaičiai. Įrodykite, kad jiems galioja
  nelygybė
  $$\frac{a^3}{a^3+b^3+abc}+\frac{b^3}{b^3+c^3+abc}+\frac{c^3}{c^3+a^3+abc}\geq1.$$
\end{pavnr}

\begin{sprendimas}
  Pakeiskime $\frac{b}{a}=x$, $\frac{c}{b}=y$, $\frac{a}{c}=z$
  ir pastebėkime, kad tada $xyz=1$. Tuomet:
  $$\frac{a^3}{a^3+b^3+abc}=\frac{1}{1+\left(\frac{b}{a}\right)^3+\frac{b}{a}\cdot\frac{c}{a}}=\frac{1}{1+x^3+\frac{x}{z}}=\frac{xyz}{xyz+x^3+x^2y}=\frac{yz}{yz+x^2+xy}.$$
  Pagal Cauchy-Schwarz nelygybę:
  $$\sum_{cyc}{\frac{yz}{yz+x^2+xy}}\geq\frac{(xy+xz+yz)^2}{\sum\limits_{cyc}{yz(yz+x^2+xy)}}.$$
  Taigi, lieka įrodyti $$(xy+yz+xz)^2\geq\sum_{cyc}{yz(yz+x^2+xy)}.$$
  Nepabijoję reiškinio išskleisti matysime, kad tai yra tapatybė.
\end{sprendimas}

\begin{pavnr}[St. Petersburg 2009]
  Duotiems teigiamiems realiesiems skaičiams galioja sąryšis
  $a+b+c=ab+bc+ac$. Įrodykite, kad jiems galioja nelygybė $$a+b+c+1\geq4.$$
\end{pavnr}

\begin{sprendimas}[Įrodymas pagal Mathias Tejs Knudsen.]
  Jei $a+b<1$, tai $a+b+c=ab+bc+ca=c(a+b)+ab< c+(a+b)(a+b) < c+a+b$ ir
  gauname prieštarą. Taigi $a+b\geq1$ ir analogiškai $b+c\geq1$ bei
  $a+c\geq1$. Įveskime keitinį $a=x+\frac{1}{2}$, $b=y+\frac{1}{2}$,
  $c=z+\frac{1}{2}$, tuomet duota sąlyga taps $ab+bc+ac=\frac{3}{4}$.
  Ankščiau gautas rezultatas bus ekvivalentus $x+y\geq0$, $x+z\geq0$,
  $y+z\geq0$, vadinasi ne daugiau kaip vienas iš skaičių $x,y,z$ yra
  neigiamas. Pakeitus, pagrindinė nelygybė pavirsta į $$8xyz\leq1.$$ Jei
  vienas iš $x,y,z$ yra neigiamas, nelygybė akivaizdi, o jei visi teigiami -
  pagal AM-GM nelygybę:
  $$\frac{3}{4}=xy+yz+xz\geq3\sqrt[3]{x^2y^2z^2}\Leftrightarrow 8xyz\geq1.$$
\end{sprendimas}

\begin{pastaba}
  Keitinys $a=\frac{1}{x}$, $b=\frac{1}{y}$, $c=\frac{1}{z}$ šiuo atveju
  irgi labai padėtų, nes tuomet duota sąlyga nepasikeistų, o pagrindinė
  nelygybė įgytų kitokią, galbūt, patogesnę formą, bet tai jau visai kitas
  sprendimas.
\end{pastaba}

Užuominos į trigonometrinius keitinius gali būti labai įvairios: sąlyga,
jog kintamieji yra intervale $[0,1]$, arba konstrukcija $\sqrt{1-x^2}$
sufleruoja apie sinusus, kosinusus, o algebrinė konstrukcija $\sqrt{1+x^2}$
- tipinis tangeto ar kotangento taikymo atvejis, kadangi
$\frac{1}{\sqrt{1+\tan^2x}}=|\cos x|$ ir $\frac{1}{\sqrt{1+\cot^2x}}=|\sin
x|$.

\begin{pavnr}[Latvia 2002]
  Teigiami realieji skaičiai $a,b,c,d$ tenkina
  $$\frac{1}{1+a^4}+\frac{1}{1+b^4}+\frac{1}{1+c^4}+\frac{1}{1+d^4}=1.$$
  Įrodykite, kad tada teisinga yra nelygybė $abcd\geq3.$
\end{pavnr}

\begin{sprendimas}
  Pakeiskime $a^2=\tan A$, $b^2=\tan B$, $c^2=\tan C$, $d^2= \tan D$, kur
  $A,B,C,D\in(0,\frac{\pi}{2}).$ Žinodami, kad
  $\tan^2\Theta+1=\frac{1}{\cos^2\Theta}$, pertvarkome duotą sąlygą į
  $$\cos^2A+\cos^2B+\cos^2C+\cos^2D=1.$$ Pagrindinė nelygybė tampa $$\tan
  A\cdot\tan B\cdot\tan C\cdot\tan D\geq9.$$ Pagal AM-GM nelygybę:
  $$\left\{
  \begin{array}{ll}
    \sin^2A=1-\cos^2A=\cos^2B+\cos^2C+\cos^2D\geq3\sqrt[3]{\cos^2B\cos^2C\cos^2D}\\
    \sin^2B=1-\cos^2B=\cos^2C+\cos^2D+\cos^2A\geq3\sqrt[3]{\cos^2C\cos^2D\cos^2A}\\
    \sin^2C=1-\cos^2C=\cos^2D+\cos^2A+\cos^2B\geq3\sqrt[3]{\cos^2D\cos^2A\cos^2B}\\
    \sin^2D=1-\cos^2D=\cos^2A+\cos^2B+\cos^2C\geq3\sqrt[3]{\cos^2A\cos^2B\cos^2C}
  \end{array}
  \right.$$
  Viską sudauginę gausime reikiamą rezultatą.
\end{sprendimas}

\subsubsection{Pokštai su trikampiu}

Dažnai, ypač rimtesnėse olimpiadose, yra mėgiami uždaviniai, susiejantys
kelias matematikos disciplinas. Šiame mažame skyrelyje nagrinėsime algebros
ir geometrijos junginį: nelygybes trikampio kraštinėms.

Pagrindinis dalykas, naudingas žinoti įrodinėjant nelygybę trikampio
kraštinėms, yra trikampio nelygybė: bet kurių dviejų kraštinių ilgių suma
yra didesnė už likusiosios ilgį.

\begin{pavnr}[Pham Kim Hung]
  Duoto trikampio kraštinių ilgiai yra $a,b,c$. Įrodykite, kad galioja
  $$\frac{1}{\sqrt{a+b-c}}+\frac{1}{\sqrt{b+c-a}}+\frac{1}{\sqrt{a+c-b}}\geq\frac{9}{ab+bc+ac},$$
  kai trikampio perimetras 3.
\end{pavnr}

\begin{proof}[Pirmas įrodymas]
  Pažymekime $x=\sqrt{a+b-c}$, $y=\sqrt{b+c-a}$, z=$\sqrt{a+c-b}$, tuomet
  $a^2+b^2+c^2=3$, o nelygybė pavirs į
  \begin{equation*}\frac{1}{x}+\frac{1}{y}+\frac{1}{z}\geq\frac{36}{(x^2+y^2+z^2)^2+x^2y^2+x^2z^2+y^2z^2}.
    \tag{Įsitikinkite!}
  \end{equation*} Kas yra ekvivalentu
  $$(xy+xz+yz)(9+x^2y^2+x^2z^2+y^2z^2)\geq36xyz.$$ Pagal trikampio
  nelygybę, gauname, kad $x,y,z$ - teigiami skaičiai, taigi, jiems galime
  taikyti AM-GM nelygybę. Iš tikrųjų: sudauginus
  $$xy+xz+yz\geq3\sqrt[3]{x^2y^2z^2}$$ ir
  $$9+x^2y^2+x^2z^2+y^2z^2\geq12\sqrt[12]{x^4y^4z^4}$$ gausime reikiamą
  rezultatą.
\end{proof}

Ypač fantastiškas yra Ravi keitinys: žinome, kad į trikampį $ABC$ įbrėžus
apskritimą, kuris kraštines $AB,$ $BC$ ir $AC$ liečia atitinkamai taškuose
$X,$ $Y$ ir $Z$, gausime $AX=AZ=p$, $BX=BY=r$ ir $CY=CZ=s$. Tuomet
$AB=p+r$, $BC=r+s$ ir $AC=p+s$. Akivaizdu, kad $p,r,s$ - teigiami dydžiai.
Toks keitinys atriša sprendėjui rankas nuo trikampio ir leidžia dirbti su
bet kokiais teigiamais skaičiais.

\begin{proof}[Antras įrodymas]
  Atlikime Ravi keitinį: $a=p+r$, $b=r+s$, $c=p+s$. Turėsime
  $p+r+s=\frac{3}{2}$. Pagrindinė nelygybė taps:
  $$\frac{1}{\sqrt{2p}}+\frac{1}{\sqrt{2r}}+\frac{1}{\sqrt{2s}}\geq\frac{9}{(p+r+s)^2+pr+rs+ps}.$$
  Tai yra ekvivalentu
  $$(\frac{9}{4}+pr+rs+ps)(\sqrt{ps}+\sqrt{pr}+\sqrt{ps})\geq9\sqrt{2}\cdot\sqrt{prs}.$$
  Pagal AM-GM nelygybę:
  $$\frac{9}{4}+pr+rs+ps\geq12\sqrt[12]{\frac{1}{4^9}\cdot p^2r^2s^2}$$ ir
  $$\sqrt{ps}+\sqrt{pr}+\sqrt{ps}\geq3\sqrt[3]{prs}.$$ Šias dvi sudauginame
  ir gauname tai, ką ir reikėjo įrodyti.
\end{proof}

\subsubsection{\textit{Cauchy Reverse Technique}}

Tokį įspūdingą pavadinimą gali turėti nebent koks nors labai sudėtingas ir
niekam nereikalingas matematinis metodas. Taip jau atsitiko, kad būtent
šitaip yra vadinamas itin paprastas ir tuo genialus nelygybių sprendimo
būdas.

Kai turime nelygybę, ir mums tiesiog niežti rankas pritaikyti AM-GM
nelygybę, bet to padaryti negalime, nes nelygybės ženklas yra priešingas,
atliekame paprastą triuką: Iš trupmenos iškeliame sveikąją dalį, kuri yra
didesnė už pradinė trupmeną. Tada prie naujo trupmeninio ,,likučio''
gausime minusą ir galėsime išlieti savo energiją ir pyktį pritaikydami
AM-GM nelygybę. Nematant, kaip tai vyksta iš tikrųjų, pagal aprašyma tai
atrodo visiškai nesuprantama, tad pereikime prie pavyzdžių, kurie
spalvingai iliustruos mintį.

\begin{pavnr}
  Įrodykite, kad su teigiamais realiaisiais skaičiais teisinga nelygybė
  $$\frac{a^4}{a^3+2b^3}+\frac{b^4}{b^3+2c^3}+\frac{c^4}{c^3+2d^3}+\frac{d^4}{d^3+2d^3}\geq\frac{a+b+c+d}{3}.$$
\end{pavnr}

\begin{sprendimas} Pertvarkykime kairės pusės dėmenis, kad jie taptų
  ,,apversti'' ir iškart taikykime AM-GM nelygybę:
  \begin{eqnarray*}
    \sum_{cyc}{\frac{a^4}{a^3+2b^3}}&=&\sum_{cyc}{\frac{a^4+2ab^3-2ab^3}{a^3+2b^3}}
    = a+b+c+d-\sum_{cyc}{\frac{2ab^3}{a^3+2b^3}}\\ &\geq&
    a+b+c+d-\sum_{cyc}{\frac{2ab^3}{3\sqrt[3]{a^3b^6}}}=a+b+c+d-\frac{2}{3}(a+b+c+d)\\&=&\frac{a+b+c+d}{3}.
  \end{eqnarray*}
\end{sprendimas}

\begin{pavnr}
  Įrodykite, kad teigiamiems realiesiems $a,b,c$, kur $a+b+c=3$, galioja
  nelygybė $$\frac{1}{1+2b^2c}+\frac{1}{1+2c^2a}+\frac{1}{1+2a^2b}\geq1.$$
\end{pavnr}

\begin{proof}[Įrodymas]
  Partvarkome ir du kartus taikome AM-GM nelygybę (stebuklinga, kad galime
  taikyti AM-GM nelygybę toje pačioje nelygybėje ir mažėjančia, ir didėjančia
  puse):
  \begin{eqnarray*}\sum_{cyc}{\frac{1}{1+2b^c}}&=&\sum_{cyc}{\frac{1+2b^2c-2b^2c}{1+2b^2c}}\\
    &\geq&3-\sum_{cyc}{\frac{2b^2c}{3\sqrt[3]{b^4c^2}}}=3-\sum_{cyc}{\frac{2\sqrt[3]{b^2c}}{3}}\\
    &\geq&3-\sum_{cyc}{\frac{2(2b+c)}{9}}=3-\frac{2\cdot3(a+b+c)}{9}=1.
  \end{eqnarray*}
\end{proof}

\subsubsection{Uždaviniai}

\begin{enumerate}
  \item \text{[Romania Junior TST 2003]} Įrodykite, kad teigiami realieji
    skaičiai, tenkinantys $abc=1$, taip pat tenkina ir
    $$1+\frac{3}{a+b+c}\geq\frac{6}{ab+bc+ac}.$$
    %Pirma mintis - atlikti homogenizuojantį keitinį $a=\frac{x}{y}$,
    %tačiau netrunkame įsitikinti kad tai nieko gero neduoda, todėl
    %tenka pasukti galvą ieškant kitokio kelio. Ir štai - keitinys
    %$a=\frac{1}{x}$, $b=\frac{1}{y}$, $c=\frac{1}{z}$ išspręs problemą.
    %Žinoma, nepamirškime, kad vistiek $xyz=1$. Nelygybė tampa
    %$$1+\frac{3}{xy+yz+zx}\geq\frac{6}{x+y+z}.$$ Kadangi
    %$$xy+yz+xz\leq\frac{1}{3}(x+y+z)^2,$$ tai belieka įrodyti:
    %$$1+\frac{9}{(x+y+z)^2}\geq\frac{6}{x+y+z},$$ kas seka iš AM-GM.

  \item \text{[Clock-Tower School Junior Competition 2009]} Teigiami
    realieji skaičiai $a,b,c$ tenkina $abc=8$. Įrodykite, kad jiems taip
    pat galioja nelygybė
    $$\frac{a-2}{a+1}+\frac{b-2}{b+1}+\frac{c-2}{c+1}\leq0.$$
    %Nesunku pamatyti, kad reikia pasikeisti $a=\frac{2x}{y}$,
    %$b=\frac{2y}{z}$, $c=\frac{2z}{a}$. Gausime nelygybę:
    %$$\frac{2x-2y}{2x+y}+\frac{2y-2z}{2y+z}+\frac{2z-2x}{2z+x}\leq0\Leftrightarrow\frac{y}{2x+y}+\frac{z}{2y+z}+\frac{x}{2z+x}\geq1.$$
    %Pagal Cauchy-Schwarz nelygybę:
    %$$\sum_{cyc}{\frac{x}{2z+x}}\geq\frac{(x+y+z)^2}{x(2z+x)+y(2x+y)+z(2y+z)}=1.$$
  \item \text{[Zhautykov Olympiad 2008]} Įrodykite, kad teigiamiems
    realiesiems skaičiams, kurie tenkina $abc=1$, galioja nelygybė
    $$\frac{1}{b(a+b)}+\frac{1}{c(b+c)}+\frac{1}{a(a+c)}\geq\frac{3}{2}.$$
    %Kadangi $abc=1$, keičiame $a=\frac{x}{y}$, $b=\frac{y}{z}$,
    %$c=\frac{z}{x}$. Tuomet gausime, kad reikia įrodyti
    %$$\sum_{cyc}{\frac{z^2}{y^2+xz}}\geq\frac{3}{2}.$$ Pritaikome
    %Cauchy-Schwarz nelygybę:
    %$$\sum_{cyc}{\frac{z^4}{z^2y^2+xz^3}}\geq\frac{(x^2+y^2+z^2)^2}{x^2y^2+x^2z^2+y^2z^2+xz^3+yx^3+zy^3}.$$
    %Belieka įrodyti
    %$$2(x^2+y^2+z^2)^2\geq3(x^2y^2+x^2z^2+y^2z^2+xz^3+yx^3+zy^3),$$
    %kas ekvivalentu šių dviejų nelygybių (kurios galioja pagal AM-GM
    %nelygybę) sumai: $$\sum_{cyc}{x^4}\geq\sum_{cyc}{x^3y}$$ ir
    %$$\sum_{cyc}{x^4+x^2y^2}\geq2\sum_{cyc}{x^3y}.$$
  \item Įrodykite nelygybę, kuri galioja su teigiamais realiaisiais
     $a,b,c,d$:
     \begin{center}$\frac{a}{b^2+c^2+d^2}+ \frac{b}{c^2+d^2+a^2}+
       \frac{c}{d^2+a^2+b^2}+
       \frac{d}{a^2+b^2+c^2} \geq\frac{3\sqrt{3}}{2}\cdot\frac{1}{\sqrt{a^2+b^2+c^2+d^2}}.$
     \end{center}
     %Duota nelygybė yra homogeninė, todėl ją įrodysime kai
     %$a^2+b^2+c^2+d^2=1$. Nelygybė tampa:
     %$$\frac{a}{1-a^2}+\frac{b}{1-b^2}+\frac{c}{1-c^2}+\frac{d}{1-d^2}\geq\frac{3\sqrt{3}}{2}.$$
     %Pagal AM-GM nelygybę:
     %$$2a^2(1-a^2)(1-a^2)\leq\left(\frac{2a^2+1-a^2+1-a^2}{3}\right)^3=\left(\frac{2}{3}\right)^3$$
     %$$\Leftrightarrow a(1-a^2)\leq\frac{2}{3\sqrt{3}}$$
     %$$\Leftrightarrow\frac{a}{1-a^2}\geq\frac{3\sqrt{3}}{2}a^2.$$ Taigi:
     %$$\frac{a}{1-a^2}+\frac{b}{1-b^2}+\frac{c}{1-c^2}+\frac{d}{1-d^2}\geq\frac{3\sqrt{3}}{2}(a^2+b^2+c^2+d^2)=\frac{3\sqrt{3}}{2}.$$
  \item \text{[USAMO 2003]} Įrodykite nelygybę, kuri teisinga su teigiamais
    realiaisiais $a,b,c$:
    $$\frac{(2a+b+c)^2}{2a^2+(b+c)^2}+\frac{(2b+a+c)^2}{2b^2+(a+c)^2}+\frac{(2c+a+b)^2}{c^2+(a+b)^2}\leq8.$$
    %Kadangi turime homogeninę nelygybę, nemažindami bendrumo tariame, kad
    %$a+b+c=3$. Pertvarkę gausime:
    %$$\frac{(3+a)^2}{2a^2+(3-a)^2}+\frac{(3+b)^2}{2b^2+(3-b)^2}+\frac{(3+c)^2}{2c^2+(3-c)^2}\leq8$$
    %$$\Leftrightarrow\frac{a^2+6a+9}{a^2-2a+3}+\frac{b^2+6b+9}{b^2-2b+3}+\frac{c^2+6c+9}{c^2-2c+3}\leq24$$
    %$$\Leftrightarrow3+\frac{8a+6}{(a-1)^2+2}+\frac{8b+6}{(b-1)^2+2}+\frac{8c+6}{(c-1)^2+2}\leq24.
    %$$ Kadangi $(x-1)^2+2\geq2$ visiems $x$, tai belieka įrodyti
    %$$8(a+b+c)+18\leq42,$$ kas pagal sąlygą $a+b+c=3$ yra tapatybė.
  \item \text{[Korea 1998]} Teigiami realieji skaičiai tenkina sąryšį
    $x+y+z=xyz$. Įrodykite, kad jiems galioja nelygybė
    $$\frac{1}{\sqrt{1+x^2}}+\frac{1}{\sqrt{1+y^2}}+\frac{1}{\sqrt{1+z^2}}\leq\frac{3}{2}.$$
    %Pasikeiskime $x=\frac{1}{a}$, $y=\frac{1}{b}$, $z=\frac{1}{c}$. Sąlyga
    %taps $xy+xz+yz=1$. Pagrindinė nelygybė:
    %$$\frac{x}{\sqrt{1+x^2}}+\frac{y}{\sqrt{1+y^2}}+\frac{z}{\sqrt{1+z^2}}\leq\frac{3}{2},$$
    %arba
    %$$\frac{x}{\sqrt{x^2+xz+xy+yz}}+\frac{y}{\sqrt{y^2+xz+xy+yz}}+\frac{z}{\sqrt{z^2+xz+xy+yz}}\leq\frac{3}{2},$$
    %arba
    %$$\frac{x}{\sqrt{(x+y)(x+z)}}+\frac{y}{\sqrt{(y+x)(y+z)}}+\frac{z}{\sqrt{(z+x)(z+y)}}\leq\frac{3}{2}.$$
    %Pagal AM-GM nelygybę:
    %\begin{eqnarray*}\sum_{cyc}{\frac{x}{\sqrt{(x+y)(x+z)}}}
    %&=&\sum_{cyc}{\frac{x\sqrt{(x+y)(x+z)}}{(x+y)(x+z)}} \\
    %&\leq&\sum_{cyc}{\frac{1}{2}\cdot\frac{x(x+y)+x(x+z)}{(x+y)(x+z)}}\\
    %&=&\frac{1}{2}\sum_{cyc}{\frac{x}{x+z}+\frac{x}{x+y}}\\&=&\frac{3}{2}.\end{eqnarray*}
  \item \text{[\textit{Crux Mathematicorum}]} Parodykite, kad teigiamiems
    realiesiems skaičiams, kurie tenkina $abcde=1$, galioja
    \begin{center}
      $\frac{a+abc}{1+ab+abcd}+\frac{b+bcd}{1+bc+bcde}+\frac{c+cde}{1+cd+cdea}+\frac{d+dea}{1+de+deab}+\frac{e+eab}{1+ea+eabc}\geq\frac{10}{3}.$
    \end{center}
    %Pasikeiskime $a=\frac{x}{y}$, $b=\frac{y}{z}$, $c=\frac{z}{t}$,
    %$d=\frac{t}{u}$, $e=\frac{u}{a}$. Tada po nedidelių pertvarkymų
    %gausime:
    %$$\sum_{cyc}{\frac{a+abc}{1+ab+abcd}}=\sum_{cyc}{\frac{\frac{1}{y}+\frac{1}{t}}{\frac{1}{x}+\frac{1}{z}+\frac{1}{u}}}.$$
    %O tada dar pakeitę $\frac{1}{x}=a_1$, $\frac{1}{y}=a_2$,
    %$\frac{1}{z}=a_3$, $\frac{1}{t}=a_4$, $\frac{1}{u}=a_5$ ir paprastumo
    %dėlei pažymėję $S=a_1+a_2+a_3+a_4+a_5$, gausime, kad reikia įrodyti
    %\begin{equation*}\sum_{cyc}{\frac{a_2+a_4}{a_1+a_3+a_5}}\geq\frac{10}{3}.\tag{1}\end{equation*}
    %Dabar taikome Cauchy-Schwarz nelygybę, nežymiai pertvarkome vardiklį
    %ir dar kartą taikome Cauchy-Schwarz nelygybę:
    %\begin{eqnarray*}\text{KAIRĖ
    %PUSĖ(1)}&\geq&\frac{4S^2}{\sum\limits_{cyc}{(a_2+a_4)(a_1+a_3+a_5)}}\\
    %&=&\frac{4S^2}{2S^2-\sum\limits_{cyc}{(a_1+a_3)^2}}\\
    %&\geq&\frac{4S^2}{2S^2-\frac{4S^2}{5}}\\
    %&=&\frac{10}{3}.\end{eqnarray*}
  \item \text{[George Tsintifas, \textit{Crux Mathematicorum}]} Įrodykite
    nelygybę teigiamiems realiesiems skaičiams:
    $$(a+b)^3(b+c)^3(c+d)^3(d+a)^3\geq16a^2b^2c^2d^2(a+b+c+d)^4.$$
    %Neprarasdami bendrumo tariame, kad $a+b+c+d=1$. Tuo naudodamiesi
    %įrodysime, kad $$(a+b)(b+c)(c+d)(d+a)\geq abc+bcd+cda+dab.$$ Tai
    %reikalauja tiesiog pertvarkyti nelygybę ir pritaikyti faktą
    %$x^2\geq0$:
    %\begin{eqnarray*}(a+b)(b+c)(c+d)(d+a)&=&a^2c^2+b^2d^2+2abcd+\sum_{cyc}{abc(a+b+c)}\\
    %&=&(ac-bd)^2+\sum_{cyc}{abc(a+b+c+d)}\\&\geq&\sum_{cyc}{abc}.\end{eqnarray*}
    %Dabar įrodysime
    %$$\left(\sum_{cyc}{abc}\right)^3\geq16a^2b^2c^2d^2(a+b+c+d).$$ Pakeitę
    %$abc=x$, $bcd=y$, $cda=z$, $dab=t$, gauname
    %$$(x+y+z+t)^3\geq16(xyz+yzt+ztx+txy).$$ Taikykime AM-GM nelygybę: \begin{eqnarray*}&&\text{KAIRĖ
    %PUSĖ}=\\&=&\sum_{cyc}{x^3}+\frac{3}{2}\sum_{sym}{x^2y}+6\sum_{cyc}{xyz}\\
    %&=&\frac{1}{3}\sum_{cyc}{x^3+y^3+z^3}+\frac{1}{4}\sum_{sym}{x^2y+x^2z+y^2x+y^2z+z^2x+z^2y}+6\sum_{cyc}{xyz}\\
    %&\geq&\sum_{cyc}{xyz}+\frac{3}{2}\sum_{sym}{xyz}+6\sum_{cyc}{xyz}\\&=&16\sum_{cyc}{xyz}.\end{eqnarray*}
  \item \text{[Romania Junior TST 2002]} Skaičiai $a,b,c$ priklauso
    intervalui $[0,1]$. Įrodykite, kad jiems galioja nelygybė
    $$\sqrt{abc}+\sqrt{(1-a)(1-b)(1-c)}<1.$$
    %Sąlyga $a,b,c\in[0,1]$ sufleruoja apie trigonometrinį keitinį. Ir
    %išties, pasikeitę $a=\sin^2x$, $b=\sin^2y$, $c=\sin^2z$, kur
    %$x,y,z\in[0,\frac{\pi}{2}]$, gauname tai, ką reikia: $$\sin x\sin
    %y\sin z+\cos x\cos y\cos z<\sin x\sin y+\cos x\cos y=\cos(x-y)<1.$$
  \item \text{[IMO 1983]}. Įrodykite, kad trikampio kraštinės $a,b,c$
    tenkina nelygybę $$a^2b(a-b)+b^2c(b-c)+c^2a(c-a)\geq0.$$
    %Pakeitę $a=y+z$, $b=x+z$, $c=x+y$, padalinę iš $xyz$ ir sutvarkę
    %nelygybę gausime, jog tereikia įrodyti
    %$$\frac{x^2}{y}+\frac{y^2}{z}+\frac{z^2}{x}+\frac{x^2}{z}+\frac{y^2}{x}+\frac{z^2}{y}\geq
    %2x+2y+2z.$$ Tačiau tai yra dviejų nelygybių, kurios tiesiogiai
    %įrodomos su Cauchy-Schwarz nelygybe, suma:
    %$$\frac{x^2}{y}+\frac{y^2}{z}+\frac{z^2}{x}\geq x+y+z$$ ir
    %$$\frac{x^2}{z}+\frac{y^2}{x}+\frac{z^2}{y}\geq x+y+z.$$
  \item \text{[Samin Riasat]} Įrodykite, kad trikampio kraštinės $a,b,c$
    tenkina nelygybę
    $$\frac{a}{3a-b+c}+\frac{b}{3b-c+a}+\frac{c}{3c-a+b}\geq1.$$
    %Nelygybę dauginame iš 4, pertvarkome, tada taikome Cauchy-Schwarz
    %nelygybės Engel formą, nes iš trikampio nelygybės seka, kad visi
    %vardikliai teigiami: \begin{eqnarray*}4\cdot\text{(KAIRĖ
    %PUSĖ)}&=&3+\frac{a+b-c}{3a-b+c}+\frac{b+c-a}{3b-c+a}+\frac{c+a-b}{3c-a+b}\\
    %&\geq&3+\frac{(a+b+c)^2}{\sum\limits_{cyc}{(a+b-c)(3a-b+c)}}\\
    %&=&3+\frac{(a+b+c)^2}{\sum\limits_{cyc}{3a^2-ab+ac+3ab-b^2+bc-3ac+bc-c^2}}\\
    %&=&4.\end{eqnarray*}
  \item Įrodykite, kad trikampio kraštinės tenkina nelygybę
    $$\sqrt{3\left(\sqrt{ab}+\sqrt{bc}+\sqrt{ac}\right)}\geq\sqrt{a+b-c}+\sqrt{a+c-b}+\sqrt{b+c-a}.$$
    %Atliekame Ravi keitinį: $a=x+y$, $b=y+z$, $c=z+x$. Gausime:
    %$$3\left(\sqrt{(x+y)(x+z)}+\sqrt{(x+y)(y+z)}+\sqrt{(z+y)(x+z)}\right)\geq2\left(\sqrt{x}+\sqrt{y}+\sqrt{z}\right)^2.$$
    %Bet pagal AM-GM nelygybę:
    %$$\sqrt{(x+y)(x+z)}=\sqrt{x^2+xy+xz+yz}\geq\sqrt{x^2+2x\sqrt{yz}+yz}=x+\sqrt{yz}.$$
    %Analogiškai pasielgę su likusiais nariais gausime naują nelygybę,
    %kuriai vėl taikome AM-GM nelygybę:
    %\begin{eqnarray*}3(x+y+z)+3(\sqrt{yz}+\sqrt{xz}+\sqrt{xy})&\geq&2(x+y+z)+4(\sqrt{yz}+\sqrt{xz}+\sqrt{xy})\\
    %&=&2\left(\sqrt{x}+\sqrt{y}+\sqrt{z}\right)^2.\end{eqnarray*}
  \item \text{[Bulgaria TST 2003]} Duoti teigiami realieji skaičiai $a,b,c$
    tenkina $a+b+c=3$. Įrodykite, kad jiems teisinga nelygybė
    $$\frac{a}{1+b^2}+\frac{b}{1+c^2}+\frac{c}{1+a^2}\geq\frac{3}{2}.$$
    %Pertvarkykime kairės pusės dėmenis, kad jie taptų ,,apversti'' ir
    %iškart taikykime AM-GM nelygybę:
    %$$\frac{a}{1+b^2}=\frac{a+ab^2-ab^2}{1+b^2}=a-\frac{ab^2}{1+b^2}\geq
    %a-\frac{ab^2}{2b}=a-\frac{ab}{2}.$$ Analogiškai pertvarkius likusius
    %dėmenis, nelygybė pavirs į $$\sum_{cyc}{\frac{a}{1+b^2}}\geq
    %a+b+c-\frac{1}{2}\sum_{cyc}{ab}\geq\frac{3}{2}.$$ Paskutinę nelygybę
    %įrodome pasinaudoję faktu $$ab+bc+ca\leq\frac{(a+b+c)^2}{3}=3.$$
  \item \text{[Pham Kim Hung]} Duoti tokie teigiami skaičiai $a,b,c,d$, kad
    $a+b+c+d=4$. Įrodykite, kad jie tenkina nelygybę
    $$\frac{a}{1+b^2c}+\frac{b}{1+c^2a}+\frac{c}{1+d^2a}+\frac{d}{1+a^2b}\geq 2.$$
    %Pertvarkome, taikome AM-GM: \begin{eqnarray*}
    %\sum_{cyc}{\frac{a+ab^2c-ab^2c}{1+b^2c}}&\geq&\sum_{cyc}{a-\frac{ab^2c}{2b\sqrt{c}}}\\
    %&=&\sum_{cyc}{a-\frac{1}{2}b\sqrt{ac\cdot a}}\\
    %&\geq&\sum_{cyc}{a-\frac{1}{4}b(ac+a)}\\&=&a+b+c+d-\frac{1}{4}\sum_{cyc}{abc}-\frac{1}{4}\sum_{cyc}{ab}.\end{eqnarray*}
    %Pagal AM-GM nelygybę: $$\sum_{cyc}{abc}\leq\frac{1}{16}(a+b+c+d)^3=4,$$ o pagal
    %Cauchy-Schwarz nelygybę:
    %$$\sum_{cyc}{ab}=(a+b+c+d)^2-(a+c)^2-(b+d)^2\leq(a+b+c+d)^2-\frac{(a+b+c+d)^2}{2}=4.$$
    %Taigi,
    %$$\frac{a}{1+b^2c}+\frac{b}{1+c^2a}+\frac{c}{1+d^2a}+\frac{d}{1+a^2b}\geq
    %a+b+c+d-2=2.$$
  \item Duoti $n$ teigiamų skaičių $a_1,a_2,a_3,\ldots,a_n$, kurių kvadratų
    suma lygi $n$. Įrodykite, kad jiems galioja nelygybė
    $$\frac{1}{a_1^3+2}+\frac{1}{a_2^3+2}+\frac{1}{a_3^3+2}+\ldots+\frac{1}{a_n^3+2}\geq\frac{n}{3}.$$
    %Pertvarkome, taikome AM-GM:
    %$$\sum_{cyc}{\frac{1}{a_n^3+2}}=\frac{n}{2}-\frac{1}{2}\sum_{cyc}{\frac{a_n^3}{a_n^3+2}}
    %\geq\frac{n}{2}-\frac{1}{2}\sum_{cyc}{\frac{a_n^3}{3a_n}}=\frac{n}{3}.$$
  \item Turime skaičius $a,b,c$, kurių suma lygi 3. Įrodykite, kad jiems
    taip pat galios nelygybė
    $$\frac{a+1}{b^2+1}+\frac{b+1}{c^2+1}+\frac{c+1}{a^2+1}\geq 4.$$
    %Naudosime \textit{Cauchy Reverse Technique}:
    %$$\sum_{cyc}{\frac{a+1}{b^2+1}}=\sum_{cyc}{a+1-\frac{ab^2+b^2}{b^2+1}}\geq\sum_{cyc}{a+1-\frac{ab+b}{2}}.$$
    %Pagal Cauchy-Schwarz nelygybę:
    %\begin{eqnarray*}\sum_{cyc}{ab}&=&\frac{1}{2}((a+b+c+d)^2-(a+c)^2-(b+d)^2)\\
    %&\leq&
    %\frac{1}{2}((a+b+c+d)^2-\frac{(a+b+c+d)^2}{2})=4.\end{eqnarray*}
    %Taigi: $$\sum_{cyc}{\frac{a+1}{b^2+1}}\geq
    %a+b+c+d+4-\frac{4+a+b+c+d}{2}=4.$$
  \item \text{[Pham Kim Hung]} Parodykite, kad teigiamiems realiesiems
    skaičiams $a,b,c$, kurie tenkina $a^2+b^2+c^2=1$, galioja
    $$\frac{1}{2-a}+\frac{1}{2-b}+\frac{1}{2-c}\geq 3.$$
    %\textit{Lema.} $x(2-x)\leq1$, su realiais $x$. \\ \noindent\textit{Lemos
    %įrodymas.} $\Leftrightarrow (x-1)^2\geq0$ \hfill{$\square$}
    %\\Pertvarkome pagrindinę nelygybę ir taikome lemą:
    %$$\sum_{cyc}{\frac{1}{2-a}}=\frac{3}{2}+\sum_{cyc}{\frac{a^2}{2a(a-2)}}\geq\frac{3}{2}+\sum_{cyc}{\frac{a^2}{2}}=3.$$
  \item Tegu $a,b,c$ bus tokie teigiami skaičiai, kad $a+b+c=1$.
    Parodykite, kad teisinga
    $$\frac{a^2}{a+2b^3}+\frac{b^2}{b+2c^3}+\frac{c^2}{c+2a^3}\geq1.$$
    %Pertvarkome ir du kartus taikome AM-GM bei nelygybę
    %$ab+bc+ac\leq\frac{(a+b+c)^2}{3}$:
    %\begin{eqnarray*}\sum_{cyc}{\frac{a^2}{a+2b^3}}&=&\sum_{cyc}{a-\frac{2b^3a}{a+2b^3}}\\
    %&\geq&\sum_{cyc}{a-\frac{2b^3a}{3\sqrt[3]{ab^6}}}=\sum_{cyc}{a-\frac{2}{3}\sqrt[3]{b^3a^2}}\\&\geq&\sum_{cyc}{a-\frac{2}{9}(ab+ab+b)}\\
    %&\geq&a+b+c-\frac{2}{27}\left(2(a+b+c)^2+3(a+b+c)\right)=1.\end{eqnarray*}
\end{enumerate}

\newpage
\subsection{Drakonų puota}

Iš tamsiausių kerčių, tolimiausių užkampių susirinko jos ir jie pasirodyti
vieni kitiems. Ne jėgos, o savo žaižaruojačios išvaizdos parodyti, emocijomis
pasidalinti atvyko. Kiekvienas svečias laukiamas, kiekvieno istorija ypatinga.
Ir suksis jie valso ritme iki ryto, kol giedoriai gaidžiai paskelbs puotos
pabaigą. Kai drakonai atsisveikinę pakils skrydžiui namo, liks čia jų letenų
įspaudai, nagų dryžiai ir neatsargių kostelėjimų apdegintų užuolaidų likučiai,
bylojantys apie šių įspūdingų padarų egzistavimą. Kas žino, galbūt kada nors
kas nors galės regėti nors vieną jų dvikovoje su piktu burtininku, kada degs
žemė, užvirs vandenynai, o dangus apsitrauks ledu.

Šiame skyrelyje skaitytoją supažindinsime su dar keliomis nelygybėsmis, kurios
uždavinių sprendimuose pasitaiko išskirtinai retai. Ne dėl to, kad šios
nelygybės yra silpnos ar neuniversalios, priešingai: dėl to, kad sunkių
uždavinių yra gerokai mažiau nei lengvųjų. Tai bus tik pažintinis skyrelis,
siekiantis parodyti artimiausias fantastikai teormas-nelygybes, todėl
nepateiksime nei pavyzdžių, nei uždavinių, tik keletą taikymo komentarų ir
leisime skaitytojui pasinerti į savą vaizduotę.

\begin{thm}[H\"{o}lder]
  Tegu $\{a_{11},\ldots,a_{n1}\}$, $\ldots$, $\{a_{1k},\ldots,a_{nk}\}$ bus k
  skaičius aibių, kur kiekviena jų turi po $n$ teigiamų elementų, o
  $\{p_1,\ldots,p_k\}$ bus teigiamų skaičių aibė, kurios visų elementų suma
  lygi 1. Tuomet
  $$(a_{11}+\ldots+a_{n1})^{p_1}\ldots(a_{1k}+\ldots+a_{nk})^{p_k}\geq
  a_{11}^{p_1}\ldots a_{1k}^{p_k}+\ldots+a_{n1}^{p_1}\ldots a_{nk}^{p_k},$$
  arba
  $$\prod_{j=1}^{k}{\left(\sum_{i=1}^{n}{a_{ij}}\right)^{p_j}}\geq\sum_{i=1}^{n}{\left(\prod_{j=1}^{k}{a_{ij}^{p_j}}\right)}.$$
\end{thm}

\noindent \textit{Komentarai ir taikymas.} Dažniausiai yra taikoma forma,
kai visi $p_j$ yra lygūs, tačiau įspūdingiausiai nelygybė ,,dirba'', kai
jie yra skirtingi. Pastebėkime, kad kai $k=2$, o $p_1=p_2=\frac{1}{2}$,
gauname Cauchy-Schwarz nelygybę, o ir visa H\"{o}lder nelygybės forma yra
tarsi Cauchy-Schwarz nelygybės apibendrinimas.

\begin{thm}[Chebyshev] Jei turime tokias aibes $a_1\leq a_2\leq\ldots\leq
  a_n$ ir $b_1\leq b_2\leq\ldots\leq b_n$, tai \begin{center}
    $\frac{a_1b_1+a_2b_2+\ldots+a_nb_n}{n}\geq\frac{a_1+a_2+\ldots+a_n}{n}\cdot\frac{b_1+b_2+\ldots+b_n}{n}\geq\frac{a_1b_n+a_2b_{n-1}+\ldots+a_{n-1}b_2+a_nb_1}{n}.$
  \end{center}
\end{thm}

\begin{thm}[Minkowski]
  Jei turime teigiamų skaičių sekas $\{a_1,a_2,\ldots,a_n\}$ ir
  $\{b_1,b_2,\ldots,b_n\}$, o $p>0$, tai
  $$\left(\sum_{i=1}^{n}{(x_i+y_i)^p}\right)^{\frac{1}{p}}\leq\left(\sum_{i=1}^{n}{x_i^p}\right)^{\frac{1}{p}}+\left(\sum_{i=1}^{n}{y_i^p}\right)^{\frac{1}{p}}.$$
\end{thm}

\begin{thm}[Schur]
  Tarkime, kad $a,b,c$ - neneigiami skaičiai, o $r>0$. Tada
  $$a^r(a-b)(a-c)+b^r(b-a)(b-c)+c^r(c-a)(c-b)\geq0.$$ Lygybė galios tada ir
  tik tada, kai $a=b=c$ arba du iš jų lygūs, o trečiasis lygus 0.
\end{thm}

\noindent \textit{Komentarai ir taikymas.} Kai $r=1$, gausime
$a^3+b^3+c^3+3abc\geq a^2(b+c)+b^2(a+c)+c^2(a+b)$, kas yra viena dažniausių
Schur'o nelygybės taikymo formų.

\begin{thm}[Perstatų nelygybė]
  Turime aibes $a_1\leq a_2\leq\ldots\leq a_n$ ir $b_1\leq b_2\leq\ldots\leq
  b_n$. Tada kiekvienai aibės $\{1,2,\ldots,n\}$ perstatai
  $\pi$ galios $$a_1b_1+\ldots+a_nb_n\geq
  a_1b_{\pi(1)}+\ldots+a_nb_{\pi(n)}\geq a_nb_1+a_{n-1}b_2+\ldots+a_1b_n.$$
  Lygybės pirmu ir antru atveju galios atitinkamai tada, kai aibės perstata
  $\pi$ bus griežtai mažėjanti ir griežtai didejanti.
\end{thm}

\noindent \textit{Komentarai ir taikymas.} Įrodinėjant ciklines ar
simetrines nelygybes, visada galima nemažinant bendrumo apsibrėžti, kokie
yra kintamųjų sąryšiai tarpusavyje (pvz. jei yra ciklinė nelygybė nuo
$a,b,c$, tai galime sakyti, kad $a\leq b\leq c$ ar panašiai). Tai leis
suformuoti reikiamas nemažėjančias sekas, kurioms galiotų perstatų
nelygybė.

\begin{thm}[Jensen]
  Tegu $f:A\rightarrow\R$ bus iškila (angl. \textit{convex}) funkcija. Tada
  bet kokiems $x_1,x_2,\ldots,x_n\in A$ ir neneigiamiems
  $w_1,w_2,\ldots,w_n$, kurių suma teigiama, galios
  $$w_1f(x_1)+\ldots+w_nf(x_n)\geq(w_1+\ldots+w_n)f\left(\frac{w_1x_1+\ldots+w_nx_n}{w_1+\ldots+w_n}\right).$$
  Kai $f$ yra išgaubta (angl. \textit{concave}), galioja atvirkščia
  nelygybė.
\end{thm}

\noindent \textit{Komentarai ir taikymas.} Funkcija intervale yra iškila,
jei jos antros eilės išvestinė tame intervale yra ne mažiau už $0$, arba
išgaubta, jei ne daugiau už $0$. Na o praktiškai tą galima pamatyti
funkcijos grafike: iškilos funkcijos grafikas tame intervale savo forma bus
,,panašus'' į funkcijos $y=x^2$ grafiką, o išgaubtos - į funkcijos $y=-x^2$
grafiką. Teoremos idėją galime suformuluoti taip: iškilos funkcijos
reikšmių vidurkis yra ne mažesnis už funkcijos nuo argumentų vidurkio
reikšmę.  Išgaubtai funkcijai, žinoma, atvirkščiai. Taikant šią nelygybę,
dažniausiai $w_1=w_2=\ldots=w_n=1$.

\newpage
