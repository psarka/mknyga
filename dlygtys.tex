%Copyrighted by Paulius Šarka, 2010. Some rights reserved.

\section{Diofantinės lygtys}

Lygtys yra vadinamos diofantinėmis, kai yra ieškoma jų sveikųjų
sprendinių. Šiame skyrelyje apžvelgsime keletą metodų padedančiu jas
spręsti. Atkreipsime dėmesį, kad, skirtingai nuo įprastų lygčių,
'spręsti' dažniausiai yra bandyti įrodyti, kad lygtis sprendinių neturi,
arba jei ir turi, tai labai specifinius.  

\subsection{Dvi lygties pusės}

Pradėsime nuo trijų pagrindinių principų, besiremiančių labai bendru
pastebėjimu:

\smallskip
\begin{center}\emph{Lygybės abi pusės yra vienodo dydžio, vienodai skaidomos
dauginamaisiais ir duoda vienodas liekanas dalijamos iš natūraliųjų
skaičių.}\end{center}
\smallskip

\subsubsection{Dydis}

Pradėkime nuo pavyzdžių. Išspręsime tris paprastas lygtis.
\begin{pav} 
  Raskite lygties $x^2 = x + 2$ sveikuosius sprendinius.
\end{pav}

\begin{sprendimas}
  Ši lygtis yra kvadratinė, ir ją galima išspręsti įprastai, tačiau minutėlei tą
  pamirškime ir pabandykime pasinaudoti tuo, kad kairioji pusė beveik visada yra didesnė
  už dešiniąją. Įvertinkime - kai $x\geq 3$, tai $x^2 \geq 3x \geq x + 6 >
  x+2$, o kai $x\leq -2$, tai $x^2 > 0 \geq x+2$, tad vienintėliai sveikieji
  skaičiai, kurių negalėjome atmesti samprotaudami apie skirtingus lygties
  pusių dydžius, yra $-1, 0, 1$ ir $2$. Lieka tik patikrinti, kurie iš jų
  tinka, ir rasti, kad lygties sprendiniai yra $-1$ ir $2$. 
\end{sprendimas}

\begin{pav}
  Raskite lygties $x^2 + y^2 = 100$ sveikuosius sprendinius.
\end{pav}

\begin{sprendimas}
  Sveikųjų skaičių kvadratai yra visuomet neneigiami ir auga palyginti
  sparčiai. Šios lygties atveju, kaip tik tuo ir pasinaudosime - jei $x$ arba
  $y$ yra moduliu didesni už $10$, tai kairioji pusė tampa didesnė už
  $100$. Atkreipę dėmesį į tai, kad jei $(x,y)$ yra sprendinys tai ir
  $(\pm x,\pm y)$ yra sprendinys gauname, kad užtenka patikrinti $x$ 
  reikšmes nuo $0$ iki $10$. Tai padaryti nesunku - randame, kad sprendiniai
  bus $(0,10)$, $(6,8)$, $(8,6)$, $(10,0)$ su visomis skirtingomis ženklų
  kombinacijomis. 
\end{sprendimas}

\begin{pav}
  Raskite lygties $xy = x+y$ sveikuosius sprendinius.
\end{pav}

\begin{sprendimas}
  Dviejų sveikųjų skaičių sandauga beveik visada yra didesnė už sumą.
  Pasinaudosime tuo, tačiau pirmiausia atmeskime neigiamus atvejus. Aišku,
  kad abu ir $x$, ir $y$, negali būti neigiami, nes tuomet sandauga bus
  teigiama, o suma neigiama. Negali būti ir vienas neigiamas, vienas
  teigiamas, pvz. $x>0$, $y<0$, nes tuomet $xy \leq y < y + x$. Tad ieškokime
  sprendinių, kuriuose $x\geq 0$ ir $y\geq 0$ ir taip pat neprarasdami
  bendrumo tarkime, kad $y$ yra nemažesnis nei $x$. Parodysime, kad
  $x$ negali būti didesnis už $2$. Iš ties, jei $x\geq 3$, tai $xy \geq 3y >
  y + x$. Vadinasi $x$ gali įgyti tik reikšmes $0$, $1$ ir $2$. Patikrinę
  randame sprendinius $(0,0)$ ir $(2,2)$.
\end{sprendimas}

Bandant įvertinti reiškinių dydžius, natūraliai praverčia algebrinės
nelygybės ir supratimas apie funkcijų didėjimą, argumentui artėjant į
begalybę (pavyzdžiui, didesnio laipsnio daugianaris nuo kažkurios reikšmės
visuomet įgis didesnes reikšmes už mažesnio laipsnio daugianarį). Puiki ir
paprasta šių idėjų iliustracija - 1988 metų Lietuvos matematikos olimpiados
uždavinys:
\begin{pav} \text{\emph{[LitMo 1988]}} Išspręskite natūraliaisiais skaičiais lygtį
  $3x^2 + 2y^2 = 4xy + 2x.$
\end{pav}
\begin{sprendimas}
  Parodysime, kad kairioji pusė beveik visada yra didesnė už dešiniąją. Iš
  ties - pagal aritmetinio-geometrinio vidurkio nelygybę $2x^2 +
  2y^2 \geq 4xy$, ir $x^2 > 2x$, kai $x>2$. Vadinasi, lieka patikrinti tik
  dvi reikšmes - $x=1$ ir $x=2$. Tinka tik antroji, randame sprendinį
  $(2,2)$.
\end{sprendimas}

Dažniausiai, kaip ir turi būti olimpiadiniuose uždaviniuose, lygybės pusių
dydžių skirtumo idėja būna užmaskuota ir reikia akylumo norint ją įžiūrėti.
Pavyzdžiui:

\begin{pav} \text{\emph{[LitKo 2009]}} Raskite lygties
  $(a^2 - 9b^2)^2 - 33b = 16$ sveikuosius neneigiamus sprendinius.
\end{pav}

\begin{sprendimas}
Šis uždavinys organizatoriams greičiausiai pasirodė kiek sunkokas, todėl
olimpiadoje buvo suformuluotas kaip dviejų dalių, pirmoji iš kurių prašė
įrodyti, kad visi sprendiniai tenkina nelygybę $|a-3b|\geq 1$. Įrodyti tai
labai paprasta, tačiau įžiūrėti užuominą gerokai sunkiau. Paprasčiausia tai
padaryti turbūt būtų išskaidant pirmąjį dėmenį dauginamaisiais: $(a^2 -
9b^2)^2 = (a-3b)^2(a+3b)^2$, tuomet
$$(a-3b)^2(a+3b)^2 \geq (a+3b)^2\geq 9b^2.$$
Vadinasi, kairioji lygties pusė yra ne mažesnė nei $9b^2 - 33b=(9b-33)b$, bet
šio reiškinio reikšmė yra didesnė už $16$ su visomis $b$ reikšmėmis
viršijančiomis $4$, vadinasi lieka patikrinti vos keletą reikšmių.

Tačiau atidėkime šį sprendimą į šalį ir dar kartą pažvelkime į lygtį,
bandydami kiek kitaip įvertinti kairiosios pusės dydį. Priežastis, dėl
kurios $(a^2-9b^2)^2$ yra beveik visada daug didesnis už $33b$ yra
ta, kad skirtumas tarp kvadratų yra pakankamai didelis. Išties, jei
$a^2$ nėra lygus $9b^2$, tai arčiausiai (tuomet skirtumas mažiausias) jis
gali būti tik tuomet, kai yra artimiausiai esantis kvadratas. O
artimiausias kvadratas yra $(3b-1)^2$, bet net tuomet skirtumas visvien yra
$6b-1$, o $(6b-1)^2 - 33b$ yra didesnis už $16$ su visomis $b$ reikšmėmis
didesnėmis už $1$! Lieka vos du atvejai, iš kurių gauname po sprendinį:
$(4,0)$ ir $(4,1)$.
\end{sprendimas}

Viena (labai svarbi!) iš samprotavimo apie dydį variacijų -„įterpimo tarp
kvadratų' triukas. Norint parodyti, kad sveikasis skaičius nėra kvadratas,
užtenka parodyti, kad jis yra tarp dviejų gretimų kvadratų ir nė vienam iš
jų nelygus. Ši strategija tinka, žinoma, ir aukštesniems laipsniams.

\begin{pav} Raskite lygties $y^2 = x^2 + x + 1$ sveikuosius sprendinius.
\end{pav}

\begin{sprendimas}
Kairioji lygties pusė yra kvadratas, o dešinioji beveik visada nėra, nes
$x^2 < x^2 + x + 1 < (x+1)^2$ (arba $(x+1)^2 < x^2 + x + 1 < x^2$, jei $x$
neigiamas).  Vienintelės $x$ reikšmės, su kuriomis šios nelygybės nėra
teisingos yra $x=-1$ ir $x=0$, gauname sprendinius $(-1, \pm 1)$ ir $(0,
\pm 1)$.
\end{sprendimas}

\subsubsection{Liekanos}

Nagrinėjant lygtį moduliu pasirinkto skaičiaus, apribojimą, kad abi lygybės
pusės turi duoti vienodą liekaną moduliu to skaičiaus, dažnai galima
perkelti į apribojimą ieškomiems sprendiniams. Kartais tas apribojimas būna
pakankamas, kad galėtume visiškai išspręsti lygtį, bet dažniau jis tampa
pagalbine informacija, kuri tampa naudinga sujungus ją su kitomis idėjomis.
Pradėkime nuo paprasčiausių atvejų, kai nagrinėjant lygtį moduliu tinkamai
parinkto skaičiaus ji išsisprendžiama iki galo.

\begin{pav} Raskite lygties $x^2 = 3y - 1$ sveikuosius sprendinius.
\end{pav}

\begin{sprendimas}
Nagrinėkime šią lygtį moduliu $3$. Norint, kad $(x,y)$ būtų sprendinys,
abiejų lygybės pusių dalybos liekana iš $3$ turi būti vienoda. Dešinės
pusės dalybos liekana bus $-1$, o kairės, priklausomai nuo $x$, arba
$0$, arba $1$. Gavome, kad su jokiais $(x,y)$ jos nesutaps, todėl lygtis
sprendinių neturi.
\end{sprendimas}

\begin{pav} Raskite lygties $x^2 = 2^n - 1$ sveikuosius sprendinius. 
\end{pav}

\begin{sprendimas}
  Nagrinėkime lygtį moduliu $4$. Dešinė pusė, kad $n>1$, lygsta $-1$ moduliu
  $4$, o kairė $0$ arba $1$. Kadangi liekanos nesutampa, tai lieka tik
  atvejai $n\leq 1$, kuriuos patikrinę ($n$ negali būti neigiamas, nes
  tuomet $2^n$ nebūtų sveikasis) randame sprendinius $n=1, x=\pm 1$ ir
  $n=0, x=0$. 
\end{sprendimas}

\begin{pav} Raskite lygties $2 + x^2 + x^3 = 6^n$ sveikuosius sprendinius
\end{pav}

\begin{sprendimas}
  Nagrinėkime lygtį moduliu $3$ arba moduliu $5$, arba moduliu $7$. Visais
  trimis atvejais lengva įsitikinti, kad abi pusės duoda skirtingas
  liekanas.
\end{sprendimas}

Kaip jau užsiminėme, lygties nagrinėjimas moduliu (arba sprendimas moduliu)
dažniausiai yra tik dalis sprendimo. Pavyzdžiui:

\begin{pav}
  \text{\emph{[Lietuvos TST 2009]}} Raskite lygties $x^3 + x^2 =
  16 + 2^y$ natūraliuosius sprendinius.
\end{pav}

\begin{sprendimas}
  Nagrinėkime lygtį moduliu $7$. Kairioji pusė gali įgyti liekanas $0, 1,
  2, 3, 5$, o dešinioji $3, 4$ ir $6$. Vienintėlė bendra liekana yra $3$, ir
  ji įgyjama kai $y$ dalijasi iš $3$. Panaudokime gautą informaciją -
  pažymėkime $y=3a$ ir perrašykime lygtį kaip $$(2^{a})^3 = x^3 + x^2 -
  16.$$ Lieka pastebėti, kad galime pritaikyti įterpimo tarp kubų įdėją: su
  visais $x>4$ turime $$x^3 < x^3 + x^2 -16 < (x+1)^3,$$ vadinasi, lieka
  patikrinti tik keturias $x$ reikšmes. Randame vienintelį sprendinį
  $(4,6)$.
\end{sprendimas}

\begin{pav}
  \text{\emph{[MEMO 2009, Aivaras Novikas]}} Raskite lygties $2^x + 2009 =
  3^y5^z$ neneigiamus sveikuosius sprendinius.
\end{pav} 
 
\begin{sprendimas}
  Pirmiausia įsitikinkime, kad $x$ negali būti mažesnis už $3$. Išties -
  įstačius reikšmes $0, 1, 2$ kairiojoje pusėje gauname $2010, 2011,
  2013$ ir nė vienas iš šių skaičių neišsiskaido tik į trejeto ir penketo
  laipsnius. Tad tarkime, kad $x\geq 3$. Įrodysime, kad visi trys
  $x, y, z$ turi būti lyginiai.
  \begin{itemize}
    \item[$x$ -] Jei $y>0$, tai nagrinėkime lygtį moduliu $3$ gausime
      $(-1)^x - 1 \equiv 0$, vadinasi $x$ lyginis. Jei $y=0$, tai
      $z>0$, tuomet nagrinėkime lygtį moduliu $5$. Gausime $2^x - 1 \equiv
      0$, vadinasi $x$ dalijasi iš $4$, t.y. yra lyginis.
    \item[$y$ -] nagrinėkime lygtį moduliu $4$. Kadangi $x>2$, tai gausime
      $1 \equiv (-1)^y$, vadinasi $y$ lyginis.
    \item[$z$ -] nagrinėkime lygtį moduliu $8$. Kadangi $x>2$ ir $y$ lyginis,
      tai gausime $1 \equiv 5^z$, vadinasi $z$ lyginis.
  \end{itemize}
  Pažymėję $x=2a$, $y=2b$, $z=2c$ galime lygtį pertvarkyti į $$2009 =
  (3^b5^c - 2^a)(3^b5^c + 2^a).$$ Kadangi $2009$ išsiskaido kaip $7^2\cdot
  41$, tai į dviejų dauginamųjų sandaugą galime jį išskaidyti tik trim
  būdais: $1\cdot 2009$, $7\cdot 287$ ir $41\cdot 49$. Vienintelis
  išskaidymas, kurio dauginamieji skiriasi per dvejeto laipsnį yra
  $41\cdot 49$, iš kur randame vienintėlį sprendinį $(4,4,2)$. 
\end{sprendimas}

Lygties sprendimą moduliu visuomet verta prisiminti sprendžiant diofantines
lygtis ir ypač tas, kuriose iš pirmo žvilgsnio nesimato jokių silpnų vietų.
Neretai verta spręsti lygtį moduliu nedidelių skaičių (pvz. $2, 3, 4, 5, 7,
8, 9$) ir akylai stebėti gaunamą informaciją. Taip pat visuomet verta gerai
įsižiūrėti į lygtį, kartais koeficientai ar dideli laipsniai gali
pasufleruoti skaičių, moduliu kurio pavyks išpešti ką nors vertingo. 

Pastebėsime, kad sprendimas moduliu dažnai būna sėkmingas, jei viena arba
abi lygties pusės įgyja nedaug liekanų moduliu nagrinėjamo skaičiaus. Kiek
liekanų įgyja reiškiniai pavidalo $x^k$ (kur $x$ kintamasis) kartais padeda
įvertinti liekanų grupių teorija. Prisiminkime, kad liekanų grupės moduliu
pirminio $p$ eilė yra $p-1$, o moduliu sudėtinio $n$ yra $\varphi(n)$. Jei
$p-1$ (ar $\varphi(n)$) ir $k$ didžiausias bendras daliklis yra didelis, tai
tuomet $x^k$ įgis nedaug reikšmių moduliu $p$ (ar $n$). Konkrečiau:

\begin{itemize}
  \item[$p=3$ -] liekanų grupės eilė $2$ - $x^2$ (ir kiti lyginiai
    laipsniai) įgys $2$ liekanas iš $3$ 
  \item[$n=4$ -] liekanų grupės eilė $2$ - $x^2$ įgis $2$ liekanas iš $4$
  \item[$p=5$ -] liekanų grupės eilė $4$ - $x^4$ įgis $2$ liekanas iš $5$
  \item[$p=7$ -] liekanų grupės eilė $6$ - $x^6$ įgis $2$, $x^3$ įgis $3$ liekanas iš $7$
  \item[$n=8$ -] liekanų grupės eilė $4$ - $x^4$ įgis $2$, $x^2$ įgis $3$ liekanas iš $8$
  \item[$n=9$ -] liekanų grupės eilė $6$ - $x^6$ įgis $2$, $x^3$ įgis $3$ liekanas iš $9$
  \item[$p=11$ -] liekanų grupės eilė $10$ - $x^{10}$ įgis $2$, $x^5$ įgis $3$ liekanas iš $11$
\end{itemize}

Žiūrint iš šio taško, 1998 metų Balkanų Matematikos Olimpiados uždavinys
atrodo labai paprastas:
\begin{pav} \text{\emph{[BMO 1998]}} Parodykite, kad lygtis $x^2 + 4 = y^5$ neturi
  sveikųjų sprendinių.
\end{pav}

\begin{sprendimas}
  Atkreipkime dėmesį į $y^5$. Šis reiškinys įgis nedaug reikšmių moduliu
  $11$, o tiksliau, kadangi $y^{10} \equiv 0,1 \m{11}$, tai $y^5 \equiv 0,
  -1, 1 \m{11}$. Tad spręskime lygtį moduliu $11$ - $x^2$ įgys reikšmes
  $0, 1, 4, 9, 5, 3$, todėl kairioji pusė įgis reikšmes $4, 5, 8, 2, 9, 7$.
  Nei viena iš jų nėra lygi $0, 1$ ar $-1$, vaidinasi lygtis sprendinių
  neturi.
\end{sprendimas}

\subsubsection{Skaidymasis}

Vėl pradžiai pateiksime porą paprastų pavyzdžių.

\begin{pav} Raskite visus sveikuosius lygties $xy = x + y$ sprendinius.
\end{pav}

\begin{sprendimas}
  Vienas iš būtinų įgūdžių norint sėkmingai taikyti skaidymosi idėjas
  yra skaidymas dauginamaisiais. Pažvelkime į du skirtingus šios jau
  matytos lygties
  pertvarkymus: $(x-1)(y-1) = 1$ ir $x(y-1) = y$. Pirmuoju atveju lygtis iš
  karto išspręsta - jei dviejų sveikųjų skaičių sandauga lygi $1$, tai jie
  arba abu lygūs $1$, arba $-1$. Antrasis išskaidymas yra iš pirmo žvilgsnio
  prastesnis, bet įdomesnis: kadangi $y-1$ ir $y$ yra tarpusavyje
  pirminiai, tai bet koks $y-1$ daliklis dalins kairę lygybės pusę, bet
  nedalins dešinės. Vadinasi $y-1$ negali turėti jokių daliklių, todėl
  yra lygus $1$ arba $-1$. Gauname sprendinius $(0, 0)$ ir $(2, 2)$.  
\end{sprendimas}

\begin{pav} Raskite visus sveikuosius lygties $x^2 = 2^n + 1$ sprendinius.
\end{pav}

\begin{sprendimas}
  Pastebėkime, kad jei $(x,n)$ yra sprendinys, tai ir $(-x,n)$ bus
  sprendinys, tad ieškokime tik teigiamų $x$.

  Išskaidykime dauginamaisiais: $(x-1)(x+1) = 2^n$. Dešinioji pusė yra
  dvejeto laipsnis, todėl kairiosios pusės abu dauginamieji taip pat turi
  būti dvejeto laipsniai. Tačiau vienintėliai dvejeto laipsniai, tarp kurių
  skirtumas yra du (o būtent toks skirtumas yra tarp daugiklių), yra
  $2$ ir $4$, vadinasi $x=3$, $n=3$.

  Alternatyviai galima samprotauti taip: kadangi didžiausias $x-1$ ir
  $x+1$ bendras daliklis yra nedidesnis už $2$, tai vienas iš dauginamųjų
  dalinsis daugiausia tik iš $2^1$, vadinasi, bus lygus $2$ arba $1$,
  vadinasi $x$ lygus $0, 1, 2$ arba $3$. Iš jų tinka tik $x=3$.
\end{sprendimas}

Kaip jau buvo matyti praeitos dalies pavyzdyje iš MEMO 2009 olimpiados,
ne visuomet iš karto pavyksta išskaidyti lygtį dauginamaisiais - kartais
pirmiausia reikia gauti papildomos informacijos apie ieškomus sprendinius.
Taip pat ne visuomet aišku, ką daryti išskaidžius. Bendros strategijos
greičiausiai nėra, bet visuomet verta atkreipti dėmesį į dauginamųjų bendrus
daliklius. Dažnai pastebėjus, kad jie jų neturi (arba jie labai riboti) galima
pasistūmėti į priekį. 

\begin{pav} \text{\emph{[IMO 2006]}} Raskite lygties $1 + 2^x + 2^{2x+1} = y^2$
  sveikuosius sprendinius.
\end{pav}
Pirmiausia pastebėkime, kad $x$ negali būti mažesnis už $-1$, nes tuomet
kairė pusė nebus sveikasis skaičius. Patikrinę $x$ reikšmes nuo
$-1$ iki $2$ randame vienintelį sprendinį $(0, \pm 2)$, tad tarkime, kad
$x\geq 3$ ir $y>0$ (iš sprendinio $(x,y)$ gausime ir sprendinį $(x,
-y)$). 

Išskaidykime dauginamaisiais: $$2^x(2^{x+1} +1) = (y-1)(y+1).$$
Kadangi $\dbd(y-1, y+1) \leq 2$, o sandauga $(y-1)(y+1)$ dalijasi iš
$2^x$, tai bent vienas iš dauginamųjų dalinisis iš $2^{x-1}$. Atkreipkite
dėmesį, kad $y\pm 1$ negali būti daug kartų didesnis už $2^{x-1}$, nes
tuomet dešinėje pusė bus didesnė už kairiąją. Lieka viską tvarkingai
pabaigti. Nagrinėkime du atvejus:
\begin{itemize}
  \item[$2^{x-1}|y-1$ -] pažymėję $y-1 = a2^{x-1}$ ir įstatę į lygtį
    gausime $2^x + 2^{2x+1} = a2^{x-1}(a2^{x-1} + 2)$ arba $1 + 8\cdot
    2^{x-2} = a^2 2^{x-2} + a.$ Aišku, kad $a<3$, bet $a=1$ ir $a=2$
    netinka.
  \item[$2^{x-1}|y+1$ -] pažymėję $y+1 = a2^{x-1}$ ir įstatę į lygtį
    gausime $1+8\cdot 2^{x-2} = a^2 2^{x-2} - a$. Aišku, kad $a<4$,
    patikrinę mažesnes reikšmes randame, kad tinka $a=3$, tuomet $x=4$ ir
    $y=23$. 
\end{itemize}
Vadinasi, visi lygties sprendiniai yra $(0, \pm 2)$ ir $(4, \pm 23)$.

\begin{pav} \text{\emph{[BMO 2009]}} Raskite lygties $3^x - 5^y = z^2$ sveikuosius
  teigiamus sprendinius.
\end{pav}

\begin{sprendimas}
  Spręskime lygtį moduliu $4$. Kairė pusė lygsta $(-1)^x - 1$, o dešinė
  $0$ arba $1$. Norint, kad jos būtų lygios $x$ turi būti lyginis. Pažymėję
  $x=2a$ gausime $$-5^y = (z-3^a)(z+3^a).$$
  Kadangi $\dbd(z-3^a, z+3^a)|2\cdot 3^a$, tai vienas iš dauginamųjų
  nesidalins iš $5$, vadinasi, bus lygus $\pm 1$. Tačiau $z+3^a > 3$, todėl
  lieka vienintelis variantas $z - 3^a = -1$ - gauname lygtį $$5^y = 2\cdot
  3^a - 1.$$
  Pastebėkime, kad $a=1, y=1$ yra sprendinys. Jei $a>1$, tai spręsdami
  moduliu $9$ gausime $5^y \equiv -1$, todėl $y$ dalijasi iš $3$. Tačiau
  tuomet $5^y + 1$ dalinsis iš $7$, o $2\cdot 3^a$ nesidalins. Radome, kad
  lygtis turi vienintelį sprendinį $(2,1,2)$.
\end{sprendimas}

Retais atvejais pavyksta panaudoti elegantiškas idėjas apie kai kurių
reiškinių pirminius daliklius. Pavyzdžiui, iš kvadratinių liekanų skyrelio
žinome, kad $x^2 + a$ negali turėti pirminio daliklio, su kuriuo
$\lez{-a}{p} = -1,$ taip pat kaip ir dviejų kvadratų suma negali dalintis iš
pirminio skaičiaus, lygstančio $3$ moduliu $4$, nelyginio laipsnio.

\begin{pav} \text{\emph{[IMO Longlist 1984]}}
  Įrodykite, kad lygtis $4mn - m - n = x^2$ neturi sveikųjų sprendinių.
\end{pav}

\begin{sprendimas}
  Išskaidykime dauginamaisiais: $$(4m-1)(4n-1) = 4x^2 + 1.$$ Kairėje pusėje
  esantys dauginamieji lygsta $3$ moduliu $4$, vadinasi dalijasi bent iš
  vieno pirminio $p$, kuris irgi lygsta $3$ moduliu $4$. Tačiau dešinė
  pusė tokio pirminio daliklio turėti negali, nes tuomet $(2x)^2 \equiv -1
  \m{p}$, ko negali būti, nes $-1$ yra kvadratinė liekana tik moduliu
  pirminių, kurie lygsta $1$ moduliu $4$.
\end{sprendimas}

\subsubsection{Uždaviniai}

\begin{enumerate}
  \item Raskite lygties $x^2 = 200 + 9y$ sveikuosius sprendinius.
    %Nagrinėkime lygtį moduliu $3$. Gausime $x^2 \equiv 2 \m{3}$, o taip
    %būti negali. Vadinasi lygtis sveikųjų sprendinių neturi.
  \item Raskite lygties $x^2 = 100 + y^2$ sveikuosius sprendinius.
    %Ieškokime tik teigiamų sprendinių, nes radę juos, rasime ir neigiamus.
    %Išskaidykime dauginamaisiais: $(x-y)(x+y)=100$. Kadangi $x-y$ ir $x+y$
    %yra vienodo lyginumo, ir jų sandauga lygi $100$, tai jie tegali būti
    %lygūs $2$ ir $50$ arba $10$ ir $10$. Gauname sprendinius $(26,24)$ ir 
    %$(10,0)$. Lieka tik pridurti, kad šie sprendiniai tiks ir paimti su
    %visomis įmanomomis ženklų kombinacijomis.
  \item Raskite lygties $x^2 + y^2 = 4z + 3$ sveikuosius sprendinius.
    %Nagrinėdami lygtį moduliu $4$ gauname, kad dviejų kvadratų suma turi
    %būti lygi trims. Kadangi kvadratai moduliu $4$ įgyja tik liekanas
    %$0$ ir $1$, tai taip niekada nebus. Lygtis sprendinių neturi.
  \item Raskite lygties $x^2 + 2x = 4y + 2$ sveikuosius sprendinius.
    %Pastebėkime, kad $x$ turi būti lyginis. Tačiau tuomet kairioji lygties
    %pusė dalinsis iš $4$, o dešinioji - ne. Sprendinių nėra.
  \item Raskite lygties $x^2 + y^2 = 2x + 3y + 4$ sveikuosius sprendinius.
    %Pastebėkime, kad jei $x>2$ arba $x<0$, tai $x^2> 2x + 2$. Taip pat,
    %jei $y>3$ arba $y<0$, tai $y^2 > 3y + 2$. Vadinasi, arba $x$ turi būti
    %lygus $0, 1, 2$, arba $y$ turi būti lygus $0, 1, 2, 3$. Patikrinę
    %randame sprendinius $(0,-1)$, $(0,4)$, $(2, -1)$ ir $(2, 4)$.
  \item \text{[LitMo 1987]} Nurodykite natūraliųjų skaičių, didesnių už
    $100$, trejetą $(x, y, z)$, tenkinantį lygybę $x^2 + yz^2-xy-xz^2 =
    1987.$
    %Išskaidykime dauginamaisiais: $(x-y)(x-z^2) = 1987.$ Iš čia nesunku
    %rasti didelį sprendinį, pvz $(100^2 + 1,100^2 - 1986, 100)$.	
  \item Raskite lygties $2^x = 3^y + 1$ sveikuosius sprendinius.
    %$3^y$ moduliu $8$ lygsta tik $3$ arba $1$, tad lygtis neturės
    %sprendinių su $x\geq 3$. Patikrinę mažesnes reikšmes randame
    %sprendinius $(2,1)$ ir $(1,0)$.
  \item Raskite lygties $2^x = 3^y - 1$ sveikuosius sprendinius.
    %Pastebėkime, kad jei $x>1$, tai $y$ turi būti lyginis, o tuomet,
    %pažymėję $y=2z$, galime išskaidyti lygties dešiniąją pusę : $2^x =
    %(3^z - 1)(3^z + 1)$. Vienas iš dauginamųjų nesidalins iš $4$ todėl
    %turės būti lygus dviems. Gauname sprendinius $(3,2)$ ir (iš atvejo
    %$x=1$) $(1,1)$.
  \item \text{[LitMo 1988]} Išspręskite natūraliaisiais skaičiais lygtį
    $x^2 + (x+y)^2 = (x+9)^2$. 
    %Kairioji pusė bus didesnė už dešiniąją, jei tik $y$ bus didesnis už
    %$9$, todėl užtenka patikrinti devynias reikšmes. Tai padaryti paprasta
    %persirašius lygtį kaip kvadratinę ($x^2 + x(2y-18) + y^2 - 81$) ir
    %suskaičiavus diskriminantą - $4\cdot 9\cdot (18-2y)$. Tiks reikšmės
    %$y=1$, $y=7$ ir $y=9$ (pastaroji netinka, nes $x$ turėtų būti
    %$0$). Gausime sprendinius $(20,1)$ ir $(8,7)$. 
  \item \text{[LitMo 1989]} Išspręskite lygtį $x^{2y} = 2^z - 1$
    natūraliaisiais skaičiais.
    %Kairioji lygybės pusė yra kvadratas, o dešinioji, jei $z>1$, duoda
    %liekaną $3$ moduliu $4$. Vadinasi $z$ gali būti lygus tik vienam, iš
    %kur randame sprendinius $(1,y,1)$, $y \in \N$.
  \item \text{[LitMo 1989]} Išspręskite sveikaisiais skaičiais lygtį
    $2x^2y^2 + y^2 - 6x^2 - 12 = 0.$
    %Išskaidykime dauginamaisiais: $(y^2 - 3)(2x^2 + 1)=9$. Dauginamasis
    %$2x^2 + 1$ dalo $9$ tik kai $x=0$, $x=\pm 1$ arba $x=\pm 2$. Tinka tik
    %pastarasis, randame sprendinį $(\pm 2, \pm 2)$.
  \item \text{[LitMo 1989]} Išspręskite natūraliaisiais skaičiais lygtį
    $13x^2 + 17y^2 = 1989^2.$
    %Išskaidę dauginamaisiais $1989 = 13\cdot17\cdot9$ matome, kad $x$ turi
    %dalintis iš $17$, o $y$ iš $13$. Pakeitę $x=17a$, $y=13b$ gauname
    %lygtį $17a^2 + 13b^2 = 17\cdot13\cdot 9^2$, iš kurios vėl gauname, kad
    %$a=13k$, $b=17l$. Įstatę ir suprastinę gauname $13k^2 + 17l^2 = 81$.
    %Pastaroji labai paprasta, randame, kad $k=1$, $l=2$, vadinasi pradinės
    %lygties sprendinys bus $(17\cdot13, 2\cdot 17\cdot 13)$.
  \item \text{[IMO Longlist 1972]} Raskite visus sveikuosius lygties
    $1+x+x^2+x^3+x^4 = y^4$ sprendinius.
    %Naudosime įterpimo tarp kvadratų (šiuo atveju ketvirtųjų laipsnių)
    %triuką. Kai $x$ teigiamas, tai $x^4<1+x+x^2+x^3+x^4<(x+1)^4$,o kai
    %$x$ neigiamas, tai $(x+1)^4<1+x+x^2+x^3+x^4\leq x^4$ (lygybė įgyjama
    %tik kai $x=-1$). Vadinasi lieka patikrinti dvi reikšmes $x=0$,
    %$x=-1$, iš kurių gauname sprendinius $(-1, \pm 1)$ ir $(0, \pm 1)$.
  \item \text{[IMO Longlist 1977]} Raskite visus sveikuosius lygties 
    $7a + 14b = 5a^2 + 5ab + 5b^2$ sprendinius.
    %Dešinė lygties pusė beveik visuomet didesnė už kairiąją. Tą paprasta
    %išnaudoti persirašius lygtį kaip kvadratinę ($5a^2 + a(5b-7)
    %+5b^2-14b)$ ir suskaičiauvus diskriminantą: $-15(5b^2-14b) +49.$ Jis
    %nebus neigiamas tik kai $b$ tenkins $0\leq b \leq 3$, patikrinę šias
    %reikšmes gauname du sprendinius - $(0,0)$ ir $(-1, 3)$.
  \item \text{[LitMo 1986]} Išspręskite lygtį $x^y = y^{x-y}$
    natūraliaisiais skaičiais.
    %Kadangi kairioji pusė sveikas skaičius, tai $x$ turi būti nemažesnis
    %už $y$. Jei jie lygus, tai tinka tik $(1,1)$, tad tarkime, kad
    %$x>y$. Tuomet gausime, kad $x-y$ turi būti didesnis už $y$, ir kad
    %$y|x$. Pažymėję $x=ky$ gauname $(ky)^{y} = y^{(k-1)y}$, arba
    %$k = y^{k-2}$. Ši lygtis turi tik du sprendinius $k=3,y=3$ ir
    %$k=4, y=2$, nes jei $k>4$ tai $k<2^{k-2}\leq y^{k-2}$. Pakeitę atgal,
    %gauname pradinės lygties sprendinius $(6,3)$ ir $(8,2)$.
  \item \text{[LitMo 1987]} Išspręskite lygtį $6!x!=y!$ natūraliaisiais
    skaičiais.
    %Uždavinys ekvivalentus tokiam - išskaidykite $6!$ į paeiliui einančių
    %skaičių sandaugą. Daugiausia jį galima išskaidyti į $6$
    %dauginamuosius, tuomet gausime sprendinį $(1,6)$. Į penkis ir keturis
    %dauginamuosius išskaidyti nepavyks, nes jei visi bus mažesni už
    %$6$, tai sandauga bus per maža, o jei didesni, tai turės arba dalintis
    %iš $7$ (arba $11$, arba $13$) arba sandauga jau bus per didelė. Į tris
    %dauginamuosius išskaidyti galima - $6! = 8\cdot 9 \cdot 10$, į du ne
    %($26\cdot27 < 720 < 27\cdot 28$), į vieną, aišku, galima. Randame dar
    %du sprendinius: $(7,10)$ ir $(6!-1, 6!)$
  \item \text{[LitKo 2007]} Raskite visus sveikųjų skaičių $x, y, z$ ir $t$
    ketvertus $(x,y,z,t)$ tenkinančius lygtį $x^2 + y^2 + z^2 + t^2 = 3(x +
    y + z + t)$.
    %Sukelkime viską į vieną lygties pusę : $x^2 - 3x + y^2 - 3y + z^2 - 3z
    %+ t^2 - 3t = 0$. Mažiausios reikšmės kurias gali įgyti reiškinys
    %$x^2 - 3x$ yra $4, 0, -2$, o visos likusios ne mažesnės už dešimt.
    %Susumavę gausime nulį tik arba atveju $0+0+0+0$ arba $0-2-2+4$, tad
    %sprendiniai bus $(0,0,0,0)$ ir visos įmanomos kombinacijos iš $0$, $1$
    %arba $2$, $1$ arba $2$, $-1$ arba $4$ (pvz. $(0,1,1,-1)$, $(4,2,0,1)$,
    %\ldots).
  \item Raskite visus natūraliuosius lygties $x^3 - y^3 = xy + 61$
    sprendinius.
    %Parodysime, kad kairioji lygties pusė yra beveik visuomet didesnė už
    %dešiniąją. Kadangi $x>y$, tai $xy+61<x^2+61$. Iš kitos pusės,
    %$x^3-y^3 \geq x^3 - (x-1)^3 = 3x^2 - 3x + 1$, kas yra daugiau už
    %$x^2 + 61$, kai $x>6$. Vadinasi, lieka patikrinti tik keletą reikšmių,
    %ką padarę randame vienintėlį sprendinį $(6,5)$.
  \item \text{[JBMO 2009]} Raskite lygties $2^a 3^b + 9 = c^2$
    natūraliuosius sprendinius.
    %Jei $b=0$, tai lygtis užrašoma kaip $2^a = (c-3)(c+3)$. Vienintėliai
    %dvejeto laipsniai besiskiriantys per $6$ yra $2$ ir $8$, randame
    %sprendinį $(4,0,5)$. Tegu $b>0$, tuomet $c$ dalijasi iš trijų ir
    %$b\geq 2$. Pažymėję $b-2 = d$, $c = 3n$ gauname $2^a3^d =
    %(n-1)(n+1)$. Kadangi $\dbd(n-1, n+1) \leq 2$, tai vienas iš dauginamųjų
    %nesidalija iš $3$. Tada jis arba yra lygus $1$, arba dalijasi iš
    %$2$. Jei lygus vienetui, tai tuomet $n=2$, randame sprendinį $(0,3,6)$.
    %Jei dalijasi iš dviejų, tai tuomet $n$ nelyginis ir $a\geq 2$. Pažymėję
    %$n=2k-1$ ir $a-2 = e$ gauname $2^e3^d = k(k+1)$. Kadangi $k$ ir
    %$k+1$ tarpusavyje pirminiai, tai arba $k=2^e$, $k+1=3^d$ arba $k=3^d$,
    %$k+1 = 2^e$. Pirmu atveju gauname lygtį $3^d = 2^e+1$, antru $2^e = 3^d
    %+ 1$.  
  \item Raskite lygties $3^x - 2^y = 1$ natūraliuosius sprendinius.
    %$3^x \equiv (-1)^x \m{4}$, todėl, jei $y>1$ ($y=1$ tinka, tuomet
    %$x=1$), tai $x$ turi būti lyginis. Pažymėję $x=2a$ gauname lygtį
    %$2^y = (3^a-1)(3^a+1)$. Abu dauginamieji esantys dešinėje pusėje turi
    %būti dvejeto laipsniai, bet besiskiriantys per du yra tik $2$ ir
    %$4$. Vadinasi $a=1$, $x=2$, $y=3$.
  \item Raskite lygties $x^2 + 3 = 12y^3 - 16y + 1$ sveikuosius
    sprendinius.
    %Išskaidykime $x^3 = 4(y-1)(3y^2 + 3y -1)$. Kadangi su visomis $y$
    %reikšmėmis $3y^2 + 3y - 1 \equiv 2 \m{3}$, tai jis turės pirminį
    %daliklį duodantį liekaną $2$ moduliu $3$. Tačiau kairioji lygties pusė
    %tokio daliklio turėti negali, nes $-3$ negali būti kvadratinė liekana
    %moduliu pirminio $p \equiv 2 \m{3}$. Vadinasi, $3y^2 + 3y - 1$ turi
    %būti lygus $-1$, todėl $y=0$ arba $y=-1$. Tinka tik pirmasis, randame
    %sprendinį $(\pm 1,0)$. 
\end{enumerate}
