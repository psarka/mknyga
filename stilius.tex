\section{Knygos stiliaus pavyzdžio failas}

Matematikos knygoje yra keturi skyriai \chapter - skaičių teorija, algebra,
kombinatorika ir geometrija. Skyriai skirstomi į sekcijas \section, o
pastarosios, jei reikia (pavyzdžiui nelygybės ar funkcinės lygtys), į
subsekcijas \subsection. Šie trys struktūriniai vienetai yra vaizduojami
turinyje. Sekcijos su tekstu arba, jei be teksto, jų subsekcijos yra
pagrindiniai "teksto vienetai", vadinami skyreliais.

Skyreliai yra galiausiai skirstomi į subsubsekcijas \subsubsection, kurios
turinyje nevaizduojamos. Didelės sekcijos \section, t.y. tos, kurios turi
subsekcijas \subsection, yra iškeliamos į atskirus failus, kurie nurodomi
komanda \input (žr. pavyzdžiui failą algebra.tex).

\subsubsection{aplinkos}

Subsubsekcijos nėra vienintėlis struktūrinis elementas, padedantis organizuoti
rašomą skyrelį. Taip pat yra naudojamos aplinkos teiginiams, teoremoms,
išvadoms, pavyzdžiams, sprendimams ir įrodymams, bei jų numeruotiems
atitikmenims. Jas visas galite rasti aprašytas faile knyga.sty, ir, jei ko
trūksta, galite pridėti savo. Štai kaip jos atrodo tekste:

\begin{thm}[Matematiko teorema]
Labai įdomios teoremos vadinamos matematiko vardu formuluotė. Ši teorema bus
nenumeruota. Numeruotos santrumpa yra thmnr.
\end{thm}

\begin{proof}[Įrodymas]
Teoremos įrodymas. (Dėl babel sulietuvinimo klaidos tenka pridėti [Įrodymas],
arba tą klaidą pataisyti. Tam, Ubuntu naudotojai gali faile
/usr/share/texmf-texlive/tex/latex/lithuanian/lithuanian.ldf pridėti eilutę
\def\proofname{\k{I}rodymas}.)
\end{proof}

\begin{pav}
Pavyzdys. Numeruotų pavyzdžių santrumpa pavnr.
\end{pav}

\begin{sprendimas}
Pavyzdžio sprendimas. Pavyzdžiai ir uždaviniai yra sprendžiami, o teoremos įrodinėjamos.
\end{sprendimas}

\begin{teig}
Labai įdomus teiginys. Numeruoti teignr.
\end{teig}

\begin{isv}
Kokio nors teiginio išvada. Numeruota išvada isvnr.
\end{isv}

\subsubsection{santrumpos}

Faile knyga.sty taip pat yra aprašytos kai kurios šiek tiek gyvenimą
palengvinančios santrumpos:

\begin{itemize}
\item Sveikųjų, natūraliųjų, kompleksinių ir t.t. aibės rašomos kaip \Z, \N, \C
ir pan, užuot rinkus {\mathbb X}.
\item Didžiausio bendro daliklio, mažiausio bendro kartotinio santrumpos
rašomos \dbd ir \mbk. Taip pat yra santrumpa ir hiperboliniam arktangentui
\arctanh, kuris kažkokiu būdu įsliūkino į knygą.
\item Komanda \lez{a}{b} pagamins Ležandro simbolį.
\item Komanda \m{p} parašys "(mod p)" (daug maž tą patį daro \pmod{p}).
\end{itemize}

\subsubsection{stilius}

TODO

\subsubsection{Uždaviniai}

Paskutinė subsubsekcija kiekviename skyrelyje yra uždavinių. Ją visuomet reikia
formatuoti taip, kaip čia, nes ją automatiškai apdoroja programa parse.py. Ji
surenka po uždaviniais esančius sprendimus, sudeda į sprendimų failą ir padaro
nuorodas, kurias galite matyti ir spaudyti spalvotoje versijoje. Taigi, po
\subsubsection{Uždaviniai} visuomet eina:

\begin{enumerate}
\item Pirmo uždavinio sąlyga. Sąlyga labai trumpa, matyt uždavinys nesudėtingas.
%Pirmo uždavinio sprendimas komentaruose. Iš ties buvo paprastas.
\item Antras nepamatuotai sunkus uždavinys su labai ilga ir nuobodžia sąlyga.
Antras nepamatuotai sunkus uždavinys su labai ilga ir nuobodžia sąlyga. Antras
nepamatuotai sunkus uždavinys su labai ilga ir nuobodžia sąlyga. Antras
nepamatuotai sunkus uždavinys su labai ilga ir nuobodžia sąlyga.
%Jo sprendimas komentaruose. Labai jau techniškas techniškas techniškas
%techniškas techniškas techniškas techniškas techniškas techniškas techniškas
%techniškas techniškas techniškas techniškas techniškas techniškas techniškas
\item Trečias uždavinys, kuris prieš tai buvo ketvirtas, bet kadangi nereikia
rašyti uždavinių numerių, tai sukeisti buvo nesudėtinga. Taip pat nereikia
ieškoti ir atitinkamame faile sukeisti sprendimų, nes juos perkėlėme kartu su
sąlyga.
%Sprendimas trečiojo uždavinio.
\item Ketvirtas uždavinys, minėtas anksčiau.
%Ir jo sprendimas.
\end{enumerate}
