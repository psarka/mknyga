\chapter{Sprendimai}
\section*{Skaičių teorija}
\subsection*{Dalumas}
\begin{enumerate}
\item
    Jei $n|3a$, tai $n|12a$ ir $n|12a + 5b - 12a = 5b$. Aišku, kad iš
    $n|5b$ seka ir $n|10b$.
\item
    Pastebėkime, kad $b$ galima išreikšti kaip $3(2a+5b) - 2(3a+7b)$, o
    $a$ kaip $5(3a+7b) - 7(2a+5b)$, vadinasi, abu jie iš $n$ dalinsis.
\item
    Ne, jos visos trys neteisingos.
    
    a) Jei $x|a+b$, tai nebūtinai $x|a$ ir $x|b$. Pavyzdžiui, $5|2+3$, bet
    $5\nmid 2$ ir $5\nmid 3$.
    
    b) Jei $x|a\cdot b$, tai nebūtinai $x|a$ arba $x|b$. Pavyzdžiui,
    $6|2\cdot 3$, bet $6\nmid 2$ ir $6\nmid 3$. Kaip bebūtų, ši savybė
    teisinga, kai $x$ pirminis (jei dviejų skaičių sandauga dalijasi iš
    pirminio skaičiaus, tai iš to pirminio dalijasi nors vienas iš
    skaičių).
    
    c) Jei $x|a$ ir $y|a$, tai nebūtinai $xy|a$. Pavyzdžiui, $4|12$ ir
    $6|12$, bet $24\nmid 12$.
\item
    Taip gauto skaičiaus skaitmenų suma yra lygi $45$, tad pagal dalumo
    požymį jis iš $9$ dalinsis.
\item
    Pritaikome dalybos iš $11$ požymį: $a-b +b-a =0$ dalijasi iš
    $11$, vadinasi ir skaičius $\overline{abba}$ dalinsis iš $11$.
\item
    a) Jei vietoje žvaigždutės įrašysime $x$, tai gauto skaičiaus
    skaitmenų suma bus lygi $15+x$. Ji dalinsis iš $9$ kai $x=3$
    
    b) Pagal dalumo požymį iš $8$ turi dalintis $45*$. Kadangi $400$
    dalijasi iš $8$ ir $56$ dalijasi iš $8$ tai vietoje žvaigždutės galime
    įrašyti $6$.
    
    c) Alternuojanti suma vietoj žvaigždutės įrašius $x$ yra lygi
    $3-x$. Ji dalinsis iš $11$, kai $x=3$.
\item
    Pakanka pastebėti, kad $10a+b$ yra lygus $10(a+4b) -13\cdot 3b$.
\item
    Atmetę lyginius ir dalius iš $5$ skaičius gauname, kad lieka
    patikrinti $181$, $183$, $187$, $189$, $191$, $193$, $197$ ir $199$. Pagal dalumo
    požymius $183$ ir $189$ dalijasi iš $3$, o $187$ iš $11$. Iš
    $7$ šitame intervale dalijasi skaičiai $182, 189$ ir $196$, o iš
    $13$ tik $182$. Vadinasi, skaičiai $181, 191, 193, 197$ ir $199$
    nesidalija iš pirminių, mažesnių už $\sqrt{199} \approx 14$, todėl
    yra pirminiai.
\item
    Išskaidykime: $n^2 + 5n + 6 = (n+2)(n+3)$. Kadangi su visomis
    natūraliosiomis $n$ reikšmėmis abu dauginamieji yra didesni už
    $1$, tai jų sandauga niekada nebus pirminis skaičius.
\item
    Išskaidykime dauginamaisiais: $$a^3 + 2a + b^3 + 2b = 2(a+b) + (a+b)(a^2
    -ab + b^2) = (a+b)(a^2 -ab+b^2+2).$$ Kadangi $a+b$ dalijasi iš $n$, tai
    ir duotas reiškinys iš $n$ dalinsis.
\item
    Pagal Euklido algoritmo išvadą tokiu būdu galima išreikšti
    $\dbd(8,5)=1$. Bet tuomet galima išreikšti ir bet kurį skaičių
    $a$ - pakanka vietoje $x$ ir $y$, naudojamų vieneto išraiškoje, imti
    $ax$ ir $ay$.
\item
    Negali. Jei jo lyginėse pozicijose esančių skaitmenų sumą pažymėsime
    $x$, o nelyginėse $y$, tai gausime, kad $x-y$ turi dalintis iš
    $11$. Kadangi $x+y=5$, tai $-11 <x-y< 11$, lieka tiktai variantas
    $x-y=0$. Bet tokiu atveju $x$ ir $y$ turėtų būti arba abu lyginiai,
    arba abu nelyginiai, o tai prieštarautų tam, kad jų suma nelyginė.
\item
    Skaičius $p_1^{\alpha_1} \cdots p_n^{\alpha_n}$ turi $(\alpha_1 +
    1)\cdots(\alpha_n+1)$ daliklių. Kad daliklių skaičius būtų nelyginis,
    visi $\alpha_1, \dots ,\alpha_n$ turi būti lyginiai. Tačiau tuomet
    skaičius bus sveikojo skaičiaus kvadratas ($p_1^{\alpha_1/2} \cdots
    p_n^{\alpha_n/2})^2$.
\item
    Jei trupmena $\frac{a}{b}$ yra suprastinama, tai $\dbd(a,b)=d>1$.
    Tada, kadangi $d|a$ ir $d|b$, tai $d|a-b$ ir $d|a+b$, vadinasi, ir
    trupmena $\frac{a-b}{a+b}$ bus suprastinama. Atvirkščias teiginys nėra
    teisingas. Iš $\dbd(a-b,a+b)=d>1$ galime gauti, kad $d|2a$ ir
    $d|2b$, o iš čia ir idėją kontrapavyzdžiui: $\frac{5-3}{5+3}$
    suprastinama, o $\frac{5}{3}$ - ne.
\item
    Pažymėkime $\dbd(a,b) = d$ ir $a=da_1$, $b=db_1$. Kadangi $d|b$, o
    $\dbd(a_1,b) = 1$, tai $\mbk(a,b) = \mbk(da_1, b)=a_1b$. Lieka
    patikrinti: $$\mbk(a,b)\cdot \dbd(a,b) = d \cdot a_1\cdot b = a\cdot b$$
\item
    Jei $11|3x+7y$ ir $11|2x+5y$, tai $11|3(2x+5y) - 2(3x + 7y)$, t.y.
    $11|y$. Tačiau jei $11|2x+5y$ ir $11|y$, tai $11|2x \implies
    11|x$. Gavome, kad $11|x$ ir $11|y$, todėl tikrai $11^2|x^2 + 3y^2$.
\item
    Pažymėkime skaičiaus skaitmenų esančių lyginėse vietose sumą
    $a$ ir nelyginėse $b$. Pagal dalumo iš $9$ požymį $a+b$ turi dalintis
    iš $9$. Pastebėkime, kad $a+b$ negali būti lygus $9$, nes tada vienas
    iš jų turėtų būti lyginis, kitas nelyginis, ir jų skirtumas $a-b$
    nebūtų lygus $0$ ir nesidalintų iš $11$ ($-11<a-b<11$). Vadinasi,
    $a+b$ turi būti lygus bent $18$.
\item
    Aišku, kad $n=12! + 2$ tenkins sąlygą.
\item
    Įvertinkime grubiai $d$ dydį. Kadangi $a$ yra šimtaženklis, tai
    $b$ neviršys $100\cdot 9$. Tuomet jo skaitmenų suma, $c$, neviršys
    $3\cdot 9$, o šio skaitmenų suma, $d$, neviršys $2 + 9 = 11$.
    Pagal dalumo iš $9$ požymį $9|a \implies 9|b \implies 9|c \implies
    9|d$. Vienintelis teigiamas skaičius besidalijantis iš $9$ ir
    nedidesnis už $11$ yra $9$.
\item
    Raskime paskutinį skaičiaus $27^{28}$ skaitmenį. $27^1$ paskutinis
    skaitmuo $7$, $27^2$ - $9$, $27^3$ - $3$, $27^4$ - $1$, $27^5$ - $7$,
    \dots. Matome, kad paskutinis skaitmuo keliant laipsniais kartojasi
    kas keturis, vadinasi, $28$ laipsnio bus toks pat kaip ir $4$, t.y.
    $1$. Tuomet $n$ paskutinis skaitmuo bus lygus $5$, vadinasi, jis
    dalinsis iš $5$. 
\item
    Kadangi $p$ pirminis, tai jokie mažesni už jį skaičiai iš $p$
    nesidalins. Tuomet iš $p$ nesidalins ir $k!$ ir $(p-k)!$. Kadangi iš
    $p$ dalinasi $p!$, t.y. trupmenos skaitiklis, bet nesidalija trupmenos
    vardiklis, tai suprastinus $p$ neišsiprastins, ir gautas skaičius iš
    $p$ dalinsis.
\item
    a) Pertvarkę $n^2 + 1 = (n-1)(n+1) + 2$ gauname, kad $\dbd(n^2 + 1, n+1)
    = \dbd(2, n+1)$. Pastarasis bus didesnis už $1$ tada, kai
    $n$ bus nelyginis, o jų iki $100$ bus $50$.
    
    b) Pertvarkę $n^2 + 1 = (n-2)(n+2) +5$ gauname, kad $\dbd(n^2 + 1,
    n+2) = \dbd(5,n+2)$. Pastarasis bus didesnis už $1$ tada, kai
    $n+2$ dalinsis iš $5$. Tokių skaičių bus $20$ - $3, 8, \dots 98$.
\item
    Perrašykime: $$\frac{n^3+3}{n^2+7} = \frac{(n^2+7)n - 7n + 3}{n^2+7} =
    n - \frac{7n-3}{n^2+7}.$$
    Matome, kad duotas skaičius bus sveikasis tik tada, kai sveikasis bus
    $\frac{7n-3}{n^2+7}.$ Pastebėkime, kad kai $|n|>6$, tai vardiklis tampa
    moduliu didesnis už skaitiklį, tad trupmena tikrai nebus sveikasis
    skaičius. Lieka patikrinti likusias reikšmes, iš kurių tinka tik
    $n=2$ ir $n=5$.
\item
    Pirma, aiškumo dėlei, parodysime, kad teiginys teisingas su $n=3$
    (su $n=2$, ir $n=1$ jis teisingas pagal dalumo iš $9$ ir $3$
    požymius). Užrašykime $$\underbrace{11\ldots1}_{27} =
    1\underbrace{00\ldots0}_{8}1\underbrace{00\ldots0}_{8}1\cdot
    \underbrace{11\ldots1}_{9}.$$ Dešinėje pusėje pirmojo dauginamojo
    skaitmenų suma lygi $3$, todėl jis dalijasi iš $3$, o antrasis
    dauginamasis dalijasi iš $9$, vadinasi, sandauga dalijasi iš $27$.
    Bendru atveju naudosime indukciją. Užrašę
    $$\underbrace{11\ldots1}_{3^n} =
    1\underbrace{00\ldots0}_{3^{n-1}
    -1}1\underbrace{00\ldots0}_{3^{n-1}-1}1\cdot
    \underbrace{11\ldots1}_{3^{n-1}}$$ ir tarę, kad
    $\underbrace{11\ldots1}_{3^{n-1}}$ dalijasi iš $3^{n-1}$ gauname, kad
    $\underbrace{11\ldots1}_{3^{n}}$ dalijasi iš $3^{n}$.
\item
    Iš sąlygos aišku, kad skaičius turi dalintis bent iš $2$, $3$ ir
    $5$. Kadangi ieškome mažiausio tokio skaičiaus, tai galime tarti, kad
    daugiau pirminių daliklių skaičius neturės, nes iš to kad $2^a3^b5^cq$
    tenkina sąlygą gautume ir kad $2^a3^b5^c$ tenkina sąlygą, o jis
    mažesnis. Pagal sąlygą $2^{a-1}3^b5^c$ turi būti kvadratas,
    $2^a3^{b-1}5^c$ - kubas, $2^a3^b5^{c-1}$ - penktasis laipsnis.
    Vadinasi $2|a-1, 2|b, 2|c, 3|a, 3|b-1, 3|c, 5|a, 5|b, 5|c-1$.
    Kiekvieno iš $a,b,c$ ieškome atskirai. Mažiausias nelyginis iš
    $3$ ir $5$ besidalijantis skaičius yra $15$, vadinasi $a=15$.
    Analogiškai $b=10$, $c=6$. Gavome, kad mažiausias skaičius tenkinantis
    sąlygą yra $2^{15}3^{10}5^6$.
\item
    Skaičius $75$ išsiskaido kaip $3\cdot 5\cdot5$, vadinasi jis turės ne
    daugiau kaip $3$ skirtingus pirminius daliklius, iš kurių du yra
    $5$ ir $3$. Mažiausias skaičius, kurį gauname dviejų pirminių daliklių
    atveju yra $3^{14}5^4$, mažiausias skaičius, kurį gauname trijų
    pirminių daliklių atveju, yra $2^43^45^2$. Šis ir bus mažiausias.
\item
    Pastebėkime, kad $5n+3$ užrašomas kaip $4(2n + 1) - (3n+1)$. Pažymėję
    $2n+1 = a^2$ ir $3n+1 = b^2$ gausime, kad $5n+3$ išsiskaido kaip
    $(2a-b)(2a+b)$. Jis nebus pirminis, jei $2a-b >1$. Patikrinkime
    atvejį, kai $2a = b+1$. Įsistatę gausime lygčių sistemą
    $$\begin{cases}
      2n+1 & = a^2, \\
      3n+1 & = (2a-1)^2.
      \end{cases}$$ 
    Išsprendę randame vienintėlį sveikąjį sprendinį $a=1$, $n=0$.
\item
    Tarkime priešingai, kad tokių pirminių skaičių yra baigtinis skaičius.
    Pažymėkime juos $p_1, p_2, \ldots, p_k$ ir nagrinėkime skaičių $4p_1p_2\cdots
    p_k - 1$. Jis nesidalins iš nė vieno pirminio $p_1, \ldots, p_k$,
    vadinasi, visi jo pirminiai dalikliai bus pavidalo $4k+1$. Tačiau tokių
    daliklių ir jų laipsnių sandauga bus pavidalo $4k+1$, vadinasi, negali
    būti lygi $4p_1p_2\cdots p_k - 1$. Prieštara.
\item
    Jei sudauginę gavome 1, tai priešpaskutinis skaičius turėjo būti
    pavidalo 1\dots11.  Įrodysime, kad tokio tipo skaičiaus negalime gauti
    daugindami skaitmenis. Tam užteks parodyti, kad jis turi pirminių
    daliklių didesnių už 7. Išties, jei 1\dots11 dalijasi iš 3, tai
    dalijasi ir iš 111, ir iš 37. Jei 1\dots11 dalijasi iš 7, tai dalijasi
    iš 111111, ir taip pat dalijasi iš 37. Jei nesidalija nei iš 3, nei iš
    7, tai tikrai dalijasi iš pirminio didesnio už juos. Vadinasi, sąlygą
    tenkina tik skaičiai pavidalo 1\dots11.
\item
    Pastebėkime, kad pirminiai $p$ ir $q$ yra panašaus dydžio, t.y.
    $p\leq q+6$ ir $q\leq p+7$. Taip pat, $p$ ir $q$ negali būti labai
    dideli, nes bet kokio skaičiaus didžiausias daliklis neskaitant paties
    skaičiaus yra bent dvigubai už jį mažesnis. Pasinaudokime tuo: kadangi $p|q+6$ ir
    $p\geq q-7$, tai arba $q-7$ bus didesnis nei pusė $q+6$ ir $p$ turės
    būti lygus $q+6$, arba $q-7$ bus nedidesnis nei $q+6$. Pirmuoju atveju
    iš $p=q+6$ gauname $q|q+13$, iš kur $q=13$, $p=19$. Antruoju atveju
    $q-7$ turi būti nedidesnis už pusę $q+6$, arba sutvarkius, $q\leq 20$.
    Vadinasi, lieka patikrinti tik $q$ reikšmes $2,3,5,7,11,13,17,19$. Tai
    padaryti nesunku: nei viena iš jų, išskyrus jau rastą $13$, netinka.
\item
    Pastebėkime, kad didžiausias $n$ daliklis neviršija $n$, antras pagal
    dydį neviršija $\frac{n}{2}$, trečias pagal dydį neviršija
    $\frac{n}{3}$ ir taip toliau. Tuomet gausime, kad $$d_k d_{k-1} +
    \cdots + d_{2}d_{1} < \frac{n}{1}\frac{n}{2} + \frac{n}{3}\frac{n}{4}
    + \dots + \frac{n}{k}\frac{n}{k+1} = n^2(\frac{1}{1}\frac{1}{2} +
    \frac{1}{2}\frac{1}{3} + \dots + \frac{1}{k}\frac{1}{k+1}).$$
    Įvertinkime sumą:
    
    $$\frac{1}{1}\frac{1}{2} + \frac{1}{2}\frac{1}{3} + \dots +
    \frac{1}{k}\frac{1}{k+1} = \frac{1}{1} - \frac{1}{2} + \frac{1}{2}
    -\frac{1}{3} + \dots + \frac{1}{k} -\frac{1}{k+1} = 1 -
    \frac{1}{k+1} < 1.$$
    
    Įrodysime, kad $d_1 d_2 + d_2 d_3 + \cdots + d_{k-1}d_{k}$ dalo
    $n^2$ tada ir tik tada, kai $n$ yra pirminis. Tarkime priešingai, tegu
    $n$ sudėtinis, ir tegu $p$ yra mažiausias pirminis $n$ daliklis.
    Tuomet $$n^2 > d_1 d_2 + d_2 d_3 + \cdots + d_{k-1}d_{k} >
    n\frac{n}{p},$$ prieštara, nes $\frac{n^2}{p}$ yra antras pagal dydį
    $n^2$ daliklis.
\end{enumerate} 
\subsection*{Lyginiai}
\begin{enumerate}
\item
    a) $1+11+111+1111+11111 = 1 + (9+2) + (108 + 3) + (1107 + 4) + (11106 +
    5) \equiv 6 \m{9}.$
    
    b) $555\cdot 777 + 666\cdot 888 \equiv 6\cdot 3 + 0\cdot 6 \equiv
    0\m{9}$.
    
    c) $3^{99} \equiv 1 \m{2}$, $\equiv 0 \m{3}$, $\equiv -1 \m{4}$,
    $\equiv 2 \m{5}$, $\equiv 3\m{6}$, $\equiv -1\m{7}$.
    
    d) $7^4 \equiv 1 \m{10} \implies 7^{777} \equiv 7 \m{10}$.
\item
    Įrodysime naudodamiesi apibrėžimu. Išskaidykime skirtumą: $$ab + cd -
    ad - bc = a(b-d) + c(d-b) = (a-c)(b-d).$$ Kadangi $a-c|(a-c)(b-d)$,
    tai iš ties $ab+cd \equiv ad + bc \m{a-c}$.
\item
    Jei skaičius lyginis, tai jo kvadrato dalybos iš $4$ liekana bus $0$, jei
    nelyginis ($\equiv \pm1\m{4}$), tai $1$.
\item
    Išskaidykime $n^5-n = (n-1)n(n+1)(n^2+1).$ Akivaizdu, kad duotas
    reiškinys dalijasi iš $2$ ir $3$. Įrodysime, kad dalijasi ir iš
    $5$. Jei $n$ lygsta 1, 0 arba -1, tai tuomet iš $5$ dalijasi
    atitinkamai $n-1$, $n$, $n+1$, o jei $n$ lygsta $\pm 2$ moduliu
    $5$ tai iš $5$ dalijasi $n^2 +1$ ($n^2 + 1\equiv 4 + 1 \equiv 0 \m{5}$). 
\item
    Dalindami kvadratą iš trijų gausime tiktai liekanas $0$ arba $1$. Jas
    sumuojant nulį galima gauti vieninteliu būdu, kai abu dėmenys lygūs
    $0$.
\item
    Dalindami kvadratą iš septynių, gausime liekanas $0$, $1$, $2$ arba
    $4$. Kaip ir praeitame uždavinyje, jas sumuojant, nulį galima gauti
    tik, kai abu dėmenys lygūs $0$.
\item
    Nelyginis skaičius moduliu $8$ gali duoti liekanas $\pm 1$ ir
    $\pm 3$. Tuomet jo kvadratas duos liekanas $(\pm1)^2 \equiv 1 \m{8}$ ir
    $(\pm 3)^2 \equiv 9 \equiv 1 \m{8}$.
\item
    Kadangi $6|x^3-x$, tai $x^3\equiv x\m{6}$. Tuomet ir $a^3 + b^3 + c^3
    \equiv a+b+c \m{6}.$
\item
    Kadangi $a$ nesidalija iš $2$ tai $a\equiv \pm1 \m{8}$ arba $a\equiv
    \pm3 \m{8}$. Abiem atvejais pakėlę abi lygybės puses kvadratu gauname
    $a^2 \equiv 1\m{8}$. Analogiškai, kadangi $a$ nesidalija iš $3$, tai
    $a\equiv \pm1 \m{3}$, vadinasi $a^2 \equiv 1 \m{3}$. Gavome, kad
    $a^2-1$ dalijasi iš $3$ ir $8$, vadinasi, $a^2 \equiv 1 \m{24}$.
\item
    Kvadratai duoda liekanas $1,0$ moduliu $4$, o dviejų nelyginių skaičių
    kvadratų suma duoda liekaną $2$.
\item
    Išskaidykime - $5\cdot3\cdot2^3 | n(n^2+1)(n+1)(n-1)$. Nesunku
    įsitikinti, kad su visomis $n$ reikšmėmis $n(n^2+1)(n+1)(n-1)$
    dalijasi iš $5$ ir $3$. Patikrinkime, kada dalijasi iš $8$. Kai
    $n$ nelyginis, tai trys dauginamieji $(n^2+1)(n+1)(n-1)$ lyginiai,
    todėl iš $8$ dalinsis. Kai $n$ lyginis, tai vienintėlis lyginis
    dauginamasis yra $n$, vadinasi sandauga dalinsis iš $8$ kai $n$
    dalinsis iš $8$. Gavome, kad $120|(n^5-n)$, kai su visomis nelyginėmis
    ir iš aštuonių besidalijančiomis reikšmėmis.
\item
    Jei abu pirminiai $p$ ir $q$ nesidalija iš $3$, tai jų kvadratai
    lygsta $1$ moduliu $3$. Tačiau tuomet $p^2 - 2q^2 \equiv 1-2 \equiv -1
    \not \equiv 1\m{3}$. Vadinasi bent vienas iš jų dalijasi iš trijų,
    t.y. yra lygus trims. Patikriname: su $q=3$ sprendinių nėra, o su  
    $p=3$ gauname $q=2$.
\item
    Iš pradžių raskime, su kuriomis $n$ reikšmėmis duotas daugianaris
    dalijasi iš $11$. Tam užtenka perrinkti $11$ liekanų - gausime, kad
    tinka tik $n\equiv 4 \m{11}$. Įstatę $n = 11k + 4$ gausime $11^2 k^2 +
    11^2k + 33$, kas su jokia $k$ reikšme nesidalija iš $121$.
\item
    Užrašykime reiškinį kaip $(10 - 1)(10^{n-1} + \cdots + 10 + 1) + 45n$.
    Padaliję iš $9$ matome, kad dalmuo dar dalijasi iš $3$:
    $10^{n-1} + \cdots + 10 + 1 + 5n \equiv 1 + 1 + \cdots 1 + 5n \equiv
    6n \equiv 0 \m{3}$.
\item
    Pastebėkime, kad su bet kuriuo $k$ yra teisinga $10^k \equiv 1^k
    \equiv 1\m{9}$. Tuomet $$\overline{a_1a_2\dots a_n}=a_1\cdot 10^{n-1}
    + a_2\cdot 10^{n-2} + \cdots + a_n-1\cdot 10 + a_n \equiv a_1 + a_2 +
    \cdots a_n \m{9}.$$ 
\item
    Perrašykime sąlygą kaip $n|a-b$. Aišku, kad jei $d|n$ ir $d|a$, tai
    $d$ turi dalinti ir $b$. Lygiai taip pat, jei $d|n$ ir $d|b$, tai $d$
    turi dalinti ir $a$. Vadinasi iš ties $\dbd(a,n) = \dbd(b,n)$.
\item
    Pastebėję, kad $899 = 900 - 1 = (30-1)(30+1)$ galime ieškoti, su
    kuriomis $n$ reikšmėmis duotas reiškinys dalijasi iš $29$ ir $31$
    atskirai.  Kadangi $36^n \equiv 7^n \m{29}$ ir $24^n \equiv
    (-5)^n\m{29}$, tai $$36^n + 24^n - 7^n -5^n\equiv (-5)^n -5^n
    \m{29},$$ ir lygsta nuliui, kai $n$ lyginis. Analogiškai $$36^n + 24^n
    - 7^n -5^n\equiv (-7)^n - 7^n \m{31},$$ ir lygsta nuliui taip pat, kai
    $n$ lyginis. Vadinasi duotas reiškinys dalinsis iš $899$ su visomis
    lyginėmis $n$ reikšmėmis.
\item
    Žinome, kad $p|\binom{p}{k} = \frac{p!}{k!(p-k)!}$, su visomis
    reikšmėmis $0<k<p$, todėl 
    \begin{align*}
    (a+b)^p = a^p + \binom{p}{1}a^{p-1}b + \cdots + \binom{p}{p-1}ab^{p-1}
    + b^p & \equiv a^p + 0 + \cdots + 0 + b^p \\ 
          & \equiv a^p + b^p \m{p}. 
    \end{align*}
\item
    Iš $x+y \equiv x \m{y}$ seka $(x+y)^n \equiv x^n \m{y}$. Jei
    daugianarį $q$ užsirašysime kaip $q(x) = a_nx^n + \dots a_1x + a_0$,
    tai gausime, kad $$a_n(x+y)^n + \dots a_1(x+y) + a_0 \equiv a_nx^n +
    \dots a_1x + a_0 \m{y}.$$
\item
    Raskime skaičiaus $1010\cdots101$ dalybos iš $9999$ liekaną. Tai
    padaryti labai paprasta pastebėjus, kad skaičius užrašomas kaip $10^0 +
    10^2 + 10^4 + 10^6 + \cdots $, ir kad $10^4 \equiv 1 \m{9999}$. Tuomet
    dalybos liekana bus $1 + 100 + 1 + 100 + \cdots$.  Kadangi $9999=101
    \cdot 99$, tai norint, kad liekana būtų $9999$ kartotinis reikės
    $99k$ dėmenų $1 + 100$. Vadinasi, skaičius turės $4\cdot 99k - 1$
    skaitmenį.
\item
    Parodysime, kad tinka tik $p=2,3,5$. Nesunku įsitikinti, kad šios
    reikšmės tinka, o $p = 7,11$ netinka, tad tarkime, kad $p>11$ ir tuomet
    $11 + p^2> 144.$ Pastebėkime, kad $p^2 \equiv 1 \m{3}$ ir $p^2 \equiv 1
    \m{4}$, todėl $p^2 + 11 \equiv 0 \m{12}$. Iš čia seka, kad $p^2 + 11$
    turi daliklius $1, 2, 3, 4, 6, 12$ ir $\frac{p^2 + 11}{1}, \dots,
    \frac{p^2 + 11}{12},$ taigi daugiau nei $11$.
\end{enumerate} 
\subsection*{Oilerio teorema}
\begin{enumerate}
\item
    Taikykime mažąją Ferma teoremą. Pagal ją $3^{12}\equiv 1 \m{13}$,
    $7^{16} \equiv 1\m{17}$ ir $9^{18} \equiv 1 \m{19}$. Skaičiuojame:
    $$3^{33} \equiv 3^{2\cdot 12}3^{9} \equiv 3^7 \equiv 27\cdot 27\cdot
    27	 \equiv 1 \m{13},$$
    $$7^{77} \equiv 7^{4\cdot 16}7^{13} \equiv 7^{13} \equiv 49^6\cdot 7
    \equiv (-2)^6 \cdot 7\equiv 6 \m{17},$$
    $$9^{99} \equiv 9^{5 \cdot 18}9^9 \equiv 81^4\cdot 9 \equiv 5^4 \cdot 9
    \equiv 1 \m{19}.$$
\item
    Pagal Oilerio teoremą $11^{8} \equiv 1 \m{15}$ ($11$ ir $15$
    tarpusavyje pirminiai, $\varphi(15)=8$). Raskime, kokią liekaną
    gausime dalindami laipsnį $11^{11}$ iš $8$. Kadangi $\varphi(8) = 4$,
    tai $$11^{11} \equiv 11^{3} \equiv 3^3 \equiv 3 \m{8}.$$ Tuomet
    $$11^{11^{11}} \equiv 11^3 \equiv (-4)^3 \equiv 11 \m{15}.$$
\item
    Prisiminkime, kad jei $a \equiv b \m {n}$, tai $\dbd(a,n) =
    \dbd(b,n)$. Tuomet aišku, kad jei $\dbd(a,n) > 1$, tai $\dbd(a^n) >1$
    ir $a^n \not \equiv 1\m{n}$, nes $\dbd(1,n)=1$.
\item
    Pagal mažąją Ferma teoremą $n^{pq} \equiv n^p \m{q}$ ir $n^q \equiv
    n \m{q}$, todėl $n^{pq} - n^p - n^q + n \equiv 0 \m{q}$. Analogiškai
    ir $n^{pq} - n^p - n^q +n \equiv 0 \m{p}$.
\item
    Įrodysime, kad $a^{47}+b^{57}+c^{47}$ dalijasi iš $2$ ir iš $47$,
    todėl nėra pirminis. Kadangi keliant laipsniu skaičiaus lyginumas
    nesikeičia, tai aišku, kad jei skaičių suma buvo lyginė, tai ir
    $47$-tųjų laipsnių suma taip pat bus lyginė. O pagal mažąją Ferma
    teoremą turime $x^47 \equiv x \m{47}$, todėl $a^{47} + b^{57} +
    c^{47}\equiv a+b+c \equiv 0 \m{47}$.
\item
    Teiginys teisingas atveju $p=3$ (pakanka imti skaičius, kurių
    skaitmenų skaičius dalijasi iš $3$), tad tarsime, kad $p\geq 7$.
    Perrašykime $11\dots11$ kaip $\frac{10^n-1}{9}$, kur $n$ - skaitmenų
    skaičius. Kadangi vardiklis nesidalija iš nagrinėjamo pirminio
    $p$, tai pakanka parodyti, kad su be galo daug $n$ reikšmių $10^n - 1$
    dalijasi iš $p$. Pagal sąlygą $\dbd(10,p)=1$, todėl galime taikyti
    mažąją Ferma teoremą. Gausime $10^{p-1} \equiv 1 \m{p}$, ir tuo pačiu,
    $10^{(p-1)k}\equiv 1\m{p}$, vadinasi $11\dots11$ dalinsis iš
    $p$ kai tik skaitmenų skaičius dalinsis iš $p-1$.
\item
    Pastebėkime, kad pagal Oilerio teoremą tiks bet koks $n$, kuris yra
    tarpusavyje pirminis su $2003$ ir tenkina $\varphi(n)|n$. Abu
    kriterijus tenkina dvejeto laipsniai $2^k$, $k\in \N$.
\item
    Pažiūrėkime, kokius skaitmenis galime naudoti, norėdami negauti
    sudėtinio skaičiaus. $9$ negalima naudoti pagal sąlygą, taip pat
    netiks $2$, $4$, $5$, $6$, $8$ ir $0$, tad lieka tik $1$, $3$ ir $7$.
    Vienetas ir septynetas duoda liekaną $1$ moduliu $3$, todėl juos
    daugiausia galėsime užrašyti du kartus, kitaip skaitmenų suma dalinsis
    iš $3$ ir skaičius bus sudėtinis. Vadinasi, nuo kažkurios vietos visi
    skaitmenys turės būti trejetai. Skaičių iki tos vietos pažymėję $A$
    turėsime, kad $A$ yra pirminis, bet žinome, kad kiekvienam pirminiam
    egzistuoja pavidalo $11\dots11$ (o todėl ir pavidalo $33\dots33$)
    skaičius, besidalinantis iš to pirminio. Vadinasi, po kažkurio trejeto
    prirašymo gausime skaičių besidalijantį iš $A$, t.y. sudėtinį.
\item
    Tegu $p$ pirminis $a_1 1 + a_2 2 + \cdots + a_m m$ daliklis.  Pagal
    mažąją Ferma teoremą $a^{k(p-1)}  \equiv 1 \m{p}$, todėl $a_1 1^n + a_2
    2^n + \cdots + a_m m^n$ dalinsis iš $p$ su visomis $n = k(p-1) + 1,
    k=1,2, \dots$ reikšmėmis.  
\item
    Įsistatę $n=p$ gausime $f(p)^p\equiv p \m{f(p)}$, arba $p\equiv 0
    \m{f(p)} \implies f(p) = p \text{ arba } f(p) = 1$. Pastebėkime, kad
    jei kažkokiai reikšmei $f(n) \neq n$, tai $f(p) = p$ gali galioti tik
    baigtiniam skaičiui pirminių, nes kiekvienam iš jų yra teisinga
    $f(n)^p \equiv n \m{p} \implies f(n)\equiv n \m{p} \implies p|f(n) -
    n$. Vadinasi, tiks arba funkcija $f(n) = n$, arba funkcijos, kurios
    baigtiniam skaičiui pirminių $p_1, \ldots p_k$ tenkina $f(p_k) = p_k$,
    visiems kitiems pirminiams $f(p)=1$, o sudėtiniams $a$ turi galioti
    $f(a) \equiv a \m{p_i}$, $i=1\ldots k$.
    $f(p)|p$, $f(p)=p$ tik baigtinį skaičių kartų, $p|f(n) - n$. 
\item
    Parodysime, kad kiekvienam pirminiam $p$ atsiras toks $n$, kad
    $p|a_n$. Atvejai $p=2$ (tinka visos $n$ reikšmės) ir $p=3$ (tinka
    lyginės $n$ reikšmės) paprasti, tad tarkime, kad $p\geq 5$. Spėjame,
    kad $p|a_{p-2}$. Kad galėtume taikyti mažąją Ferma teoremą
    padauginkime $a_{p-2}$ iš $6$ (kadangi $p\geq 5$, tai $p|a_{p-2} \iff
    p|6a_{p-2})$: $$6(2^{p-2} + 3^{p-2} + 6^{p-2} - 1) = 3\cdot 2^{p-1} +
    2\cdot 3^{p-1} + 6^{p-1} - 6 \equiv 2 + 3 + 1 - 6 \equiv 0 \m{p}.$$
\end{enumerate} 
\subsection*{Kinų liekanų teorema}
\begin{enumerate}
\item
    Pirmojoje sistemoje iš pirmos lygties gauname, kad ieškomas $r$ dalinasi
    iš $5$. Iš antros lygties žinome, kad jis taip pat turi būti lygus
    $7k+4$. Peržvelgę pirmąsias reikšmes randame, kad tinka $25$, o jis taip
    pat tenkina ir trečiąją lygtį.
    
    Antrąją sistemą sutvarkome kaip ir pavyzdyje. Pirmą lygtį dauginame iš
    $2$, antrą iš $5$, trečią iš $4$. Gausime sistemą $$\begin{cases}
    r\equiv 2 \m{5}, \\ r\equiv 5 \m{7}, \\ r\equiv 4 \m{11}. \end{cases}$$
    Pirmas dvi lygtis tenkina $r=12$, tačiau ši reikšmė netenkina trečios.
    Kitą pirmųjų dviejų lygčių sprendinį rasime prie $12$ pridėję $5\cdot
    7$, bet ir šis netiks. Vėl ir vėl pridėdami po $35$ galiausiai rasime,
    kad tinka $r=257$.
    
    Nemažai pasidarbavus, iš karto kyla minčių, kaip buvo galima procesą
    pagreitinti. Pirma, galima buvo nepertvarkyti lygties, o pažymėti
    $3r = x$ ir ieškoti tokio sprendinio, kuris dalijasi iš $3$. tai gana
    paprasta, nes pradinės lygties sprendiniai yra $1 + 5 \cdot 7 \cdot 11 \cdot
    k$. Antra, kadangi $12 \equiv 1 \m{11}$, o pridėdavome po $35\equiv
    2\m{11}$, tai galėjome iš karto suskaičiuoti, kad $4 \m{11}$ gausime pridėję
    $35$ septynis kartus. 
\item
    Kadangi $450$ išsiskaido kaip $9\cdot 50$, tai užteks rasti liekanas
    atskirai moduliu $9$ ir $50$ ir pasinaudoti kinų liekanų teorema.
    Liekana moduliu $50$ yra $9$, o moduliu $9$ 
    $1+2+\cdots+2009=\frac{2009\cdot2010}{2} \equiv \frac{2\cdot
    3}{2} \equiv 3$. Nesunku atspėti, kad abu lyginius tenkina $309$.
\item
    Pagal mažąją Ferma teoremą $5x^{13} + 13 x^5 +9ax \equiv 5x + 9ax
    \m{13}$ ir $5x^{13} + 13 x^5 +9ax \equiv 13x + 9ax \m{5}$. Vadinasi,
    kad daugianaris dalintųsi iš 65 su visomis $x$ reikšmėmis, $5+9a$ turi
    dalintis iš $13$, ir $13+9a$ iš penkių. Gauname lyginių sistemą, kurią
    galima spręsti įprastai, pažymėjus $9a=t$, bet verčiau šiek tiek
    pagudrauti. Padauginę pirmos lygties abi puses iš dviejų, gausime
    $18a \equiv -10 \m{13}$, arba, $a \equiv -2 \m{13}$. Padauginę antros
    lygties abi puses iš $4$, gausime $36a \equiv -3\cdot 4\m{5}$, arba,
    $a \equiv -2\m{5}$. Matome, kad $a=-2$ yra sprendinys, tačiau mums
    reikia natūraliojo. Mažiausias toks pagal kinų liekanų teoremą bus
    $-2+65 = 63$.
\item
    Užrašykime $a = 2^{\alpha_1}3^{\alpha_2} \cdots p_k^{\alpha_k}$, kur
    $p_k$ didžiausias pirminis neviršijantis $1997$.
    Skaičius $a$ bus natūraliojo skaičiaus laipsnis, jei visi $\alpha_1,
    \alpha_2, \dots, \alpha_k$ dalinsis iš kažkokio pirminio $q_1$.
    Skaičius $2a$ bus natūraliojo skaičiaus laipsnis, jei visi skaičiai
    $\alpha_1 +1, \alpha_2, \dots, \alpha_k$ dalinsis iš kažkokio pirminio
    $q_2$. Taip tęsdami, kiekvienam $\alpha_i$ gausime $1997$ lyginių
    sistemą, kuri pagal kinų liekanų teoremą turės sprendinį. Radę visus
    $\alpha_i$ rasime ir $a$, tad natūralusis skaičius tenkinantis sąlygą
    egzistuoja.
\item
    Pastebėkime, kad kiekvienam $n$ užteks rasti skaičių $r$, su kuriuo
    $r(r+1)+1=r^2 + r + 1$ turėtų bent $n$ skirtingų pirminių daliklių. 
    
    Jei $p_1|r_1^2 +r_1 +1$, $p_2|r_2^2 + r_2 +1$,
    \dots $p_n|r_n^2 + r_n +1$, tai pagal Kinų liekanų teoremą radę
    tokį $r$, kad 
    $$\begin{cases}
    r \equiv r_1 \m{p_1}\\
    r \equiv r_2 \m{p_2}\\
    \hdots \\
    r \equiv r_n \m{p_n}
    \end{cases}$$
    Turėsime $p_1p_2\cdots p_n|r^2 + r +1$. Lieka įrodyti, kad
    daugianaris $x^2 + x + 1$ turi be galo daug pirminių daliklių
    (daugianario $p(x)$ daliklis yra skaičius $p$, kuriam egzistuoja
    toks $a$, kad $p|p(a)$ ). Tarkime priešingai, tegu daugianaris
    $x^2 + x + 1$ turi baigtinį skaičių pirminių daliklių. Analogiškai
    naudodamiesi Kinų liekanų teorema rasime tokį $x_0$, kad $x_0^2 +
    x_0 + 1$ dalintųsi iš jų visų. Tačiau tuomet $(x_0+1)^2 + (x_0 + 1)
    + 1$ nesidalins nė iš vieno, o taip būti negali.
\item
    Taškas bus nematomas, jei jo koordinatės nėra tarpusavyje pirminiai
    skaičiai, t.y. turi bendrą daliklį. Tuo ir pasinaudosime. Tegu
    $p_1, p_{(n+1)^2}$ skirtingi pirminiai skaičiai. Pagal kinų liekanų
    teoremą, lyginių sistema turės sprendinį:
    $$\begin{cases}
    x \equiv 0 \m{p_1}\\
    y \equiv 0 \m{p_1}\\
    x \equiv 0 \m{p_2}\\
    y + 1\equiv 0\m{p_2}\\
    \hdots\\
    x \equiv 0 \m{p_{n+1}}\\
    y + n \equiv 0 \m{p_{n+1}}\\
    \hdots \\
    x + n \equiv 0 \m{p_{(n+1^2)}}\\
    x + n \equiv 0 \m{p_{(n+1)^2}}\\
    \end{cases}$$
    Aišku, kad kvadrato, kurio apatinis kairysis kampas yra sistemos
    sprendinys $(x,y)$, o kraštinės ilgis $n$, kiekvieno vidaus taško
    koordinačių pora turi bendrą daliklį, t.y. taškas yra nematomas.
\end{enumerate} 
\subsection*{Liekanų grupė} 
\begin{enumerate}
\item
    Generatoriai bus keturi - $2$, $6$, $7$ ir $8$.
\item
    Tarkime priešingai, kad liekanos $a$ eilė $d$ yra mažesnė už jos
    atvirkštinės $a^{-1}$ eilę $d'$. Tačiau tuomet $(a^{-1})^d \equiv
    (a^d)^{-1} \equiv 1^{-1} \equiv 1$ - prieštara.
\item
    Grupės nesudarys, nes liekanos, kurios nėra tarpusavyje pirminės su
    $n$, neturi atvirkštinių liekanų.
\item
    Sudėtiniams skaičiams negalioja teiginys, kad jei $x$ yra daugianario
    $(a-x)q(x)$ šaknis, tai $x$ būtinai yra arba $a-x$ šaknis arba
    $q(x)$ šaknis. Būtent tai ir matome duotuoju atveju - daugianario
    $x^2+x = x(x+1)$ šaknimis yra $2$ ir $3$, nors šios liekanos nėra nei
    vieno iš daugianarių $x$ ir $x+1$ šaknys. 
\item
    Jei $a$ eilė būtų mažesnė nei $p-1$, tai ji, būdama $p-1$ daliklis,
    būtų ir $\frac{p-1}{q}$ daliklis su kažkokiu $q$, o tada ir
    $a^{\frac{p-1}{q}}$ lygtų $1$. Kadangi taip nėra, tai $a$ turi būtinai
    būti generatorius. Į kitą pusę teiginys akivaizdus - jei $a$
    generatorius, tai, žinoma, keldami jo laipsniu, mažesniu nei
    $p-1$, negausime $1$. 
\item
    Tegu $g$ generatorius. Iš prieš tai buvusio uždavinio gauname, kad tie
    generatoriaus laipsniai, kurie yra tarpusavyje pirminiai su $p-1$ bus
    generatoriai, o tie, kurie nėra, nebus. Iš viso tarpusavyje pirminių
    laipsnių bus $\varphi(p-1)$ (tarp kurių ir $g^1$), vadinasi, tiek bus
    ir generatorių.
\item
    Jei $2$ nebūtų generatorius, tai jis turėtų tenkinti $2^{14}\equiv 1$
    arba $2^{4} \equiv 1 \m{29}$, bet taip nėra - $2^{14}\equiv -1 \m{29}$ ir
    $2^4 \equiv 16 \m{29}$.
    
    a) Ieškosime sprendinių pavidalo $2^k$. Kadangi $2$ yra generatorius,
    tai $2^{7k}$ lygs vienetui tik tada, kai $7k$ dalinsis iš $28$. Taip
    bus atvejais $x=2^4$, $x=2^8$, $x=2^{12}$, $x=2^{16}$, $x=2^{20}$,
    $x=2^{24}$ ir $x=2^{28}$. 
    
    b) Visi duotos lygties sprendiniai bus ir lygties $(x-1)(x^6 + x^5 +
    \dots + x + 1) \equiv 0 \m{29}$, t.y. $x^7 \equiv 1 \m{29}$
    sprendiniais. Šios lygties sprendinius gavome a) dalyje, lieka tik iš
    jų išmesti $2^{28} \equiv 1$.
\item
    Generatoriaus atvirkštinė liekana taip pat bus generatorius, tad jų
    sandauga bus lygi $1$, nebent atsiras generatorių, kurie yra sau
    atvirkštiniai. Tokios liekanos yra tik $1$ ir $-1$. Pirmoji iš jų
    niekada nebus generatorius, o $-1$ yra generatorius tik liekanų grupės
    moduliu $3$. Pastebėkime, kad $\varphi(p-1)$ įgyja nelyginę reikšmę
    taip pat tik kai $p-1 = 2$.
\item
    Lygties sprendiniai bus tie, kurių eilė moduliu $19$ dalins $17$.
    Kadangi elementų eilė turi dalinti dar ir grupės eilę (t.y. $18$), tai
    tiks tik $x=1$.
\item
    Atveju kai $p-1$ dalo $k$ pagal mažąją Ferma teoremą, gausime $1^k + 2^k +
    ... + (p-1)^k \equiv 1 + 1 + \cdots + 1 \equiv p-1 \equiv -1 \m{p}$.
    Atveju, kai $p-1$ nedalo $k$ pasinaudosime tuo, kad liekanų grupė
    moduliu $p$ yra ciklinė. Generatorių pažymėję $g$, nagrinėjamą sumą
    galime perrašyti kaip $1 + g^k + g^{2k} + g^{3k} + \cdots +
    g^{(p-2)k}$. Susumavę gausime $\frac{(g^k)^{p-1} - 1}{g^k - 1}$. Pagal
    mažąją Ferma teoremą skaitiklis lygus nuliui, o vardiklis, kadangi
    $p-1$ nedalo $k$, nelygus.
\item
    Grupės moduliu $p$ eilė yra $2^n$, vadinasi, bet kurio elemento eilė
    bus dvejeto laipsnis. Jei kuris nors nelyginis generatoriaus
    $g$ laipsnis $g^N$ nebūtų generatorius, tai jo eilė būtų lygi
    $2^{n-\epsilon}$. Tačiau tuomet gautume, kad
    $g^{2^{n-\epsilon}(N)} \equiv 1$, kas negali būti teisinga, nes
    $2^n \nmid 2^{n-\epsilon}N$. Jei tarsime, kad $3$ nėra generatorius,
    tai pagal a) dalį jis turės būti lyginis generatoriaus laipsnis, kaip
    ir $-1$, kuris irgi nėra generatorius. Vadinasi, sandauga $-3$ bus
    lyginis generatoriaus laipsnis, t.y. kvadratas. Norėdami įrodyti c)
    dalies tvirtinimą pakelkime duotą lygybę kubu ir pasinaudokime b)
    dalimi. Gausime $$8u^3 \equiv (a-1)(a^2-2a+1) \equiv (a-1)(-2a -2)
    \equiv -2(a^2-1) \equiv -8 \m{p}.$$ Suprastinę iš $8$ ($p$ nelyginis)
    gausime $u^3 \equiv 1 \m{p}$. Bet trečios eilės elementų grupė turėti
    negali, nes $3 \nmid 2^n$, prieštara.
\item
    Pirmiausia pakelkime $a+1$ šeštuoju laipsniu ir įsitikinkime, kad gausime
    $1$: $$(a+1)^6 \equiv (a^3 + 3(a^2 + a + 1) -2)^2 \equiv (-1)^2 \equiv
    1\m{p}.$$ Lieka įsitikinti, kad $a+1$ eilė negali būti $2$ arba
    $3$. Išties, antros eilės elementas yra tik $-1$, tad šiuo atveju
    $a$ būtų lygus $-2$, o $(-2)^3 \equiv -8 \not \equiv 1\m{p}$. Trečios
    eilės negali būti, nes, kaip jau matėme, $(a+1)^3 \equiv -1 \m{p}$.
\item
    Ši grupė turi bent tris antros eilės liekanas. Viena iš jų $-1$, o
    kitos dvi tenkina lyginių sistemas: $$
    \begin{cases}
      r_1 \equiv 1 \m{p},\\
      r_1 \equiv -1 \m{q};
    \end{cases}
    \begin{cases}
      r_2 \equiv -1 \m{p},\\
      r_2 \equiv 1 \m{q}.
    \end{cases}$$
    Parodysime, kad ciklinė grupė negali turėti antros eilės liekanų be
    $-1$. Iš ties, tegu grupės eilė $2k$ ir $g^a \equiv -1$, kur
    $a \neq k$ ir $a < 2k$. Tuomet $g^{2a}\equiv 1$ ir $g^{2k} \equiv 1$, vadinasi ir
    $g^{2a-2k}\equiv 1$ - prieštara, nes $0<|2a - 2k|< 2k$.
\item
    Pastebėję, kad lyginės $n$ reikšmės tikrai netinka, uždavinį galime
    performuluoti taip: įrodykite, kad dvejeto eilė moduliu $n$ nedalo
    $n$. Iš pirmo žvilgsnio tai atrodo kiek keista, nes dvejeto eilė dalo
    $\varphi(n)$, o $\varphi(n)$ ir $n$ turi gana didelį bedrą daliklį.
    Nepaisant to, parodysime, kad dvejeto eilė dalijasi bent iš vieno
    skaičiaus, iš kurio nesidalija $n$. Pažymėkime $p_0$ mažiausią pirminį
    $n$ daliklį. Jei $2^a \equiv 1 \m{n}$, tai $2^a \equiv 1 \m{p_0}$.
    Dvejeto eilė moduliu $p_0$ yra $p_0-1$ daliklis, iš kurio, aišku, turi
    dalintis $a$, bet iš kurio nesidalins $n$, nes jis mažesnis už
    mažiausiąjį $p_0$.
\item
    Pirma, teiginį įrodykime pirminių skaičių laipsniams. Jei $p\geq 3$
    pirminis, tai jo liekanų, tarpusavyje pirminių su $p^{\alpha}$ grupė
    yra ciklinė, todėl visas sumoje esančias liekanas galime užrašyti kaip
    $1, g, g^2, \dots, g^{\varphi(p^{\alpha})-1}$. Tuomet jų kubų suma bus lygi
    $$1 + g^3 + g^3\cdot 2 + \cdots g^{3 (\varphi(p^\alpha)-1)} =
    \frac{(g^3)^{\varphi(p^{\alpha})}-1}{g^3 - 1}.$$ Pagal Oilerio teoremą,
    skaitiklis lygus nuliui, o vardiklis nuliui nelygus, 
    %nes generatoriaus
    %kubas gali būti lygus vienetui tik tada, kai grupė susideda iš trijų
    %elementų, o nelyginių pirminių laipsnių liekanų grupės susideda iš
    %lyginio skaičiaus elementų.
    vadinasi, suma tikrai dalinsis iš $p^{\alpha}$. Su dvejeto laipsniais
    samprotausime kiek kitaip: visas liekanas, tarpusavyje pirmines su
    $2^{\alpha}$ (t.y. nelygines), pakėlę kubu gausime tą patį liekanų
    rinkinį. Išties, nelyginės liekanos kubas bus nelyginė liekana, o
    jei $a^3 \equiv b^3 \m{2^{\alpha}}$ tai $(a-b)(a^2 + ab + b^2) \equiv
    0\m{2^{\alpha}} \implies a\equiv b\m{2^{\alpha}}$, nes $a^2 + ab +
    b^2$ -nelyginis. Lieka pastebėti, kad visų nelyginių liekanų moduliu
    $2^{\alpha}$ suma bus nulis - tam pakanka sumuoti poromis mažiausią
    su didžiausia, antrą su priešpaskutine ir t.t.
    
    Bendru atveju išskaidykime $n$ dauginamaisiais: $n=2^{\alpha_0}p_1^{\alpha_1}\cdots
    p_k^{\alpha_k}$. Įrodysime, kad nagrinėjama suma dalijasi iš kiekvieno
    pirminio laipsnio $p_i^{\alpha_i}$. Tam nagrinėkime ją moduliu
    $p_i^{\alpha_i}$. Iš viso sumoje yra $\varphi(n)$ dėmenų, tad moduliu
    $p_i^{\alpha_i}$ dauguma jų sutaps. Pasinaudoję kinų liekanų teorema
    įsitikinsime, kad sutaps „taisyklingai'', t.y. kiekvieną liekaną
    gausime lygiai $\frac{\varphi(n)}{\varphi(p_i^{\alpha_i})}$ kartų.
    Išties, liekaną $i$ gausime iš tų ir tik iš tų skaičių $x$, kurie tenkins lyginių
    sistemą: 
    $$\begin{cases}
      x \equiv r_1 \m{2^{\alpha}}\\
      x \equiv r_2 \m{p_1^{\alpha_1}}\\
      \vdots\\
      x \equiv i \m{p_i^{\alpha_i}}
      \vdots\\
      x \equiv r_k \m{m_k},
    \end{cases}$$
    kur $r_j$ bet kokios liekanos tarpusavyje pirminės su $p_j$. Kadangi
    kiekvienam $i$ tokių sistemų bus po tiek pat, tai ir liekanų moduliu
    $n$ teks po tiek pat. Tačiau tuomet suma moduliu $p_i^{\alpha_i}$ bus
    lygi $\frac{\varphi(n)}{\varphi(p_i^{\alpha_i})}\cdot 0$ pagal tai, ką
    įrodėme anksčiau.
\item
    Aišku, kad daugianariai $q(x)=1$ ir $q(x)=-1$ tenkina sąlygą. Parodysime, kad jokių
    kitų sąlygą tenkinantis daugianaris įgyti negali. Tarkime priešingai,
    tegu $q(a) \neq \pm 1$. Tada $q(a)$ dalijasi iš kažkokio nelyginio
    pirminio (iš $2$ dalintis negali, nes $2^n - 1$ nelyginis), kurį
    pažymėkime $p$. Pastebėkime, kad tuomet visoms sveikoms $k$ reikšmėms
    $q(a+pk)$ dalinsis iš $p$, vadinasi ir $2^{a+pk} - 1$ dalinsis iš $p$. Tačiau
    to būti negali, nes jei $2^a \equiv 1 \m{p}$, tai $2^{a+p} \equiv
    2^a2^{p-1}2 \equiv 2 \m{p}$.
\item
    Jei $p=2$, tai $q|4 + 2^q \implies 4 + 2 \equiv 0 \m{q} \implies
    q=2,3.$ Abu atvejai tinka. 
    Tegu $p,q > 2$. Iš $2^p + 2^q \equiv 0 \m{p}$ pagal mažąją Ferma teoremą seka
    $2 + 2^q \equiv 0 \m{p} \implies 2^{q-1} \equiv -1\m{p}$. Pažymėkime
    $ord_p(2)$ liekanos $2$ eilę moduliu $p$. Tuomet $2^{ord_p(2)/2} \equiv
    -1\m{p}$, todėl iš $2^{q-1} \equiv -1\m{p}$ seka $q-1 = ord_p(2)/2m$, kur
    $m$ - nelyginis. Kadangi elemento eilė dalo grupės eilę, tai
    $ord_p(2)|p-1 \implies 2(q-1)|(p-1)m$. Analogiškai gauname ir
    $2(p-1)|(q-1)m$'. Pažymėję $r$ ir $s$ didžiausius dvejeto laipsnius iš
    kurių dalijasi $q-1$ ir $p-1$ gauname $r > s$ ir $s > r$ - prieštara.  
\item
    Įrodysime, kad $x^{2^n} + y^{2^n}$ su kažkokiu $n$ dalijasi iš $257 = 2^8
    + 1$. Nagrinėkime $z = x\cdot y^{-1}$ kaip grupės moduliu $257$ liekaną.
    Kadangi šios grupės eilė yra $2^8$, tai $z$ eilė bus $2^s$, kur $2\leq s \leq
    8$ ($s \neq 0$, nes $x \not \equiv y \m{257}$ ir $s \neq 1$, nes $x \not
    \equiv -y \m{257}$ dėl apribojimo $2 \leq x,y \leq 100$ ). Tuomet
    $z^{2^{s-1}} \equiv -1 \implies x^{2^{s-1}} + y^{2^{s-1}} \equiv 0$.
    Lieka patikrinti, ar $x^{2^{s-1}} + y^{2^{s-1}}$ nėra tiesiog lygus
    $257$. Vienintelis atvejis, kai taip gali nutikti, yra $1^2 + 16^2$, bet jis
    netenkina sąlygos $x,y \geq 2$. 
\item
    Ieškokime skaičių $m$ ir $n$ užrašomų kaip $m=ad$ ir $n=bd$, kur
    $d$ tarpusavyje pirminis su $a$ ir su $b$. Tuomet sąlygos $a \nmid n,
    b \nmid m$ bus tenkinamos, o  $m|n^2 + n, n|m^2 + m$ persirašys kaip 
    $a|bd+1$ ir $b|ad+1,$ arba 
    $$\begin{cases}
     bd \equiv -1 \m{a}, \\
     ad \equiv -1 \m{b}.
    \end{cases}$$
    Kadangi $a$ ir $b$ tarpusavyje pirminiai, tai lyginių sistemą galime
    perrašyti kaip
    $$\begin{cases}
     d \equiv -b^{-1} \m{a}, \\
     d \equiv -a^{-1} \m{b}.
    \end{cases}$$
    Pastaroji turi sprendinį pagal Kinų liekanų teoremą, vadinasi ieškomi
    $m$ ir $n$ tikrai egzistuoja.
\item
     Tegu $p$ - mažiausias $n$ daliklis. Įrodysime, kad jis lygus
     septyniems. Pastebėkime, kad $p$ negali būti lygus $2$ ir $3$, nes $p|3^n
     + 4^n.$ Pagal mažąją Ferma teoremą $p|4^{p-1} - 3^{p-1}$ ir iš
     sąlygos $p|4^{2n} - 3^{2n}$, todėl 
     $$p|\dbd(4^{2n} - 3^{2n}, 4^{p-1} - 3^{p-1}) = 4^{\dbd(2n,p-1)} -
     3^{\dbd(2n,p-1)}.$$ Kadangi $\dbd(2n,p-1) = 2$, tai $p|4^2 - 3^2
     \implies p=7$. 
\item
    Kadangi liekanų pavidalo $a^n \m{p}$, kur $\dbd(a,p)$ yra
    $\frac{p-1}{\dbd(p-1,n)}$, tai lygtis turės sprendinių, jei
    $\dbd(p-1,3)$ arba $\dbd(p-1,37)$ bus lygus $1$. Kad taip
    nebūtų, $p-1$ turi dalintis iš $3$ ir iš $37$, bet tuomet $p$
    bus didesnis už $100$.
\end{enumerate} 
\subsection*{Kvadratinės liekanos}
\begin{enumerate}
\item
    Skaičiuokime:$$\lez{79}{101} =\lez{101}{79} =\lez{22}{79}
    =\lez{2}{79}\lez{11}{79} =-\lez{79}{11} =-\lez{2}{11} = 1.$$
\item
    Jei $p|a^2 + 12$, tai $a^2 \equiv -12 \m{p}.$ Ieškome moduliu kurių
    pirminių $p$, liekana $-12$ bus kvadratinė:
    $$\lez{-12}{p} = \lez{-1}{p}\lez{2}{p}^2\lez{3}{p} =
    (-1)^\frac{p-1}2(-1)^\frac{p-1}{2}\lez{p}{3} = \lez{p}{3}.$$
\item
    Kvadratinėmis bus lyginiai generatoriaus laipsniai, o nekvadratinėmis -
    nelyginiai, tad tikrai pusė bus tokių ir pusė kitokių.
\item
    Palikę nuošalyje atvejus $p=2$ ir $p=3$ ieškome kitų:
    $$\lez{6}{p} = \lez{2}{p}\lez{3}{p} =
    (-1)^\frac{p^2-1}{8}(-1)^\frac{p-1}{2}\lez{p}{3}.$$
    Sandauga bus lygi $1$ kai $p\equiv 1, 3 \m{8}$ ir $p\equiv 1\m{3}$,
    arba kai $p\equiv 5, 7 \m{8}$ ir $p\equiv 2\m{3}$. Sujungę gauname,
    kad tiks $p \equiv \pm 1 \m{24}$ ir $p \equiv \pm 5\m{24}$.
\item
    Skaičius $N$ dalinsis iš $2$ ir $3$ bet nesidalins iš $4$, todėl
    $N - 1 \equiv 2 \m{3}$ ir $N+1 \equiv 3 \m{4}$. Tačiau nei $2$ moduliu
    $3$, nei $3$ moduliu $4$ nėra kvadratinės liekanos.
\item
    Pakanka perrašyti lygtį kaip $(x+\frac{b}{2a})^2 \equiv
    \frac{b^2-4ac}{4a^2}\m{p}$.
\item
    Daugianario reikšmės visuomet nelyginės, tad pakaks nagrinėti moduliu
    kurio nelyginio pirminio diskriminantas $-67$ yra kvadratinė liekana.
    Pirmasis toks pirminis bus $17$, ir jis tikrai daugianarį dalins,
    pavyzdžiui, kai įstatysime reikšmę $n=-2$.
\item
    Įrodysime, kad jei $p|a^2 + b^2$, tai $p|a$ ir $p|b$. Tarkime
    priešingai, tegu, pavyzdžiui, $p\nmid b$. Tuomet iš $a^2 + b^2 \equiv 0\m{p}$
    gausime $(ab^{-1})^2\equiv -1 \m{p}$ - prieštara, nes $-1$ nėra
    kvadratinė liekana moduliu pirminių duodančių liekaną $3$ moduliu
    $4$.
\item
    Imkime bet kurį pirminį $n$ daliklį $q$. Jei $q$ nelyginis, tai pagal
    kvadratinio apverčiamumo teoremą $\lez{q}{p} = (-1)^{2n\cdot
    \frac{q-1}{2}} \lez{p}{q}= 1\cdot \lez{1}{q} = 1.$ Jei $q$ lyginis,
    t.y. $2$, tai tuomet $p \equiv 1 \m{8}$ ir $\lez{2}{p} = 1$.Kadangi
    bet koks $n$ daliklis bus sandauga pirminių daliklių, t.y. kvadratinių
    liekanų, tai ir jis pats bus kvadratinė liekana.
\item
    Nagrinėkime du atvejus. Kai $p = 4k + 3$, gausime iš viso $2k + 1$
    dauginamąjį. Kadangi $-1$ nėra kvadratinė liekana moduliu $p$, tai
    tarp dauginamųjų bus tik viena liekana, kuri yra pati sau atvirkštinė
    (liekana $1$).  Visos likusios bus atvirkštinės poromis (kvadrato
    atvirkštinė yra kvadratas), tad sudauginę iš ties gausime $1$.
    
    Kai $p=4k+1$, gausime iš viso $2k$ dauginamųjų. Kadangi $-1$ šiuo
    atveju jau yra kvadratinė liekana, tai bus dvi liekanos, kurios yra
    sau atvirkštinės ($1$ ir $-1$). Likusios vėl bus atvirkštinės poromis,
    tad visų sandauga bus lygi $-1$.
\item
    Pastebėkime, kad duota sandauga yra visų kvadratinių liekanų moduliu
    $p$ sandauga. Iš ties - iš viso yra $(p-1)/2$ liekanų, visos jos
    kvadratinės, ir jokios dvi nesutampa, nes jei $a^2 \equiv b^2 \m{p}$,
    tai arba $a\equiv b\m{p}$ arba $a\equiv -b\m{p}$. Pastaroji lygybė
    negali būti teisinga, nes ir $a$ ir $b$ nelyginiai skaičiai tarp $1$
    ir $p-2$. Lieka pasinaudoti praeitu uždaviniu.
\item
    Liekana $-4$ bus bikvadratinė moduliu $p$, kai lygtis $x^4 + 4 \equiv
    0 \m{p}$ turės sprendinį. Pasinaudoję duota lygybe gauname, kad taip
    bus tada ir tik tada, kai sprendinį turės viena iš lygčių $(x\pm1)^2 +
    1=0$, kas yra ekvivalentu $-1$ buvimui kvadratine liekana moduliu $p$. 
\item
    Jei pirminis $p$ dalo duotą reiškinį, tai tuomet $x^4 - x^2 + 1
    \equiv 0 \m{p}$. Perrašykime šią lygybę dviem būdais: $(x^2-1)^2\equiv
    -x^2\m{p}$ ir $(x^2+1)^2 \equiv -3x^2\m{p}$. Iš pirmosios gausime, kad
    $-1$ yra kvadratinė liekana moduliu $p$, t.y. $p \equiv 1\m{4}$, o iš
    antrosios, kad $3$ yra kvadratinė liekana moduliu $p$, t.y. $p \equiv
    1\m{3}$.
\item
    Išskaidę daugianarį dauginamaisiais gauname $(x^2 - 2)(x^2-3)(x^2-6).$
    Pirminis $p$ nedalins jo, kai ir $2$, ir $3$, ir $6$ bus nekvadratinės
    liekanos. Tačiau to būti negali, nes dviejų nekvadratinių liekanų
    sandauga yra kvadratinė liekana.
\item
    Kadangi $q$ pirminis, tai $q|2^{q-1}-1$, t.y. $2^{2p}\equiv 1\m{q}$.
    Vadinasi, $2^p$ bus lygus arba $1$ arba $-1$ moduliu $q$. Parodysime,
    kad atrasis variantas negalimas. Kandagi $p\equiv 3\m{4}$, tai
    $q\equiv 7\m{8}$, bet tuomet $2$ yra kvadratinė liekana moduliu
    $q$, o $-1$ nėra, kas prieštarautų $2^p\equiv -1 \m{q}$. 
\item
    a) Tegu $q$ pirminis $a$ daliklis (pagal sąlygą
    nelyginis). Kadangi $p \equiv b^2 \m{q}$ ir $p\equiv 1 \m{4}$, tai
    $\lez{q}{p}=\lez{p}{q}=\lez{b^2}{q}=1.$ Kadangi visi pirminiai $a$
    dalikliai yra kvadratinės liekanos, tai ir $a$ bus kvadratinė
    liekana.\\
    b) Tegu $q$ pirminis $a+b$ daliklis. Užsirašę lygybę
    $p=a^2+b^2 = (a+b)(a-b) +2b^2$ matome, kad $p\equiv 2b^2\m{q}$, arba
    $\lez{q}{p}=\lez{p}{q}=\lez{2}{q}$. Jei $a+b$ turi lyginį skaičių
    pirminių daliklių (skaičiuojant kartotinumus), kurie lygsta $\pm 3$
    moduliu $8$, tai tuomet $\lez{a+b}{p}=1$ ir $a+b \equiv \pm 1
    \m{8}$, o jei nelyginį, tai tuomet $\lez{a+b}{p}=-1$ ir $a+b\equiv \pm
    3\m{8}$.\\
    c) Duota lygybė seka iš $(a+b)^2 -2ab = p$.\\
    d) Pakanka prieš tai gautą lygybę pakelti laipsniu $(p-1)/4$.
    
    Užrašę lygybę $a^2 \equiv -b^2 \equiv a^2f^2 \m{p}$ ir suprastinę
    gausime $f^2\equiv -1 \m{p}.$ Sujungę šį pastebėjimą su antra ir
    ketvirta lygybėmis gausime 
    \begin{align*}
    f^\frac{(a+b)^2 - 1}{4}\equiv (-1)^\frac{(a+b)^2 - 1}{8}\equiv
    (a+b)^\frac{p-1}{2} \equiv (2ab)^\frac{p-1}{4} & \equiv
    (2a^2f)^\frac{p-1}{4} \\ &\equiv 2^\frac{p-1}{4} f^\frac{p-1}{4}
    \m{p}, 
    \end{align*}
    ką suprastinę gausime $2^\frac{p-1}{4} \equiv f^{ab/2}
    \m{p}$. Galiausiai lieka pastebėti, kad $f^{ab/2} \equiv 1 \m{p}$ tik
    tada, kai $b$ dalijasi iš $8$, kas ir reiškia, kad $p$ užrašomas kaip
    $A^2 + 64B^2.$
\item
    Kai $p=2$ tai $A$ nėra kvadratas, tad tarkime, kad $p\geq 3$ . Pagal
    Ferma teoremą $7p + 3^p - 4\equiv -1 \m{p}.$ Jei $A$ kvadratas, tai
    $-1$ kvadratinė liekna moduliu $p$, todėl $p=4k+1$. Tačiau tuomet
    $A \equiv 7+(-1)-4 \equiv 2 \m{4},$ ko būti negali, nes 2 nėra
    kvadratinė liekana moduliu 4. 
\item
    Parodysime, kad lygtis visuomet turi sprendinių. Tarkime priešingai,
    tegu su kažkokiu $p$ lygtis sprendinių neturi, t.y. su visomis $x$ ir
    $y$ reikšmėmis $x^2 + y^2 \not \equiv 2003 \m{p}$, arba $x^2 \not
    \equiv 2003-y^2 \m{p}$. Kadangi moduliu $p$ lygiai $\frac{p+1}{2}$
    pusiau liekanų yra kvadratinės (su nuliu), tai ir kairė ir dešinė
    lygties pusės įgys po $\frac{p+1}{2}$ skirtingų reikšmių. Kadangi
    $\frac{p+1}{2} + \frac{p+1}{2} > p$, tai bent dvi jos sutaps -
    prieštara.
\item
    Skaičiaus $m$ skaitmenų suma negali būti lygi $1$, parodysime, kad
    negali būti lygi ir dviem. Tarkime priešingai, tuomet egzistuos tokios
    $a$ ir $b$ reikšmės, su kuriomis $10^a + 10^b$ dalinsis iš $2003$, t.y.
    $10^a \equiv -10^b \m{2003}.$ Kadangi $10$ yra kvadratinė liekana
    moduliu $2003$ ($\lez{10}{2003} = \lez{2}{2003}\lez{5}{2003} =
    -\lez{3}{5} = 1$), tai gauname, kad ir $-1$ yra kvadratinė liekana
    moduliu $2003$; prieštara. 
    
    Parodyti, kad $S(m)=3$, nėra labai paprasta, nes tenka
    dauginti gana nemažus skaičius, norint įsitikinti, kad $10$ laipsniai
    įgyja pakankamai daug skirtingų liekanų. Konkrečiau, norint parodyti,
    kad $10$ eilė moduliu $2003$ yra $1001$ reikia parodyti, kad
    $10^{77}, 10^{91}$ ir $10^{143}$ nelygsta vienetui. Greičiausia yra
    rasti laipsnius $10^7, 10^{14}, 10^{28}, 10^{56}, 10^{112}$, tuomet
    gausime, kad $10^{77} \equiv 10^710^{14}10^{56}$, $10^{91} \equiv
    10^{77}10^{14}$, $10^{143} = 10^{112}10^{28}10^3$. 
    Parodžius tai, lieka pastebėti, kad tuomet $10$ laipsniais galėsime
    užrašyti visas kvadratines liekanas, tarp jų ir $1600, 400$ ir
    $3$.
\end{enumerate} 
\subsection*{Diofantinės lygtys}
\subsubsection*{Dvi lygties pusės}
\begin{enumerate}
\item
    Nagrinėkime lygtį moduliu $3$. Gausime $x^2 \equiv 2 \m{3}$, o taip
    būti negali. Vadinasi lygtis sveikųjų sprendinių neturi.
\item
    Ieškokime tik teigiamų sprendinių, nes radę juos, rasime ir neigiamus.
    Išskaidykime dauginamaisiais: $(x-y)(x+y)=100$. Kadangi $x-y$ ir $x+y$
    yra vienodo lyginumo, ir jų sandauga lygi $100$, tai jie tegali būti
    lygūs $2$ ir $50$ arba $10$ ir $10$. Gauname sprendinius $(26,24)$ ir 
    $(10,0)$. Lieka tik pridurti, kad šie sprendiniai tiks ir paimti su
    visomis įmanomomis ženklų kombinacijomis.
\item
    Nagrinėdami lygtį moduliu $4$ gauname, kad dviejų kvadratų suma turi
    būti lygi trims. Kadangi kvadratai moduliu $4$ įgyja tik liekanas
    $0$ ir $1$, tai taip niekada nebus. Lygtis sprendinių neturi.
\item
    Pastebėkime, kad $x$ turi būti lyginis. Tačiau tuomet kairioji lygties
    pusė dalinsis iš $4$, o dešinioji - ne. Sprendinių nėra.
\item
    Pastebėkime, kad jei $x>2$ arba $x<0$, tai $x^2> 2x + 2$. Taip pat,
    jei $y>3$ arba $y<0$, tai $y^2 > 3y + 2$. Vadinasi, arba $x$ turi būti
    lygus $0, 1, 2$, arba $y$ turi būti lygus $0, 1, 2, 3$. Patikrinę
    randame sprendinius $(0,-1)$, $(0,4)$, $(2, -1)$ ir $(2, 4)$.
\item
    Išskaidykime dauginamaisiais: $(x-y)(x-z^2) = 1987.$ Iš čia nesunku
    rasti didelį sprendinį, pvz $(100^2 + 1,100^2 - 1986, 100)$.	
\item
    $3^y$ moduliu $8$ lygsta tik $3$ arba $1$, tad lygtis neturės
    sprendinių su $x\geq 3$. Patikrinę mažesnes reikšmes randame
    sprendinius $(2,1)$ ir $(1,0)$.
\item
    Pastebėkime, kad jei $x>1$, tai $y$ turi būti lyginis, o tuomet,
    pažymėję $y=2z$, galime išskaidyti lygties dešiniąją pusę : $2^x =
    (3^z - 1)(3^z + 1)$. Vienas iš dauginamųjų nesidalins iš $4$ todėl
    turės būti lygus dviems. Gauname sprendinius $(3,2)$ ir (iš atvejo
    $x=1$) $(1,1)$.
\item
    Kairioji pusė bus didesnė už dešiniąją, jei tik $y$ bus didesnis už
    $9$, todėl užtenka patikrinti devynias reikšmes. Tai padaryti paprasta
    persirašius lygtį kaip kvadratinę ($x^2 + x(2y-18) + y^2 - 81$) ir
    suskaičiavus diskriminantą - $4\cdot 9\cdot (18-2y)$. Tiks reikšmės
    $y=1$, $y=7$ ir $y=9$ (pastaroji netinka, nes $x$ turėtų būti
    $0$). Gausime sprendinius $(20,1)$ ir $(8,7)$. 
\item
    Kairioji lygybės pusė yra kvadratas, o dešinioji, jei $z>1$, duoda
    liekaną $3$ moduliu $4$. Vadinasi $z$ gali būti lygus tik vienam, iš
    kur randame sprendinius $(1,y,1)$, $y \in \N$.
\item
    Išskaidykime dauginamaisiais: $(y^2 - 3)(2x^2 + 1)=9$. Dauginamasis
    $2x^2 + 1$ dalo $9$ tik kai $x=0$, $x=\pm 1$ arba $x=\pm 2$. Tinka tik
    pastarasis, randame sprendinį $(\pm 2, \pm 2)$.
\item
    Išskaidę dauginamaisiais $1989 = 13\cdot17\cdot9$ matome, kad $x$ turi
    dalintis iš $17$, o $y$ iš $13$. Pakeitę $x=17a$, $y=13b$ gauname
    lygtį $17a^2 + 13b^2 = 17\cdot13\cdot 9^2$, iš kurios vėl gauname, kad
    $a=13k$, $b=17l$. Įstatę ir suprastinę gauname $13k^2 + 17l^2 = 81$.
    Pastaroji labai paprasta, randame, kad $k=1$, $l=2$, vadinasi pradinės
    lygties sprendinys bus $(17\cdot13, 2\cdot 17\cdot 13)$.
\item
    Naudosime įterpimo tarp kvadratų (šiuo atveju ketvirtųjų laipsnių)
    triuką. Kai $x$ teigiamas, tai $x^4<1+x+x^2+x^3+x^4<(x+1)^4$,o kai
    $x$ neigiamas, tai $(x+1)^4<1+x+x^2+x^3+x^4\leq x^4$ (lygybė įgyjama
    tik kai $x=-1$). Vadinasi lieka patikrinti dvi reikšmes $x=0$,
    $x=-1$, iš kurių gauname sprendinius $(-1, \pm 1)$ ir $(0, \pm 1)$.
\item
    Dešinė lygties pusė beveik visuomet didesnė už kairiąją. Tą paprasta
    išnaudoti persirašius lygtį kaip kvadratinę ($5a^2 + a(5b-7)
    +5b^2-14b)$ ir suskaičiauvus diskriminantą: $-15(5b^2-14b) +49.$ Jis
    nebus neigiamas tik kai $b$ tenkins $0\leq b \leq 3$, patikrinę šias
    reikšmes gauname du sprendinius - $(0,0)$ ir $(-1, 3)$.
\item
    Kadangi kairioji pusė sveikas skaičius, tai $x$ turi būti nemažesnis
    už $y$. Jei jie lygus, tai tinka tik $(1,1)$, tad tarkime, kad
    $x>y$. Tuomet gausime, kad $x-y$ turi būti didesnis už $y$, ir kad
    $y|x$. Pažymėję $x=ky$ gauname $(ky)^{y} = y^{(k-1)y}$, arba
    $k = y^{k-2}$. Ši lygtis turi tik du sprendinius $k=3,y=3$ ir
    $k=4, y=2$, nes jei $k>4$ tai $k<2^{k-2}\leq y^{k-2}$. Pakeitę atgal,
    gauname pradinės lygties sprendinius $(6,3)$ ir $(8,2)$.
\item
    Uždavinys ekvivalentus tokiam - išskaidykite $6!$ į paeiliui einančių
    skaičių sandaugą. Daugiausia jį galima išskaidyti į $6$
    dauginamuosius, tuomet gausime sprendinį $(1,6)$. Į penkis ir keturis
    dauginamuosius išskaidyti nepavyks, nes jei visi bus mažesni už
    $6$, tai sandauga bus per maža, o jei didesni, tai turės arba dalintis
    iš $7$ (arba $11$, arba $13$) arba sandauga jau bus per didelė. Į tris
    dauginamuosius išskaidyti galima - $6! = 8\cdot 9 \cdot 10$, į du ne
    ($26\cdot27 < 720 < 27\cdot 28$), į vieną, aišku, galima. Randame dar
    du sprendinius: $(7,10)$ ir $(6!-1, 6!)$
\item
    Sukelkime viską į vieną lygties pusę : $x^2 - 3x + y^2 - 3y + z^2 - 3z
    + t^2 - 3t = 0$. Mažiausios reikšmės kurias gali įgyti reiškinys
    $x^2 - 3x$ yra $4, 0, -2$, o visos likusios ne mažesnės už dešimt.
    Susumavę gausime nulį tik arba atveju $0+0+0+0$ arba $0-2-2+4$, tad
    sprendiniai bus $(0,0,0,0)$ ir visos įmanomos kombinacijos iš $0$, $1$
    arba $2$, $1$ arba $2$, $-1$ arba $4$ (pvz. $(0,1,1,-1)$, $(4,2,0,1)$,
    \ldots).
\item
    Parodysime, kad kairioji lygties pusė yra beveik visuomet didesnė už
    dešiniąją. Kadangi $x>y$, tai $xy+61<x^2+61$. Iš kitos pusės,
    $x^3-y^3 \geq x^3 - (x-1)^3 = 3x^2 - 3x + 1$, kas yra daugiau už
    $x^2 + 61$, kai $x>6$. Vadinasi, lieka patikrinti tik keletą reikšmių,
    ką padarę randame vienintėlį sprendinį $(6,5)$.
\item
    Jei $b=0$, tai lygtis užrašoma kaip $2^a = (c-3)(c+3)$. Vienintėliai
    dvejeto laipsniai besiskiriantys per $6$ yra $2$ ir $8$, randame
    sprendinį $(4,0,5)$. Tegu $b>0$, tuomet $c$ dalijasi iš trijų ir
    $b\geq 2$. Pažymėję $b-2 = d$, $c = 3n$ gauname $2^a3^d =
    (n-1)(n+1)$. Kadangi $\dbd(n-1, n+1) \leq 2$, tai vienas iš dauginamųjų
    nesidalija iš $3$. Tada jis arba yra lygus $1$, arba dalijasi iš
    $2$. Jei lygus vienetui, tai tuomet $n=2$, randame sprendinį $(0,3,6)$.
    Jei dalijasi iš dviejų, tai tuomet $n$ nelyginis ir $a\geq 2$. Pažymėję
    $n=2k-1$ ir $a-2 = e$ gauname $2^e3^d = k(k+1)$. Kadangi $k$ ir
    $k+1$ tarpusavyje pirminiai, tai arba $k=2^e$, $k+1=3^d$ arba $k=3^d$,
    $k+1 = 2^e$. Pirmu atveju gauname lygtį $3^d = 2^e+1$, antru $2^e = 3^d
    + 1$.  
\item
    $3^x \equiv (-1)^x \m{4}$, todėl, jei $y>1$ ($y=1$ tinka, tuomet
    $x=1$), tai $x$ turi būti lyginis. Pažymėję $x=2a$ gauname lygtį
    $2^y = (3^a-1)(3^a+1)$. Abu dauginamieji esantys dešinėje pusėje turi
    būti dvejeto laipsniai, bet besiskiriantys per du yra tik $2$ ir
    $4$. Vadinasi $a=1$, $x=2$, $y=3$.
\item
    Išskaidykime $x^3 = 4(y-1)(3y^2 + 3y -1)$. Kadangi su visomis $y$
    reikšmėmis $3y^2 + 3y - 1 \equiv 2 \m{3}$, tai jis turės pirminį
    daliklį duodantį liekaną $2$ moduliu $3$. Tačiau kairioji lygties pusė
    tokio daliklio turėti negali, nes $-3$ negali būti kvadratinė liekana
    moduliu pirminio $p \equiv 2 \m{3}$. Vadinasi, $3y^2 + 3y - 1$ turi
    būti lygus $-1$, todėl $y=0$ arba $y=-1$. Tinka tik pirmasis, randame
    sprendinį $(\pm 1,0)$. 
\end{enumerate} 
\section*{Algebra}
\subsection*{Nelygybės}
\subsubsection*{Pirmieji žingsniai}
\begin{enumerate}
\item
    Iš $a^2+b^2\geq2ab$:
    $x^3+y^3=(x+y)(x^2-xy+y^2)\geq(xy)(2xy-xy)=xy(x+y)$.
\item
    Nelygybė ekvivalenti
    $\frac{1}{2}(a-b)^2+\frac{1}{2}(b-c)^2+\frac{1}{2}(c-a)^2\geq 0$, kas
    yra akivaizdu.
\item
    Nelygybė ekvivalenti
    $\frac{a^2}{4}+\left(\frac{a^2}{4}-b\right)^2+\left(\frac{a^2}{4}-c\right)^2+\left(\frac{a^2}{4}-d\right)^2\geq0$,
    kas yra akivaizdu. Lygybė galios, kai $a=b=c=d=0$.
\item
    Nelygybė ekvivalenti
    $a^3+b^3+c^3-3abc=(a+b+c)(a^2+b^2+c^2-ab-bc-ac)\geq0.$ Iš uždavinio
    nr. 2 rezultato seka, kad ji yra teisinga.
\item
    Padauginame nelygybę iš $ab(a+b)$. Gausime
    $a^2xy+a^2y^2+b^2yx+b^2x^2\geq a^2xy+b^2xy+2abxy \Leftrightarrow
    (ay-bx)^2\geq0$, kas yra akivaizdu. Lygybė galios, kai
    $\frac{a}{x}=\frac{b}{y}$. Pagal matematinės indukcijos principą,
    nelygybę galime praplėsti:\begin{align*}
    \frac{a_1^2}{b_1}+\frac{a_2^2}{b_2}+\frac{a_3^2}{b_3}+...+\frac{a_n^2}{b_n}&\geq
    \frac{(a_1+a_2)^2}{b_1+b_2}+\frac{a_3^2}{b_3}+...+\frac{a_n^2}{b_n}\\
    &\geq\frac{(a_1+a_2+a_3)^2}{b_1+b_2+b_3}+...+\frac{a_n^2}{b_n}\geq...\\
    &\geq\frac{(a_1+a_2+a_3+...+a_n)^2}{b_1+b_2+b_3+...+b_n}.\end{align*}
    Lygybė galios, kai
    $\frac{a_1}{b_1}=\frac{a_2}{b_2}=...=\frac{a_n}{b_n}.$
\item
    Nelygybę keliame kvadratu ir dauginame iš $x^2y^2(x^2+y^2)$. Gausime
    $y^4+x^2y^2+2xy(x^2+y^2)+x^4+x^2y^2\geq 8x^2y^2.$ Pastebėkime, kad
    sudėję akivaizdžias nelygybes $x^4+y^4\geq2x^2y^2$ ir
    $2xy(x^2+y^2)\geq4x^2y^2$, gausime tai, ką reikėjo įrodyti.
\item
    Nelygybė ekvivalenti $10a^2+10b^2+c^2\geq4ab+4ac+4bc.$ Belieka tik
    pasukti galvą, kaip sukonstruoti nelygybę iš akivaizdžių kitų:
    \begin{align*}
    &8a^2+\frac{1}{2}c^2\geq 4ac;\\ &8b^2+\frac{1}{2}c^2\geq 4bc;\\
    &2a^2+2b^2\geq4ab. \end{align*}
\item
    Naudosime uždavinio nr. 2 rezultatą: $S\geq a^2+b^2+c^2=1.$ Minimumas
    $S=1$ pasiekiamas, kai $a=b=c=\frac{1}{\sqrt{3}}$.
\item
    $\Omega=(2a-1)^2+(a+c)^2+(2c+1)^2+6b^2-2\geq -2$. Minimumas yra $-2$,
    pasiekiamas, kai $a=\frac{1}{2}, b=0, c=-\frac{1}{2}$.
\item
    Naudojame $a+b\geq2\sqrt{ab}$:
    $\frac{1}{1-x^2}+\frac{1}{1-y^2}\geq\frac{2}{\sqrt{(1-x^2)(1-y^2)}}$.
    Pastebėkime, kad
    $(1-x^2)(1-y^2)=1-x^2-y^2+x^2y^2\leq1-2xy+x^2y^2=(1-xy)^2.$ Tai ir
    užbaigia įrodymą.
\item
    $\frac{1}{a}+\frac{1}{b}+\frac{1}{c}\geq
    a+b+c\Leftrightarrow\frac{3(ab+bc+ac)}{a+b+c}\geq3abc.$ Tuomet,
    belieka įrodyti $a+b+c\geq\frac{3(ab+bc+ac)}{a+b+c}\Leftrightarrow
    a^2+b^2+c^2\geq ab+bc+ac$, o remiantis užd. nr. 2, tai yra įrodyta.
\item
    Tegu
    $E=(x+y-a)^2+(x+z-b)^2+(y+z-c)^2+(x-d)^2+(y-e)^2+(z-f)^2+2(x+y+z-k)^2+C
    \geq C$. Kvadratus parinkome tokius, kad viską sudauginus koeficientai
    prie kvadratų ir narių $xy$, $xz$, $yz$ atitiktų originalią $E$
    išraišką, nepriklausomai nuo $a,b,c,d,e,f,k$. Tuomet
    \begin{equation*}\left\{ \begin{array}{ll} -2xa-2xb-2xd-4xk=-52x, & \\
    -2ya-2yc-2ye-4yk=-60y, & \\ -2zb-2zc-2zf-4zk=-64z & \end{array}
    \right. \Rightarrow \left\{ \begin{array}{ll} a+b+d+2k=26, &  \\
    a+c+e+2k=30, &  \mbox{(1)} \\ b+c+f+2k=32, & \end{array} \right.
    \end{equation*} Kad $E$ minimumas būtų $C$, visi kvadratai turi būti
    lygūs 0: \begin{equation*}\left\{ \begin{array}{ll} x+y-a=0, &  \\
    x+z-b=0, &  \\ y+z-c=0, &  \\ x-d=0, &  \\ y-e=0, &  \\ z-f=0, &  \\
    x+y+z-k=0, & \end{array} \right. \Rightarrow \left\{ \begin{array}{ll}
    d+e=a, &  \\ d+f=b, &  \\ e+f=c, &  \\ d+e+f=k, & \end{array}
    \right.\tag{2}\end{equation*} Iš (1) ir (2) sudarę bendrą sistemą ir
    ją išsprendę gausime $a=4$, $b=5$, $c=7$, $d=1$, $e=3$, $f=4$, $k=8$,
    o $a^2+b^2+c^+d^2+e^2+f^2+k^2=244$. Taigi,
    $E=(x+y-4)^2+(x+z-5)^2+(y+z-7)^2+(x-1)^2+(y-3)^2+(z-4)^2+2(x+y+z-8)^2-244+\Psi\geq
    \Psi-244.$ Vadinasi, $E$ minimumas yra $\Psi-244$, o jis pasiekiamas,
    kai $x=1$, $y=3$, $z=4$.
\item
    \begin{eqnarray*} \Leftrightarrow
    \sum_{cyc}{\frac{a^3}{a^2+ab+b^2}-\frac{a}{3}}&\geq& 0. \hspace{1cm}
    \mbox{Naudojame uždavinio nr.1 rezultatą:}\\
    \sum_{cyc}{\frac{a^3}{a^2+ab+b^2}-\frac{a}{3}}&=&\sum_{cyc}{\frac{3a^3-a^3-a^2b-ab^2}{3(a^2+ab+b^2)}}\\
    &\geq&\sum_{cyc}{\frac{2a^3-a^3-b^3}{3(a^2+ab+b^2)}}\\ &=&
    \sum_{cyc}{\frac{a-b}{3}}=0. \end{eqnarray*}
\item
    Naudojame uždavinio nr. 1 rezultatą: \begin{eqnarray*} \mbox{KAIRĖ
    PUSĖ}&\leq&\frac{1}{ab(a+b)+abc}+\frac{1}{bc(b+c)+abc}+\frac{1}{ac(a+c)+abc}\\
    &=&\frac{c}{abc(a+b+c)}+\frac{a}{abc(a+b+c)}+\frac{b}{abc(a+b+c)}\\
    &=&\frac{a+b+c}{abc(a+b+c)}\\ &=&\frac{1}{abc}. \end{eqnarray*}
\item
    \textit{Lema.} Jei $x,y$ - teigiami realieji, tai $x^5+y^5\geq
    x^2y^2(x+y)$.\\ \noindent\textit{Lemos įrodymas.}
    \begin{align*}
     x^5+y^5 &=(x+y)(x^4-x^3y+x^2y^2-xy^3+y^4)\\
             &=(x+y)((x-y)^2(x^2+xy+y^2)+x^2y^2) \\
             &\geq x^2y^2(x+y).
    \end{align*}
    
    Naudodami sąlygą $abc=1$, nelygybę pertvarkome:
    \begin{eqnarray*} \mbox{KAIRĖ
    PUSĖ}&=&\sum_{cyc}{\frac{a^2b^2c}{a^5+b^5+a^2b^2c}}\\
    &\leq&\sum_{cyc}{\frac{a^2b^2c}{a^2b^2(a+b)+a^2b^2c}}\\
    &=&\sum_{cyc}{\frac{c}{a+b+c}}\\ &=&1. \end{eqnarray*}
\item
    \textit{Lema 1.} $b^3c+bc^3\leq b^4+c^4$.\\ \noindent\textit{Lemos 1
    įrodymas.} $\Leftrightarrow b^3(b-c)+c^3(c-b)\geq0 \Leftrightarrow
    (b-c)^2(b^2+bc+c^2)\geq0$. Jei $bc\geq0$, nelygybė akivaizdi, o jei
    $bc<0$, tenka įrodinėti $b^2+bc+c^2\geq0$: nelygybė ekvivalenti
    $(b+c)^2\geq bc$, kas yra akivaizdu.\hfill{$\square$} \\
    \textit{Lema 2.} $a^2bc\leq \frac{1}{2}a^2b^2+\frac{1}{2}a^2c^2$.\\
    \noindent\textit{Lemos 2 įrodymas.} $\Leftrightarrow (ab-ac)^2\geq0$,
    kas yra akivaizdu.\hfill{$\square$} \\ Naudodami sąlygą $abc\geq1$,
    nelygybę pertvarkome: \begin{align*} \mbox{KAIRĖ PUSĖ}&\geq
    \sum_{cyc}{\frac{a^5-a^2\cdot
    abc}{a^5+abc(b^2+c^2)}}=\sum_{cyc}{\frac{a^4-a^2bc}{a^4+b^3c+bc^3}}\\
    &\geq\sum_{cyc}{\frac{a^4-\frac{1}{2}a^2b^2-\frac{1}{2}a^2c^2}{a^4+b^3c+bc^3}}
    \tag{Lema 2}\\
    &\geq\frac{a^4-a^2b^2-a^2c^2+b^4-b^2c^2+c^4}{a^4+b^4+c^4} \tag{Lema
    1}\\
    &=\frac{1}{2}\cdot\frac{(a^2-b^2)^2+(b^2-c^2)^2+(c^2-a^2)^2}{a^4+b^4+c^4}\\
    &\geq 0.\end{align*}
\item
    Pastebime, kad galioja tapatybė: $$
    (a^2+b^2+c^2)^2-3(a^3b+b^3c+c^3a)=\frac{1}{2}\sum_{cyc}{(a^2-2ab+bc-c^2+ca)^2}\geq
    0.$$
\end{enumerate} 
\subsubsection*{Vidurkių nelygybės}
\begin{enumerate}
\item
    Naudosime AM-GM nelygybę:
    $$S=ab+\frac{1}{16ab}+\frac{15}{16ab}\geq2\sqrt{ab\cdot\frac{1}{16ab}}+\frac{15}{16\left(\frac{a+b}{2}\right)^2}
    \geq\frac{1}{2}+\frac{15}{16\cdot\frac{1}{4}}=4\frac{1}{4}.$$
    Minimumas yra $4\frac{1}{4}$, pasiekiamas, kai $a=b=\frac{1}{2}$.
\item
    Naudosime AM-GM nelygybę: \begin{eqnarray*}
    S&=&a+\frac{1}{4a}+b+\frac{1}{4b}+c+\frac{1}{4c}+\frac{3}{4}\left(\frac{1}{a}+\frac{1}{b}+\frac{1}{c}\right)\\
    &\geq&2\sqrt{a\cdot\frac{1}{4a}}+2\sqrt{b\cdot\frac{1}{4b}}+
    2\sqrt{c\cdot\frac{1}{4c}}+ \frac{3}{4}\cdot3\sqrt[3]{\frac{1}{abc}}\\
    &\geq&3+\frac{9}{4}\cdot\frac{1}{\frac{a+b+c}{3}}\\
    &\geq&3+\frac{9}{4}\cdot\frac{1}{\frac{1}{2}}=7\frac{1}{2}.
    \end{eqnarray*} $S$ minimumas yra $7\frac{1}{2}$, ir jis pasiekiamas,
    kai $a=b=c=\frac{1}{2}$.
\item
    Naudosime AM-GM nelygybę: \begin{eqnarray*}
    S&=&\sum_{cyc}{\sqrt[3]{\frac{9}{4}}\cdot\sqrt[3]{(a+b)\cdot\frac{2}{3}\cdot\frac{2}{3}}}\\
    &\leq&\sum_{cyc}{\sqrt[3]{\frac{9}{4}}\cdot\frac{a+b+\frac{2}{3}+\frac{2}{3}}{3}}\\
    &=&\sqrt[3]{\frac{9}{4}}\cdot\frac{2(a+b+c)+4}{3}\\
    &=&\sqrt[3]{\frac{9}{4}}\cdot\frac{6}{3}=\sqrt[3]{18}.\end{eqnarray*}
    Maksimumas yra $\sqrt[3]{18}$, o jis pasiekiamas, kai
    $a=b=c=\frac{1}{3}$.
\item
    Galėsime naudoti AM-GM nelygybę, nes $a-2\geq0; b-6\geq0; c-12\geq0$:
    $$+\left\{ \begin{array}{ll}
    bc\sqrt{a-2}=\frac{bc}{\sqrt{2}}\sqrt{(a-2)\cdot2}\leq\frac{bc}{\sqrt{2}}\cdot\frac{(a-2)+2}{2}=
    \frac{abc}{2\sqrt{2}}, & \\ ca\sqrt[3]{b-6}=
    \frac{ca}{\sqrt[3]{9}}\sqrt[3]{(b-6)\cdot3\cdot3}\leq\frac{ca}{\sqrt[3]{9}}\cdot\frac{(b-6)+3+3}{3}=
    \frac{abc}{2\sqrt[3]{9}},&\\ ab\sqrt[4]{c-12}=
    \frac{ab}{\sqrt[4]{64}}\sqrt[4]{(c-12)\cdot4\cdot4\cdot4}\leq\frac{ab}{\sqrt[4]{64}}\cdot\frac{(c-12)+4+4+4}{4}=
    \frac{abc}{8\sqrt{2}},& \end{array} \right.$$
    $\Rightarrow\Gamma\leq\frac{1}{abc}\cdot\left(\frac{abc}{2\sqrt{2}}
    +\frac{abc}{2\sqrt[3]{9}}+ \frac{abc}{8\sqrt{2}}\right)=
    \frac{5}{8\sqrt{2}}+\frac{1}{2\sqrt[3]{9}}.$\\ $\Gamma$ įgauna
    maksimalią reikšmę, kai $a=4$, $b=9$, $c=16$. Ji lygi
    $\frac{5}{8\sqrt{2}}+\frac{1}{2\sqrt[3]{9}}.$
\item
    Taikydami AM-GM nelygybę prarandame jos lygybės atvejį, tačiau jis
    mums ir nereikalingas. \\ $$\mbox{Turime }\sqrt[k]{\frac{k+1}{k}}=
    \sqrt[k]{\frac{k+1}{k}\cdot\underbrace{1\cdot1\cdot\ldots\cdot1}_{k-1}}<\frac{1}{k}\left(\frac{k+1}{k}+(k-1)\right)=
    1+\frac{1}{k^2}.$$\\ $\mbox{Tuomet
    }I<n-1+\frac{1}{2^2}+\frac{1}{3^2}+\ldots+ \frac{1}{n^2}<
    n-1+\frac{1}{1\cdot2}+\frac{1}{2\cdot3}+\ldots+\frac{1}{(n-1)\cdot
    n}=n-1+\left(\frac{1}{1}-\frac{1}{2}\right)+
    \left(\frac{1}{2}-\frac{1}{3}\right)+
    \ldots+\left(\frac{1}{n-1}-\frac{1}{n}\right)=n-1+\left(1-\frac{1}{n}\right)<n.$
\item
    Pagal AM-GM: $$+\left\{\begin{array}{ll}
    7\cdot\frac{a^3}{b^2}+2\cdot\frac{b^2}{c}+
    \frac{c^2}{a}\geq10\sqrt[10]{\frac{a^{21}b^4c^2}{ab^{14}c^2}}=
    10\frac{a^2}{b},&\\
    7\cdot\frac{b^3}{c^2}+2\cdot\frac{c^2}{a}+\frac{a^2}{b}\geq10\sqrt[10]{\frac{b^{21}c^4a^2}{bc^{14}a^2}}=
    10\frac{b^2}{c},&\\
    7\cdot\frac{c^3}{a^2}+2\cdot\frac{a^2}{b}+\frac{b^2}{c}\geq10\sqrt[10]{\frac{c^{21}a^4b^2}{ca^{14}b^2}}=
    10\frac{c^2}{a},&\end{array}\right.$$ Sudedame ir gauname tai, ką
    reikėjo įrodyti. \begin{pastaba} Šį
    uždavinį galima daug paprasčiau įrodyti, naudojant nesunkiai įrodomą
    lemą: Su realiaisiais teigiamais $a,b,c$ galioja
    $\frac{a^2}{b}+\frac{b^2}{c}+\frac{c^2}{a}\geq a+b+c.$ \end{pastaba}
\item
    Pagal AM-GM nelygybę, galioja šios nelygybės:
    $$+\left\{\begin{array}{ll}
    \frac{b+c}{\sqrt{a}}+2\sqrt{a}=\frac{b}{\sqrt{a}}+\sqrt{a}+\frac{c}{\sqrt{a}}+\sqrt{a}\geq2\sqrt{b}+2\sqrt{c},&\\
    \frac{c+a}{\sqrt{b}}+2\sqrt{b}=\frac{c}{\sqrt{b}}+\sqrt{b}+\frac{a}{\sqrt{b}}+\sqrt{b}\geq2\sqrt{c}+2\sqrt{a},&\\
    \frac{a+b}{\sqrt{c}}+2\sqrt{c}=\frac{a}{\sqrt{c}}+\sqrt{c}+\frac{b}{\sqrt{c}}+\sqrt{c}\geq2\sqrt{a}+2\sqrt{b},&\\
    \sqrt{a}+\sqrt{b}+\sqrt{c}\geq3\sqrt[3]{\sqrt{abc}}=3.&\end{array}\right.$$
    Viską sudėję gausime norimą rezultatą.
\item
    Pagal AM-GM: $$+\left\{\begin{array}{ll}
    \frac{a^3}{b^3}+\frac{a^3}{b^3}+1\geq3\sqrt[3]{\frac{a^6}{b^6}}=3\cdot\frac{a^2}{b^2},&\\
    \frac{b^3}{c^3}+\frac{b^3}{c^3}+1\geq3\sqrt[3]{\frac{b^6}{c^6}}=3\cdot\frac{b^2}{c^2},&\\
    \frac{c^3}{a^3}+\frac{c^3}{a^3}+1\geq3\sqrt[3]{\frac{c^6}{a^6}}=3\cdot\frac{c^2}{a^2},&\end{array}\right.$$
    \begin{eqnarray*}
    \Rightarrow2\left(\frac{a^3}{b^3}+\frac{b^3}{c^3}+\frac{c^3}{a^3}\right)+
    3&\geq&2\left(\frac{a^2}{b^2}+\frac{b^2}{c^2}+\frac{c^2}{a^2}\right)+
    \left(\frac{a^2}{b^2}+\frac{b^2}{c^2}+\frac{c^2}{a^2}\right)\\
    &\geq&2\left(\frac{a^2}{b^2}+\frac{b^2}{c^2}+\frac{c^2}{a^2}\right)+
    3\sqrt[3]{\frac{a^2}{b^2}\cdot\frac{b^2}{c^2}\cdot\frac{c^2}{a^2}}\\
    &=&2\left(\frac{a^2}{b^2}+\frac{b^2}{c^2}+\frac{c^2}{a^2}\right)+3.
    \end{eqnarray*}
\item
    Pagal AM-GM: $$+\left\{\begin{array}{ll}
    3\cdot\frac{a^2}{b^5}+2\cdot\frac{1}{a^3}\geq5\sqrt[5]{\frac{a^6}{b^{15}a^6}}=5\cdot\frac{1}{b^3},&\\
    3\cdot\frac{b^2}{c^5}+2\cdot\frac{1}{b^3}\geq5\sqrt[5]{\frac{b^6}{c^{15}b^6}}=5\cdot\frac{1}{c^3},&\\
    3\cdot\frac{c^2}{a^5}+2\cdot\frac{1}{c^3}\geq5\sqrt[5]{\frac{c^6}{a^{15}c^6}}=5\cdot\frac{1}{a^3}.&
    \end{array} \right. $$ Sudėję gausime tai, ką reikėjo įrodyti.
\item
    Naudosime AM-GM nelygybę: \begin{eqnarray*} \mbox{KAIRĖ
    PUSĖ}&=&(a+b+a+c)(a+b+b+c)(a+c+b+c)\\
    &\geq&2\sqrt{(a+b)(a+c)}\cdot2\sqrt{(a+b)(b+c)}\cdot2\sqrt{(a+c)(b+c)}\\
    &=&8(a+b)(a+c)(b+c)\\ &=&8(1-a)(1-b)(1-c). \end{eqnarray*}
\item
    Duota nelygybė ekvivalenti $\frac{a}{b}+\frac{b}{c}+\frac{c}{a}+\frac{b}{a}+\frac{c}{b}+\frac{a}{c}\geq\frac{2(a+b+c)}{\sqrt[3]{abc}}$.
    Pagal AM-GM nelygybę: $$+\left\{\begin{array}{ll}
    \frac{a}{b}+\frac{a}{b}+\frac{b}{c}\geq3\sqrt[3]{\frac{a}{b}\cdot\frac{a}{b}\cdot\frac{b}{c}}=\frac{3a}{\sqrt[3]{abc}},&\\
    \frac{b}{c}+\frac{b}{c}+\frac{c}{a}\geq3\sqrt[3]{\frac{b}{c}\cdot\frac{b}{c}\cdot\frac{c}{a}}=\frac{3b}{\sqrt[3]{abc}},&\\
    \frac{c}{a}+\frac{c}{a}+\frac{a}{b}\geq3\sqrt[3]{\frac{c}{a}\cdot\frac{c}{a}\cdot\frac{a}{b}}=\frac{3c}{\sqrt[3]{abc}},&
    \end{array}\right.$$ \begin{equation*}
    \Rightarrow\frac{a}{b}+\frac{b}{c}+\frac{c}{a}\geq\frac{a+b+c}{\sqrt[3]{abc}}.\tag{1}\end{equation*}
    Taip pat: $$+\left\{\begin{array}{ll}
    \frac{b}{a}+\frac{b}{a}+\frac{a}{c}\geq3\sqrt[3]{\frac{b}{a}\cdot\frac{b}{a}\cdot\frac{a}{c}}=\frac{3b}{\sqrt[3]{abc}},&\\
    \frac{c}{b}+\frac{c}{b}+\frac{b}{a}\geq3\sqrt[3]{\frac{c}{b}\cdot\frac{c}{b}\cdot\frac{b}{a}}=\frac{3c}{\sqrt[3]{abc}},&\\
    \frac{a}{c}+\frac{a}{c}+\frac{c}{b}\geq3\sqrt[3]{\frac{a}{c}\cdot\frac{a}{c}\cdot\frac{c}{b}}=\frac{3a}{\sqrt[3]{abc}},&
    \end{array}\right.$$ \begin{equation*}
    \Rightarrow\frac{b}{a}+\frac{c}{b}+\frac{a}{c}\geq\frac{a+b+c}{\sqrt[3]{abc}}.\tag{2}\end{equation*}
    Sudėję (1) ir (2) gausime tai, ką reikėjo įrodyti.
\item
    Pagal AM-GM: $$\left\{\begin{array}{ll}
    1+\frac{2a}{3b}=\frac{1}{3}+\frac{1}{3}+\frac{1}{3}+\frac{a}{3b}+\frac{a}{3b}\geq5\sqrt[5]{\left(\frac{1}{3}\right)^3\cdot\left(\frac{a}{3b}\right)^2}=\frac{5}{3}\left(\frac{a}{b}\right)^{\frac{2}{5}},&\\
    1+\frac{2b}{3c}=\frac{1}{3}+\frac{1}{3}+\frac{1}{3}+\frac{b}{3c}+\frac{b}{3c}\geq5\sqrt[5]{\left(\frac{1}{3}\right)^3\cdot\left(\frac{b}{3c}\right)^2}=\frac{5}{3}\left(\frac{b}{c}\right)^{\frac{2}{5}},&\\
    1+\frac{2c}{3d}=\frac{1}{3}+\frac{1}{3}+\frac{1}{3}+\frac{c}{3d}+\frac{c}{3d}\geq5\sqrt[5]{\left(\frac{1}{3}\right)^3\cdot\left(\frac{c}{3d}\right)^2}=\frac{5}{3}\left(\frac{c}{d}\right)^{\frac{2}{5}},&\\
    1+\frac{2d}{3a}=\frac{1}{3}+\frac{1}{3}+\frac{1}{3}+\frac{d}{3a}+\frac{d}{3a}\geq5\sqrt[5]{\left(\frac{1}{3}\right)^3\cdot\left(\frac{d}{3a}\right)^2}=\frac{5}{3}\left(\frac{d}{a}\right)^{\frac{2}{5}},&
    \end{array}\right.$$ $\Rightarrow
    S=\left(1+\frac{2a}{3b}\right)\left(1+\frac{2b}{3c}\right)\left(1+\frac{2c}{3d}\right)\left(1+\frac{2d}{3a}\right)\geq\frac{625}{81}\cdot\left(\frac{a}{b}\cdot\frac{b}{c}\cdot\frac{c}{d}\cdot\frac{d}{a}\right)^{\frac{2}{5}}=\frac{625}{81}.$
    $S$ Minimumas yra $\frac{625}{81}$. Jis pasiekiamas, kai $a=b=c=d>0.$
\item
    Naudodami sąlygą, verčiame nelygybę homogenine:
    $\frac{a^3}{b(2c+a)}+\frac{b^3}{c(2a+b)}+\frac{c^3}{a(2b+c)}\geq
    \frac{a+b+c}{3}.$ Pagal AM-GM: $$+\left\{\begin{array}{ll}
    \frac{9a^3}{b(2c+a)}+3b+(2c+a)\geq3\sqrt[3]{\frac{9a^3}{b(2c+a)}\cdot3b(2c+a)}=9a,&\\
    \frac{9b^3}{c(2a+b)}+3c+(2a+b)\geq3\sqrt[3]{\frac{9b^3}{b(2a+b)}\cdot3c(2a+b)}=9b,&\\
    \frac{9c^3}{a(2b+c)}+3a+(2b+c)\geq3\sqrt[3]{\frac{9c^3}{b(2b+c)}\cdot3a(2b+c)}=9c,&
    \end{array}\right.$$ Sudęję ir sutvarkę nelygybę ir gausime tai, ką
    reikėjo įrodyti.
\item
    Nelygybę galime paversti homogenine, naudodami duotą sąlygą:
    \begin{align*}
    &\frac{c(a+b)+ab}{a(a+b)}+\frac{a(b+c)+bc}{b(b+c)}+\frac{b(c+a)+ca}{c(c+a)}\geq\frac{9}{2}\\
    \Leftrightarrow&\frac{a}{b}+\frac{b}{c}+\frac{c}{a}+\frac{b}{a+b}+\frac{c}{b+c}+\frac{a}{c+a}\geq\frac{9}{2}\\
    \Leftrightarrow&\frac{a+b}{b}+\frac{b+c}{c}+\frac{c+a}{a}+\frac{b}{a+b}+\frac{c}{b+c}+\frac{a}{c+a}\geq\frac{15}{2}
    \tag{1} \end{align*} Naudosime AM-GM nelygybę:
    \begin{eqnarray*}\mbox{KAIRĖ PUSĖ(1)}
    &=&\frac{a+b}{4b}+\frac{b+c}{4c}+\frac{c+a}{4a}+\frac{b}{a+b}+\frac{c}{b+c}+\frac{a}{c+a}\\
    &&\hspace{1cm}+\frac{3}{4}\left(\frac{a+b}{b}+\frac{b+c}{c}+\frac{c+a}{a}\right)\\
    &\geq&6\sqrt[6]{\frac{a+b}{4b}\cdot\frac{b+c}{4c}\cdot\frac{c+a}{4a}\cdot\frac{b}{a+b}\cdot\frac{c}{b+c}\cdot\frac{a}{c+a}}\\
    &&\hspace{1cm}+\frac{3}{4}\left(3\sqrt[3]{\frac{a}{b}\cdot\frac{b}{c}\cdot\frac{c}{a}}+3\right)\\
    &=&\frac{15}{2}. \end{eqnarray*}
\item
    Pagal AM-GM nelygybę: $$3=ab+bc+ac\geq3\sqrt[3]{a^2b^2c^2}\Rightarrow
    abc\leq1.$$ Pagal duotą sąlygą ir turimą rezultatą:
    \begin{eqnarray*}\text{KAIRĖ
    PUSĖ}&=&\sum_{cyc}{\frac{1}{1+a(3-bc)}}=\sum_{cyc}{\frac{1}{1+3a-abc}}\\&\leq&\sum_{cyc}{\frac{1}{3a}}=\frac{ab+ac+bc}{3abc}=\frac{1}{abc}.\end{eqnarray*}
\item
    Nelygybę keliame kvadratu ir sutvarkome:
    $\Leftrightarrow\frac{a^2b^2}{c^2}+\frac{b^2c^2}{a^2}+\frac{c^2a^2}{b^2}\geq
    a^2+b^2+c^2.$ Pagal AM-GM: $$+\left\{\begin{array}{ll}
    \frac{a^2b^2}{c^2}+\frac{b^2c^2}{a^2}\geq2\sqrt{\frac{a^2b^2}{c^2}\cdot\frac{b^2c^2}{a^2}}=2b^2,&\\
    \frac{b^2c^2}{a^2}+\frac{c^2a^2}{b^2}\geq2\sqrt{\frac{b^2c^2}{a^2}\cdot\frac{c^2a^2}{b^2}}=2c^2,&\\
    \frac{c^2a^2}{b^2}+\frac{a^2b^2}{c^2}\geq2\sqrt{\frac{c^2a^2}{b^2}\cdot\frac{a^2b^2}{c^2}}=2a^2,&\end{array}\right.
    $$
    $\Rightarrow\frac{a^2b^2}{c^2}+\frac{b^2c^2}{a^2}+\frac{c^2a^2}{b^2}\geq
    a^2+b^2+c^2.$
\item
    Pagal AM-GM nelygybę: $$(x+y)(x+z)=xy+(x^2+zy)+xz\geq
    xy+2x\sqrt{yz}+xz=(\sqrt{xy}+\sqrt{xz})^2.$$ Taigi,
    $$\sum_{cyc}{\frac{x}{x+\sqrt{(x+y)(x+z)}}}\leq\sum_{cyc}{\frac{x}{x+\sqrt{xy}+\sqrt{xz}}}=
    \sum_{cyc}{\frac{\sqrt{x}}{\sqrt{x}+\sqrt{y}+\sqrt{z}}}=1.$$
\item
    Padauginę iš 2 ir prie abiejų nelygybės pusių pridėję $x^2+y^2+z^2$,
    gausime $$x^2+2\sqrt{x}+y^2+2\sqrt{y}+z^2+2\sqrt{z}\geq3.$$ Iš AM-GM
    nelygybės:
    $$\sum_{cyc}{x^2+\sqrt{x}+\sqrt{x}}\geq\sum_{cyc}{3\sqrt[3]{x^3}}=9.$$
    Tą ir reikėjo įrodyti.
\item
    Pagal AM-GM: $$+\left\{\begin{array}{ll}
    \frac{a^3}{(1+b)(1+c)}+\frac{1+b}{8}+\frac{1+c}{8}\geq3\sqrt[3]{\frac{a^3}{(1+b)(1+c)}\cdot\frac{1+b}{8}\cdot\frac{1+c}{8}}=\frac{3a}{4},&\\
    \frac{b^3}{(1+c)(1+a)}+\frac{1+c}{8}+\frac{1+a}{8}\geq3\sqrt[3]{\frac{b^3}{(1+c)(1+a)}\cdot\frac{1+c}{8}\cdot\frac{1+a}{8}}=\frac{3b}{4},&\\
    \frac{c^3}{(1+a)(1+b)}+\frac{1+a}{8}+\frac{1+b}{8}\geq3\sqrt[3]{\frac{c^3}{(1+a)(1+b)}\cdot\frac{1+a}{8}\cdot\frac{1+b}{8}}=\frac{3c}{4},&
    \end{array}\right.$$
    $\Rightarrow\frac{a^3}{(1+b)(1+c)}+\frac{b^3}{(1+c)(1+a)}+\frac{c^3}{(1+a)(1+b)}\geq\frac{a+b+c}{2}-\frac{3}{4}\geq\frac{3\sqrt[3]{abc}}{2}-\frac{3}{4}=\frac{3}{4}.$
\item
    Naudodami AM-GM nelygybę gauname:\\
    $\left(\frac{a^6}{b^3}+\frac{b^6}{c^3}+4\right)+\left(\frac{b^6}{c^3}+\frac{c^6}{a^3}+4\right)+\left(\frac{c^6}{a^3}+\frac{a^6}{b^3}+4\right)\geq6\left(\sqrt[6]{\frac{a^6b^3}{c^3}}+\sqrt[6]{\frac{b^6c^3}{a^3}}+\sqrt[6]{\frac{c^6a^3}{b^3}}\right)=18$\\
    $\Leftrightarrow2\left(\frac{a^6}{b^3}+\frac{b^6}{c^3}+\frac{c^6}{a^3}\right)+12\geq18\Leftrightarrow\frac{a^6}{b^3}+\frac{b^6}{c^3}+\frac{c^6}{a^3}\geq3.$
\item
    Pastebėkime, kad \\
    $(a-b+c-d)^2\geq0\Leftrightarrow(a+b+c+d)^2\geq4(ab+bc+cd+da)=4\Leftrightarrow
    a+b+c+d\geq2.$ Pagal AM-GM: $$+\left\{\begin{array}{ll}
    \frac{36a^3}{b+c+d}+2(b+c+d)+6a+3\geq4\sqrt[4]{\frac{36a^3}{b+c+d}\cdot2(b+c+d)\cdot6a\cdot3}=24a,&\\
    \frac{36b^3}{c+d+a}+2(c+d+a)+6b+3\geq4\sqrt[4]{\frac{36b^3}{c+d+a}\cdot2(c+d+a)\cdot6b\cdot3}=24b,&\\
    \frac{36c^3}{d+a+b}+2(d+a+b)+6c+3\geq4\sqrt[4]{\frac{36c^3}{d+a+b}\cdot2(d+a+b)\cdot6c\cdot3}=24c,&\\
    \frac{36d^3}{a+b+c}+2(a+b+c)+6d+3\geq4\sqrt[4]{\frac{36d^3}{a+b+c}\cdot2(a+b+c)\cdot6d\cdot3}=24d,&
    \end{array}\right.$$ $$\Rightarrow\mbox{KAIRĖ
    PUSĖ}\geq{\frac{a+b+c+d}{3}}-\frac{1}{3}\geq\frac{2}{3}-\frac{1}{3}=\frac{1}{3}.$$
\item
    Pagal AM-GM: $$+\left\{\begin{array}{ll}
    \frac{bc}{a^2}=\sqrt[3]{\frac{b^7}{a^2c^2}\cdot\frac{c^7}{a^2b^2}\cdot\frac{1}{a^2b^2c^2}}\leq\frac{1}{3}\left(\frac{b^7}{a^2c^2}+\frac{c^7}{a^2b^2}+\frac{1}{a^2b^2c^2}\right),&\\
    \frac{ca}{b^2}=\sqrt[3]{\frac{c^7}{b^2a^2}\cdot\frac{a^7}{b^2c^2}\cdot\frac{1}{a^2b^2c^2}}\leq\frac{1}{3}\left(\frac{c^7}{b^2a^2}+\frac{a^7}{b^2c^2}+\frac{1}{a^2b^2c^2}\right),&\\
    \frac{ab}{c^2}=\sqrt[3]{\frac{a^7}{c^2b^2}\cdot\frac{b^7}{c^2a^2}\cdot\frac{1}{a^2b^2c^2}}\leq\frac{1}{3}\left(\frac{a^7}{c^2b^2}+\frac{b^7}{c^2a^2}+\frac{1}{a^2b^2c^2}\right),&\\
    abc=\sqrt[3]{\frac{b^7}{a^2c^2}\cdot\frac{c^7}{a^2b^2}\cdot\frac{a^7}{b^2c^2}}\leq\frac{1}{3}\left(\frac{b^7}{a^2c^2}+\frac{c^7}{a^2b^2}+\frac{a^7}{b^2c^2}\right),&
    \end{array}\right.$$ Sudėję gausime tai, ką reikėjo įrodyti.
\end{enumerate} 
\subsubsection*{Cauchy-Schwarz nelygybė}
\begin{enumerate}
\item
    Pažymime $a_1=\alpha$, $a_2+a_3=\beta$, $a_4+a_5+a_6=\gamma$ ir
    $a_7+a_8+a_9+a_{10}=\delta$. Tuomet $\alpha+\beta+\gamma+\delta=1$ ir
    $\alpha\geq\beta\geq\gamma\geq\delta$. Pagal Cauchy-Schwarz nelygybę:
    \\$Z=a_1^2+a_2^2+...+a_{10}^2\geq\alpha^2+\frac{\beta^2}{2}+\frac{\gamma^2}{3}+\frac{\delta^2}{4}\Leftrightarrow12Z\geq12\alpha^2+6\beta^2+4\gamma^2+3\delta^2.$
    Pastebime, kad $\alpha\geq\frac{1}{4}$, $\alpha+\beta\geq\frac{1}{2}$,
    $\alpha+\beta+\gamma\geq\frac{3}{4}$, be to
    $\alpha+\beta+\gamma+\delta=1$. Teisingai padauginę ir sudėję gausime
    $12\alpha+6\beta+4\gamma+3\delta\geq\frac{25}{4}.$ Na o pagal
    Cauchy-Schwarz nelygybę:
    $$12Z\geq\frac{(12\alpha+6\beta+4\gamma+3\delta)^2}{25}\geq\frac{25^2}{4^2\cdot25}=\frac{25}{16}.$$
    Taigi $Z$ minimumas yra $\frac{25}{192}$, o jis pasiekiamas, kai
    $a_1=\frac{1}{4}$, $a_2=a_3=\frac{1}{8}$, $a_4=a_5=a_6=\frac{1}{12}$
    ir $a_7=a_8=a_9=a_{10}=\frac{1}{16}$.
\item
    Pažymėkime $\mbox{Ž} = \sqrt{a} + \sqrt{b} + \sqrt{c} + \sqrt{d}.$
    Pagal Cauchy-Schwarz nelygybės Engel formą:
    $$\frac{\mbox{Ž}^2}{10}=\frac{(\sqrt{a}+\sqrt{b}+\sqrt{c}+\sqrt{d})^2}{10}\leq
    a+\frac{b}{2}+\frac{c}{3}+\frac{d}{4}$$
    \begin{eqnarray*}\Leftrightarrow12\cdot\frac{\mbox{Ž}^2}{10}&\leq&12a+6b+4c+3d\\
    &=&3(a+b+c+d)+(a+b+c)+2(a+b)+6a\\
    &\leq&3\cdot30+14+2\cdot5+6\cdot1=120
    \end{eqnarray*}$$\Leftrightarrow\mbox{Ž}\leq10.$$
\item
    Nelygybę transformuojame naudodami duotą sąlygą ir tada sprendžiame
    naudodami Cauchy-Schwarz nelygybę: \begin{align*} \mbox{KAIRĖ
    PUSĖ}&=\frac{b^2c^2}{a(b+c)}+\frac{a^2c^2}{b(a+c)}+\frac{a^2b^2}{c(a+b)}\\
    &\geq\frac{(ab+bc+ac)^2}{2(ab+bc+ac)}=\frac{ab+bc+ac}{2}\\
    &\geq\frac{3\sqrt[3]{a^2b^2c^2}}{3} \tag{AM-GM}\\
    &=\frac{3}{2}.\end{align*}
\item
    Pagal Cauchy-Schwarz nelygybę: \begin{eqnarray*} \text{KAIRĖ PUSĖ}
    &=&\sum_{cyc}{\sqrt{x_n\left(3x_1+x_2\right)}}\\
    &\leq&\sqrt{\left(\sum_{cyc}{x_n}\right)\left(\sum_{cyc}{3x_1+x_2}\right)}\\
    &=&\sqrt{4\left(\sum_{cyc}{x_n}\right)^2}=2(x_1+x_2+...+x_n).\end{eqnarray*}
\item
    Pertvarkę taikome Cauchy-Schwarz nelygybę:
    $$(a+b+c)\left(\frac{a}{(b+c)^2}+\frac{b}{(a+c)^2}+\frac{c}{(a+b)^2}\right)\geq\left(\frac{a}{b+c}+\frac{b}{a+c}+\frac{c}{a+b}\right)^2\geq\frac{9}{4}.$$
    Paskutinė nelygybė remiasi Nesbitt'o nelygybe, o tai ir užbaigia
    įrodymą.
\item
    Pagal Cauchy-Schwarz nelygybę:
    \begin{eqnarray*}\frac{x}{ay+bz}+\frac{y}{az+bx}+\frac{z}{ax+by}&\geq&\frac{(x+y+z)^2}{x(ay+bz)+y(az+bx)+z(ax+by)}\\
    &=&\frac{(x+y+z)^2}{(xy+yz+xz)(a+b)}\\&\geq&\frac{3}{a+b}.\end{eqnarray*}
    Paskutinė nelygybė teisinga pagal $$(x+y+z)^2\geq3(xy+yz+xz).$$
\item
    Nelygybę pertvarkome, tada taikome Cauchy-Schwarz nelygybę, tada vėl
    pertvarkome: $$\text{KAIRĖ
    PUSĖ}=\sum_{cyc}{\frac{a^2}{a+ab^2c}}\geq\frac{(a+b+c)^2}{a+b+c+abc(a+b+c)}=\frac{a+b+c}{abc+1}.$$
    Belieka įrodyti $$2(a+b+c)\geq3abc+3,$$ kas pagal duotą sąlygą yra
    ekvivalentu $$(a+b+c)^3\geq27abc,$$ kas seka iš AM-GM nelygybės.
\item
    Įrodymas remiasi matematine indukcija. Akivaizdu, kad jei nelygybė
    teisinga su $n=k$, tai teisinga ir su $n=k+1$. Taigi, belieka įrodyti
    kai $n=2$:
    $$\sqrt{a_1^2+b_1^2}+\sqrt{a_2^2+b_2^2}\geq\sqrt{(a_1+a_2)^2+(b_1+b_2)^2}.$$
    Atskliaudus ir sutvarkius:
    $$\Leftrightarrow(a_1^2+a_2^2)(b_1^2+b_2^2)\geq(a_1b_1+a_2b_2)^2,$$
    kas yra tiesiog Cauchy-Schwarz nelygybė.
    Šią nelygybę taip pat galima įrodyti naudojantis Pitagoro teoremą, čia
    įrodymo nepateiksime, bet galite pabandyti jį patys atrasti.
\item
    \textit{Lemma.} $3(a^3+b^3+c^3)\geq(a+b+c)(a^2+b^2+c^2)$.\\
    \noindent\textit{Lemos įrodymas.} Naudojame AM-GM nelygybę:
    $3(a^3+b^3+c^3)=\sum\limits_{cyc}{a^3}+\sum\limits_{sym}{\frac{a^3+a^3+b^3}{3}}\geq\sum\limits_{cyc}{a^3}+\sum\limits_{sym}{\frac{3a^2b}{3}}=(a+b+c)(a^2+b^2+c^2).$\hfill{$\square$}
    \\Pagal Cauchy-Schwarz nelygybę ir lemą:
    \begin{eqnarray*}(\text{DEŠINĖ
    PUSĖ})^2&\leq&(a^2+b^2+c^2)((b+c)+(a+c)+(a+b))\\
    &=&2(a^2+b^2+c^2)(a+b+c)\\ &\leq&6(a^3+b^3+c^3)=6(\text{KAIRĖ
    PUSĖ}).\end{eqnarray*} Kita vertus, pagal AM-GM:
    \begin{eqnarray*}\text{DEŠINĖ
    PUSĖ}&\geq&3\sqrt[3]{abc\sqrt{(b+c)(a+c)(a+b)}}\\&\geq&3{\sqrt[3]{abc\sqrt{8abc}}}\\&=&3\sqrt[3]{2\cdot\sqrt{8\cdot2}}=6.\end{eqnarray*}
    Gauname, kad:\\ $6(\text{DEŠINĖ PUSĖ})\leq(\text{DEŠINĖ
    PUSĖ})^2\leq6(\text{KAIRĖ PUSĖ})\Rightarrow$\\ $\text{KAIRĖ
    PUSĖ}\geq\text{DEŠINĖ PUSĖ}$, ką ir reikėjo įrodyti.
\item
    Pagal Cauchy-Schwarz nelygybę:
    $$(x+y)(z+x)\geq(\sqrt{xy}+\sqrt{xz})^2.$$ Taip sumažinę visų trupmenų
    vardiklius gausime:
    $$\sum_{cyc}{\frac{x}{x+\sqrt{(x+y)(x+z)}}}\leq\sum_{cyc}{\frac{x}{x+\sqrt{xy}+\sqrt{xz}}}=\sum_{cyc}{\frac{\sqrt{x}}{\sqrt{x}+\sqrt{y}+\sqrt{z}}}=1.$$
\item
    Pagal Cauchy-Schwarz nelygybę: \\$\mbox{KAIRĖ PUSĖ} =
    \sum\limits_{cyc}\frac{a^2}{ab+ac} \geq
    \frac{(a+b+c+d+e+f)^2}{ab+ac+bc+bd+cd+ce+de+df+ef+ea+fa+fb}.$
    \\Pavadinkime gautą vardiklį $V$. Tada:
    $$2V=(a+b+c+d+e+f)^2-(a+d)^2-(b+e)^2-(c+f)^2.$$ Tačiau vėl iš
    Cauchy-Schwarz nelygybės:
    $$\left(1+1+1\right)\left((a+d)^2+(b+e)^2+(c+f)^2\right)\geq(a+b+c+d+e+f)^2.$$
    Taigi, $V\leq\frac{1}{3}\cdot(a+b+c+d+e+f)^2$, kas užbaigia įrodymą.
\item
    Cauchy-Schwarz nelygybę naudosime dukart. Pirmiausia,
    $$ax+by+cz\leq\sqrt{a^2+b^2+c^2}\cdot\sqrt{x^2+y^2+z^2}.$$ Taigi,
    \begin{eqnarray*}\text{KAIRĖ
    PUSĖ}&\leq&\sqrt{\sum_{cyc}{a^2}\cdot}\sqrt{\sum_{cyc}{x^2}}+\sqrt{2\sum_{cyc}{ab}}\cdot\sqrt{2\sum_{cyc}{xy}}\\
    &\leq&\sqrt{\sum_{cyc}{x^2}+2\sum_{cyc}{xy}}\cdot\sqrt{\sum_{cyc}{a^2}+2\sum_{cyc}{ab}}\\
    &=&(a+b+c)(x+y+z)\\ &=&a+b+c.\end{eqnarray*}
\item
    Pirmiausia pertvarkome:
    \begin{eqnarray*}&\Leftrightarrow&\sum_{cyc}\frac{a+b+c}{b+c}+\sum_{cyc}\frac{a}{b+c}\leq\sum_{cyc}\frac{2a}{b}\Leftrightarrow3+2\sum_{cyc}\frac{a}{b+c}\leq\sum_{cyc}\frac{2a}{b}\\
    &\Leftrightarrow&\sum_{cyc}{\frac{a}{b}-\frac{a}{b+c}}\geq\frac{3}{2}\Leftrightarrow\frac{ac}{b(b+c)}+\frac{ab}{c(a+c)}+\frac{bc}{a(a+b)}\geq\frac{3}{2}\\
    &\Leftrightarrow&\frac{a^2c^2}{abc(b+c)}+\frac{a^2b^2}{abc(a+c)}+\frac{b^2c^2}{abc(a+b)}\geq\frac{3}{2}.\end{eqnarray*}
    Paskutinei nelygybei pritaikę Cauchy-Schwarz nelygybę gausime:
    $$\frac{a^2c^2}{abc(b+c)}+\frac{a^2b^2}{abc(a+c)}+\frac{b^2c^2}{abc(a+b)}\geq\frac{(ab+bc+ac)^2}{2abc(a+b+c)}\geq\frac{3}{2}.$$
    Paskutinei nelygybei įrodyti naudojome gerai žinomą faktą, kad
    realiesiems $x,y,z$ galioja $(x+y+z)^2\geq3(xy+xz+yz)$.
\item
    Padauginę nelygybę iš -2 ir prie abiejų pusių pridėję po 3, gausime
    ekvivalenčią nelygybę
    \begin{equation*}\frac{a^2+b^2}{2+a^2+b^2}+\frac{a^2+c^2}{2+a^2+c^2}+\frac{c^2+b^2}{2+c^2+b^2}\geq\frac{3}{2}.\tag{1}\end{equation*}
    Naudosimes Cauchy-Schwarz nelygybe: \begin{eqnarray*}\text{KAIRĖ PUSĖ
    (1)}&\geq&\frac{\left(\sqrt{a^2+b^2}+\sqrt{a^2+c^2}+\sqrt{b^2+c^2}\right)^2}{6+2(a^2+b^2+c^2)}\\&=&\frac{2(a^2+b^2+c^2)+2\sum\limits_{cyc}{\sqrt{(a^2+b^2)(a^2+c^2)}}}{6+2(a^2+b^2+c^2)}\\
    &\geq&\frac{2(a^2+b^2+c^2)+2\sum\limits_{cyc}{(a^2+bc)}}{6+2(a^2+b^2+c^2)}\\
    &=&\frac{(a+b+c)^2+3(a^2+b^2+c^2)}{6+2(a^2+b^2+c^2)}\\&=&\frac{3(3+a^2+b^2+c^2)}{2(3+a^2+b^2+c^2)}=\frac{3}{2}.\end{eqnarray*}
\item
    Visur taikysime Cauchy-Schwarz nelygybę. Pastebėkime, kad
    \begin{eqnarray*}
    (a^2+2)(b^2+2)&=&(a^2+1)(1+b^2)+a^2+b^2+3\\&\geq&(a+b)^2+\frac{(a+b)^2}{2}+3\\&=&\frac{3}{2}((a+b)^2+2).\end{eqnarray*}
    Tuomet
    \begin{eqnarray*}(a^2+2)(b^2+2)(c^2+2)&\geq&\frac{3}{2}((a+b)^2+2)(2+c^2)\\&\geq&\frac{3}{2}(\sqrt{2}(a+b)+\sqrt2c)^2\\&=&3(a+b+c)^2.\end{eqnarray*}
\end{enumerate} 
\subsubsection*{Specialios technikos}
\begin{enumerate}
\item
    Pirma mintis - atlikti homogenizuojantį keitinį $a=\frac{x}{y}$,
    tačiau netrunkame įsitikinti kad tai nieko gero neduoda, todėl
    tenka pasukti galvą ieškant kitokio kelio. Ir štai - keitinys
    $a=\frac{1}{x}$, $b=\frac{1}{y}$, $c=\frac{1}{z}$ išspręs problemą.
    Žinoma, nepamirškime, kad vistiek $xyz=1$. Nelygybė tampa
    $$1+\frac{3}{xy+yz+zx}\geq\frac{6}{x+y+z}.$$ Kadangi
    $$xy+yz+xz\leq\frac{1}{3}(x+y+z)^2,$$ tai belieka įrodyti:
    $$1+\frac{9}{(x+y+z)^2}\geq\frac{6}{x+y+z},$$ kas seka iš AM-GM.
\item
    Nesunku pamatyti, kad reikia pasikeisti $a=\frac{2x}{y}$,
    $b=\frac{2y}{z}$, $c=\frac{2z}{a}$. Gausime nelygybę:
    $$\frac{2x-2y}{2x+y}+\frac{2y-2z}{2y+z}+\frac{2z-2x}{2z+x}\leq0\Leftrightarrow\frac{y}{2x+y}+\frac{z}{2y+z}+\frac{x}{2z+x}\geq1.$$
    Pagal Cauchy-Schwarz nelygybę:
    $$\sum_{cyc}{\frac{x}{2z+x}}\geq\frac{(x+y+z)^2}{x(2z+x)+y(2x+y)+z(2y+z)}=1.$$
\item
    Kadangi $abc=1$, keičiame $a=\frac{x}{y}$, $b=\frac{y}{z}$,
    $c=\frac{z}{x}$. Tuomet gausime, kad reikia įrodyti
    $$\sum_{cyc}{\frac{z^2}{y^2+xz}}\geq\frac{3}{2}.$$ Pritaikome
    Cauchy-Schwarz nelygybę:
    $$\sum_{cyc}{\frac{z^4}{z^2y^2+xz^3}}\geq\frac{(x^2+y^2+z^2)^2}{x^2y^2+x^2z^2+y^2z^2+xz^3+yx^3+zy^3}.$$
    Belieka įrodyti
    $$2(x^2+y^2+z^2)^2\geq3(x^2y^2+x^2z^2+y^2z^2+xz^3+yx^3+zy^3),$$
    kas ekvivalentu šių dviejų nelygybių (kurios galioja pagal AM-GM
    nelygybę) sumai: $$\sum_{cyc}{x^4}\geq\sum_{cyc}{x^3y}$$ ir
    $$\sum_{cyc}{x^4+x^2y^2}\geq2\sum_{cyc}{x^3y}.$$
\item
     Duota nelygybė yra homogeninė, todėl ją įrodysime kai
     $a^2+b^2+c^2+d^2=1$. Nelygybė tampa:
     $$\frac{a}{1-a^2}+\frac{b}{1-b^2}+\frac{c}{1-c^2}+\frac{d}{1-d^2}\geq\frac{3\sqrt{3}}{2}.$$
     Pagal AM-GM nelygybę:
     $$2a^2(1-a^2)(1-a^2)\leq\left(\frac{2a^2+1-a^2+1-a^2}{3}\right)^3=\left(\frac{2}{3}\right)^3$$
     $$\Leftrightarrow a(1-a^2)\leq\frac{2}{3\sqrt{3}}$$
     $$\Leftrightarrow\frac{a}{1-a^2}\geq\frac{3\sqrt{3}}{2}a^2.$$ Taigi:
     $$\frac{a}{1-a^2}+\frac{b}{1-b^2}+\frac{c}{1-c^2}+\frac{d}{1-d^2}\geq\frac{3\sqrt{3}}{2}(a^2+b^2+c^2+d^2)=\frac{3\sqrt{3}}{2}.$$
\item
    Kadangi turime homogeninę nelygybę, nemažindami bendrumo tariame, kad
    $a+b+c=3$. Pertvarkę gausime:
    $$\frac{(3+a)^2}{2a^2+(3-a)^2}+\frac{(3+b)^2}{2b^2+(3-b)^2}+\frac{(3+c)^2}{2c^2+(3-c)^2}\leq8$$
    $$\Leftrightarrow\frac{a^2+6a+9}{a^2-2a+3}+\frac{b^2+6b+9}{b^2-2b+3}+\frac{c^2+6c+9}{c^2-2c+3}\leq24$$
    $$\Leftrightarrow3+\frac{8a+6}{(a-1)^2+2}+\frac{8b+6}{(b-1)^2+2}+\frac{8c+6}{(c-1)^2+2}\leq24.
    $$ Kadangi $(x-1)^2+2\geq2$ visiems $x$, tai belieka įrodyti
    $$8(a+b+c)+18\leq42,$$ kas pagal sąlygą $a+b+c=3$ yra tapatybė.
\item
    Pasikeiskime $x=\frac{1}{a}$, $y=\frac{1}{b}$, $z=\frac{1}{c}$. Sąlyga
    taps $xy+xz+yz=1$. Pagrindinė nelygybė:
    $$\frac{x}{\sqrt{1+x^2}}+\frac{y}{\sqrt{1+y^2}}+\frac{z}{\sqrt{1+z^2}}\leq\frac{3}{2},$$
    arba
    $$\frac{x}{\sqrt{x^2+xz+xy+yz}}+\frac{y}{\sqrt{y^2+xz+xy+yz}}+\frac{z}{\sqrt{z^2+xz+xy+yz}}\leq\frac{3}{2},$$
    arba
    $$\frac{x}{\sqrt{(x+y)(x+z)}}+\frac{y}{\sqrt{(y+x)(y+z)}}+\frac{z}{\sqrt{(z+x)(z+y)}}\leq\frac{3}{2}.$$
    Pagal AM-GM nelygybę:
    \begin{eqnarray*}\sum_{cyc}{\frac{x}{\sqrt{(x+y)(x+z)}}}
    &=&\sum_{cyc}{\frac{x\sqrt{(x+y)(x+z)}}{(x+y)(x+z)}} \\
    &\leq&\sum_{cyc}{\frac{1}{2}\cdot\frac{x(x+y)+x(x+z)}{(x+y)(x+z)}}\\
    &=&\frac{1}{2}\sum_{cyc}{\frac{x}{x+z}+\frac{x}{x+y}}\\&=&\frac{3}{2}.\end{eqnarray*}
\item
    Pasikeiskime $a=\frac{x}{y}$, $b=\frac{y}{z}$, $c=\frac{z}{t}$,
    $d=\frac{t}{u}$, $e=\frac{u}{a}$. Tada po nedidelių pertvarkymų
    gausime:
    $$\sum_{cyc}{\frac{a+abc}{1+ab+abcd}}=\sum_{cyc}{\frac{\frac{1}{y}+\frac{1}{t}}{\frac{1}{x}+\frac{1}{z}+\frac{1}{u}}}.$$
    O tada dar pakeitę $\frac{1}{x}=a_1$, $\frac{1}{y}=a_2$,
    $\frac{1}{z}=a_3$, $\frac{1}{t}=a_4$, $\frac{1}{u}=a_5$ ir paprastumo
    dėlei pažymėję $S=a_1+a_2+a_3+a_4+a_5$, gausime, kad reikia įrodyti
    \begin{equation*}\sum_{cyc}{\frac{a_2+a_4}{a_1+a_3+a_5}}\geq\frac{10}{3}.\tag{1}\end{equation*}
    Dabar taikome Cauchy-Schwarz nelygybę, nežymiai pertvarkome vardiklį
    ir dar kartą taikome Cauchy-Schwarz nelygybę:
    \begin{eqnarray*}\text{KAIRĖ
    PUSĖ(1)}&\geq&\frac{4S^2}{\sum\limits_{cyc}{(a_2+a_4)(a_1+a_3+a_5)}}\\
    &=&\frac{4S^2}{2S^2-\sum\limits_{cyc}{(a_1+a_3)^2}}\\
    &\geq&\frac{4S^2}{2S^2-\frac{4S^2}{5}}\\
    &=&\frac{10}{3}.\end{eqnarray*}
\item
    Neprarasdami bendrumo tariame, kad $a+b+c+d=1$. Tuo naudodamiesi
    įrodysime, kad $$(a+b)(b+c)(c+d)(d+a)\geq abc+bcd+cda+dab.$$ Tai
    reikalauja tiesiog pertvarkyti nelygybę ir pritaikyti faktą
    $x^2\geq0$:
    \begin{eqnarray*}(a+b)(b+c)(c+d)(d+a)&=&a^2c^2+b^2d^2+2abcd+\sum_{cyc}{abc(a+b+c)}\\
    &=&(ac-bd)^2+\sum_{cyc}{abc(a+b+c+d)}\\&\geq&\sum_{cyc}{abc}.\end{eqnarray*}
    Dabar įrodysime
    $$\left(\sum_{cyc}{abc}\right)^3\geq16a^2b^2c^2d^2(a+b+c+d).$$ Pakeitę
    $abc=x$, $bcd=y$, $cda=z$, $dab=t$, gauname
    $$(x+y+z+t)^3\geq16(xyz+yzt+ztx+txy).$$ Taikykime AM-GM nelygybę: \begin{eqnarray*}&&\text{KAIRĖ
    PUSĖ}=\\&=&\sum_{cyc}{x^3}+\frac{3}{2}\sum_{sym}{x^2y}+6\sum_{cyc}{xyz}\\
    &=&\frac{1}{3}\sum_{cyc}{x^3+y^3+z^3}+\frac{1}{4}\sum_{sym}{x^2y+x^2z+y^2x+y^2z+z^2x+z^2y}+6\sum_{cyc}{xyz}\\
    &\geq&\sum_{cyc}{xyz}+\frac{3}{2}\sum_{sym}{xyz}+6\sum_{cyc}{xyz}\\&=&16\sum_{cyc}{xyz}.\end{eqnarray*}
\item
    Sąlyga $a,b,c\in[0,1]$ sufleruoja apie trigonometrinį keitinį. Ir
    išties, pasikeitę $a=\sin^2x$, $b=\sin^2y$, $c=\sin^2z$, kur
    $x,y,z\in[0,\frac{\pi}{2}]$, gauname tai, ką reikia: $$\sin x\sin
    y\sin z+\cos x\cos y\cos z<\sin x\sin y+\cos x\cos y=\cos(x-y)<1.$$
\item
    Pakeitę $a=y+z$, $b=x+z$, $c=x+y$, padalinę iš $xyz$ ir sutvarkę
    nelygybę gausime, jog tereikia įrodyti
    $$\frac{x^2}{y}+\frac{y^2}{z}+\frac{z^2}{x}+\frac{x^2}{z}+\frac{y^2}{x}+\frac{z^2}{y}\geq
    2x+2y+2z.$$ Tačiau tai yra dviejų nelygybių, kurios tiesiogiai
    įrodomos su Cauchy-Schwarz nelygybe, suma:
    $$\frac{x^2}{y}+\frac{y^2}{z}+\frac{z^2}{x}\geq x+y+z$$ ir
    $$\frac{x^2}{z}+\frac{y^2}{x}+\frac{z^2}{y}\geq x+y+z.$$
\item
    Nelygybę dauginame iš 4, pertvarkome, tada taikome Cauchy-Schwarz
    nelygybės Engel formą, nes iš trikampio nelygybės seka, kad visi
    vardikliai teigiami: \begin{eqnarray*}4\cdot\text{(KAIRĖ
    PUSĖ)}&=&3+\frac{a+b-c}{3a-b+c}+\frac{b+c-a}{3b-c+a}+\frac{c+a-b}{3c-a+b}\\
    &\geq&3+\frac{(a+b+c)^2}{\sum\limits_{cyc}{(a+b-c)(3a-b+c)}}\\
    &=&3+\frac{(a+b+c)^2}{\sum\limits_{cyc}{3a^2-ab+ac+3ab-b^2+bc-3ac+bc-c^2}}\\
    &=&4.\end{eqnarray*}
\item
    Atliekame Ravi keitinį: $a=x+y$, $b=y+z$, $c=z+x$. Gausime:
    $$3\left(\sqrt{(x+y)(x+z)}+\sqrt{(x+y)(y+z)}+\sqrt{(z+y)(x+z)}\right)\geq2\left(\sqrt{x}+\sqrt{y}+\sqrt{z}\right)^2.$$
    Bet pagal AM-GM nelygybę:
    $$\sqrt{(x+y)(x+z)}=\sqrt{x^2+xy+xz+yz}\geq\sqrt{x^2+2x\sqrt{yz}+yz}=x+\sqrt{yz}.$$
    Analogiškai pasielgę su likusiais nariais gausime naują nelygybę,
    kuriai vėl taikome AM-GM nelygybę:
    \begin{eqnarray*}3(x+y+z)+3(\sqrt{yz}+\sqrt{xz}+\sqrt{xy})&\geq&2(x+y+z)+4(\sqrt{yz}+\sqrt{xz}+\sqrt{xy})\\
    &=&2\left(\sqrt{x}+\sqrt{y}+\sqrt{z}\right)^2.\end{eqnarray*}
\item
    Pertvarkykime kairės pusės dėmenis, kad jie taptų ,,apversti'' ir
    iškart taikykime AM-GM nelygybę:
    $$\frac{a}{1+b^2}=\frac{a+ab^2-ab^2}{1+b^2}=a-\frac{ab^2}{1+b^2}\geq
    a-\frac{ab^2}{2b}=a-\frac{ab}{2}.$$ Analogiškai pertvarkius likusius
    dėmenis, nelygybė pavirs į $$\sum_{cyc}{\frac{a}{1+b^2}}\geq
    a+b+c-\frac{1}{2}\sum_{cyc}{ab}\geq\frac{3}{2}.$$ Paskutinę nelygybę
    įrodome pasinaudoję faktu $$ab+bc+ca\leq\frac{(a+b+c)^2}{3}=3.$$
\item
    Pertvarkome, taikome AM-GM: \begin{eqnarray*}
    \sum_{cyc}{\frac{a+ab^2c-ab^2c}{1+b^2c}}&\geq&\sum_{cyc}{a-\frac{ab^2c}{2b\sqrt{c}}}\\
    &=&\sum_{cyc}{a-\frac{1}{2}b\sqrt{ac\cdot a}}\\
    &\geq&\sum_{cyc}{a-\frac{1}{4}b(ac+a)}\\&=&a+b+c+d-\frac{1}{4}\sum_{cyc}{abc}-\frac{1}{4}\sum_{cyc}{ab}.\end{eqnarray*}
    Pagal AM-GM nelygybę: $$\sum_{cyc}{abc}\leq\frac{1}{16}(a+b+c+d)^3=4,$$ o pagal
    Cauchy-Schwarz nelygybę:
    $$\sum_{cyc}{ab}=(a+b+c+d)^2-(a+c)^2-(b+d)^2\leq(a+b+c+d)^2-\frac{(a+b+c+d)^2}{2}=4.$$
    Taigi,
    $$\frac{a}{1+b^2c}+\frac{b}{1+c^2a}+\frac{c}{1+d^2a}+\frac{d}{1+a^2b}\geq
    a+b+c+d-2=2.$$
\item
    Pertvarkome, taikome AM-GM:
    $$\sum_{cyc}{\frac{1}{a_n^3+2}}=\frac{n}{2}-\frac{1}{2}\sum_{cyc}{\frac{a_n^3}{a_n^3+2}}
    \geq\frac{n}{2}-\frac{1}{2}\sum_{cyc}{\frac{a_n^3}{3a_n}}=\frac{n}{3}.$$
\item
    Naudosime \textit{Cauchy Reverse Technique}:
    $$\sum_{cyc}{\frac{a+1}{b^2+1}}=\sum_{cyc}{a+1-\frac{ab^2+b^2}{b^2+1}}\geq\sum_{cyc}{a+1-\frac{ab+b}{2}}.$$
    Pagal Cauchy-Schwarz nelygybę:
    \begin{eqnarray*}\sum_{cyc}{ab}&=&\frac{1}{2}((a+b+c+d)^2-(a+c)^2-(b+d)^2)\\
    &\leq&
    \frac{1}{2}((a+b+c+d)^2-\frac{(a+b+c+d)^2}{2})=4.\end{eqnarray*}
    Taigi: $$\sum_{cyc}{\frac{a+1}{b^2+1}}\geq
    a+b+c+d+4-\frac{4+a+b+c+d}{2}=4.$$
\item
    \textit{Lema.} $x(2-x)\leq1$, su realiais $x$. \\ \noindent\textit{Lemos
    įrodymas.} $\Leftrightarrow (x-1)^2\geq0$ \hfill{$\square$}
    \\Pertvarkome pagrindinę nelygybę ir taikome lemą:
    $$\sum_{cyc}{\frac{1}{2-a}}=\frac{3}{2}+\sum_{cyc}{\frac{a^2}{2a(a-2)}}\geq\frac{3}{2}+\sum_{cyc}{\frac{a^2}{2}}=3.$$
\item
    Pertvarkome ir du kartus taikome AM-GM bei nelygybę
    $ab+bc+ac\leq\frac{(a+b+c)^2}{3}$:
    \begin{eqnarray*}\sum_{cyc}{\frac{a^2}{a+2b^3}}&=&\sum_{cyc}{a-\frac{2b^3a}{a+2b^3}}\\
    &\geq&\sum_{cyc}{a-\frac{2b^3a}{3\sqrt[3]{ab^6}}}=\sum_{cyc}{a-\frac{2}{3}\sqrt[3]{b^3a^2}}\\&\geq&\sum_{cyc}{a-\frac{2}{9}(ab+ab+b)}\\
    &\geq&a+b+c-\frac{2}{27}\left(2(a+b+c)^2+3(a+b+c)\right)=1.\end{eqnarray*}
\end{enumerate} 
\subsection*{Funkcinės lygtys}
\subsubsection*{Įsistatykime $x=0$}
\begin{enumerate}
\item
    Įsistatykime $y=0$, gausime $f(x)=x^2$. Patikrinę matome, kad ši
    funkcija tinka.
\item
    Įsistatykime $y=0.$ Gausime, kad su visais $x$ turi būti $f(x)=1,$
    tačiau ši funkcija lygties netenkina. Sprendinių nėra.
\item
    Įsistatę $x=0$ gauname $f(y) = (y+1)f(0)$, t.y. vienintelės funkcijos
    kurios galėtų tikti yra $f(x)=c(x+1)$. Patikrinę gauname, kad tinka tik
    $c=0$, t.y. $f(x)=0$.
\item
    Įsistatykime vietoje $y$ bet kokį nelygų nuliui skaičių, pavyzdžiui
    $1$. Gausime $f(x)=f(1)x$, vadinasi, ieškomos funkcijos bus pavidalo
    $f(x)=cx$, kur $c$ reali konstanta. Patikrinę gauname, kad visos
    tokios funkcijos tinka.
\item
    Įsistatykime $y=-1$, gausime $f(x+f(-1))=0$. Kadangi $f(-1)$ yra
    konkretus skaičius, tai $x=f(-1)$ įgyja visas realias reikšmes, iš kur
    gauname, kad funkcija turi tenkinti $f(x)=0$. Patikrinę matome, kad
    šis sprendinys tinka.
\item
    Įsistatykime $x=-x.$ Gausime $-xf(-x) + f(x) + 1 = 0.$ Iš pradinės
    lygties išsireiškę $f(-x)$ ir įsistatę gausime
    $f(x)=-\frac{1+x}{1+x^2},$ kas ir yra sprendinys.
\item
    Įsistatę $x=\frac{x-1}{x}$ ir $x=\frac{1}{1-x}$ kartu su pradine turime
    tris lygtis, iš kurių paplušėję išsireiškiame $f(x)$. Gauname
    $f(x)=\frac{x^3-3x^2+2x-1}{2x-2x^2}$.
\item
    Įsistatę $x=1$ ir $z=1$ gauname $f(t) = tf(1)$, t.y. funkcija gali
    būti tiktai pavidalo $f(x)=ax, a\in \R$. Patikrinę matome, kad visos
    tokios funkcijos tinka.
\item
    Įsistatykime $y=0$ ir $y=1$. Iš gautų lygybių gauname, kad $f(x)$ -
    tiesinė funkcija ($f(x) = (f(1)-f(0))x +f(0)$). Patikrinę matome, kad
    funkcijos $f(x)=ax+b$ tinka su visais $a,b\in \R.$
\item
    Įstatę vietoje t bet kokią reikšmę, su kuria $f(t)\neq 0$ gauname, kad
    $f(x)$ - tiesinė funkcija. Patikrinę gauname, kad tinka tik $f(x) = x
    + 1$.
\item
    Taip: $$f(x) = \begin{cases}-x,& x<0, \\ x^3,& x\geq 0.\end{cases}$$
\item
    Įsistatę $x=y$ gauname $f(f(0)) = -x^2 - f(x)^2$. Įsistatę $x=y=0$
    gauname $f(f(0)) = -f(0)^2$. Iš šių dviejų lygybių gauname $f(0)^2 -
    x^2 = f(x)^2 \geq 0$, kas negalioja su visais $x \in \R$. Vadinasi
    funkcijų tenkinančių lygtį nėra.
\item
    Kadangi visiems realiesiems $a$, $b$ egzistuoja tokie $x$,$y$, kad
    $x+y=a$ ir $x-y=b$, tai lygtį galime užrašyti $b^2f(a)=a^2f(b)$. Jei
    egzistuoja toks $b_0$, kad $f(b_0) \neq 0$, tai jį įstatę vietoje $b$
    gauname $f(a)=ca^2$, kur $c$ - konstanta (jei neegzistuoja, tai
    $f(x)=0$). Įstatę gauname, kad $c$ tegali būti lygus $1$, vadinasi,
    sprendiniai yra $f(x)=x$ ir $f(x)=0$.
\item
    Įstatę $x=0$ ir įstatę $y=0$ gauname $f(f(x))=f(f(0)) +x$ ir
    $f(x+f(0))=f(f(x))$, iš kur $f(x+f(0))=f(f(0))+x\implies
    f(x)=x+c$.
\item
    Įsistatykime $x=y$ ir $y=x$ (t.y sukeiskime kintamuosius vietomis).
    Gausime $f(x+y)=3^x\cdot f(y)+2^y\cdot f(x)$. Atėmę iš šios lygybės
    pradinę ir įsistatę $y=1$ gausime $f(x) = 3^x-2^x$. Patikriname -
    tinka.
\item
    Rasti bent vieną funkciją nėra visai paprasta, tačiau kiek pamąstę
    matome, kad $f(x)=x+\frac{1}{x}$. Taip pat tiks ir
    $f(x)=x^2+\frac{1}{x^2}$ ir $f(x)=x^3+\frac{1}{x^3}$.
    
    Įstatykime $x=1$ ir $y=1$. Gausime $f(1)^2=2f(1)$. Kadangi funkcija
    įgyja tik teigiamas reikšmes, tai $f(1)=2$.
    
    Įstatykime $y=x$. Gausime, kad $f(x)^2 = f(x^2) + 2$. Kadangi su
    visais $x\in \R$ $f(x^2)\geq 0$, tai su visais $x \in \R$ $f(x)\geq
    \sqrt{2}$. Dabar, kadangi su visais $x\in \R$ $f(x^2)\geq \sqrt{2}$,
    tai su visais $x \in \R$ $f(x)\geq \sqrt{2+\sqrt{2}}$. Taip tęsdami,
    gauname, kad su kiekvienu $x$ ir kiekvienu $n\in \N$ $$f(x)\geq
    \underbrace{\sqrt{2+\sqrt{2+\cdots +\sqrt{2+\sqrt{2}}}}}_{n}.$$
    
    Kadangi, kai $n$ artėja į begalybę, $\underbrace{\sqrt{2+\sqrt{2+\cdots
    +\sqrt{2+\sqrt{2}}}}}_{n}$ artėja į $2$ (seka akivaizdžiai didėjanti,
    mažesnė už $2$ $\Rightarrow$ turi ribą. Ją randame išsprendę
    $\sqrt{2+x}=x \Rightarrow x=2$), tai $f(x) \geq 2$.  \\
    c.) dalyje užtenka pasinaudoti įsistačius $f(x)^2 = f(x^2) + 2$,
    ir norint įsitikinti, kad $f^2(x)-2\geq 0$ - b.) dalyje gauta nelygybe.
\item
    Atkreipsime dėmesį, kad jei nebūtų lygties apribojimo vien teigiamiems
    skaičiams, tai įsistatę $y = 0$ iš karto gautume, kad $f(x)=c$. Tačiau
    apribojimas yra, todėl suktis teks kiek kitaip.
    
    Fiksuokime sumą ir žiūrėkime, kaip kinta sandauga. T.y. įsistatykime
    pvz., $y=2-x$. Gausime $f(x(2-x))=f(2)$. Kadangi galime statyti tik
    teigiamas reikšmes, tai ši lygybė yra teisinga tik, kai $x\in (0,2)$.
    Šiame intervale, kintant $x$ reikšmei, reiškinio $x(2-x)$ reikšmė kinta
    nuo $0$ iki $1$, t.y. $x(2-x)\in(0,1]$. Tad gauname, kad $f(x)$ yra
    pastovi intervale $(0,1]$. Lieka pastebėti, kad ji periodinė: įstatę
    $y=1$ gausime $f(x)=f(x+1)$, todėl pastovi ir visur.
\item
    Fiksuokime sumą. Tegu $x=2-y$. Tuomet $f(2)=f(\frac{2}{x(2-x)})$.
    Kai $x$ kinta nuo $-\infty$ iki $\infty$ reiškinys
    $\frac{2}{x(2-x)}$ kinta intervaluose
    $(-\infty,0)\cup[\frac{1}{2},\infty)$, kartu ir funkcija tuose
    intervaluose pastovi. Likusią dalį $(0,\frac{1}{2})$ galime
    prijungti naudodami $f(x+\frac{1}{2}) = f(\frac{1}{x}+2)$.
    Pastebėsime, kad funkcijos reikšmė taške 0 taip ir lieka
    neapibrėžta. Atsakymas $$f(x) = \begin{cases}a,& x\neq0 \\ b,& x= 0
    \text{      \ \ \ \ \   } a,b\in\R\end{cases}$$
\item
    Atsikratykime penketo: įsistatykime $f(x) = g(x) -\frac{5}{2}$.
    Gausime, kad $g$ tenkina lygtį $g(2x+1)=g(x)$. Pabandę pirmas keletą
    reikšmių gausime, kad $$g(1)=g(0), g(3)=3g(0), g(7)=3^2g(0), g(15) =
    3^3g(0).$$ Įsižiūrėję pamatysime, kad taip tęsdami gausime
    $$g(2^n-1)=3^{n-1}g(0).$$ Pažymėję $2^n-1 = x$ gauname $n =
    \log_2(x+1)$ arba $g(x)=3^{\log_2\frac{x+1}{2}}\cdot g(0).$ Iš
    $f(0)=0$ seka, kad $g(0)=\frac{5}{2}$ ir susitvarkę su neigiamų
    skaičių keliamais nepatogumais gauname, kad  $$f(x) =
    3^{\log_2\frac{|x+1|}{2}}\cdot \frac{5}{2} - \frac{5}{2}$$ tenkina
    lygtį.
\item
    Įsistatykime $y=0$ ir $y=1$:
    \begin{eqnarray*}
      f(x^3) &=& (x^2 + x + 1)(f(x)-f(1))+f(1),\\
      f(x^3) &=& x^2 (f(x)-f(0)) + f(0).
    \end{eqnarray*}
    Lygybės teisingos su visomis $x$ reikšmėmis, tad sulyginę dešiniąsias
    puses gausime $$f(x)=xf(1) + (1-x)f(0),$$ t.y. funkcija tiesinė.
    Patikrinę matome, kad lygtį tenkina visos funkcijos $f(x)=ax+b$, kur
    $a,b\in \R.$
\item
    Šis uždavinys, nors ir paprastas, yra gerai žinomi spąstai. Iš pirmo
    žvilgsnio padaryta išvada, kad sprendinai yra tik $f(x)=1$ ir
    $f(x)=-1$ nėra teisinga. Atidžiau pažvelgus tampa aišku, kad viskas, ką
    galima pasakyti apie funkciją, yra tai, kad bet kuriame taške ji įgyja
    reikšmę $1$ arba $-1$. Užrašius tą matematiškiau, sprendiniai atrodo
    kaip $$f(x) = \begin{cases}1,& x\in A, \\ -1,& x\not\in A,\end{cases}$$
    kur $A$ bet koks $\R$ poaibis.
\end{enumerate} 
\subsubsection*{Funkcijų tipai}
\begin{enumerate}
\item
    Jei funkcija griežtai didėjanti, tai visiems skirtingiems $a>b$
    turėsime $f(a)>f(b)$, todėl funkcija neįgis vienodų reikšmių. 
    
    Bijektyvi funkcija nebūtinai turi būti monotoniška. Pavyzdžiui,
    $f(x)=\frac{1}{x}$, kai $x\neq 0$, ir $f(0)=0$.
\item
    Negali, nes, pavyzdžiui, įstatę $x=0$ matome, kad $f(y)=f(-y)$ su visais $y$.
\item
    Įstatę $x=-x$ ir $y=-y$ gauname $f(x+y) = -f(-x-y)\Rightarrow
    f(t)=-f(-t)$ $\forall t \in \R.$
\item
    Lyginės monotoninės yra tik $f(x)=c$, $f:A\to \R$, lyginė injektyvi
    tik $f(0)=c$, $f:\{0\}\to \R$ (jei dar bent viena taškų pora
    priklausytų apibrėžimo sričiai iš karto gautume neinjektyvią). Lyginės
    surjektyvios pavyzdys gali būti $f(x)=ln|x|$, kai $x\neq 0$, $f(0)=0$.
\item
    Tegu $a>b$. Įstatę $x =b$, $y = a-b$ gausime $f(a)f^2(a-b)=f(b)$.
    Kadangi $f(a-b)^2\leq 1$, tai $f(a)\geq f(b)$.
\item
    Jei žinome, kad funkcija yra surjektyvi, tai egzistuoja toks $a$, kad
    $f(a)=1$. Įstatę gauname $f(a\cdot 1) =a \Rightarrow 1 = a$, vadinasi
    $f(1)=1$.
\item
    Įrodysime, kad $f(x)=x$. Tarkime priešingai - tegu egzistuoja toks
    $a$, kad $f(a)>a$. Tuomet, kad kadangi $f$ yra didėjanti, tai $$f(a)>a
    \Rightarrow f(f(a))>f(a) \Rightarrow f(f(a))>f(a)>a \Rightarrow
    f(f(a))\neq a \text { - prieštara}.$$ Tardami, kad $f(a)<a$, prieštarą
    gauname analogiškai.
\item
    Įstatę $x=0$ ir $x=1$ gauname, kad $f(a+b)=f(b)$, todėl pasinaudoję injektyvumu gauname $a = 0$.
    Įstatę $x=b$ gauname $f(b)f(1-b)=f(b)$, tad arba $f(1-b)=1$, arba
    $f(b)=0$. Tačiau $f(b)$ negali būti lygus nuliui, nes gautume
    $f(x)f(1-x)=0$, o iš čia be galo daug reikšmių, su kuriomis funkcija
    lygi nuliui, kas prieštarauja injektyvumui.
    Galiausiai pastebėkime, kad funkcija nulio iš vis neįgyja, nes jei,
    tarkime, $f(c)=0$, tai įstatę gauname $f(b)=0$, ko negali būti. Taigi
    ji nėra surjektyvi.
\item
    Įsistatykime $x=y$, $y=x$, gausime $f(x+f(y))=f(y+f(x))$. Kadangi $f$
    yra griežtai didėjanti, tai ji injektyvi, tai $x+f(y)=y+f(x)
    \Rightarrow f(x)=x+c$. Įstatę randame $c=2005$.
\item
    Įsistatykime $x=y$, gausime $(y+y)(f(y)y)=y^2f(f(y)+f(y))$. Jei
    $f(x)=f(y)$, tai iš abiejų lygybių gauname $\frac{x^2}{x+y} =
    \frac{y^2}{y+y} \Rightarrow x=y$. Gavome kad funkcija injektyvi.
    Įstatykime $y=1$ ir $x=\frac{1+\sqrt{5}}{2}=\lambda$, t.y. lygties
    $x+1=x^2$ sprendinį. Tuomet gausime, kad
    $f(f(\lambda))=f(f(1)+f(\lambda))\Rightarrow f(1)=0$, o taip būti
    negali.
\item
    Nesunku pastebėti, kad funkcija yra injektyvi. Įsistatę $x=1,y=1$
    gauname $f(f(1))=f(1) \implies f(1)=1$. Įsistatę $x=1$ gauname
    $(y+1)f(y)=f(y)+1 \implies f(y)=\frac{1}{y}.$
\item
    Kadangi $g$ yra surjektyvi ir $f(y)+x$ įgyja visas realiąsias reikšmes,
    tai iš lygybės gauname, kad ir $f$ surjektyvi. Tegu $a$ toks, kad
    $g(a) = 0$. Įstatę $x=a$ gauname $f(y)=g(f(y)+a)$. Kadangi $f$
    surjektyvi, tai $g(x)=x-a$ su visas $x\in \R$. Įstatę $g$ išraišką į
    pradinę lygybę gauname, kad $f(x+y-a)=f(y)+x-a$. Įstatę $y=a$ gauname
    $f(x)=x+b$. Vadinasi, sprendiniai yra $g(x) =x+a$, $f(x)=x+b$, kur $a,b
    \in \R$.
\item
    Įrodykime, kad $f$ injektyvi. Naudodami keitinį $x+y =a$, $xy =b$
    gauname lygtį $f(a+f(b))=f(f(a)) + b$.  Tačiau ji galioja ne visiems
    $a$ ir $b$, o tik tenkinantiems sąlygą $4b\leq a^2$, nes kitaip
    sistema $x+y =a$, $xy =b$ neturi sprendinių.  Bet tai ne bėda -
    kiekvieniem $b_1$ ir $b_2$ galime paimti $a$ tokį, kad $4b_1 \leq a^2$
    ir $4b_2 \leq a^2$. Tuomet galime naudotis lygtimi ir iš
    $f(b_1)=f(b_2)$ gauname $b_1 = b_2$ - injektyvumas įrodytas.
    Įstatykime į pradinę lygtį $y=0$, gausime $f(x + f(0))=f(f(x))$, iš
    injektyvumo $f(x)=x+c$.
\item
    Iš lygybės $g(f(x))=x^3$ seka, kad $f$ yra injektyvi ir kad
    $f(g(f(x)))=f(x^3)\Rightarrow f^2(x)=f(x^3)$.  Įsistatę $x=-1,0,1$
    gauname, kad $f(-1)$, $f(0)$ ir $f(1)$ gali įgyti tik reikšmes $0$
    arba $1$, kas prieštarauja injektyvumui.
\item
    Pastebėkime, kad $f$ injektyvi. Įstatykime $x = 0$, gausime
    $f(f(0))=\frac{f(0)}{2}$.  Įstatykime $x = f(0)$, gausime $4f(f(f(0)))
    = 2f(f(0)) + f(0) = f(0) + f(0) = 2f(0)$, iš kur
    $f(f(f(0)))=\frac{f(0)}{2} = f(f(0))$. Naudodamiesi injektyvumu
    gauname $$f(f(f(0)))=f(f(0)) \Rightarrow f(f(0)) = f(0) \Rightarrow
    f(0) = 0.$$ Kadangi funkcija injektyvi, tai išties $f(x)=0
    \Leftrightarrow x=0$.
\item
    Patyrinėkime keitinį $y=\frac{x}{f(x)-1}$. Iš pradžių gali pasirodyti,
    kad jis yra nuleistas iš dangaus, bet viskas daug paparasčiau - jis
    tiesiog kyla iš natūralaus noro sulyginti $f(yf(x))$ ir $f(x+y)$
    ($yf(x)=x+y\implies y=\frac{x}{f(x)-1}$). Tačiau prisiminkime, kad
    funkcijos apibrėžimo sritis yra teigiami skaičiai. Tuomet tenka
    samprotauti taip: jei egzistuoja toks $x$, kad $f(x)>1$, tai galime
    įsistatyti $y=\frac{x}{f(x)-1}$ ir gausime
    $f(x)f(\frac{xf(x)}{f(x)-1})=f(\frac{xf(x)}{f(x)-1}) \Rightarrow
    f(x)=1$ - prieštara!
    
    Vadinasi gavome, kad $f(x)\leq 1$ ir, iš pradinės lygybės, $f$ yra
    nedidėjanti ($f(x+y)=f(x)f(yf(x))\leq f(x)$).
    
    Nagrinėkime injektyvumą: Jei egzistuoja tokie $a < b$, kad
    $f(a)=f(b)$, tai gauname, kad $f(a+y)=f(b+y)$ su visais $y$, todėl
    $f(y)=f(b-a+y)$ su visais $y>a$, vadinasi, funkcija yra monotoniška ir
    periodinė $\Rightarrow f(x)= c$ su visais $x >a$. Įsistatę į pradinę
    lygtį pakankamai didelius $x$ ir $y$ gauname $c = 1$, o įsistatę $x$
    pakankamai didelį gauname $f(y)=1$ su visais $y$.
    Lieka atvejis, kai funkcija yra injektyvi. Pakeitę $y=\frac{z}{f(x)}$
    gausime $f(x)f(z)=f(x+\frac{z}{f(x)})$ su visais $z,x >0$. Sukeitę $x$
    ir $z$ vietomis bei pasinaudoję injektyvumu gauname $x +
    \frac{z}{f(x)} = z + \frac{x}{f(z)}$, iš kur lengvai randame
    $f(x)=\frac{1}{1+cx}$, kur $c\in \R^+$.
\item
    Tegu $f(x_0)=1$, tada įsistatę $x=x_0$ gauname $f(x_0+y)=f(y)$,
    vadinasi, funkcija yra periodinė ir vienetą įgis be galo daug kartų, o
    to būti negali, vadinasi, $f(x)\neq 1, \forall x\in \R^+$.
    
    Tegu $f(a)=f(a+b)$, tuomet įsistatykime $x=a, y=\frac{b}{f(a)}$,
    gausime $1=f(\frac{b}{f(a)})$, prieštara, vadinasi funkcija injektyvi.
    
    Pradinėje lygtyje įstatykime $x=y$, $y=x$, gausime
    $f(x+yf(x))=f(x)f(y) =f(y+xf(y))$. Kadangi $f$ injektyvi, tai
    $x+yf(x)=y+xf(y) \rightarrow f(x)=kx+1$. Patikrinę matome, kad tinka.
\item
    Pastebėkime, kad $f$ injektyvi.  Įstatę $x=0$, $y=0$ ir pažymėję
    $f(0)+f(f(0))=u$ gauname $f(u)=u$.  Įstatę $y=u$ gauname
    $f(x+u)=f(f(x)+u) \Rightarrow f(x)+u=x+u \Rightarrow f(x)=x$.
\item
    Funkcija akivaizdžiai bijektyvi, todėl egzistuoja toks $x_0$, kad
    $f(x_0)=0$. Įsistatę $x=x_0$ gauname $f(f(y))=y$.  Įsistatę $x=f(x)$
    gauname $f(x^2+f(y))=xf(x)+y=f(f^2(x)+f(y))$. Kadangi $f$ injektyvi,
    tai $f^2(x)=f(x)^2$. Vadinasi, kiekviename taške $x$ funkcija lygi arba
    $x$, arba $-x$. Tegu egzistuoja du nenuliniai taškai, kuriuose $f(x)=x$
    ir $f(y)=-y$. Tuomet gauname $f(x^2+y)=x^2-y$, kas yra neįmanoma
    ($x^2+y = x^2-y\implies y=0$, $-x^2-y = x^2-y\implies x=0$ ).
    Vadinasi, tinka tik $f(x)=x$ ir $f(x)=-x$.
\item
    Funkcija bijektyvi. Įstatykime $x=0$ ir $x=a$, kur $a$ toks, kad
    $f(a)=0$. Gausime lygybes $f(f(y))=f^2(0) + y$ ir $f(f(y))=y$, iš kur
    $f(0) = 0$ ir $a=0$.
    
    Įstatykime $x = f(x)$ ir pasinaudokime lygybe $f(f(x))=x$. Gausime
    $f(f(x)x + f(y))=x^2 + y$, vadinasi $f^2(x)=x^2$ su visais $x$.
    
    Tegu $x$ ir $y$ tokie, kad $f(x)=x$ ir $f(y)=y$ ir $x$,$y$ $\neq 0$.
    Tada iš pradinės lygties gauname $f(x^2 - y) = x^2 + y$. Kadangi
    $f(x^2 - y)$ gali būti lygus tik $x^2 -y$ arba $y-x^2$, tai gauname,
    kad arba $y=0$ arba $x=0$ - prieštara. Vadinasi, sprendiniai yra
    $f(x)=x$ ir $f(x)=-x$.
\item
    Įstatę $y=z=0$ gauname $f(h(g(x)))=x+h(0)$. Įstatę $y=0$ gauname
    $g(z+f(0))=g(f(z))$. Kadangi $g$ injektyvi (atkreipkite dėmesį į
    kintamąjį $x$ pradinėje lygtyje), tai $f(x)=x+a$.
    
    Įsistatę gauname lygtį $h(g(x)) + y + a = h(y) + x$, iš kurios
    akivaizdžiai $h(x)=x+b$, ir $g(x)=x-a$.
\item
    Funkcija injektyvi. Raskime $f(0)$: $x=0 \Rightarrow f(f(y))=y +
    f^2(0) \Rightarrow f(f(0))=f^2(0)$. Pažymėkime $f(0)=a$, tuomet
    paskutinioji lygybė pavirsta į $f(a)=a^2$. Įstatykime $x=0$, $y=a$ ir
    $x=a$, $y=0$. Gausime $f(a^2)=a^2+a$ ir $f(a^2+a)=a^4+a^2$. Iš čia
    $f(f(a^2))=f(a^2+a)\implies 2a^2 = a^4 + a^2 \implies a=-1$,$0$
    arba $1$.
    
    Jei $f(0)=1$, tai tuomet iš $f(f(y))=y + f^2(0)$ gauname $f(1)=1$ -
    prieštara injektyvumui.
    
    Jei $f(0)=-1$, tai iš $f(f(y))=y + f^2(0)$ gauname $f(-1)=1
    \Rightarrow f(1)=0 \Rightarrow f(0)=2$ - prieštara.
    
    Vadinasi, $f(0)=0$. Tuomet $f(f(x))=x$ ir $f(x^2)=f^2(x)$. Įstatę
    $x=-y$ gauname $f(f(y))=f^2(-y) -yf(y) + y \Rightarrow y = f((-y)^2) -
    yf(y) + y\Rightarrow f^(y)=yf(y)$. Kadangi funkcija injektyvi, tai
    $f(y)=0$ tik kai $y=0 \Rightarrow f(y)=y$.
\item
    $f(x)=0$ yra sprendinys, ieškosime likusių. Įrodykime, kad $f$ turi
    būti lyginė. Pastebėkime, kad $f(f(x)-f(y))=f(f(y)-f(x))$, todėl
    užtenka įrodyti, kad $f(x)-f(y)$ įgyja visas reikšmes. Išties, tegu
    $a$ toks, kad $f(a)\neq 0$. Įstatykime $y= a-x \implies
    f(f(x)-f(a-x))=(2x-a)^2f(a)$. Iš čia matome, kad $f$ įgyja visas
    teigiamas arba visas neigiamas reikšmes (priklausomai nuo $f(a)$),
    vadinasi, $f(x)-f(y)$ tikrai įgyja visas realiąsias reikšmes.
    
    Įstatykime $y=-y$. Gausime $f(f(x)-f(y))=(x+y)^2f(x-y) \implies
    (x-y)^2f(x+y)=(x+y)^2f(x-y)$. Kadangi visiems realiesiems $a$, $b$
    egzistuoja tokie $x$,$y$, kad $x+y=a$ ir $x-y=b$, tai lygtį galime
    užrašyti $b^2f(a)=a^2f(b) \implies f(x)=cx^2$. Patikrinę gauname
    $c=1$, vadinasi, sprendiniai yra $f(x)=x$ ir $f(x)=0$.
\item
    Įstatykime $x = 0$, gausime $f(0)+y=f(g(y))$, vadinasi $f$ surjektyvi,
    $g$ injektyvi.
    
    Įrodykime, kad $g(1)=1$. Įstatykime $y = 0$, gausime $f(xg(1)) = xf(0)
    + f(x + g(0))$. Jei $g(1)\neq 1$, tai galime sulyginti $xg(1)=x+g(0)$
    paėmę $x = \frac{g(0)}{g(1)-1}$. Tuomet gauname
    $\frac{g(0)f(0)}{g(1)-1}=0 \Rightarrow f(0)=g(0)=0$ (pasinaudojus
    antrąja sąlyga). Įsistatę $y = -1$ gauname $f(x)=ax$ ir
    $g(x)=\frac{x}{a}$. Patikrinę gauname, kad $a =1$, taigi $f(x)=g(x)=x$
    - prieštara prielaidai $g(1)\neq 1$.
    
    Iš $f$ surjektyvumo žinome, kad egzistuoja toks $u$, kad $g(u)=0$.
    Įrodykime, kad $u = 0$.  Tegu $u \neq 0$, tada $g(u+1)\neq g(1)=1$ (iš
    $g$ injektyvumo).  Įstatykime $x = \frac{g(u)}{g(u+1)-1}$ ir $y = u$,
    gausime $u = 0$ - prieštara.
    Taigi gavome, kad $f(0)=0$ ir $g(0)=0$, ir iš čia jau žinome, kad
    gaunasi $f(x)=g(x)=x$.
\item
    Įstatykime $x = 0$. Gausime $f(f(y))=y$. Įstatykime $x=f(x)$, gausime
    $f(f^2(x) + f(y)) = y + f(x)x = f(x^2 + f(y))$. Kadangi funkcija
    tenkinanti lygtį yra akivaizdžiai bijektyvi, tai gauname
    $f^2(x)=x^2 \Rightarrow f(x)=\pm x$.
    
    Tegu $x$ ir $y$ tokie, kad $f(x)=x$ ir $f(y)=-y$ bei $x$,$y$ $\neq 0$.
    Tada pradinė lygtis tampa $f(x^2 - y)=y+x^2$. Kadangi $f(x^2-y)$ =
    $x^2 - y$ arba $f(x^2-y) = y -x^2$, tai gauname $y=0$ arba $x = 0$ -
    prieštara. Vadinasi, sprendiniai yra tik $f(x)=x$ ir $f(x)=-x$.
\item
    Funkcija bijektyvi, todėl egzistuoja toks $a$, kad $f(a)=0$. Įsistatę
    $x=y=a$ gauname $f(a^2)=a \Rightarrow f(f(a^2))=0$. Tačiau kadangi
    $f(f(y))=y + f^2(0)$, tai $a^2+f^2(0)=0 \Rightarrow a=0$ ir $f(0)=0$.
    
    Tuomet iš pradinės lygties gauname, kad $f(x^2)=f^2(x)=f(-x)^2$. Dėl
    injektyvumo $f(x)\neq f(-x)$, todėl $f(x)=-f(-x)$. Iš čia ir iš
    $f(x^2)=f^2(x)$ gauname, kad $f(x)>0$, kai $x>0$ ir $f(x)<0$, kai
    $x<0$.
    
    Galiausiai įstatę $y=-x^2$ gauname, kad $f(x^2-f^2(x))=-(x^2-f^2(x))
    \Rightarrow f^2(x)=x^2 \Rightarrow f(x)=x$.
\item
    $f(x)=0$ yra sprendinys, nagrinėkime galimus likusius. Tegu $f(a)=0$,
    tuomet įstatę $x=a$ gausime $0=af(y) \implies a=0$, vadinasi, jei
    $0$ yra įgyjamas, tai tik taške $0$. Įstatę $x=y=-1$ gausime
    $f(f(1)-1)=0 \implies f(0)=0, f(1)=1$.
    
    Įstatę $x=1$ gauname $f(f(y)+1)=f(y)+1(*)$ iš kur $f(n)=n$ visiems
    $n\in \N$. Įrodysime, kad $f$-injektyvi. Pažymėję $xy=a$ gauname
    $f(x+f(a))=f(x)+xf(\frac{a}{x})$. Jei $f(a)=f(b)$, tai visiems $x$
    teisinga $f(\frac{a}{x}) = f(\frac{b}{x})$. Pakeitę $x=\frac{b}{y}$ ir pažymėję
    $\frac{a}{b}=m$, gausime, kad su visomis $y$ reikšmėmis $f(ym)=f(y)$. Iš čia
    randame $f(m)=1$. Įstatę $x=m$ gauname $f(m + f(y))=1+mf(y)$, iš kur
    $f(m+1)=m+1$ $(y=1)$ ir $f(m+2)=1+2m$ $(y=2)$. Tačiau pagal (*)
    $f(m+2)=m+2$ $(y=m+1)$, todėl $1+2m = m+2 \implies m=1 \implies
    a=b \implies f$ injektyvi.
    
    Įstatykime $x=y=-2 \implies f(-2)=-2 \implies f(-1)=-1$.
    Įstatykime $x=-1 \implies f(-1+f(-y))=-1-f(y)$.
    
    Naudodamiesi $f(-1+f(-y))=-1-f(y)$ ir $f(f(y)+1)=f(y)+1$ gausime, kad
    kiekvienam $x$ egzistuoja toks $y$, kad $f(x)=f(y)+1$. Išties: jei
    $a$ priklauso $f$ vaizdui $\implies -1 -a$ priklauso vaizdui
    $\implies -a$ priklauso vaizdui $\implies -1 + a$ priklauso
    vaizdui. Kadangi kiekvienam $x$ $f(x)$ priklauso vaizdui, tai $f(x)-1$
    priklauso vaizdui, todėl egzistuoja toks $y$, kad $f(y)=f(x)-1$.
    Įstatę į $f(f(y)+1)=f(y)+1$ gauname, kad kiekvienam $x$
    $f(f(x))=f(x)$. Kadangi $f$ injektyvi, tai $f(x)=x$.
\item
    Pastebėkime, kad $f(0)=0$ ir $f(xf(x))=x^2$(*). Įstatę $x=1$ gauname
    $f(f(1))=1$, įstatę $x=f(1)$ gauname $1=f(1)^2$. Jei $f(1)=1$, tai
    $f(x)+f(f(x))=2x$ - $f$ injektyvi. Jei $f(1)=-1$, tai įstatę $x=y=1$
    gauname $f(-1)=1$ ir įstatę $y=-1$ gauname $f(x)+f(-f(x))=-2x$ - $f$
    injektyvi.
    
    Įrodysime, kad $f(\frac{1}{x})= \frac{1}{f(x)}$. Įstatykime
    $y=f(\frac{1}{x})\frac{1}{x}$: $$f(xf(f(\frac{1}{x})\frac{1}{x}))+
    f(f(\frac{1}{x})\frac{1}{x}f(x))=2f(\frac{1}{x}).$$
    Iš (*) gauname, kad $$f(f(\frac{1}{x})\frac{1}{x}) = \frac{1}{x^2},$$
    todėl lygybę galime perrašyti į
    $$f(f(\frac{1}{x})\frac{1}{x}f(x))=f(\frac{1}{x}).$$ Lieka pasinaudoti
    injektyvumu ir gauname $f(x)=\frac{1}{x}$.
    
    Jei $f(1)=1$, tai įstatę $x=\frac{1}{x}$ į $f(x)+f(f(x))=2x$ gauname
    $$\frac{1}{f(x)} + \frac{1}{f(f(x))}= \frac{2}{x} \Rightarrow \\
    \frac{1}{f(x)} + \frac{1}{2x - f(x)}= \frac{2}{x} \\ \Rightarrow
    (f(x)-x)^2=0 \Rightarrow f(x)=x.$$
    Jei $f(-1)=-1$, tai $x = \frac{1}{x}$ statome į $f(x)+f(-f(x))=-2x$ ir
    analogiškai gauname $f(x) = -x$.
%noparse
\end{enumerate} 
\subsubsection*{Cauchy funkcinė lygtis}
\begin{enumerate}
\item
    Pasižymėkime $f(x)=g(x)+x^2$. Gausime $g(x+y)=g(x)+g(y)$. Tada
    $g(x)=kx$, ir $f(x)=kx+x^2$. Nesunku patikrinti, kad sprendinys tiks.
\item
    Pasižymėkime $f(x)=g(x)+1$. Gausime $g(x+1+g(y))=g(x+1)+y$, arba
    $g(t+g(y))=g(t)+y$. Iš čia nesunku įsitikinti, kad funkcija bijektyvi.
    Įstatę $t=y=0$, gausime $g(0)=0$, o paskui įstatę $t=0$ - $g(g(y))=y$.
    Tada prieš tai gautoje lygtyje pakeitę $y=g(y)$, gausime Cauchy
    funkcinę lygtį, iš kur $g(x)=kx$. Nesunku patikrinti, kad tiks tik
    $k=1$ arba -1. Randame sprendinius $f(x)=x+1$ arba $f(x)=1-x$.
\item
    Statykime $x=y=0$. Gausime $h(0)=f(0)-g(0)$. Paimkime pradinėje
    lygtyje $y=0$. Tada turėsime $g(x)=f(x)-h(0)=f(x)+g(0)-f(0)$. Paimkime
    pradinėje lygtyje $x=0$. Gausime $h(y)=f(y)-g(0)$. Įstatę gautas
    $g(x)$ ir $h(y)$ išraiškas į pradinę lygtį gausime:
    $f(x+y)=f(x)+f(y)-f(0)$. Įsivedę keitinį $i(x)=f(x)-f(0)$, gausime,
    kad $i$ tenkina Cauchy funkcinę lygtį ir yra tolydi vadinasi
    $i(x)=kx$. Tada, jei pažymėsime $f(0)=a$ ir $g(0)=b$, gausime
    $f(x)=kx+a$, $g(x)=kx+b-a$, $h(x)=kx-b$
    
    Įstatę į pradinę lygtį, gausime, kad $a=0$, o $k$ ir $b$ - bet kokios
    realiosios konstantos.
\item
    Pasižymėkime $f(x)=g(x)+1$. Tada pradinė lygtis virs
    $g(xy)+g(x+y)=g(x)g(y)+g(x)+g(y)$. Įsistatę $x=y=0$ gausime $g(0)=0$.
    Tada, pažymėję $g(1)=k$, po nesudėtingos indukcijos gausime
    $$g(n)=k^{n}+k^{n-1}+...+k, \forall \, n\in\N.$$ Jei $g(1)=1$, tai gausime
    $g(n)=n$, kitu atveju $g(n)=\frac{k^{n+1}-1}{k-1}-1$. Įstatę į
    prieš tai turėtą lygtį ir išprastinę gausime
    $$k^{xy+2}-k^{xy+1}-k^{x+y+1}=k^{2}-k^{x+1}-k^{y+1}.$$ Čia galime
    statyti bet kokius naturaliuosius $x$ ir $y$. Tą darydami, nesunkiai
    gausime $k=1,0,-1$. Kai $k=0$ gausime sprendinį $f(x)=1$. Kai $k=-1$,
    nesunkiai gausime prieštarą. Kai $k=1$, pradinėje lygtyje įstatę $x=1,
    y=-1$ gausime $g(-1)=-1$. Tada pradinėje lygtyje paėmę $y=-1$, o
    paskui $x=-x, y=1$ ir sudėję abi gautas lygybes gausime $-g(x)=g(-x)$
    visiems $x\in\R$. Tada pradinėje lygybėje paėmę $x=-x,y=-y$ ir
    pritaikę paskutiniąją lygybę gausime:
    $g(xy)-g(x+y)=g(x)g(y)-g(x)-g(y)$. Sudėję su pradine lygybę
    gausime $g(x+y)=g(x)+g(y)$ ir $g(xy)=g(x)g(y)$, iš ko, kaip jau matėme
    pavyzdyje, gausime $g(x)=x$. Taigi, šios lygties sprendiniai yra
    $f(x)=x+1$ ir $f(x)=1$.
\item
    Nesunku atspėti, kad $f(x)=x^2$ yra lygties sprendinys. Iš čia kyla
    idėja įsivesti keitinį $f(x)=g(x^2)$, visiems $x\geq 0$. Gausime
    $g((x^2-y^2)^2+(2xy)^2)=g((x^2-y^2)^2)+g((2xy)^2)$. Kita vertus, jei
    pažymėsime $a=x^2-y^2$ ir $b=2xy$, tai nesunku įsitikinti, kad lygčių
    sistema
    $$\left\{\begin{array}{cc}
    a&=x^2-y^2\\
    b&=2xy\\
    \end{array}  \right.$$ visados turės sprendinių, kad ir kokius $a$ ir
    $b$ pasirinktume (tiesiog išsireikštume iš antros lygties $x$,
    įstatytume į pirmą ir gautume kvadratinę lygtį $y^2$ atžvilgiu, kurios
    diskriminantas tikrai teigiamas). Tada gautą funkcinę lygtį galime
    pasikeisti į $g(a^2+b^2)=g(a^2)+g(b^2)$ , arba į $g(z+t)=g(z)+g(t)$,
    kur $z$ ir $t$ bet kokie neneigiami realieji. Kadangi turime
    $f:\R\rightarrow[0,+\infty)$, funkcija $g$ bus aprėžta iš apačios ir
    galime teigti, kad $g(x)=kx$ visiems neneigiamiems $x$ (nors funkcija
    Cauchy lygtį tenkina tik neneigiamiems skaičiams, nesunku įsitikinti,
    kad aprėžtumo vistiek užteks). Tada $f(x)=kx^2$ visiems teigiamiems
    $x$, bet pradinėje lygtyje paėmę $y=0$, gausim $f(0)=0$, o tada vėl
    pradinėje lygtyje paėmę $x=0$ gautume $f(y)=f(-y)$ visiems $y$, taigi
    $f(x)=kx^2$ ir neigiamiems $x$.
\item
    Nesunku įsitikinti, kad funkcija bijektyvi. Įstačius $x=0$, $y=x^n$,
    gausime: $f(f(x^n))=x^n+f(0)^n$. Iš bijektyvumo aišku, kad egzistuoja
    toks $t$, kad  $f(t)=0$ ir $z$, kad  $f(z)=t$. Tada įsistatę pradinėje
    lygtyje $y=t$ gausime: $f(x^n)=f(x)^n+t$. Panaudoję tai
    ankščiau gautoje lygtyje gauname $f(f(x)^n+t)=x^n+f(0)^n$. Dabar pradinėje
    lygtyje pakeitę $x$ į $f(x)$ ir $y$ į $z$, gausime, kad
    $f(f(x)^n+t)=f(f(x))^n+z$ ir, sulyginę tai su prieš tai gauta lygtimi,
    gausime $f(f(x))^n+z=x^n+f(0)^n$. Galiausiai pagrindinėje lygtyje
    paėmę $x=0$ ir $y=x$, gausime $f(f(x))=x+f(0)^n$. Šią $f(f(x))$
    išraišką įstatę į prieš tai gautą lygtį gauname:
    $(x+f(0)^n)^n+z=x^n+f(0)^n$ visiems $x$, iš kur lengvai gauname
    $f(0)=t=z=0$. Tai įstatę į prieš tai turėtas lygtis gausime $f(f(x))=x$
    ir $f(x^n)=f(x)^n$. Tada pradinėje lygtyje pakeitę $y$ į $f(y)$
    turėsime $f(x^n+y)=f(x^n)+f(y)$, kas jau labai panašu į Cauchy
    funkcinę lygtį.  Lyginiams $n$ $f(x+y)=f(x)+f(y)$, kur $x$ teigiamas, o
    $y$ - betkoks. Tada paėmę $y=-x$ gauname, kad $f(-x)=-f(x)$ $ \forall
    x\geq 0$ ir taip $f(x+y)=f(x)+f(y)$ bet kokiems realiesiems $x,y$.
    Tačiau ankščiau turėjome $f(x^n)=f(x)^n$, taigi $f(x)\geq 0$ visiems
    $x\geq 0$ ir funkcija yra aprėžta intervale, vadinasi, - pavidalo $kx$.
    Patikrinę pradinėje lygtyje gauname, kad tiks tik $k=1$, taigi kai $n$
    - lyginis gauname sprendinį $f(x)=x$.
    
    Kai $n$ - nelyginis, tai iškart gauname, kad $f(x+y)=f(x)+f(y)$ bet
    kokiems realiesiems $x,y$. Be to, turėjome, kad $f(x^n)=f(x)^n$, tada
    $f(1)=f(1)^n$ ir $f(1)=1,-1$ ($0$ netiks, nes $f$ - injektyvi). Tada
    gauname du atvejus: $f(p)=p$ arba $-p$ $\forall p\in\Z$ ir abiem
    atvejais galios $f(px)=pf(x)$. Pažymėkime $b_{k}=f(x^k)$,
    $k=2,3,...,n$ ir $q=f(x)$. Iš ankščiau gauto rezultato galios
    $f((x+p)^n)=(f(x+p))^n=(f(x)+f(p))^n$. Čia galime statyti bet kokį
    sveiką $p$ ir tai yra tiesinė lygtis bet kurio  $b_{k}$ atžvilgiu. Tada
    keisdami įvairias $p$ reikšmes galime gauti $n-1$ neekvivalenčių
    lygčių su $n-1$ kintamųjų $b_{k}$. Tada aišku, kad tokia tiesinių
    lygčių sistema turės daugiausiai tik vieną sprendinį. Nesunku
    patikrinti, kad pirmam atvejui tiks sprendinys $b_{k}=q^k$, o antram
    $b_{k}=-q^k$, lyginiams $k$ ir $b_{k}=q^k$ nelyginiams $k$. Tada pirmu
    atveju gausime  $f(x^2)=f(x)^2$, o antru $f(x^2)=-f(x)^2$. Iš čia
    funkcija ir vėl aprėžta ir gausime, kad kai $n$ - nelyginis, tiks tik
    tiesiniai sprendiniai $f(x)=x$ ir $f(x)=-x$.
\item
    Įstatę duotojoje lygtyje $x=0$, gausime $f(0)\not=0$, nes kitaip
    $f(y)=0$ visiems $y$, bet $f$ -  nekonstanta. Taigi
    $g(y)=1-\frac{f(y)}{f(0)}$. Įstatę pradinėję lygtyje $x=y=1$ ir
    panaudoję a), gausime $f(1)=0$ ir tada galime pažymėti $f(0)=-k$, kur
    $k$ - kažkoks teigiamas skaičius. Tada įstatę $g$ išraišką į pradinę
    lygtį gausime $f(xy)=f(x)+f(y)+\frac{f(x)f(y)}{k}$, arba:
    $k+f(xy)=(\sqrt{k}+\frac{f(x)}{\sqrt{k}})(\sqrt{k}+\frac{f(y)}{\sqrt{k}})$.
    Pakeitę $h(x)=\sqrt{k}+\frac{f(x)}{\sqrt{k}}$, gausime
    $\sqrt{k}h(xy)=h(x)h(y)$, o tada pakeitę $h(x)=\sqrt{k}i(x)$:
    $i(xy)=i(x)i(y)$. Monotoniškumas niekur nedingo ir šią lygtį jau esamę
    sprendę, tad nesunku gauti atsakymą:
    
    $$f(x)=-k+k\cdot sgn(x)\cdot |x|^a \text{ ir } g(y)=sgn(y)\cdot |y|^a,$$
    kur $sgn(x)$ - $x$ ženklo funkcija.
\item
    Įstatę į pradinę lygtį $x=y=0$, gausime, kad $f(0)=0$ (jei $u(0)=0$,
    $f$ - konstanta). $f$ - griežtai monotoninė, taigi $0$ ji neįgys su
    jokia kita argumento reikšme. Iš pradinės lygties
    $u(y)f(x)+f(y)=f(x+y)=u(x)f(y)+f(x)$. Čia fiksavę $y$, gausime:
    $u(x)=\frac{u(y)-1}{f(y)}f(x)+1=Af(x)+1$. Jei $u(z)=1$, visiems
    realiesiems $z$, tai egzistuos $f(x)=x$, tenkinanti pradines sąlygas.
    Kitu atveju: $f(x+y)=Af(x)f(y)+f(x)+f(y)$. Tada pakeitę $h(x)=Af(x)+1$
    gausime $$h(x+y)=h(x)h(y).$$ Tai viena iš Cauchy tipo lygčių, kurias
    sutikome ankščiau. Kadangi $f$ monotoninė, $h$ irgi monotoninė ir
    $h(x)=b^x$, kur $b>0$. Tada $f(x)=A^{-1}(b^x-1)$ ir $u(y)=b^y$ bus
    sprendiniai. Viską apibendrinus, $u(x)=b^x$, kur $b>0$ (įskaitant ir
    $b=1$), bus vienintelės sąlygas tenkinančios funkcijos.
\item
    Pirmiausiai darykime keitinį $f(x)=g(x)|1+x|$. Pradinė lygtis taps:
    $g(\frac{x+y}{1+xy})=g(x)g(y)$. Pastebėkime, kad reiškinys
    $\frac{x+y}{1+xy}$ primena tangentų sumos formulę, tačiau tangentas
    nepaprastas, o - hiperbolinis. Hiperbolinis tangentas - tai funkcija
    $\tanh (x)=\frac{e^{2x}-1}{e^{2x}+1}$. Nesunku įsitikinti, kad irgi
    galios panaši į tangentų sumos formulė, t.y. $\tanh(x+y)=\frac{\tanh
    (x)+\tanh (y)}{1+\tanh (x) \tanh (y)}$. Taigi keičiame lygtyje
    $x=\tanh(x)$, $y=\tanh(y)$. Gausime $g(\tanh(x+y))=g(\tanh(x))g(\tanh(y))$.
    Įsiveskime keitinį $h(x)=g(\tanh(x))$. Gausime lygtį $h(x+y)=h(x)h(y)$.
    Galime nesunkiai įsitikinti, kad tolydumas niekur nedingo, tai viena
    iš Cauchy tipo lygčių, kurios sprendiniai bus $h(x)=a^{x}$, kur $a\geq
    0$. Tada $a^{x}=g(\tanh(x))\implies g(x)=a^{\arctanh(x)}$,
    $f(x)=a^{\arctanh(x)}|1+x|$.
\end{enumerate} 
\section*{Kombinatorika}
\subsection*{Matematiniai žaidimai}
\begin{enumerate}
\item
Vienu ėjimu galime sumažinti tik vieną iš parametrų (ilgį arba plotį).
Nagrinėdami paprastesnius atvejus pastebime, kad atvejais $0\times 0$,
$1\times 1$ ir $2\times 2$ laimi $B$. Natūralu galvoti, kad atveju $n\times
n$ visada laimės $B$. $A$ atlieka ėjimą su kvadratu ir $B$ gauna ne kvadratinę
plytelę iš kurios visada gali padaryti kvadratą ir taip išsaugoti savo
laiminčiąją poziciją. Atveju $m=n$ laimi $B$, kitais atvejais laimi $A$.
\item
Purpurinūsiui tereikia dėti žirgą į langelį, kuris yra simetriškas
Žaliaūsio užimtam lentos horizantaliosios (arba vertikaliosios) ašies
atžvilgiu. 
\item
 Pirmu ėjimu $A$ prideda 1 ir gauna $n=3$. Dabar $A$ visada galės paeiti
 taip, kad $B$ gautų nelyginį skaičių, o po šio ėjimo $A$ atitektų
 lyginis. $B$ galės pridėti nedaugiau negu vieną trečiąją turimo
 skaičiaus, o $A$ visada galės pridėti bent pusę. Taigi $A$ ramiai stebi
 priešininko agoniją tol, kol gauna $n\geq 1328$. Jis, pridėdamas pusę šio
 skaičiaus, pasieks skaičių nemažesnį už $1990$
\item
1) Lentą galima padalinti į stačiakampius $2\times 1$. $A$ tereikia
pereiti į gretimą to pačio stačiakampio langelį. $B$ tada turės pereiti į
kitą stačiakampį ir $A$ visada galės atlikti dar vieną ėjimą.  

2) Lentą galima padalinti į stačiakampius $2\times 1$ neįtraukiant
apatinio kairiojo kampo. Tada analogiškai žaisdamas laimi $B$.

3) Čia $B$ jau bejėgis. Lyginiams $n$ strategija analogiška (1). Kitu
atveju lentą padaliname į stačiakampius $2\times 1$, bet neįtraukiame
apatinio kairiojo kampo. Lentą nuspalviname įprastiniu būdu. Pastebime,
kad apatinis kairys langelis $B$ yra nepasiekiamas, tad $A$ laimi
pajudėdamas į gretimą stačiakampio langelį.
\item
Suskirstome lentą į stačiakampius  $2\times 4$. Pastebime, kad iš bet
kurio stačiakampio langelio žirgo ėjimu galime patekti tik į vieną to
stačiakampio langelį. $A$ padeda žirgą į vieną iš stačiakampių, $B$ tereikia
paeiti į langelį esantį tame pačiame stačiakampyje. Kitu ėjimu $A$ būtinai
turės pereiti į kitą stačiakampį, taip sudarydamas galimybę $B$ judėti to
stačiakampio viduje. Žaidimą visada laimės $B$.
\item
 Jei nors vienoje iš krūvelių yra nelyginis akmenukų skaičius, laimi $A$.
 Jam tereikia pirmu ėjimu akmenukų skaičius paversti lyginiais abejose
 krūvelėse. Tada po $B$ ėjimo nors vienoje krūvelėje tikrai bus nelyginis
 akmenukų skaičius ir $A$ galės tęsti savo spektaklį. Kitu atveju
 analogiškai žaisdamas laimės $B$. 
\item
1) $A$ tereikia atlaužti kvadratą $ m-1\times m-1 $ ir tada laužti
simetriškai įstrižainei.\\
2) $A$ tereikia visada laužti kampinį langelį.\\
3) Laiminti CHOMP žaidimo strategija nėra žinoma bendru atveju, tai atvira
problema. Jei manote, kad uždavinį išsprendėte, tai dar kartelį
peržvelkite savo sprendimą : ] 
\item
$A$ renkasi pusiaukraštinių susikirtimo tašką, o $B$ brėžia per jį tiesę,
lygiagrečią vienai kraštinių, ir gauna $\frac{5}{9}$ pyrago. Brėždamas kitą
tiesę per $X$ jis gautų mažiau, o jei $X$ nebūtų šis taškas, tai $B$ tikrai
galėtų gauti daugiau (įrodykite tai geometriškai). 
\item
$A$ pirmu ėjimu rašo $-1$ prie $x$. $B$ rašo $a$, o $A$ atsako $-a$. $x^3-a x^2-1
x+a=0$ turi šaknis $-1$, $1$ ir $a$. Tai sveikieji skaičiai.
\item
Įrodysime, kad visiems $N>1$, antrasis žaidėjas laimi tada ir tik tada,
jei $N=2^m$. Tokiu atveju pirmasis žaidėjas paima $2^a(2b+1)$ akmenukų,
kur $a\geq 0$ ir $b\geq 0$. Tada antrasis žaidėjas paima $2^a$, o kitais
ėjimais kopijuoja pirmojo žaidėjo veiksmus (įsitikinkite, kad tai
garantuoja pergalę). Jei $N=2^a(2b+1)$, kur $a\geq 0$ ir $b\geq 1$ tada
laimi pirmasis žaidėjas pirmu ėjimu paimdamas $2^a$ akmenukų ir kitais
ėjimais kopijuodamas antrojo žaidėjo veiksmus.
\item
Kryžiukams-nuliukams. Įsitikinkite tuo!
\item
Prisiminkite NIM : ] 
\item
 Tokius uždavinius jau mokame spręsti bendru atveju. Akmenukų skaičiams
 $0$, $1$, $2$, $3$, $4$, $5$, $6$, $7$, $8$, $9$, $10$ atitinkamai
 priskiriame NIM vertes lygias $0$, $1$, $0$ , $1$, $0$, $1$, $0$, $1$,
 $2$, $3$, $1$. Atlikdami šią procedūrą didesniems skaičiams pastebime,
 kad NIM vertės kinta periodiškai, periodo ilgis $11$. Nulines vertes turi
 visi akmenukų skaičiai, kurių forma yra $11n$, $11n+2$, $11n+4$ arba $11n+6$,
 kur $n$ sveikasis neneigiamas. Jei Matemagikas pradeda pozicijoje, kurios
 vertė nėra (0), tai jis visada gali pereiti į poziciją (0), o
 priešininkas negalės pereiti iš (0) į (0). Matemagikas laimės. Jei jis
 pradeda pozicijoje (0), analogiškai žaisdamas laimi $B$.
\item
Žaidimo pabaigos pozicija yra $1$ akmenukas. Akmenukų skaičiams $1$, $2$,
$3$, $4$, $5$, $6$, $7$ priskiriame NIM vertes lygias $0$, $1$, $0$, $2$,
$1$, $3$, $0$. Priskyrus vertes didesniems skaičiams nesunku pastebėti ir
įrodyti, kad nulines vertes turės skaičiai, kurių forma $2^k-1$, kur $k$
sveikasis neneigiamas. Taigi jei $n=2^k-1$, tada laimi $B$, kitais
atvejais pergalę švenčia $A$. 
\item
1994 vektorių suma yra $\vec a$. Pirmasis žaidėjas žaidžia tokioje
kordinačių sistemoje, kur $x$ ašis sutampa su $\vec a$ kryptimi. Jei $\vec
a=0$, tada kryptis gali būti bet kokia. Kiekvienu ėjimu žaidėjas renkasi
vektorių, kurio projekcija į $x$ ašį didžiausia. Galų gale pirmojo žaidėjo
vektoriaus projekcija į $x$ ašį bus nemažesnė už antrojo, o abiejų žaidėju
vektorių projekcijos į $y$ ašį bus lygios (jų suma lygi nuliui) Taigi
pirmasis žaidėjas niekada nepralaimės.
\item
(Sprendimas Nr. 1) Imame dvi viršutines eilutes ir sunumeruojame langelius iš kairės į
dešinę. Brėžiame rodyklę iš apatinio trečio langelio į viršutinį pirmą, iš
apatinio $5$ į viršutinį $3$ ir t.t. Imame dvi žemensnes eilutes ir brėžiame
rodykles iš viršutinio antro langelio į apatinį ketvirtą, iš viršutinio
ketvirto į apatinį $6$ ir t.t. Dar dvi žemesnes eilutes pažymime kaip pirmas
dvi ir t.t. Matome, kad rodyklė atitinka horizontalų žirgo ėjimą, o
vertikaliu žirgo ėjimu iš rodyklės smaigalio visada atsiduriama rodyklės
pradžioje. Žaidėjui $A$ tereikia žirgą pastatyti rodyklės pradžioje ir
paeiti į smaigalį. Tada $B$ būtinai paeis į kitos rodyklės pradžią ir $A$ galės
paeiti į rodyklės smaigalį. 

(Sprendimas Nr. 2) Susižymėkime lentelės langelius kaip kordinates $(x,y)$, kur $x$, $y$ yra
teigiami sveikieji. Tarkime, kad žirgo pastatymas $(1,1)$ langelyje ir
paėjimas į langelį $(3,2)$ įstumia $A$ į pralaiminčią poziciją (kitu atveju
įrodymas jau yra baigtas). Tada $B$ savo ėjimu peina į langelį $(X,Y)$ taip,
kad $A$ vėl atsidurtų pralaiminčioje pozicijoje. Pastebime, kad jei $A$ pirmu
ėjimu pastato žirgą į $(2,3)$, tada ėjimas į $(Y,X)$ garantuoja $A$ pergalę.
Dabartinė situacija nuo pirmosios skiriasi tik tuo, kad žirgas nepabuvojo
langelyje $(1,1)$. Tačiau šis langelis yra nepasiekiamas $B$, tad tai nedaro įtakos baigčiai.
\item
Atveju $N=2$ antrajam žaidėjui pakanka nuspalvinti tašką simetrišką
raudonajam centro atžvilgiu. Kad ir kokį didelį lanką atsiriektų pirmasis
žaidėjas antrojo ėjimo metu, antrasis visada galės atriekti didesnį (taškų
ant pasirinkto apskritimo lanko yra be galo daug). Nagrinėdami atvejį
$N=3$ vėl bandome spalvinti taškus simetriškai centro atžvilgiu, bet
pastebime, kad tai nieko gero neduoda. Galimų strategijų skaičius nėra jau
toks didelis ir įgudusi akis greit pastebės, kad atveju $N=2$ pasiteisino
strategija spalvinti taisyklingojo dvikampio viršūnes. Tai praktiškai ir
yra visas uždavinio sprendimas. 

Antrasis žaidėjas tol spalvina taisyklingojo $N$-kampio, kurio viršūnė yra
pirmasis raudonas taškas, viršūnes, kol gali. Jis nuspalvina $a$ viršūnių.
$N$-kampis yra suskirstytas bent į $N$ lankų, vadinsime šiuos lankus
pagrindiniais. Yra nedaugiau negu $N-a-1$ pagrindinių lankų, kurių abu
galai yra raudoni ir pirmasis žaidėjas gali visuose juose nuspalvinti po
tašką ir jam dar lieka vienas ėjimas. Jei jam lieka daugiau ėjimų, tai jis
spalvina taškus lankuose, kurių abu galai raudoni, kol lieka vienintelis.
Taip ilgiausias antrojo žaidėjo lankas bus tikrai trumpesnis už
pagrindinį. Kada visos $N$-kampio viršūnės nuspalvinamos, yra bent $a+1$
lankų, kurių nors vienas galas yra mėlynas; vadinsime šiuos lankus
melsvais. Pirmasis žaidėjas jau atliko bent $N-a$ ėjimų (nuspalvino $N-a$
taisyklingojo $N$-kampio viršūnių), tad jam liko ne daugiau $a$ ėjimų ir jis
negali sudarkyti visų melsvų lankų. Prieš paskutinį ėjimą tikrai nėra nė
vieno pagrindinio lanko, kurio abu galai raudoni ir yra nors vienas
melsvas lankas. Antrasis žaidėjas gali užsitikrinti lanką mėlynais galais,
kurio ilgis kaip norima artimas pagrindinio lanko ilgiui. Šis lankas bus
tikrai ilgesnis už ilgiausią raudoną lanką. Antrasis žaidėjas turi
laiminčiąją strategiją.
\item
Tegu $a$ ir $b$ yra $A$ ir $B$ skaičiai, o $x<y$ - teisėjo skaičiai.
Tarkime, kad žaidimas begalinis. $A$ žino, kad   $y\geq b\geq0$  ir sako
„ne". Kitu žingsniu $B$ suvokia, kad $A$ suprato, jog $y\geq b\geq0$ ,
tačiau, jei $a>x$, tada $A$ žinotų, kad $a+b=y$ ir pasakytų „taip", taigi
$B$ supranta, kad $x\geq a\geq0$ ir žaidimas tęsiasi. 

Tarkime, kad $n$-tuoju žingsniu $A$ žino, jog $B$ suvokė, kad
$s_{n-1}\geq a\geq r_{n-1}$. Jei $b> x-r_{n-1}$, $B$ žinotų, kad $a+b>x$,
t.y. $a+b=y$.  Jei $b<y-s_{n-1}$, $B$ žinotų, kad $a+b<y$, t.y. $a+b=x$.
Abiem atvejais $B$	galėtų aptspėti $A$, bet jis pasako „ne", taigi
$x-r_{n-1}\geq b\geq y-s_{n-1}$. Dabar $r_{n}=y-s_{n-1}$ ir
$s_{n}=x-r_{n-1}$. Kitu žingsniu $B$ analogiškai suvokia, kad
$r_{n+1}=y-s_{n}$ ir $s_{n+1}=x-r_{n}$.  Pastebime, jog abiem atvejais
$s_{i+1}-r_{i+1}=s_{i}-r_{i}-(y-x)$. Kadangi $y-x>0$, tai egzistuoja $m$,
kuriam galioja $s_{m}-r_{m}<0$. Prieštara.
\item
Taip gali. $P(x)$ yra daugianaris žaidimo pabaigoje. Prieš paskutinį $B$
ėjimą turime daugianarį $F(x)$. Žaidėjas $B$ gali užsitikrinti, kad $A$
paskutiniu ėjimu keis tritaškį prie nelyginio laipsnio. Tada
$P(x)=F(x)+ax^m+bx^{2p+1}$. $P(-2)=F(-2)+a(-2)^m-b2^{2p+1}$,
$cP(1)=cF(1)+ca+cb$. Jei $c=2^{2p+1}$, gauname
$2^{2p+1}P(1)+P(-2)=2^{2p+1}F(1)+F(-2)+2^{2p+1}a+a(-2)^m$. Jei
$2^{2p+1}P(1)+P(-2)=0$, tai $P(x)$ tikrai turės realią šaknį tarp $1$ ir
$-2$.  $2^{2p+1}F(1)+F(-2)+2^{2p+1}a+a(-2)^m=0$, tada
$a=\frac{-cF(1)-f(-2)}{c+(-2)^m}$. Paskutiniu ėjimu $B$ tereikia parašyti
$a$ taip, kad $A$ reiktų rašyti koeficientą prie nelyginio laipsnio. Tada
$P(x)$ turės šaknį tarp $1$ ir $-2$.
\item
Kai $k=1$, žaidėjui $A$ tereikia nuspalvinti tris taškus esančius vienoje
tiesėjė lygiagrečioje ašims taip, kad vienas gulėtų lygiai per vidurį tarp
kitų dviejų, nutolęs nuo jų atstumu $X$ ir, trys taškai, nutolę nuo pirmųjų
trijų atstumu $X$ vertikaliai į viršų arbą į apačią, būtų laisvi. Pabandžius
nesunku įsitikinti, kad tai įmanoma ir tai pasiekus $A$ lengvai gali
laimėti. 

Bandydami atvejį $k=2$ pastebime, kad plokštumos begalinumas sprendžiant
šį uždavinį yra kertinis faktorius. Kuo daugiau $A$ nuspalvina taškų, tuo
daugiau galimų kvadratų turi užblokuoti $B$. Čia ir atsiranda nuojauta, kad
pirmasis žaidėjas gali laimėti su bet kokiu $k$.

Įrodinėdami uždavinį pasinaudosime keletu paprastų faktų:

$(1)$ $A$ gali nuspalvinti kaip nori daug taškų vienoje tiesėje, nes taškų
skaičius begalinis.\\
$(2)$ $A$ visada suras tuščią tiesę lygiagrečią ašims, kurioje nėra
nuspalvintas dar nė vienas taškas, nes tiesių skaičius begalinis.

Pirmasis žaidėjas nuspalvina $Z$ taškų $x$ ašyje ir brėžia per kiekvieną
tašką tiesę lygiagrečią ašiai $y$ (vadinsime šias tieses statiniais). Tada
susiranda tuščią tiesę lygiagrečią $x$ ašiai ir spalvina šios tiesės
sankirtas su statinėmis. $A$ naujojoje tiesėje nuspalvins ne mažiau negu $
\frac{N}{Z+1} $ sankirtų. Kitu žingsniu $A$ nutrina visus statinius,
kurių sankirtų šioje tiesėje nenuspalvino. $A$ tęsia žaidimą išsirinkdamas
tuščią tiesę, nuspalvindamas sankirtas ir nutrindamas nepanaudotus
statinius. Pastebime, kad statiniai su pasirinktomis tiesėmis sudaro
stačiakampę gardelę, kurios visos sankirtos nuspalvintos raudonai.
Pasirinkdamas pakankamai didelį $Z$, $A$ gali gauti tokią gardelę  $ a\times b
$, kokios tik užsigeidžia. 

Nuspalvinęs pakankamai didelę gardelę (pakankamumo sąlygos radimą
paliksime skaitytojui), žaidėjas $A$ spalvina $x$ ašyje tašką $Q$ ir brėžia per
jį tiesę $d$ sudarančią $45\,^{\circ}$  kampą su $x$ ašimi taip, kad visi
nuspalvintieji taškai gulėtų kairiau šios tiesės.

Prasitęsiame $a$ gardelės horizontalių tiesių ir spalviname šių tiesių
sankirtas su $d$. $A$ galės nuspalvinti bent $ \frac{a}{k+1} $ sankirtų
$(1)$.  Po šių  $ \frac{a}{k+1} $ ėjimų liks bent $b-a$ nenuspalvintų
sankirtų tarp $b$ pratęstų gardelės vertikalių  ir tiesės $d$, $A$ gali
nuspalvinti bent jau $ \frac{b-a}{k+1} $ šių sankirtų $(2)$. 

Dabar nagrinėsime $r= \frac{a}{k+1} $ horizontalių tiesių, kurios kerta
$d$ raudonuose taškuose $(1)$ ir $s= \frac{b-a}{k+1} $ vertikalių tiesių,
kurios kerta $d$ raudonuose taškuose $(2)$. Pastebime, kad bet kuriems $2$
raudoniems taškams iš $(1)$ ir $(2)$ gardelėje atsiras jau nuspalvintas
raudonai taškas, kuris su jais sudaro tris kvadrato, kurio kraštinės
lygiagrečios ašims, viršūnes. $A$ gali pasirinkti  $ r\times s $ skirtingų
kvadratų, kurių tris viršūnes jau yra nuspalvinęs. Jam lieka nuspalvinti
vieną iš  $  r\times s $ taškų dešiniau linijos $d$ ir taip laimėti
žaidimą. Parodysime, kad jis visada galės tai padaryti.

Nuo $d$ linijos nubrėžimo $B$ nuspalvino nedaugiau nei $b$ taškų iš
nagrinėjamų $ r\times s $. Taigi $A$ tereikia pasirinkti tokius $a$ ir
$b$, kad $a-b$ bei $b$ būtų pakankamai dideli, $ r\times s $ $r\times s >
b$. ($r,s$ išreikškiamii per $a,b,k$ ir nesunku apskaičiuoti kiek $b$ turi
būti didesnis už $a$). Kadangi žaidimo pradžioje $A$ gali spalvinti tiek
taškų, kiek tik širdis geidžia, tad tikrai galės pasirinkti pakankamus $a$
ir $b$. Patariame skaitytojui pačiam išsiaiškinti, kokie gi $a$ ir $b$ yra
pakankami.

Uždavinys gracingai neigia nusistovėjusias normas. Vienas begaliniame
lauke - puikiausiais karys.
\end{enumerate} 
