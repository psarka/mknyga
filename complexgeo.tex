\documentclass[11pt,a4paper,twoside]{book}

\usepackage[utf8x]{inputenc}
\usepackage[L7x]{fontenc}
\usepackage[lithuanian]{babel}
\usepackage[pdftex]{graphicx}
\usepackage{lmodern}
\usepackage{amssymb, amsmath, amsthm}

\def\leq{\leqslant}
\def\geq{\geqslant}
\def\C{\mathbb C}
\def\F{\mathbb F}
\def\N{\mathbb N}
\def\R{\mathbb R}
\def\Q{\mathbb Q}
\def\Z{\mathbb Z}
\def\eps{\varepsilon}
\def\ro{\varrho}


\DeclareMathOperator{\dbd}{dbd}
\DeclareMathOperator{\mbk}{mbk}
\DeclareMathOperator{\im}{Im}
\DeclareMathOperator{\re}{Re}
\DeclareMathOperator{\Arg}{Arg}


\newenvironment{sprendimas}{\noindent \textit{Sprendimas.}}{\hfill $\triangle$}
\newcounter{foo}[subsection]
\newtheorem*{teig}{Teiginys}
\newtheorem*{thm}{Teorema}
\newtheorem*{isv}{Išvada}
\newtheorem*{pav}{Pavyzdys}
\newtheorem{teignr}{Teiginys}
\newtheorem{thmnr}{Teorema}
\newtheorem{isvnr}{Išvada}
\newtheorem{pavnr}[foo]{Pavyzdys}
\theoremstyle{definition} \newtheorem*{api}{Apibrėžimas}
\theoremstyle{remark} \newtheorem*{pastaba}{Pastaba}

\begin{document}

\chapter {Kompleksiniai skaičiai geometrijoje}

Autoriai tiki, kad aukštesnio lygio konkursuose, kaip LMMO ir atrankos į tarptautines olimpiadas, be įrodymų galima naudoti visas šioje knygos dalyje pateikiamas teoremas.

Kompleksinius skaičius dažniausiai žymėsime mažosiomis raidėmis $a, b, c$, o atkarpas ir figūras didžiosiomis, pvz., atkarpa $AB$, trikampis $ABC$, o juos atitiks kompleksiniai skaičiai $a, b, c$. Skaičių koordinates kompleksinėje plokštumoje žymėsime $(x, y)=(Re (a), Im (a))$, kur $a = x + yi \in \mathbb{C}$ arba  $(r, \theta)=(|a|, Arg(a))$, kur $a = re^{i\theta} \in \mathbb{C}$. 




\subsection*{Sudėtis ir daugyba iš realiojo skaičiaus}

Prisiminkime iš vektorinės geometrijos, kad sudėdami vektorius sudedame jų atitinkamas koordinates. Jeigu turime vektorius $\overrightarrow{v_1}=(x_1, y_1)$ ir $\overrightarrow{v_2}=(x_2, y_2)$, tai jų suma bus $\overrightarrow{v_1}+\overrightarrow{v_2}=(x_1+x_2, y_1+y_2)$. Jeigu dar turime realųjį skaičių $r \in \mathbb{R}$, tai daugindami jį iš vektoriaus dauginame jį su visomis koordinatėmis $r  \overrightarrow{v_1}=(r  x_1, r  x_2)$.

Lygiai taip pat sudedame kompleksinius skaičius ir randame kompleksinio ir realaus skaičiaus sandaugą. Įsivaizduokime kiekvieną Euklidinės plokštumos tašką kaip vektorių, kurio pradžia yra $(0, 0)$. Kiekvieną kompleksinį skaičių $a = x + yi$ galime įsivaizduoti kaip vektorių $(x, y)$, kurio pradžios taškas yra $(0, 0)$ ir kuris atvaizduotas kompleksinėje plokštumoje. Taip mes gauname analogiją tarp kompleksinių skaičių ir vektorių. Todėl vektorių sudėties ir sandaugos iš realiojo skaičiaus geometrines interpretacijas galime taikyti ir kompleksiniams skaičiams. 


Geometriškai dviejų kompleksinių skaičių sumą randame naudodami lygiagretainio taisyklę, o sudauginę realųjį skaičių su kompleksiniu, pakeičiame tik kompleksinio skaičiaus nuotolį nuo koordinačių pradžios taško, išlaikydami kryptį (arba pakeisdami ją į priešingą). 






PAVEIKSLIUKAS1

\begin{thmnr}
Turime trikampį $ABC$, kraštinių vidurio taškai yra $A_1, B_1, C_1$ atitinkamai priešais viršūnes $A, B, C$, o pusiaukraštinių susikirtimo taškas $T$. Tada
$a_1 = \frac{b+c}{2}$,
$b_1 = \frac{c+a}{2}$,
$c_1 = \frac{a+b}{2}$,
$t = \frac{a+b+c}{3}$.
\end{thmnr}
\begin{sprendimas}
Pastebime, kad $a_1$ galime rasti prie $b$ pridėję $\frac{1}{2}$ vektoriaus iš viršūnės $b$ į viršūnę $c$. Toks vektorius iš koordinačių pradžios taško bus $c-b$, todėl $a_1 = b + \frac{c-b}{2}=\frac{b+c}{2}$. Taškus $b_1, c_1$ randame analogiškai.
Žinome, kad pusiaukraštinių susikirtimo taškas $T$ pusiaukraštines dalina santykiu $2:1$, todėl jį randame, prie viršūnės $a$ pridėję $\frac{2}{3}$ vektoriaus iš $a$ į $a_1$, todėl $t=a+\frac{2}{3}(a_1-a)=a+\frac{2}{3}(\frac{b+c}{2}-a)=\frac{a+b+c}{3}$.
\end{sprendimas}

PAVEIKSLIUKAS2




\begin{thmnr}
Turime trikampį $ABC$, apibrėžtinio apskritimo centras $O$, aukštinių susikirtimo taškas $H$. Tada galioja sąryšis
$h + 2o = a + b + c$.
\end{thmnr}
\begin{sprendimas}
Atsidėkime tašką $p=c+a-o$. Pastebėkime, kad taškai $AOCP$ sudaro rombą (atkarpa $AO$ yra lygiagreti atkarpai $CP$, be to $|a-o|=|c-o|$). Žinome, kad tiesė $BH$ yra statmena tiesei $AC$, bet pagal rombo savybes, tiesė $OP$ yra statmenta tiesei $AC$, todėl $OP$ lygiagreti $BH$.Vadinasi, $h-b=(p-o)r$, kur $r \in \mathbb{R}$. Persirašome paskutinę tapatybę į $h+2ro=b+r(a+c)$.

PAVEIKSLIUKAS3

Pabaigai naudosime kiek subtilesnį argumentą. Žinome, kad trikampio $ABC$ aukštinių susikirtimo tašką $H$ vienareikšmiškai galime rasti žinant visas tris viršūnes. Pažymėkime $h=f(a, b,c)$. Bet trikampio $ABC$ aukštinių susikirtimo taškas sutampa su trikampio $BCA$, todėl $f(a, b,c)=f(b,c,a)$. Todėl aukštinės $H$ išraiška turi būti simetriška $A, B, C$ atžvilgiu. Analogiškai samprotaudami gauname, kad ir $o$ išraiška turi būti simetriška $a, b, c$ atžvilgiu, todėl $h+2ro=b+r(a+c)$ yra simetriška, vadinasi $r$ turi būti lygus 1.
\end{sprendimas}

\textit {Pastaba.} Sprendžiant uždavinius, kuriuose reikalingi apibrėžtinio apskritimo centras ir aukštinių susikirtimo taškas, atrodytų paprasčiausia būtų tiesiog rasti šių taškų išraiškas per $a, b, c.$ Bet jos yra gan komplikuotos, todėl dažniausiai daug paprasčiau yra tarti, kad $a, b, c$ yra ant vienetinio apskritimo, tada $o=0, h=a+b+c$ ir turime $|a|=|b|=|c|=1$.






\subsection*{Kompleksinių skaičių sandauga}
Pagrindinis kompleksinių skaičių privalumas geometrijoje yra tas, kad kompleksinių skaičių sandauga yra labai patogi apsirašyti įvairiausius geometrinius taškus. 






\begin{pavnr}
Duotas keturkampis $ABCD.$ Ant visų kraštinių nubrėžti išoriniai kvadratai, jų centrai yra $P, Q, R, S$, atitinkamai ant $AB, BC, CD$ ir $DA.$ Įrodykite, kad $PR$ yra lygi ir statmena $QS$. 
\end{pavnr}
\begin{sprendimas}
Nagrinėkime kvadratą $ABUV$ (viršūnės eina prieš laikrodžio rodyklę). Tada $p=\frac{v+b}{2}$, bet $v=e^{i\frac{\pi}{2}}(b-a)+a$. Todėl $$p=\frac{a+b+i(b-a)}{2}.$$
PAVEIKSLIUKAS4

Analogiškai žinome, kad $$q=\frac{b+c+i(c-b)}{2},$$ $$r=\frac{c+d+i(d-c)}{2},$$ $$s=\frac{d+a+i(a-d)}{2}.$$
Todėl $$p-r=\frac{a+b-c-d+i(b+c-a-d)}{2},$$ $$q-s=\frac{b+c-a-d+i(c+d-a-b)}{2}.$$
Beliko pastebėti, kad $i(q-s)=p-r$, todėl QS yra lygi ir statmena PR.
\end{sprendimas}





\begin{thmnr}
Taškai $a, b, c, d$ priklauso vienam apskritimui, tada ir tik tada, kai: 
$$\frac{a - c}{b - c}:\frac{a-d}{b-d}\in\mathbb{R}.$$
\end{thmnr}
\begin{sprendimas}
Iš Euklidinės geometrijos, žinome, kad taškas $D$ priklauso apskritimui apibrėžtam apie trikampį $ABC$ tada ir tik tada, kai arba $\angle BCA=\angle BDA$ arba $\angle BCA+\angle BDA=\pi $. Pirmu atveju, taškai $C, D$ turi būti toje pačioje tiesės $AB$ pusėje, antru - skirtingose. Nagrinėkime pirmąjį variantą (antrasis įrodomas analogiškai). 

Nagrinėkime skaičius $$z_1 =\frac{a - c}{b - c}, z_2 =\frac{a - d}{b - d}.$$ 
Nesunku pamatyti, kad arba $Arg (z_1) = \angle BCA$ ir $Arg (z_2) = \angle BDA$ (kai $A, B, C$ ir $A, B, D$ išdėstyti pagal laikrodžio rodyklę) arba $Arg (z_1) = -\angle BCA$ ir $Arg (z_2) =- \angle BDA$  (kai $A, B, C$ ir $A, B, D$ išdėstyti prieš laikrodžio rodyklę). Bet kuriuo atveju $z_1/z_2$ bus ant realiosios ašies tada ir tik tada kai $\angle BCA=\angle BDA$, ką ir reikėjo įrodyti.

PAVEIKSLIUKAS5

\textit {Pastaba.} Jeigu norime įrodyti, kad $z \in \mathbb{C}$ yra ant realiosios ašies, dažniausiai patogu yra įrodyti ekvivalenčią sąlygą: $z=\overline{z}$.
\end{sprendimas}






\subsection*{Tiesės lygtis}

Šiame skyriuje išmoksime apsirašyti tiesės lygtis, kurios padės surasti taškus esančius ant lygiagrečių, statmenų ir susikertančių tiesių.

\begin{thmnr}
Tiesės lygtis, sudarančios kampą $\theta$ su $Re$ ašimi ir einančios per tašką $x \in \mathbb{C}$: $$z-x=e^{2i\theta}(\overline{z}-\overline{x}).$$
\end{thmnr}
\begin{sprendimas}
Pastebėkime, kad taškas $x$ tenkina lygtį, todėl visi $ z \neq x$ tenkina $$\frac{z-x}{\overline{z}-\overline{x}}=e^{2i\theta}.$$ Kadangi $|z-x|=|\overline{z}-\overline{x}|$, todėl $|(z-x)/(\overline{z}-\overline{x})|=1$. Be to, $Arg (z-x) =\phi$, kur $\phi$ yra kampas, kurį sudaro tiesė ZX su $Re$ ašimi. Todėl $Arg ((z-x)/(\overline{z}-\overline{x})) =2\phi$. Vadinasi 
$$\frac{z-x}{\overline{z}-\overline{x}}=e^{2i\theta}$$ tada ir tik tada, kai $e^{2i\phi}=e^{2i\theta}$ $\Leftrightarrow 2\phi =2\theta + 2\pi n$, kažkokiam $n \in \mathbb{Z}.$ Paskutinė tapatybė yra ekvivalenti $\phi = \theta + \pi n$. Vadinasi gavome, kad $z$ tenkina tiesės lygtį tada ir tik tada, kai $\phi = \theta$ arba $\phi = \theta + \pi$, bet antru atveju tai reiškia, kad vektorius $z-x$ yra paimtas priešinga kryptimi, todėl tiesės lygtis yra teisinga.
\end{sprendimas}




\bigskip
Dabar įrodysime vieną svarbesnių teoremų, padedančių aprašyti daugumą geometrinių taškų.

\begin{thmnr}\
\begin{enumerate}
\item Tiesė AB yra lygiagreti tiesei CD tada ir tik tada, kai $\frac{a - b}{\overline{a} -\overline{b}}=\frac{c - d}{\overline{c} - \overline{d}}$.
\item Taškai A, B ir C priklauso vienai tiesei tada ir tik tada, kai $\frac{a - b}{\overline{a} -\overline{b}}=\frac{a-c}{\overline{a} - \overline{c}}$.
\item Tiesė AB yra statmena tiesei CD tada ir tik tada, kai $\frac{a - b}{\overline{a} -\overline{b}}=-\frac{c - d}{\overline{c} - \overline{d}}$.
\end{enumerate}
\end{thmnr}
\begin{sprendimas} Kad įrodytume pirmąją dalį, iš tiesės lygties nesunku pastebėti, kad tiesė $AB$ sudaro tokį pat kampą su $Re$ ašimi kaip ir tiesė $CD$ tada ir tik tada, jeigu tenkinama duota sąlyga. O tiesės yra lygiagrečios tada ir tik tada, kai jos sudaro vienodą kampą su $Re$ ašimi (ar bet kuria kita tiese).

Antroji dalis seka gana akivaizdžiai iš pirmosios teoremos dalies.

Beliko įrodyti trečiąją dalį. Tarkime, kad $AB$ sudaro kampą $\theta$ su $Re$ ašimi, o $CD$ sudaro kampą $\phi$. Tada gauname, kad $\frac{a - b}{\overline{a} -\overline{b}}=-\frac{c - d}{\overline{c} - \overline{d}} \Leftrightarrow e^{2i\theta}=e^{i\pi}e^{2i\phi} \Leftrightarrow 2\theta = \pi + 2\phi + 2\pi n \Leftrightarrow \theta = \phi + \frac{\pi}{2} + \pi n $, kur $n$ bet koks sveikas skaičius. Nagrinėdami paskutinę tapatybę, matome, kad ji yra tenkinama tada ir tik tada, kai tiesės $AB$ ir $CD$ yra statmenos.

\end{sprendimas}

\begin{pavnr}
Duotas trikampis $ABC$. Tegu $AH$ yra trikampio aukštinė. Tada $h=\frac{a(\overline{b}-\overline{c})+\overline{a}(b-c)+\overline{b}c-b\overline{c}}{2(\overline{b}-\overline{c})}.$
\end{pavnr}
\begin{sprendimas}
Naudosimės penkta teorema. $H$ priklauos tiesei $BC$, todėl $\frac{b - c}{\overline{b} -\overline{c}}=\frac{h-b}{\overline{h} - \overline{b}}$. Be to, $AH$ yra statmena $BC$, todėl $\frac{a - h}{\overline{a} -\overline{h}}=-\frac{b - c}{\overline{b} - \overline{c}}$. Iš pirmosios tapatybės gauname $\overline{h}(b-c)=h(\overline{b} - \overline{c}) - \overline{b}c+b\overline{c}$, o iš antrosios $\overline{h}(b-c)=a(\overline{b}-\overline{c})+\overline{a}(b-c)-h(\overline{b}-\overline{c})$. Sulyginę gauname išraišką $h(\overline{b} - \overline{c}) - \overline{b}c+b\overline{c}=a(\overline{b}-\overline{c})+\overline{a}(b-c)-h(\overline{b}-\overline{c})$, kurią sutvarkę gauname $h=\frac{a(\overline{b}-\overline{c})+\overline{a}(b-c)+\overline{b}c-b\overline{c}}{2(\overline{b}-\overline{c})}$.

\textit {Pastaba.} Dažniausiai uždaviniuose išraiškas galima padaryti paprastesnėmis, pvz., neprarandant bendrumo tariant, kad $a=0$ arba kad trikampis $ABC$ yra ant vienetinio apskritimo, kas duoda $|a|^2=a\overline{a}=b\overline{b}=c\overline{c}=1$.
\end{sprendimas}


\subsection*{Keli patarimai sprendžiant uždavinius}

Šiame skyrelyje apžvelgsime metodus, kaip spręsti uždavinius. Sprendžiant uždavinius reikia išnaudoti visas geometrines sąlygas duotas uždavinyje. Kiekvienas uždavinys yra unikalus ir nuo to  kuriuos taškus pasirinksime laisvai ir per kuriuos išreikšime visus likusius, labai priklauso kaip lengvai mes uždavinį išspręsime. Sprendžiant geometrinius uždavinius šiuo metodu ir neatsargiai pasirenkant taškus, galima gauti labai komplikuotas išraiškas, kurias bus sunku suprastinti. Todėl svarbu surasti geriausią būdą kaip aprašyti taškus.



Pavyzdžiui, jeigu uždavinyje duotas bet koks kvadratas $ABCD$, neužtenka tiesiog įsivesti $4$ kompleksinių skaičių ant plokštumos. Kaip matėme pirmame pavyzdyje, tokiu atveju galime paimti bet kokius du taškus $a, b$, o kitas viršūnes išsireikšti per kompleksines konstantas tardami, kad viršūnės eina prieš laikrodžio rodyklę, tada $c=(a-b)i^3+b,  d=(b-a)i + a$.

Jeigu duotas bet koks lygiakraštis trikampis $ABC$, galime pasirinkti dvi viršūnes laisvai $a, b$, o trečiąją išreikšti $c=(b-a)e^{i\frac{\pi}{3}}+a$. Antroje teoremoje matėme pavyzdį, kaip galime išsisukti turint rombą. Su įvairiomis kitomis figūromis kaip lygiašonis trikampis, lygiagretainis ar lygiašonė trapecija gali būti kiek sudėtingiau. Kelis tokius atvejus apžvelgsime nagrinėdami konkrečius uždavinius.

Keli svarbūs triukai sprendžiant uždavinius yra tokie, kad neprarasdami bendrumo visada galime pasirinkti:
\begin{enumerate}
\item Bet kokį vieną tašką $z \in \mathbb{C}$ koordinačių pradžios tašku, t.y., prilyginti nuliui. Taikydami šią plokštumos transformaciją, visus plokštumos taškus pastumiame per $-z$, todėl taškas $z$ nukeliauja į nulį, bet visas geometrinis brėžinys lieka neiškraipytas.
\item Pasirinkti bet kokio vieno taško ilgį (ne koordinačių pradžios taško), pvz., $|u|=1, u \in \mathbb{C}$. Ši transformacija visus plokštumos taškus padaugina iš realaus teigiamo skaičiaus, kuri irgi neiškraipo geometrinio brėžinio, tik jį proporcingai padidina arba sumažina. Ši transformacija dažniausiai naudojama, kai koordinačių pradžios tašku pasirenkamas apibrėžtinio apskritimo apie trikampį $ABC$ centras, o viršūnių $A, B, C$ ilgiai prilyginami vienetui. Tada $o=0$ ir visos viršūnės tenkina $|a|^2=1, \Leftrightarrow a\overline{a}=1 \Leftrightarrow a =\frac{1}{\overline{a}}$, kas dažnai supaprastina daug išraiškų.
\item Pasukti visą plokštumą kampų $\theta$, t.y., padauginti visus plokštumos taškus iš $e^{i\theta}$. Taip bet kokį tašką ant vienetinio apskritimo galime prilyginti, pvz., $1$ ar $i$. Iš tiesų ši transformacija retai kada būna naudinga. Viena, prilygindami $a=1$ mes supaprastinsime visas išraiškas, kur turime $a$ sandaugą su kitais skaičiais, bet dažnai turėsime narius tokius kaip $a-b$, kurie susiprastins tiesiog iki $1-b$ ir realios naudos iš to negausime. Antra, jeigu, pavyzdžiui, uždavinio sąlyga yra simetriška trikampio viršūnių $ABC$ atžvilgiu, kartais neapsimoka prarasti simetrijos šių taškų atžvilgiu, nes tada galime lengviau rasti įvairių taškų išraiškas tiesiog pakeisdami raides ir galime pasitikrinti ar skaičiavimuose kur nepadarėme algebrinės klaidos.
\end{enumerate}


Skyrių užbaigiame dar vienu triuku, kuris parodys, kad kartais užtenka atspėti sprendinį ir patikrinti, kad jis yra teisingas, taip išvengdami komplikuotų skaičiavimų.

\begin{pavnr}
Duotas trikampis $ABC$. Jo apibrėžtinis apskritimas $S$. Šio apskritimo simetriški atspindžiai tiesių $AB$, $BC$, $CA$ atžvilgiu atitinkamai yra $S_{AB}, S_{BC}, S_{CA}$. Įrodykite, kad šie trys apskritimai kertasi viename taške.
\end{pavnr}
\begin{sprendimas}
Neprarasdami bendrumo, tarkime, kad $S$ yra vienetinis apskritimas, t.y., $o=0$ ir $|a|=|b|=|c|=1$. Tada $S_{AB}$ centras yra $o_{AB}=a+b$, nes $AB$ yra statmena $O_{AB}O$. Analogiškai, $S_{BC}$ ir $S_{CA}$ centrai atitinkamai yra $b+c$ ir $c+a$. Taškas $z$ priklauso visiems trims apskritimams tada ir tik tada, jeigu jis yra sprendinys $|z-(a+b)|=|z-(b+c)|=|z-(c+a)|=1$. Belieka pastebėti, kad aukštinių susikirtimo taškas $h=a+b+c$ (naudojantis antra teorema) yra šios lygties sprendinys. Vadinasi, šie trys apskritimai kertasi viename taške, kuris yra trikampio $ABC$ aukštinių susikirtimo taškas.

\end{sprendimas}





















\chapter{Vienetinis apskritimas}

Kaip pamatysime šio skyriaus teoremoje, taškai ant vienetinio apskritimo turi daug gražių savybių, kurios supaprastina daug skaičiavimų. Todėl uždaviniuose yra labai svarbu protingai pasirinkti apskritimą, kurį laikysime vienetiniu.

\begin{thmnr}  Jei $A, B, C, D$ priklauso vienetiniam apskritimui, tada:\
\begin{enumerate}
\item $\frac{a - b}{\overline{a} -\overline{b}}=-ab$, analogiškai su kitomis poromis.
\item Jei $X$ priklauso tiesei AB, tai $\overline{x}=\frac{a + b - x}{ab}$.
\item Jei $X$ yra tiesių AB ir CD susikirtimo taškas, tai $x=\frac{ab(c+d) - cd(a+b)}{ab - cd}$.
\end{enumerate}
\end{thmnr}
\begin{sprendimas}
\begin{enumerate} 
\item Naudojamės savybe, kad ant vienetinio apskritimo taškai tenkina $|a|^2=a\overline{a}=1$ ir gauname $\frac{a - b}{\overline{a} -\overline{b}}=\frac{a - b}{\frac{1}{a} -\frac{1}{b}}=-ab$.
\item Naudojamės penktąja teorema ir šios teoremos pirmąja dalimi. Taškas $X$ priklauso tiesei $AB$, tada ir tik tada, kai $\frac{a-x}{\overline{a}-\overline{x}}=\frac{a-b}{\overline{a}-\overline{b}}$. Panaudoję $\frac{a - b}{\overline{a} -\overline{b}}=-ab$ ir suprastinę išraišką gauname $\overline{x}=\frac{a + b - x}{ab}$.
\item Naudojamės šios teoremos antrąja dalimi. Gauname $\overline{x}=\frac{a + b - x}{ab}$ ir $\overline{x}=\frac{c+ d - x}{cd}$. Sulyginę išraiškas gauname $x=\frac{ab(c+d) - cd(a+b)}{ab - cd}$.
\end{enumerate}
\end{sprendimas}

\textit {Pastaba.} Jeigu norime rasti liestinės prie $A$ ir $CD$ susikirtimo tašką, paprastas, bet matematiškai negriežtas būdas yra įsistatyti $b=a$ į teoremos paskutinės dalies formulę ir gauti $x=\frac{a^2(c+d) - 2acd}{a^2 - cd}$. Ieškodami liestinių prie $A$ ir $C$ susikirtimo taško, įsistatome $b=a, d=c$ ir gauname $x=\frac{2ac}{a+c}$. Šių formulių griežti įrodymai yra panašūs pateiktam įrodymui.
\bigskip

Dabar išspręsime kelis sudėtingesnius uždavinius.



  

\begin{pavnr}

Apie trikampį $ABC$ apibrežto apskritimo skersmuo - $AD$, aukštinių susikirtimo
taškas (dar žinomas kaip ortocentras) - $H$, o $M$ - kraštinės $BC$ vidurio taškas. Įrodykite, jog taškas M yra atkarpos HD vidurys.
\end{pavnr}
\begin{sprendimas}
 Tegu vienetinis apskritimas būna apibrėžtas apie $ABC$ apskritimas. Pagal antrąją teoremą mes turime, kad $$ h = a+b+c.$$ Kadangi M yra BC vidurio taškas, tai; $$m= \frac{b+c}{2}.$$   Kadangi skersmens vidurio taškas yra koordinačių pradžios taškas ir jis yra lygus nuliui, tai taškas $d$  tenkina $d= -a.$ Galiausiai pastebime, kad $HD$ vidurio taškas yra $$ \frac{d+h}{2} = \frac {(-a)+a+b+c}{2}= \frac{b+c}{2}.$$ O juk tai ir yra mūsų jau surastas taškas $M$, ką mums ir reikėjo įrodyti.
\end{sprendimas}






\begin{pavnr} 
Duotas apie smailųjį trikampį $ABC$ apibrėžtas apskritimas, kurio liestinės iš taškų $A$ ir $B$ susikerta taške $X$. Jei $M$ yra kraštinės $AB$ vidurio taškas, įrodykite, kad $\angle ACX = \angle BCM$.
\end{pavnr}
\begin{sprendimas}
Pagal šeštą teoremą turime, kad $ x= \frac {2ab}{a+b}$. Pasižymime $\theta = \angle ACX$ ir $\phi = \angle BCM$. Toliau išreiškiame kampus:
$$ e^{ 2i\theta} = \frac { a-c}{ \bar a - \bar c} : \frac {x-c}{\bar x - \bar c}=
- \frac { 2ab -ac-bc}{2bc - ab- b^2}.$$
Naudojame $m = \frac {a+b}{2}$ ir analogiškai randame:
$$ e^{2i\phi} = \frac { m-c}{ \bar m - \bar c} : \frac {b-c}{\bar b - \bar c}=
- \frac { 2ab -ac-bc}{2bc - ab- b^2}.$$
Taigi gauname $ e^{ 2i\theta} = e^{2i\phi}$. Iš to seka, kad arba $ \theta = \pi + \phi$, kas nėra įmanoma, nes $\Delta abc$ yra smailusis trikampis, arba $\theta =\phi$, ką mums ir reikėjo įrodyti.
\end{sprendimas}





\begin{pavnr}
(USAMO 2010) 

Duotas iškilas apibrėžtinis penkiakampis $AXYZB$ toks, kad $AB$ yra apibrėžtojo apskritimo skersmuo. Pažymėkime $P, Q, R, S$ statmenimis iš taško $Y$ tiesėms  $AX, BX, AZ, BZ$ atitinkamai. Įrodykite, kad smailusis kampas tarp tiesių $PQ$ ir $RS$ yra dvigubai mažesnis nei kampas $\angle XOZ$.
\end{pavnr}
\begin{sprendimas}
  Neprarasdami bendrumo tarkime, kad apie penkiakampį apibrėžtasis apskritimas yra vienetinis. Tuomet $b = -a$, nes $ab$ yra skersmuo. 

$P$ priklauso tiesei $AX$, todėl naudodamiesi teorema 6.2, gauname $$\overline{p}=\frac{a+x-p}{ax}.$$ Tiesė $PY$ yra statmena tiesei $AX$, todėl iš teoremos 5.3 gauname $$\frac{p-y}{\overline{p}-\overline{y}}=-\frac{a-x}{\overline{a}-\overline{x}} (=ax).$$ 
Įsistatę $\overline{p}$ į paskutinę tapatybę gauname 
$$p = \frac {1}{2}( a+x+y- \frac {ax}{y}).$$
Analogiškai randame
$$q = \frac {1}{2}( a+z+y- \frac {az}{y}),$$ 
$$r = \frac {1}{2}( -a+x+y- \frac {ax}{y}),$$
$$s = \frac {1}{2}( -a+z+y- \frac {az}{y}).$$

Iš tiesų yra ganėtinai sudėtinga kompleksiniais skaičiais įrodyti kampų lygybes, kai kampai nėra lygūs. Šiuo atveju skaičiavimus palengvina Euklidinės geometrijos panaudojimas: pagal įbrėžtinius kampus turime $\angle XOZ = 2\angle XAZ$, todėl, mums lieka įrodyti, jog  $\angle XAZ=\theta$ yra lygus kampui tarp tiesių $PQ$ ir $RS$, kurį pažymėkime $\phi$. Šie kampai yra lygūs tada ir tik tada, kai tenkinama 
$$e^{2i\theta}=\frac {x-a} {\bar x - \bar a} : \frac{z-a}{\bar z - \bar a} = \frac{p-r}{\bar p - \bar r} : \frac{q-s}{\bar q - \bar s}=e^{2i\phi}.$$
Įsistačius $p, q, r, s$ žinomas išraiškas per $a, x, y, z,$ gauname, kad lygybė teisinga, todėl uždavinys išspęstas.
\end{sprendimas}
     





\begin{pavnr}
(IMO 2009)

Tegul $O$ yra apie trikampį $ABC$ apibrėžto apskritimo centras. Taškai $P$ ir $Q$ atitinkamai yra atkarpų $CA$ ir $AB$ vidiniai taškai. Tegul $\Gamma$ yra apskritimas, einantis per atkarpų $BP$, $CQ$ ir $PQ$ vidurio taškus $K, L$ ir $M$, o tiesė $PQ$ yra apskritimo $\Gamma$ liestinė. Įrodykite, kad $OP = OQ$.
\end{pavnr}
\begin{sprendimas}

Tegu apie $\bigtriangleup ABC$ apibrėžtas apskritimas yra vienetinis. Kadangi $P$ ir $Q$ priklauso $CA$ ir $AB$, atitinkamai, tai
$$\overline{p}=\tfrac{a+c-p}{ac},$$
$$\overline{q}=\tfrac{a+b-q}{ab}.$$
Mums reikia įrodyti:
$$(p-o)(\overline{p-o})=(q-o)(\overline{q-o}) \Leftrightarrow p\overline{p}=q\overline{q} \Leftrightarrow \tfrac{p(a+c-p)}{ac}=\tfrac{q(a+b-q)}{ab}.$$

Liko neišnaudota viena sąlyga - tiesė $PQ$ liečia apskritimą $\Gamma$. Galima bandyti susirasti apskritimo centrą ir įrodyti kad statmuo iš $M$ tiesei $PQ$ eina per $\Gamma$ centrą, arba naudotis tuo, kad $\angle QMK=\angle MLK$ tada ir tik tada, kai tiesė $PQ$ yra $\Gamma$ liestinė taške $M$. Pasižymime $\omega=\angle QMK=\angle MLK$, tada:
$$\tfrac{m-q}{|m-q|}\omega=\tfrac{m-k}{|m-k|} \Rightarrow \tfrac{m-q}{\overline{m}-\overline{q}}\omega^2=\tfrac{m-k}{\overline{m}-\overline{k}} \Leftrightarrow \omega^2=\tfrac{m-k}{\overline{m}-\overline{k}} \tfrac{\overline{m}-\overline{q}}{m-q}.$$
Analogiškai ir su m, l, k:
$$\tfrac{l-m}{|l-m|}\omega=\tfrac{l-k}{|l-k|} \Rightarrow  \omega^2=\tfrac{l-m}{\overline{l}-\overline{m}}\tfrac{\overline{l}-\overline{k}}{l-k}$$
Sulyginame reiškinius, įsistatome m, k ir l reikšmes, tada p ir q jungtinius:

\begin{equation*}
 \tfrac{m-k}{\overline{m}-\overline{k}} \tfrac{\overline{m}-\overline{q}}{m-q}=\tfrac{l-m}{\overline{l}-\overline{m}}\tfrac{\overline{l}-\overline{k}}{l-k} \Leftrightarrow 
\end{equation*}
\begin{equation*}
 \Leftrightarrow (q-b)(\overline{p}-\overline{q})(c-p)(\overline{c}+\overline{q}-\overline{b}-\overline{p})=(p-q)(\overline{q}-\overline{b})(c+q-b-p)(\overline{c}-\overline{p}) \Leftrightarrow
\end{equation*}
\begin{equation*}
\Leftrightarrow (q-b)(\tfrac{ab-ac+qc-pb}{abc})(c-p)(\tfrac{pb-qc}{abc})=(p-q)(\tfrac{b-q}{ab})(\tfrac{p-c}{ac})(c+q-b-p),
\end{equation*}
gautą lygybę galima nesunkiai suprastinti ir pertvarkyti į reiškinį, kurį mums reikėjo įrodyti.
\end{sprendimas}


\end{document}