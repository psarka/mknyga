\documentclass[11pt,a4paper,twoside]{book}

\usepackage[BW]{optional}
\usepackage{esvect}
\usepackage{knyga}

\begin{document}

\chapter {Kompleksinių skaičių geometrija}


Geometrija kaip olimpiadinės matematikos sritis, tikriausiai turi daugiausiai skirtingų sprendimo metodų. Geometriniai uždaviniai nebūtinai turi daugiausia skirtingų sprendimų, bet metodų, kaip spręsti pasirinktą uždavinį, yra apstu. 

Klasikinis būdas yra naudotis Euklidinės geometrijos teoremomis. Šis būdas yra turbūt vienas paprasčiausių - žvelgiant iš tos pusės, kad mokant kelias pagrindines trikampių panašumų ir apskritimų teoremas, galima išspręsti daugumą uždavinių. Taip pat, šis sprendimo būdas suteikia gilesnį uždavinio supratimą, nes uždavinio mintis nepasislepia algebrinėse išraiškose ir reiškinių pertvarkymuose. Bet iš kitos pusės, sprendžiant sudėtingesnius uždavinius, dažnai reikia labai daug patirties ir geros intuicijos, kuri padėtų pasipildyti brėžinį reikiamais taškais ar tiesėmis, kurie vestų iki sprendimo. Įgyti tokią intuiciją nėra lengva, todėl yra daug kitų būdų, kurie gali palengvinti spręsti geometrinius uždavinius.

Vienas alternatyvių būdų yra naudotis kompleksiniais skaičiais. Šis metodas reikalauja nemenko pasiruošimo - reikia mokėti ne tik laisvai daryti pertvarkymus su kompleksiniais skaičiais, suprasti geometrines šių operacijų interpretacijas, bet ir išmokti tvarkyti kompleksinių skaičių išraiškas taip, kad jos būtų kuo paprastesnės ir algebriniai reiškiniai netaptų nesuprastinami. Bet tinkamai pasiruošus, daug sudėtingų geometrinių uždavinių pasidaro įveikiami. Perrašius uždavinį algebrinėmis išraiškomis, užtenka jas tiesiog išprastinti ir uždavinys bus išspręstas. 

Kompleksinių skaičių geometrijos skyrių pradėsime nuo pagrindinių geometrijos
sampratų, tokių kaip lygiagretumas, kampų lygumas ar trikampių panašumas,
perrašymo kompleksinių skaičių kalba. Tolimesniuose skyreliuose supažindinsime
su tiesės lygtimi, plotų skaičiavimu ir apskritimo geometrija -- stipriausia
kompleksinių skaičių geometrijos puse.

\section{Skaičiai, taškai ir vektoriai}

Kompleksinė geometrija prasideda nuo kompleksinių skaičių plokštumos. Kiekvienas
kompleksinis skaičius turi realiąją ir menamąją dalis, kurias laikydami
įprastinėmis $x$ ir $y$ koordinatėmis galime jį atidėti koordinačių plokštumoje.
Ir atvirkščiai, turėdami taškus plokštumoje, galime (“uždėję“ ant jos
koordinates) jiems priskirti kompleksinius skaičius. Šiame skyriuje taškus, kaip
įprasta, žymėsime didžiosiomis raidėmis (pvz. $A$, $B$, $C$), o juos
atitinkančius kompleksinius skaičius mažosiomis (atitinkamai $a$, $b$, $c$).
Kompleksinės plokštumos pradžią žymėsime $O$ ir ją atitiks kompleksinis skaičius
$0 + 0i = 0$.

Kompleksinio skaičiaus $a = x + iy$ padėtį plokštumoje galima apibūdinti ne tik
nurodant jo realiąją ir menamąją dalis ($x$ ir $y$, arba $\Re(a)$ ir $\Im(a)$),
bet taip ir nurodant jo atstumą nuo koordinačių pradžios bei kampą, kurį atkarpa
$OA$ sudaro su realiąja ašimi. Šie du dydžiai atitinka kompleksinio skaičiaus
modulį $|a|$ ir argumentą $\Arg(a)$.  Pažymėję juos atitinkamai $r$ ir $\theta$,
ir naudodamiesi Oilerio lygybe, kompleksinį skaičių $a$ galime užrašyti dviem
būdais:

$$
a = x + iy = r e^{i\theta}.
$$

\begin{figure}[h]
  \centering
  \begin{asy}
    size(300);
    graph.xaxis("$\Re$");
    graph.yaxis("$\Im$");

    pair A=(75,50);
    pair I=(0,-20);
    pair II=(-40,20);
    pair P=1.5*A;
    pair O=origin;

    add(anglem((1,0),O,A,300,red,0));

    draw(A--(A.x,0),dotted);
    draw(O--(A.x,0));
    draw(A--(0,A.y),dotted);
    draw(O--A);

    dot(A,blue);
    dot(P,white);
    dot(I);
    dot(II);

    label("$a$",A,NE,blue);
    label("$\theta$",O,2*(3,1),blue);
    label("$r$",A/2,N,blue);
    label("$O$",O,SW,blue);
    label("$x$",(A.x,0),S,blue);
    label("$y$",(0,A.y),W,blue);
    label("$-i$",I,W,blue);
    label("$-2+i$",II,NW,blue);
  \end{asy}
  \caption{Kompleksinėje plokštumoje pažymėti trys kompleksiniai skaičiai:
    $-i$, $-2+i$ ir $a=x+yi=re^{i\theta}$.}
\end{figure}

Lygiai taip pat, kaip suformulavome kompleksinių skaičių ir taškų atitikmenį,
galima suformuluoti ir kompleksinių skaičių bei vektorių atitikmenį:
kompleksinio skaičiaus $a = x+yi$ realiąją ir menamąją dalis $x$ ir $y$ galime
laikyti vektoriaus koordinatėmis ir sakyti, kad kompleksinis skaičius $x + yi$ atitinka
vektorių $(x', y')$, jei $x=x'$ ir $y=y'$.

\begin{figure}[h]
  \centering
  \begin{asy}
    size(300);
    graph.xaxis("$\Re$");
    graph.yaxis("$\Im$");

    pair A=(60,20);
    pair B=(20,40);
    pair O=origin;
    pair II=(-40,20);

    draw(A--B,Arrow);
    draw(A*1.2--B*1.2,invisible);

    dot(A,blue);
    dot(B,blue);
    dot(II);

    label("$-2+i$",II,NW,blue);
    label("$A$",A,N,blue);
    label("$B$",B,N,blue);
    label("$O$",O,SW,blue);
  \end{asy}
  \caption{Kompleksinis skaičius $-2+i$ atitinka vektorių $\protect\vv{AB}=(-2,1)$,
einantį iš taško $A(3,1)$ į tašką $B(1,2)$.}
\end{figure}

Šie kompleksinių skaičių bei taškų ir vektorių atitikmenys leidžia geometrinius
sąryšius išreikšti naudojantis kompleksinių skaičių aritmetika. Pagrindinės
artimetinės operacijos -- sudėtis, daugyba ir kompleksinis jungtinumas -- turi
aiškius geometrinius atitikmenis ir sudaro kompleksinių skaičių geometrijos
pagrindą.

\subsection{Kompleksinių skaičių sudėtis kaip postūmis}

Kompleksinių skaičių sudėtis yra ekvivalenti vektorių sudėčiai. Jei taškus $A$
ir $B$ atitinka kompleksiniai skaičiai $a$ ir $b$, tai jų sumą $a + b 
= s$ atitiks taškas $S$ (ir tuo pačiu vektorius $\vv{OS}$), tenkinantis
$$
\vv{OA} + \vv{OB} = \vv{OS}.
$$
Lygiai tas pats galioja ir skirtumui: kompleksinių skaičių skirtumą $a - b =
d$ atitiks taškas $D$, tenkinantis $\vv{OA} - \vv{OB} = \vv{OD}$.
Pastebėkime, kad vektorius $\vv{OA} - \vv{OB}$ yra lygus vektoriui
$\vv{BA}$, tad

$$
\text{skirtumas } a - b \text{ atitinka vektorių }
\vv{BA}.
$$

\begin{figure}[h]
  \centering
  \begin{asy}
    size(200);
    graph.xaxis("$\Re$");
    graph.yaxis("$\Im$");

    pair A=(60,20);
    pair B=(20,40);
    pair O=origin;

    draw(O--A,Arrow);
    draw(O--B,Arrow);
    draw(O--A+B,dotted,Arrow);
    draw(O--A-B,dotted,Arrow);
    draw(B--A,Arrow);
    draw((A+B)*1.2,invisible);

    dot(A,blue);
    dot(B,blue);
    dot(A+B,blue);
    dot(A-B,blue);

    label("$A$",A,NE,blue);
    label("$B$",B,NE,blue);
    label("$a+b$",A+B,NE,blue);
    label("$a-b$",A-B,NE,blue);
    label("$O$",O,SW,blue);
  \end{asy}
  \caption{}
\end{figure}

Naudodamiesi šia kompleksinių skaičių ir vektorių atitikmenimi, galime
rasti atkarpos vidurį, bei trikampio pusiaukraštinių susikirtimo tašką:

\begin{teig}
  Atkarpos $AB$ vidurio tašką atitinka kompleksinis skaičius $\frac{a+b}{2}$.
\end{teig}

\begin{proof}
  Atkarpos vidurio tašką rasime prie vektoriaus $\vv{OA}$ pridėję pusę
  vektoriaus $\vv{AB}$, ką atitiks suma
  $$
  a + \frac{b - a}{2} = \frac{a + b}{2}.
  $$
\end{proof}

\begin{teig}
  Trikampio $ABC$ pusiaukraštinės susikirta viename taške, kurį atitinka
  kompleksinis skaičius $\frac{a + b + c}{3}$. Jis dalija kiekvieną iš
  pusiaukraštinių santykiu $2:1$.
\end{teig}

\begin{proof}
  Imkime tašką $M$, kurį atitinka kompleksinis skaičius $\frac{a + b + c}{3}$.
  Parodysime, kad šis taškas priklauso pusiaukraštinei $AM_{A}$ ir dalija ją
  santykiu $2:1$. Tam pakanka parodyti, kad 
  $$
  \vv{AM} = 2\vv{MM_{A}},
  $$
  arba, perrašius kompleksiniais skaičiais,
  $$
  \frac{a + b + c}{3} - a = 2\left(\frac{b+c}{2} - \frac{a + b + c}{3}\right).
  $$
  Ši lygybė akivaizdžiai teisinga, ir po tokią pačią galima gauti ir likusioms
  dviem pusiaukraštinėms.
\end{proof}

Panašiai į kompleksinių skaičių kalbą iš vektorių geometrijos galima perkelti ir
daugiau teiginių. Pavyzdžiui:

%\begin{teig}
%  Atkarpos $AB$ ir $CD$ yra lygiagrečios, jei egistuoja toks realusis skaičius
%  $k$, kad
%  $$
%  b-a = k(d-c).
%  $$
%\end{teig}
%
%\begin{teig}
%  Atkarpos $AB$ ir $CD$ yra vienodo ilgio, jei
%  $$
%  |b-a| = |d-c|.
%  $$
%\end{teig}

\begin{thm}[Oilerio tiesės teorema]
  Trikampio $ABC$ pusiaukraštinių
  susikirtimo taškas $M$ priklauso atkarpai $OH$, kur $O$ apibrėžtinio
  apskritimo centras,  ir $H$ aukštinių susikirtimo taškas, bei dalija ją
  santykiu $OM:MH = 1:2$, t.y.
  $$
  h  + 2o = a + b + c.
  $$
\end{thm}

\begin{sprendimas}
  Pirma įsitikinkime, kad kompleksinių skaičių lygybė tikrai nusako norimą trijų
  taškų priklausimą vienai tiesei bei atkarpų ilgių santykį. Tam pakanka
  prisiminti, kad pusiaukraštinių susikirtimo tašką atitinka kompleksinis
  skaičius $\frac{a+b+c}{3}$, ir perrašyti lygybę kaip
  $$
  h - \frac{a+b+c}{3} = 2\left(\frac{a+b+c}{3} - o\right).
  $$
  Lieka norimą lygybę įrodyti. Tą galima padaryti elegantiškai, parodant, kad
  taškas atitinkantis $a+b+c - 2o$ priklauso visoms trims trikampio aukštinėms.
  Imkime vektorių, einantį į šį tašką iš viršūnės $A$. Jį atitiks kompleksinis
  skaičius 
  $$
  a+b+c - 2o - a = (b-o) + (c-o),
  $$
  tad jis bus lygus vektorių $\vv{OB}$ ir $\vv{OC}$ sumai. O kandangi $O$ yra
  apibrėžtinio apskirtimo centras, tai ši vektorių suma bus statmena vektoriui
  $\vv{BC}$. Vadinasi taškas atitinkantis $a+b+c - 2o$ priklauso aukštinei
  $AA_H$, ir analogiškai aukštinėms $BB_H$ ir $CC_H$.
  \begin{figure}[h]
    \begin{center}
      \begin{asy}
	size(300);

	pair A = (35,60); 
	pair B = (100,0);
	pair C = (0,0);
	pair O = circumcenter(A,B,C);
	pair H = A + B + C - 2*O;
	pair X = (C + B - 2*O) + O;

    	add(rightanglem(O,(B+C)/2,B,125));

	dot(O,blue);
	dot(H,blue);

	draw(A--B--C--cycle);
	draw(O--C, Arrow);
	draw(O--B, Arrow);
	draw(A--H, dotted, Arrow);
	draw(O--X, Arrow);

	add(pathticks(O--B,2,0.5,0,125,red)); 
	add(pathticks(O--C,2,0.5,0,125,red)); 

	label("$A$",A,N,blue);
	label("$B$",B,E,blue);
	label("$C$",C,W,blue);
	label("$O$",O,N,blue);
	label("$a+b+c-2o$",H,S,blue);
	label("$\overrightarrow{OC} + \overrightarrow{OB}$",2*X/3+O/3,2*E,blue);
      \end{asy}
    \end{center}
  \end{figure}
\end{sprendimas}


\subsection{Kompleksinių skaičių daugyba kaip posūkis}

Priešingai nei kompleksinių skaičių sudėtis ir atimtis, daugybos ir dalybos
geometrinės interpretacijos tiesioginės atitikties vektorių geometrijoje neturi.
Dėka jų, naudojantis kompleksiniais skaičiais galima aprašyti kur kas
sudėtingesnius sąryšius.

Tegu $a=r_a e^{i\theta_a}$ ir $b=r_b e^{i\theta_b}$ du kompleksiniai skaičiai,
išreikšti per modulį ir argumentą. Šios išraiškos labai patogios norint juos
sudauginti:
$$
ab = r_a r_b e^{i(\theta_a + \theta_b)}.
$$
Iš šios lygybės matome, kad sandaugos $ab$ modulis lygus $a$ ir $b$ modulių
sandaugai, o argumentas lygus argumentų sumai. Kitaip tariant, daugybą iš
kompleksinio skaičiaus atitinka posūkis ir pailgėjimas (ar patrumpėjimas).
Posūkis dažnai bus gerokai svarbesnis, tad dažnai dauginsime iš kompleksinio
skaičiaus, kurio modulis lygus vienetui. Pavyzdžiui daugybą iš $i = e^{i\pi/2}$
atitinka posūkis stačiu kampu (aplink koordinačių pradžią, prieš laikrodžio
rodyklę):

\begin{figure}[h]
  \centering
  \begin{asy}
    size(200);
    graph.xaxis("$\Re$");
    graph.yaxis("$\Im$");

    pair O=origin;
    pair A=rotate(45,O)*(75,0);
    pair B=A*(0,1);
    pair C=rotate(-30, O)*A/2;

    add(rightanglem(A,O,B,200));
    add(anglem(C,O,A,300,red,0));

    draw(O--A);
    draw(O--B);
    draw(O--C);

    dot(A,blue);
    dot(B,blue);
    dot(C,blue);
    dot(1.2*C,invisible);

    label("$a$",A,NE,blue);
    label("$a\cdot \frac{1}{2}e^{-i\frac{\pi}{6}}$",C,NE,blue);
    label("$a\cdot i$",B,N,blue);
    label("$O$",O,SW,blue);
    label("$\frac{\pi}{6}$",O,(6,3.2),blue);
  \end{asy}
  \caption{}
\end{figure}

Neretai norint aprašyti geometrinį brėžinį reikia ne posūkio aplik koordinačių
pradžią, o posūkio aplink kurį nors kitą tašką. Kiek pamąsčius, tą padaryti
nesudėtinga.

\begin{pav}
  Kvadrato $ABCD$ viršūnės išvardintos prieš laikrodžio rodyklę. Jo viršūnes $C$
  ir $D$ atitinkančius kompleksinius skaičius $c$ ir $d$ išreikškite per $a$ ir
  $b$.
\end{pav}

\begin{sprendimas}
  Tašką $C$ gausime prie taško $B$ pridėję vektorių $\vv{BC}$, kuris yra lygus
  vektoriui $\vv{AB}$, pasuktam $90$ laipsnių kampu prieš laikrodžio rodyklę.
  Vektorių $AB$ atitinka kompleksinis skaičius $b-a$, o pasuktą $90$ laipsnių
  kampu, kompleksinis skaičius $(b-a)i$. Tad tašką $C$ atitiks komplesinis
  skaičius $b + (b-a)i$. Analogiškai, tašką $D$ atitiks kompleksinis skaičius
  $a + (b-a)i$.
  \begin{figure}[h]
    \centering
    \begin{asy}
      size(200);
      graph.xaxis("$\Re$");
      graph.yaxis("$\Im$");

      pair O=origin;
      pair A=(25, 10);
      pair B=A+(10,6);
      pair C=(B-A)*(0,1) + B;
      pair D=(B-A)*(0,1) + A;

      draw(A--B--C--D--A);
      draw(O--B-A);
      draw(O--(B-A)*(0,1));

      dot(A,blue);
      dot(B,blue);
      dot(C,blue);
      dot(D,blue);
      dot(B-A,blue);
      dot((B-A)*(0,1),blue);

      label("$O$",O,SW,blue);
      label("$A$",A,S,blue);
      label("$B$",B,E,blue);
      label("$b+(b-a)i$",C,N,blue);
      label("$a+(b-a)i$",D,W,blue);
      label("$b-a$",B-A,N,blue);
      label("$(b-a)i$",(B-A)*(0,1),N,blue);
    \end{asy}
    \caption{}
  \end{figure}
\end{sprendimas}

Naudojantis šiuo kvadrato viršūnių aprašymu galima išspręsti ganėtinai
nepaprastą uždavinį:

\begin{pav}
  Duotas keturkampis $ABCD$. Ant jo kraštinių $AB$, $BC$, $CD$ ir $DA$ nubrėžti
  išoriniai kvadratai, kurių centrai yra $P, Q, R, S$.  Įrodykite, kad atkarpa
  $PR$ yra lygi ir statmena atkarpai $QS$. 
\end{pav}

\begin{sprendimas}
  Tarkime, kad keturkampio $ABCD$ viršūnės yra išvardintos prieš laikrodžio
  rodyklę. Nagrinėkime išorinį kvadratą nubrėžta ant kraštinės $AB$. Iš praeito
  pavyzdžio žinome, kad jo įstrižai priešingą $A$ viršūnę atitiks kompleksinis
  skaičius $b + (a-b)i$ (kodėl ne $b + (b-a)i$ ?). Vadinasi jo centrą
  (įstrižainės vidurį) $P$ atitiks kompleksinis skaičius 
  $$
  p=\frac{a+b+(a-b)i}{2}.
  $$
  Analogiškai randame
  \begin{align*}
    q&=\frac{b+c+(b-c)i}{2}, \\
    r&=\frac{c+d+(c-d)i}{2},\\
    s&=\frac{d+a+(d-a)i}{2}.
  \end{align*}
  O iš čia galime rasti kompleksinius skaičius, atitinkančius vektorius $\vv{PR}$
  ir $\vv{QS}$:
  \begin{align*}
    r-p&=\frac{c+d-a-b+i(b+c-a-d)}{2},\\
    s-q&=\frac{a+d-b-c+i(c+d-a-b)}{2}.
  \end{align*}
  Matome, kad $s-q=(r-p)i$, todėl atkarpa $QS$ yra lygi ir statmena atkarpai $PR$.
  \begin{figure}
    \centering
    \begin{asy}
      size(200);

      pair A=(25, 10);
      pair B=A+(10,6);
      pair C=B+(-6,10);
      pair D=C+(-14,-4);
      pair P=(A+B+(A-B)*(0,1))/2;
      pair Q=(B+C+(B-C)*(0,1))/2;
      pair R=(C+D+(C-D)*(0,1))/2;
      pair SS=(D+A+(D-A)*(0,1))/2;

      draw(A--B--C--D--A);
      draw(P--R, dotted);
      draw(Q--SS, dotted);

      dot(A,blue);
      dot(B,blue);
      dot(C,blue);
      dot(D,blue);

      dot(P,blue);
      dot(Q,blue);
      dot(R,blue);
      dot(SS,blue);

      label("$A$",A,S,blue);
      label("$B$",B,E,blue);
      label("$C$",C,N,blue);
      label("$D$",D,W,blue);
      label("$\frac{a+b+(a-b)i}{2}$",P,S,blue);
      label("$\frac{b+c+(b-c)i}{2}$",Q,N,blue);
      label("$\frac{c+d+(c-d)i}{2}$",R,N,blue);
      label("$\frac{d+a+i(d-a)}{2}$",SS,S,blue);
    \end{asy}
  \end{figure}
\end{sprendimas}

Kompleksnių skaičių dalyba, kaip įprasta, yra ekvivalenti daugybai iš
atvirkštinio (kompleksinio skaičiaus $re^{i\theta}$ atvirkštinis  yra
$\frac{1}{r}e^{-i\theta}$). Todėl ją dažniau pasitelksime ne norėdami pasukti
turimą kompleksinį skaičių, o norėdami rasti (išreikšti) kampą ir modulių
santykį tarp jau turimų. Pavyzdžiui:

\begin{teig}
  Trikampiai $A_1B_1C_1$ ir $A_2B_2C_2$ (kurių viršūnės išvardintos prieš
  laikrodžio rodyklę) yra panašūs jei
  $$
  \frac{a_1-b_1}{c_1-b_1} = \frac{a_2-b_2}{c_2-b_2}.
  $$
\end{teig}

\begin{proof}
  Tam, kad trikampiai būtų panašūs, pakanka, kad jie turėtų po vieną lygų kampą
  $$
  \angle A_1B_1C_1 = A_2B_2C_2
  $$
  ir vienodo ilgių santykio gretimas kampui kraštines
  $$
  A_1B_1:C_1B_1 = A_2B_2:C_2B_2.
  $$
  Užrašę kairįjį santykį kaip kompleksinį skaičių
  $$
  \frac{a_1-b_1}{c_1-b_1} = r_1e^{i\theta_1}
  $$
  matome, kad $\theta_1$ kaip tik ir lygus kampui $\angle A_1B_1C_1$, o
  $r_1$ santykiui $A_1B_1:C_1B_1$. Analogiškai ir su dešiniuoju santykiu, tad
  jei gauti skaičiai $r_1e^{i\theta_1}$ ir $r_2e^{i\theta_2}$ lygūs, tai
  trikampiai panašūs.
\end{proof}

\begin{thm}
Taškai $a, b, c, d$ priklauso vienam apskritimui tada ir tik tada, kai: 
$$\frac{a - c}{b - c}:\frac{a-d}{b-d}\in\mathbb{R}.$$
\end{thm}

\begin{proof}
Iš Euklidinės geometrijos žinome, kad taškas $D$ priklauso apskritimui, apibrėžtam apie trikampį $ABC$, tada ir tik tada, kai arba $\angle BCA=\angle BDA$, arba $\angle BCA+\angle BDA=\pi $. Pirmu atveju, taškai $C, D$ turi būti toje pačioje tiesės $AB$ pusėje, antru - skirtingose. Nagrinėkime pirmąjį variantą (antrasis įrodomas analogiškai). 

\vspace{1cm}
\mbox{} % Thanks, Obama!
\vspace{-1cm}

\begin{figure}[h!]
  \centering
  \begin{asy}
    size(200);

    pair A=(25, 10);
    pair B=A+(10,0);
    pair C=B+(-1,7);
    pair X=(A-C)/(B-C);
    pair D=(A-B*X*0.3)/(1-X*0.3);

    add(anglem(A,C,B,30,red,0));
    add(anglem(A,D,B,30,red,0));

    draw(A--B--C--A);
    draw(B--D--A);
    draw(circumcircle(A,B,C));

    dot(A,blue);
    dot(B,blue);
    dot(C,blue);
    dot(D,blue);

    label("$A$",B,SE,blue);
    label("$B$",A,SW,blue);
    label("$D$",C,NE,blue);
    label("$C$",D,NW,blue);
    label("$Arg(\frac{a-d}{b-d})$",C,(-4,-10),blue);
    label("$Arg(\frac{a-c}{b-c})$",D,(4,-8),blue);
  \end{asy}
\end{figure}

Nagrinėkime teoremoje minimus santykius: 
$$
z_1 =\frac{a - c}{b - c}, \quad z_2 =\frac{a - d}{b - d}.
$$ 
Nesunku pamatyti, kad arba $Arg (z_1) = \angle BCA$ ir $Arg (z_2) = \angle BDA$ (kai $A, B, C$ ir $A, B, D$ išdėstyti pagal laikrodžio rodyklę), arba $Arg (z_1) = -\angle BCA$ ir $Arg (z_2) =- \angle BDA$  (kai $A, B, C$ ir $A, B, D$ išdėstyti prieš laikrodžio rodyklę). Bet kuriuo atveju $z_1/z_2$ bus ant realiosios ašies tada ir tik tada, kai $\angle BCA=\angle BDA$, ką ir reikėjo įrodyti.
\end{proof}

\subsection{Kompleksinės plokštumos parinkimas}

Šiame skyrelyje pamatėme, kad geometriniams taškams galime priskirti
kompleksinius skaičius ir geometrinius sąryšius perrašyti naudodamiesi
kompleksinių skaičių algebra. 

Kaip konkrečiai priskirsime taškams skaičius priklauso nuo mūsų ir beveik visada
neprarasdami bendrumo kompleksinę plokštumą galime parinkti laisvai. Tai yra:
pirma, bet kurį vieną tašką galime sutapatinti su kompleksinės plokštumos
pradžia, antra, likusius taškus galime pasukti norimu kampu aplink plokštumos
pradžią, ir trečia, visus taškus galime padauginti iš norimo realaus skaičiaus.

Pavyzdžiui, jei sąlygoje duotas lygiakraštis trikampis $ABC$, tai jį,
priklausomai nuo uždavinio ir kitų taškų, kompleksinėje plokštumoje galime
atidėti bent keliais prasmingais būdais. Galime parenkti kompleksinę plokštumą
bet kaip, galime sutapatinti kompleksinės plokštumos pradžią su vienu iš taškų,
arba galime visoms trims viršūnėms priskirti konkrečius kompleksinius skaičius
(lygties $z^3 = 1$ sprendinius), priklausančius vienetiniam apskritimui.

\begin{figure}[h]
  \centering
  \begin{asy}
    size(100);
    graph.xaxis("$\Re$");
    graph.yaxis("$\Im$");

    pair A=(10, 6);
    pair B=A+(10,2);
    pair C=A+(B-A)*(1/2, sqrt(3)/2);

    draw(A--B--C--A);

    dot(A,blue);
    dot(B,blue);
    dot(C,blue);
    dot((30,30),invisible);

    label("$A$",A,SW,blue);
    label("$B$",B,SE,blue);
    label("$a+(b-a)\omega$",C,N,blue);
  \end{asy}
  \hspace{1cm}
  \begin{asy}
    size(100);
    graph.xaxis("$\Re$");
    graph.yaxis("$\Im$");

    pair A=origin;
    pair B=A+(10,2);
    pair C=A+(B-A)*(1/2, sqrt(3)/2);

    draw(A--B--C--A);

    dot(A,blue);
    dot(B,blue);
    dot(C,blue);
    dot((20,20),invisible);

    label("$O$",A,SW,blue);
    label("$A$",B,SE,blue);
    label("$a\omega$",C,N,blue);
  \end{asy}
  \hspace{1cm}
  \begin{asy}
    size(100);
    graph.xaxis("$\Re$");
    graph.yaxis("$\Im$");

    pair omega = (-1/2, sqrt(3)/2);
    pair A=omega;
    pair B=A*omega;
    pair C=B*omega;

    draw(A--B--C--A);

    dot(A,blue);
    dot(B,blue);
    dot(C,blue);
    dot((2,2),invisible);

    label("$\omega$",A,NW,blue);
    label("$\omega^2$",B,SW,blue);
    label("$1$",C,NE,blue);
  \end{asy}
  \caption{Lygiakraštis trikampis plokštumoje: a.) laisvai; b.) viena viršūnė
    sutapatinta su plokštumos pradžia; c.) viršūnės ant vienetinio apskritimo,
    sutapatintos su kubinėmis šaknimis iš vieneto.}
\end{figure}

Šie skirtingi kompleksinės plokštumos parinkimo būdai leidžia sąryšius aprašyti
paprastėmis kompleksinių skaičių išraiškomis. Su tuo dažnai susidursime
tolimesniuose skyreliuose, o šį baigsime tinkamo plokštumos parinkimo naudą
iliustruojančiu pavyzdžiu.

\begin{pav}
Duotas trikampis $ABC$ ir aplink jį apibrėžtas apskritimas $S$. Šio apskritimo simetriški atspindžiai tiesių $AB$, $BC$, $CA$ atžvilgiu atitinkamai yra $S_{AB}, S_{BC}, S_{CA}$. Įrodykite, kad šie trys apskritimai kertasi viename taške.
\end{pav}

\begin{sprendimas}
  Neprarasdami bendrumo, tarkime, kad $S$ yra kompleksinės plokštumos vienetinis
  apskritimas, t.y., jo centrą $O$ atitinka plokštumos centras ($o=0$) ir jo
  spindulio ilgis lygus vienetui: 
  $$
  |a|=|b|=|c|=1.
  $$
  Tada $S_{AB}$ centras yra $o_{AB}=a+b$, nes $AB$ yra statmena $O_{AB}O$.
  Analogiškai, $S_{BC}$ ir $S_{CA}$ centrai atitinkamai yra $b+c$ ir $c+a$.
  Taškas $z$ priklauso visiems trims apskritimams tada ir tik tada, jeigu jis
  yra lygties $|z-(a+b)|=|z-(b+c)|=|z-(c+a)|=1$ sprendinys. Belieka pastebėti,
  kad šią lygybę tenkina aukštinių susikirtimo taškas $h=a+b+c$.
\end{sprendimas}

\subsection{Uždaviniai}

\begin{enumerate}
  \item Perrašykite kompleksiniais skaičiais teiginį „$ABCD$ yra
    lygiagretainis“. 
  \item Įrodykite, kad trikampio vidurio linija -- atkarpa, jungianti dviejų
    kraštinių vidurio taškus -- yra lygiagreti ir lygi pusei likusiosios
    kraštinės.
  \item Lygiakraščio trikampio $ABC$ viršūnę $C$ atitinkantį kompleksinį skaičių
    išreikškite per $a$ ir $b$.
  \item Tegu $AD$ apie trikampį $ABC$ apibrėžto apskritimo skersmuo, $H$ --
    aukštinių susikirtimo taškas, o $M$ -- kraštinės $BC$ vidurio taškas.
    Įrodykite, jog taškas $M$ yra atkarpos $DH$ vidurio taškas.
%Tegu vienetinis apskritimas būna apibrėžtas apie apskritimą $ABC$. Pagal Oilerio tiesės teoremą teoremą turime, kad 
%$$
%h = a+b+c.
%$$ 
%Kadangi $M$ yra $BC$ vidurio taškas, tai 
%$$
%m = \frac{b + c}{2}.
%$$
%Kadangi skersmens vidurio taškas yra koordinačių pradžios taškas ir jis yra lygus nuliui, tai taškas $D$ tenkina $d = -a.$ Galiausiai pastebime, kad $HD$ vidurio taškas yra 
%$$
%\frac{d + h}{2} = \frac {(-a) + a + b + c}{2}= \frac{b + c}{2}.
%$$
%O tai ir yra mūsų jau surastas taškas $M$.
\end{enumerate}


\section{Tiesės lygtis}
\subsection*{Tiesės lygtis}

Šiame skyriuje išmoksime apsirašyti tiesės lygtis, kurios padės surasti taškus, esančius ant lygiagrečių, statmenų ir susikertančių tiesių.

\begin{thmnr}
Tiesės lygtis, sudarančios kampą $\theta$ su $Re$ ašimi ir einančios per tašką $x \in \mathbb{C}$: $$z-x=e^{2i\theta}(\overline{z}-\overline{x}).$$
\end{thmnr}
\begin{sprendimas}
Pastebėkime, kad taškas $x$ tenkina lygtį, todėl visi $ z \neq x$ tenkina $$\frac{z-x}{\overline{z}-\overline{x}}=e^{2i\theta}.$$ Kadangi $|z-x|=|\overline{z}-\overline{x}|$, todėl $|(z-x)/(\overline{z}-\overline{x})|=1$. Be to, $Arg (z-x) =\phi$, kur $\phi$ yra kampas, kurį sudaro tiesė ZX su $Re$ ašimi. Todėl $Arg ((z-x)/(\overline{z}-\overline{x})) =2\phi$. Vadinasi, 
$$\frac{z-x}{\overline{z}-\overline{x}}=e^{2i\theta}$$ tada ir tik tada, kai $e^{2i\phi}=e^{2i\theta}$ $\Leftrightarrow 2\phi =2\theta + 2\pi n$, kažkokiam $n \in \mathbb{Z}.$ Paskutinė tapatybė yra ekvivalenti $\phi = \theta + \pi n$. Vadinasi, gavome, kad $z$ tenkina tiesės lygtį tada ir tik tada, kai $\phi = \theta$ arba $\phi = \theta + \pi$, bet antru atveju tai reiškia, kad vektorius $z-x$ yra paimtas priešinga kryptimi, todėl tiesės lygtis yra teisinga.
\end{sprendimas}




\bigskip
Dabar įrodysime vieną iš svarbesnių teoremų, padedančią aprašyti daugumą geometrinių taškų.

\begin{thmnr}\textit{(Lygiagretumo ir statmenumo teorema)}\
\begin{enumerate}
\item Tiesė AB yra lygiagreti tiesei CD tada ir tik tada, kai $\frac{a - b}{\overline{a} -\overline{b}}=\frac{c - d}{\overline{c} - \overline{d}}$.
\item Taškai A, B ir C priklauso vienai tiesei tada ir tik tada, kai $\frac{a - b}{\overline{a} -\overline{b}}=\frac{a-c}{\overline{a} - \overline{c}}$.
\item Tiesė AB yra statmena tiesei CD tada ir tik tada, kai $\frac{a - b}{\overline{a} -\overline{b}}=-\frac{c - d}{\overline{c} - \overline{d}}$.
\end{enumerate}
\end{thmnr}
\begin{sprendimas} Kad įrodytume pirmąją dalį, iš tiesės lygties nesunku pastebėti, kad tiesė $AB$ sudaro tokį pat kampą su $Re$ ašimi kaip ir tiesė $CD$ tada ir tik tada, jeigu tenkinama duota sąlyga. O tiesės yra lygiagrečios tada ir tik tada, kai jos sudaro tokį patį kampą su $Re$ ašimi (ar bet kuria kita tiese).

Antroji dalis gana akivaizdžiai seka iš pirmosios teoremos dalies.

Beliko įrodyti trečiąją dalį. Tarkime, kad $AB$ sudaro kampą $\theta$ su $Re$ ašimi, o $CD$ sudaro kampą $\phi$. Tada gauname, kad $\frac{a - b}{\overline{a} -\overline{b}}=-\frac{c - d}{\overline{c} - \overline{d}} \Leftrightarrow e^{2i\theta}=e^{i\pi}e^{2i\phi} \Leftrightarrow 2\theta = \pi + 2\phi + 2\pi n \Leftrightarrow \theta = \phi + \frac{\pi}{2} + \pi n $, kur $n$ - bet koks sveikas skaičius. Nagrinėdami paskutinę tapatybę matome, kad ji yra tenkinama tada ir tik tada, kai tiesės $AB$ ir $CD$ yra statmenos.
\end{sprendimas}

\begin{pavnr}
Duotas trikampis $ABC$. Tegu $AH$ yra trikampio aukštinė. Tada $h=\frac{a(\overline{b}-\overline{c})+\overline{a}(b-c)+\overline{b}c-b\overline{c}}{2(\overline{b}-\overline{c})}.$
\end{pavnr}
\begin{sprendimas}
Naudosimės lygiagretumo ir statmenumo teorema. $H$ priklauso tiesei $BC$, todėl $\frac{b - c}{\overline{b} -\overline{c}}=\frac{h-b}{\overline{h} - \overline{b}}$. Be to, $AH$ yra statmena $BC$, todėl $\frac{a - h}{\overline{a} -\overline{h}}=-\frac{b - c}{\overline{b} - \overline{c}}$. Iš pirmosios tapatybės gauname $\overline{h}(b-c)=h(\overline{b} - \overline{c}) - \overline{b}c+b\overline{c}$, o iš antrosios $\overline{h}(b-c)=a(\overline{b}-\overline{c})+\overline{a}(b-c)-h(\overline{b}-\overline{c})$. Sulyginę gauname išraišką $h(\overline{b} - \overline{c}) - \overline{b}c+b\overline{c}=a(\overline{b}-\overline{c})+\overline{a}(b-c)-h(\overline{b}-\overline{c})$, kurią sutvarkę gauname $h=\frac{a(\overline{b}-\overline{c})+\overline{a}(b-c)+\overline{b}c-b\overline{c}}{2(\overline{b}-\overline{c})}$.
\end{sprendimas}

Dažniausiai uždaviniuose išraiškas galima padaryti paprastesnėmis, pvz., neprarandant bendrumo tariant, kad $a=0$ arba kad trikampis $ABC$ yra ant vienetinio apskritimo, o tai duoda $|a|^2=a\overline{a}=b\overline{b}=c\overline{c}=1$.

\begin{pavnr}
Tegul $F$ yra taškas ant trapecijos $ABCD$ pagrindo $AB$, toks, kad $DF=CF$. Tegul E - trapecijos įstrižainių susikirtimo taškas, o taškai $O_1$ ir $O_2$ - apibrėžtų apie $ \Delta ADF$ ir $\Delta FBC$ apskritimų centrai. Įrodykite, kad $ FE \bot O_1 O_2 $.
\end{pavnr}
\begin{sprendimas}
Raktas į šio uždavinio sprendimą yra teisingas koordinačių centro parinkimas. 
Šiuo atveju, reiškiniai gražiai tvarkosi, jei koordinačių pradžios tašku pasirenkame tašką $F$. Tarkime, kad $Re$ (realiųjų skaičių ašis) yra statmena trapecijos pagrindams, o tada $d = \bar c ,$ nes $DF=CF$. Pastebime, jog tuomet taškai $a$ ir $b$ priklauso menamųjų skaičių ašiai $Im$, todėl gauname jog $ \bar a = - a $ bei $ \bar b = - b$.

Suraskime $O_1$ bei $O_2$ taškus. Šie taškai yra kraštinių vidurio statmenų sankirtos, todėl, naudojantis tiesės lygiagretumo ir statmenumo teoremos formulėmis, randame
$$ o_1 = \frac { ad( \bar a - \bar d) } { \bar a d - a \bar d } = \frac {\bar c ( a + c) } { \bar c + c },$$  
$$  o_2 = \frac { bc( \bar b - \bar c) } { \bar b c - b \bar c } = \frac { c( b + \bar c) } { \bar c + c }.$$

Liko surasti tašką $E$. Jį apibrėšime per dvi lygtis, naudojantis ta pačia teorema. Taškai $A, E, C$ bei $B, E, D$ priklauso vienai tiesei, todėl gauname
$$ \frac { a- c} {\bar a-\bar c} = \frac {e-a}{ \bar e - \bar a},$$
$$ \frac { b-d} {\bar b-\bar d} = \frac {e-b}{ \bar e - \bar b}.$$
\\ Išsireiškę iš šių dviejų lygčių  $ \bar e$ ir juos sulyginę, gauname
\\ $$ e = \frac { a \bar c - bc}{ a+ \bar c -b-c}.$$
Galiausiai liko įrodyti, kad $FE \bot O_1 O_2 $. Vėl naudojamės ta pačia teorema ir gauname, kad tai ekvivalentu
 $$  \frac {o_1 - o_2} { \bar o_1 - \bar o_2}= - \frac { f-e}{\bar f - \bar e}.$$
Šią lygybę įrodome įsistatę gautas išraiškas.
\end{sprendimas}


\section{Vienetinis apskritimas}

Kaip pamatysime šio skyriaus teoremoje, taškai ant vienetinio apskritimo turi daug gražių savybių, kurios supaprastina daug skaičiavimų. Todėl uždaviniuose yra labai svarbu protingai pasirinkti apskritimą, kurį laikysime vienetiniu. Pasirinktas apskritimas nebūtinai turi būti prasmingas iš uždavinio sąlygos perspektyvos. Vienetiniu apskritimu galima pasirinkti bet kurio trikampio apibrėžtinį (ar įbrėžtinį) apskritimą, o trikampį gali sudaryti bet kurie 3 taškai.

\begin{thmnr}  \textit{(Pagrindinė vienetinio apskritimo teorema)} Jei $A, B$ priklauso vienetiniam apskritimui, tada:\
\begin{enumerate}
\item $\frac{a - b}{\overline{a} -\overline{b}}=-ab$.
\item Jei $X$ priklauso tiesei AB, tai $\overline{x}=\frac{a + b - x}{ab}$.
\item Jei $C, D$ irgi priklauso vienetiniam apskritimui, tada $X$ yra tiesių AB ir CD susikirtimo taškas, tai $x=\frac{ab(c+d) - cd(a+b)}{ab - cd}$.
\end{enumerate}
\end{thmnr}
\begin{sprendimas}
\begin{enumerate} 
\item Naudojamės savybe, kad ant vienetinio apskritimo taškai tenkina $|a|^2=a\overline{a}=1$ ir gauname $\frac{a - b}{\overline{a} -\overline{b}}=\frac{a - b}{\frac{1}{a} -\frac{1}{b}}=-ab$.
\item Naudojamės tiesių lygiagretumo ir statmenumo teorema. Taškas $X$ priklauso tiesei $AB$ tada ir tik tada, kai $\frac{a-x}{\overline{a}-\overline{x}}=\frac{a-b}{\overline{a}-\overline{b}}$. Panaudoję $\frac{a - b}{\overline{a} -\overline{b}}=-ab$ ir suprastinę išraišką, gauname $\overline{x}=\frac{a + b - x}{ab}$.
\item Naudojamės šios teoremos antrąja dalimi. Gauname $\overline{x}=\frac{a + b - x}{ab}$ ir $\overline{x}=\frac{c+ d - x}{cd}$. Sulyginę išraiškas gauname $x=\frac{ab(c+d) - cd(a+b)}{ab - cd}$.
\end{enumerate}
\end{sprendimas}

Jeigu norime rasti liestinės prie $A$ ir $CD$ susikirtimo tašką, paprastas, bet matematiškai negriežtas būdas yra įsistatyti $b=a$ į teoremos paskutinės dalies formulę ir gauti $x=\frac{a^2(c+d) - 2acd}{a^2 - cd}$. Ieškodami liestinių prie $A$ ir $C$ susikirtimo taško, įsistatome $b=a, d=c$ ir gauname $x=\frac{2ac}{a+c}$. Šių formulių griežti įrodymai yra panašūs pateiktam įrodymui.
\bigskip

Dabar išspręsime kelis sudėtingesnius uždavinius.







\begin{pavnr} 
Duotas apie smailųjį trikampį $ABC$ apibrėžtas apskritimas, kurio liestinės iš taškų $A$ ir $B$ susikerta taške $X$. Jei $M$ yra kraštinės $AB$ vidurio taškas, įrodykite, kad $\angle ACX = \angle BCM$.
\end{pavnr}
\begin{sprendimas}
Naudodamiesi pagrindinę vienetinio apskritimo teorema, gauname, kad $ x= \frac {2ab}{a+b}$. Pasižymime $\theta = \angle ACX$ ir $\phi = \angle BCM$. Toliau išreiškiame kampus:
$$ e^{ 2i\theta} = \frac { a-c}{ \bar a - \bar c} : \frac {x-c}{\bar x - \bar c}=
- \frac { 2ab -ac-bc}{2bc - ab- b^2}.$$
Naudojame $m = \frac {a+b}{2}$ ir analogiškai randame:
$$ e^{2i\phi} = \frac { m-c}{ \bar m - \bar c} : \frac {b-c}{\bar b - \bar c}=
- \frac { 2ab -ac-bc}{2bc - ab- b^2}.$$
Taigi gauname $ e^{ 2i\theta} = e^{2i\phi}$. Iš to seka, kad arba $ \theta = \pi + \phi$, kas nėra įmanoma, nes $\Delta abc$ yra smailusis trikampis, arba $\theta =\phi$, ką mums ir reikėjo įrodyti.

\end{sprendimas}






\begin{pavnr}
(USAMO 2010) 

Duotas iškilas apibrėžtinis penkiakampis $AXYZB$ toks, kad $AB$ yra apibrėžtojo apskritimo skersmuo. Pažymėkime $P, Q, R, S$ statmenimis iš taško $Y$ tiesėms  $AX, BX, AZ, BZ$ atitinkamai. Įrodykite, kad smailusis kampas tarp tiesių $PQ$ ir $RS$ yra dvigubai mažesnis nei kampas $\angle XOZ$.
\end{pavnr}
\begin{sprendimas}
  Neprarasdami bendrumo tarkime, kad apie penkiakampį apibrėžtasis apskritimas yra vienetinis. Tuomet $b = -a$, nes $ab$ yra skersmuo. 

$P$ priklauso tiesei $AX$, todėl naudodamiesi pagrindine vienetinio apskritimo teorema, gauname $$\overline{p}=\frac{a+x-p}{ax}.$$ Tiesė $PY$ yra statmena tiesei $AX$, todėl iš tiesių lygiagretumo ir statmenumo teoremos gauname $$\frac{p-y}{\overline{p}-\overline{y}}=-\frac{a-x}{\overline{a}-\overline{x}} (=ax).$$ 
Įsistatę $\overline{p}$ į paskutinę tapatybę gauname 
$$p = \frac {1}{2}( a+x+y- \frac {ax}{y}).$$
Analogiškai randame
$$q = \frac {1}{2}( a+z+y- \frac {az}{y}),$$ 
$$r = \frac {1}{2}( -a+x+y- \frac {ax}{y}),$$
$$s = \frac {1}{2}( -a+z+y- \frac {az}{y}).$$

Iš tiesų, kompleksiniais skaičiais įrodyti kampų lygybes yra ganėtinai sudėtinga, kai kampai nėra lygūs. Šiuo atveju skaičiavimus palengvina Euklidinės geometrijos panaudojimas: pagal įbrėžtinius kampus turime $\angle XOZ = 2\angle XAZ$, todėl mums lieka įrodyti, jog  $\angle XAZ=\theta$ yra lygus kampui tarp tiesių $PQ$ ir $RS$, kurį pažymėkime $\phi$. Šie kampai yra lygūs tada ir tik tada, kai tenkinama 
$$e^{2i\theta}=\frac {x-a} {\bar x - \bar a} : \frac{z-a}{\bar z - \bar a} = \frac{p-r}{\bar p - \bar r} : \frac{q-s}{\bar q - \bar s}=e^{2i\phi}.$$
Įsistačius $p, q, r, s$ žinomas išraiškas per $a, x, y, z,$ gauname, kad lygybė teisinga, todėl uždavinys išspęstas.
\end{sprendimas}
     



\begin{pavnr}
(Baltic Way 2002) 

Tegu $ABC$ yra smailusis trikampis, toks, kad $ \angle BAC > \angle BCA$, ir tegu $D$ yra taškas ant kraštinės $AC$, kad $AB = BD$. Dar daugiau, $F$ yra toks taškas ant apibrėžto apie $ABC$ apskritimo, kad tiesė $FD$ yra statmena $BC$ bei taškai $F, B$ yra ant priešingų kraštinės $AC$ pusių. Įrodykite, kad tiesė $FB$ yra statmena kraštinei $AC$.
\end{pavnr}

\begin{sprendimas}
Tegu apie trikampį $ABC$ apibrėžtas apskritimas yra vienetinis. Išspręsime uždavinį iš kito galo. Paimsime tašką $F$ tokį, kad $BF$ statmena $CA$ ir $F$ priklauso apskritimui, apibrėžtam apie $ABC$. Tada $D$ bus toks kraštinės $AC$ taškas, kad $FD$ statmena $BC$. Tada įrodysime, kad $AB=BD$. Kadangi visi taškai randami vienareikšmiškai, tai uždavinys bus išspręstas.

Naudodami tai, kad $BF$ statmena $CA$, ir tai, kad $F$ yra ant vienetinio apskritimo, gauname
$$ f'= - \frac {ac}{b}.$$
Iš pagrindinės vienetinio apskritimo teoremos, gauname 
$$\bar d' = \frac {c+a-d}{ca}.$$
Naudodami tai, kad $FD$ statmena $BC$, randame
$$ d= b+ \frac {c}{b}(b-a).$$
Liko įrodyti, kad $AB=BD$, tad įsistatę $d$ išraišką, gauname:

$$ (d-b)( \bar d - \bar b)=( b+ \frac {c}{b}(b-a)-b)(( \bar b+ \frac {\bar c}{\bar b}(\bar b-\bar a)-\bar b)=(a-b)( \bar a - \bar b).$$
\end{sprendimas}




\begin{pavnr}
(IMO 2009)

Tegul $O$ yra apie trikampį $ABC$ apibrėžto apskritimo centras. Taškai $P$ ir $Q$ atitinkamai yra atkarpų $CA$ ir $AB$ vidiniai taškai. Tegul $\Gamma$ yra apskritimas, einantis per atkarpų $BP$, $CQ$ ir $PQ$ vidurio taškus $K, L$ ir $M$, o tiesė $PQ$ yra apskritimo $\Gamma$ liestinė. Įrodykite, kad $OP = OQ$.
\end{pavnr}
\begin{sprendimas}

Tegu apie $\bigtriangleup ABC$ apibrėžtas apskritimas yra vienetinis. Kadangi $P$ ir $Q$ priklauso atitinkamai $CA$ ir $AB$, tai
$$\overline{p}=\frac{a+c-p}{ac},$$
$$\overline{q}=\frac{a+b-q}{ab}.$$
Mums reikia įrodyti:
$$(p-o)(\overline{p-o})=(q-o)(\overline{q-o}) \Leftrightarrow p\overline{p}=q\overline{q} \Leftrightarrow \frac{p(a+c-p)}{ac}=\frac{q(a+b-q)}{ab}.$$

Liko neišnaudota viena sąlyga - tiesė $PQ$ liečia apskritimą $\Gamma$. Galima bandyti susirasti apskritimo centrą ir įrodyti, kad statmuo iš $M$ tiesei $PQ$ eina per $\Gamma$ centrą, arba naudotis tuo, kad $\angle QMK=\angle MLK$ tada ir tik tada, kai tiesė $PQ$ yra $\Gamma$ liestinė taške $M$. Pasižymime $\omega=\angle QMK=\angle MLK$, tada:
$$\frac{m-q}{|m-q|}\omega=\frac{m-k}{|m-k|} \Rightarrow \frac{m-q}{\overline{m}-\overline{q}}\omega^2=\frac{m-k}{\overline{m}-\overline{k}} \Leftrightarrow \omega^2=\frac{m-k}{\overline{m}-\overline{k}} \frac{\overline{m}-\overline{q}}{m-q}.$$
Analogiškai ir su m, l, k:
$$\frac{l-m}{|l-m|}\omega=\frac{l-k}{|l-k|} \Rightarrow  \omega^2=\frac{l-m}{\overline{l}-\overline{m}}\frac{\overline{l}-\overline{k}}{l-k}$$
Sulyginame reiškinius, įsistatome m, k ir l reikšmes, tada p ir q jungtinius:

\begin{equation*}
 \frac{m-k}{\overline{m}-\overline{k}} \frac{\overline{m}-\overline{q}}{m-q}=\frac{l-m}{\overline{l}-\overline{m}}\frac{\overline{l}-\overline{k}}{l-k} \Leftrightarrow 
\end{equation*}
\begin{equation*}
 \Leftrightarrow (q-b)(\overline{p}-\overline{q})(c-p)(\overline{c}+\overline{q}-\overline{b}-\overline{p})=(p-q)(\overline{q}-\overline{b})(c+q-b-p)(\overline{c}-\overline{p}) \Leftrightarrow
\end{equation*}
\begin{equation*}
\Leftrightarrow (q-b)(\frac{ab-ac+qc-pb}{abc})(c-p)(\frac{pb-qc}{abc})=(p-q)(\frac{b-q}{ab})(\frac{p-c}{ac})(c+q-b-p),
\end{equation*}
gautą lygybę galima nesunkiai suprastinti ir pertvarkyti į reiškinį, kurį mums reikėjo įrodyti.
\end{sprendimas}








\begin{pavnr}
Į trikampį $ABC$ įbrėžtas apskritimas su centru $I$ liečia kraštines $BC, CA, AB$ taškuose $D, E$ ir $F$ atitinkamai. Tegu $M$ ir $N$ yra atkarpų $AB$ ir $BC$ vidurio taškai, o $X$ yra taškas, kur susikerta tiesės $NM$ ir $DF$. Įrodyti (a) tiesės $IC, NM, FD$ eina per vieną tašką ir (b) $\angle AXC = \frac{\pi}{2}$. 
\end{pavnr}

\begin{sprendimas}
Pradėkime nuo (a) dalies. Įbrėžtinis į $\Delta ABC$ apskritimas bus vienetinis, todėl mes stengsimės visus taškus išreikšti per $d$, $e$ ir $f$. Tarkime, kad $X$ yra taškas, kuriame kertasi $DF$ ir $IC$. Tuomet užduotis prašo mūsų įrodyti, kad $X$ priklauso tiesei $NM$. 

Trikampio kraštinės yra įbrėžto trikampio liestinės, todėl turime, jog:
$$ a =\frac{2ef}{e+f},$$
$$ b=\frac{2df}{d+f},$$
$$ c=\frac{2ed}{e+d}.$$
Be to, galioja lygybės $\bar{d}= \frac{1}{d}, \bar{e}= \frac{1}{e}, \bar{f}= \frac{1}{f}$, nes taškai $d, e, f$ priklauso vienetiniam apskritimui. Iš to gauname
$$ \bar a =\frac{2}{e+f}, \bar b=\frac{2}{d+f}, \bar c=\frac{2}{e+d}.$$
Akivaizdu, kad
$$ m=\frac{a+b}{2}, \phantom{a} n=\frac{c+b}{2}.$$
Taškas $X$ priklauso tiesei $IC$:
$$ \frac{c-i}{\bar{c}-\overline{i}}=\frac{x-i}{\bar{x} - \overline{i}} {\phantom{a}}\Leftrightarrow \phantom{a} \bar{x}= \frac{\bar{c}}{c}\cdot  x $$ 
Taškas $X$ taip pat priklauso tiesei $FD$, todėl turime:
$$\frac{f-d}{\bar{f}-\bar{d}}=\frac{x-f}{\bar{x} - \bar{f}} \Leftrightarrow \bar{x}= -\frac{1}{fd}\cdot  (x-f)+\bar{f} $$
Sulyginame abi $\bar{x}$ išraiškas ir gauname, kad 
$$ x = e \cdot \frac{d+e}{f+e}.$$
Liko parodyti, jog gautas taškas $X$ priklauso tiesei $NM$, todėl naudojamės tiesių lygiagretumo ir statmenumo teorema:
$$ \frac{m-n}{\bar{m}-\bar{n}}=\frac{x-n}{\bar{x} - \bar{n}}.$$
Paskutinę lygybę įrodome įsistatę $m$, $n$ ir $x$ išraiškas per $d, e$ ir $f$.

Dalyje (b) tereikia įrodyti, jog galioja lygybė: $\frac{a-x}{\bar{a}-\bar{x}}= - \frac{x-c}{\bar{x} - \bar{c}}$, ką nesunkiai padarome įsistatę reikiamas išraiškas.
\end{sprendimas}










\begin{pavnr} 
(MEMO 2010)

Duotas keturkampis $ABCD$, apie kurį galima apibrėžti apskritimą. $E$ yra toks įstrižainės $AC$
taškas, kad $AD = AE$ ir $CB = CE$. $M$ yra apskritimo $k$, apibrėžto apie trikampį $BDE$,
centras. Apskritimas $k$ kerta tiesę $AC$ taškuose $E$ ir $F$ . Įrodykite, kad tiesės $FM, AD$ ir $BC$ kertasi viename taške.
\end{pavnr}
\begin{sprendimas}
Tegu apie $B, D, E$ apibrėžtas apskritimas yra vienetinis apskritimas kompleksinėje plokštumoje, tada $F$ irgi priklauso šiam apskritimui, o $m=0$. Kadangi $A\in EF$, iš šeštos teoremos gauname: $\overline{a}=\frac{e+f-a}{ef}$. $AE = AD \Leftrightarrow |a-e|=|a-d| \Leftrightarrow (a-e)\overline{(a-e)}=(a-d)\overline{(a-d)}$. Iš šių lygčių randame
$$a=\frac{d(e+f)}{d+f}.$$
Analogiškai randame
$$c=\frac{b(e+f)}{b+f}.$$
Kadangi $A, B, C, D$ priklauso vienam apskritimui, atlikę prastinimus gauname
$$\frac{a-c}{b-c}:\frac{a-d}{b-d}\in \mathbb{R} \Leftrightarrow \frac{a-c}{b-c}:\frac{a-d}{b-d}= \overline{\bigg(\frac{a-c}{b-c}:\frac{a-d}{b-d}\bigg)} \Leftrightarrow bed=f^3.$$
Tarkime, kad $FM$ ir $AD$ kertasi taške $x_1$, o $FM$ ir $BC$ taške $x_2$. Kadangi $X_1, X_2 \in FM$, tai iš tiesių lygiagretumo ir statmenumo teoremos gauname $\frac{x_i-m}{(\overline{x_i}-\overline{m})}=\frac{f-m}{(\overline{f}-\overline{m})} \Leftrightarrow          \overline{x_i}=\frac{x_i}{f^2}$ ($i \in \{1, 2\})$.
Kadangi $X_1 \in AD$, gauname $\frac{x_1-d}{(\overline{x_1}-\overline{d})}=\frac{a-d}{(\overline{a}-\overline{d})} \Rightarrow x_1=\frac{df^2(e+f)}{(f^3+d^2e)}$,
analogiškai $x_2=\frac{bf^2(e+f)}{(f^3+b^2e)}$.
Iš čia $bed=f^3 \Rightarrow x_1=x_2.$
\end{sprendimas}





\begin{pavnr}
(IMO 2009)

Tegul $ABC$ yra lygiašonis trikampis, kuriame $AB =
AC$. Kampo $CAB$ pusiaukampinė kerta kraštinę $BC$ taške $D$, o kampo
$ABC$ pusiaukampinė kerta kraštine $CA$ taške $E$. Taškas $K$ yra įbrėžto
į trikampį $ADC$ apskritimo centras, o $\angle BEK = 45^{\circ}$. Raskite visas
įmanomas $\angle CAB$ reikšmes.
\end{pavnr}
\begin{sprendimas}
Spręsdami geometrinius uždavinius kompleksiniais skaičiais, dažnai turime performuluoti sąlygą taip, kad sprendžiant gautume paprastesnius reiškinius ir lygybės netaptų pernelyg komplikuotos. 

Pažymime tiesių $AD$ ir $CK$ susikirtimo tašką $G$, o $\angle ACD = 2\alpha$. Tada $\angle KEC=3(45^{\circ}-\alpha).$ Sąlygas, kad $AB=AC$ ir kampas $BEK=45^\circ$, išnaudosime pamatę, kad kampas $ADC=90^\circ$ ir tuo, kad $G$ yra pusiaukampinių susikirtimo taškas. Todėl spręsdami uždavinį, tašką $B$ galime pamiršti, o suradę $\alpha$ reikšmes, nesunkiai surasime ir ieškomojo kampo reikšmes.

Tegu įbrėžtinis į trikampį $ADC$ apskritimas yra vienetinis ir liečia kraštines $DA, AC, CD$  taškuose $P, Q$ ir $R$. Pasižymime $\phi=e^{i45^{\circ}}$ ir $\omega=e^{i\alpha}$. Tada $r=p\phi^2$, nes $\angle PKR=90^{\circ},$ o taip yra dėl to, kad $PKRD$ yra kvadratas.
Kadangi $G$ priklauso liestinei taške $P$, tai $\overline{g}=\frac{2p-g}{p^2}$, ir $G\in CK$, todėl $\frac{c}{\overline{c}}=\frac{g}{\overline{g}}$. Be to, $c=\frac{2pq\phi^2}{p\phi^2+q}$, tai
$$g=\frac{2pq\phi ^2}{p+q\phi^2}.$$
Kadangi $E$ priklauso liestinei taške $Q$, tai $\overline{e}=\frac{2q-e}{q^2},$ ir $\angle KEQ=3(45^{\circ}-\alpha) \Rightarrow  \frac{e}{\overline{e}}\frac{\phi^6}{\omega^6}= \frac{e-q}{\overline{e}-\overline{q}}$ ir $\phi^4=-1$, tai
$$e=\frac{2q\omega^6}{\omega^6 +\phi ^2}.$$
\begin{equation*}
\angle GEK = 45^{\circ} \Rightarrow \frac{e-g}{\overline{e}-\overline{g}}\phi^2=\frac{e}{\overline{e}}.\tag{*}
\end{equation*}
Apskaičiuojame:
$$\frac{e}{\overline{e}}= \frac{\omega^6q^2}{\phi^2}$$
$$\frac{1}{2}(e-g)=\frac{q(q\omega^6\phi^2 +p\omega^6 -p\omega^6\phi^2 +p)}{(\omega^6+\phi^2)(p +q\phi^2)}$$
$$\frac{1}{2}(\overline{e}-\overline{g})=\frac{p\phi^2 -q\omega^6 -q\phi^2 - q}{q(\omega^6+\phi^2)(p +q\phi^2)}.$$
Įsistatome į (*) ir suprastinę gauname:
\begin{equation*}
(q\omega^6 -p)(\omega^6+1)=0. \tag{**}
\end{equation*}
Išsprendę $(\omega^6+1)=0$ ir atsižvelgę į tai, kad $ \alpha < 45^{\circ}$, gauname vienintelę reikšmę $\alpha = 30^{\circ}$. Tada mūsų ieškomas kampas yra $60^{\circ}$.
Antras atvejis, kai $q\omega^6 =p$. Nesunkiai apskaičiuojame (tiesiog susižymėję kampus), kad $\angle PKQ=90^{\circ} +2\alpha$, todėl gauname:
\begin{equation*}
\omega^6=e^{i6\alpha}=e^{i(90^{\circ} +2\alpha)} \Leftrightarrow 6\alpha=90^{\circ} +2\alpha +180^{\circ}k \text{ (kažkokiam } k\in \mathbb{Z}),
\end{equation*}
Matome, kad tinka tik viena $\alpha$ reikšmė, kai $4\alpha=90^{\circ}$, tad ieškomas kampas yra lygus $90^{\circ}$.

Kadangi iš pradinių sąlygų gavome (**), tai dar nereiškia, kad abu sprendiniai tenkina pradinę sąlygą, todėl turime juos abu įsistatyti ir patikrinti, ar trikampiai, tenkinantys visas sąlygas, egzistuoja. Galima tarti, kad $\angle CAB= 60^{\circ},90^{\circ}$ ir parodyti, kad $\angle BEK =45^{\circ}$.
\end{sprendimas}




















\section{Vienetinis apskritimas kitu kampu}

Iš tiesų, šio skyriaus uždaviniai mažai skiriasi nuo praeitojo, nes mes vėl turime vieną pagrindinį apskritimą, kurį laikome vienetiniu. Visgi formulės, kurias čia naudosime, yra kiek subtilesnės. Išmoksime metodą, kaip rasti apskritimo lankų vidurio taškus, ir kaip rasti įbrėžtinio apskritimo centrą. Taip susiesime apibrėžtinį apskritimą su įbrėžtiniu.

\begin{thmnr} \textit{(Kvadratinė vienetinio apskritimo teorema)} Jei $a, b$ ir $c$ priklauso vienetiniam apskritimui, tada egzistuoja $u, v, w$ tokie kad $a=u^2, b=v^2, c=w^2$. Be to:\
\begin{enumerate}
\item $-uv$, $-vw$, $-wu$ yra $\smile ab, \smile bc, \smile ca$ vidurio taškai, kuriems nepriklauso taškai $c, a, b$ atitinkamai.
\item Jei i yra $\bigtriangleup$abc įbrėžtinio apskritimo centras, tai $i= - (uv + vw + wu)$.
\end{enumerate}
\end{thmnr}

\begin{sprendimas}
Neprarasdami bendrumo tarkime, kad $a=e^{i\theta_a}, b=e^{i\theta_b}, c=e^{i\theta_c}$, kur $0 \leq \theta_a < \theta_b < \theta_c < 2\pi$. Tada paimkime $u=e^{i\frac{1}{2}\theta_a}, v=e^{i(\frac{1}{2}\theta_b+\pi)}, w=e^{i\frac{1}{2}\theta_c}$. Nesunku pastebėti, kad $a=u^{2}, b=v^{2}, c=w^{2}$.

\begin{enumerate}
\item Tada $-uv=e^{i\frac{1}{2}(\theta_a+\theta_b)}, -vw=e^{i\frac{1}{2}(\theta_b+\theta_c)}, -wu=e^{i(\frac{1}{2}(\theta_a+\theta_b)+\pi)}$ ir yra atitinkamų lankų vidurio taškai. Pastebėkime, kad paskutinio nario laipsnyje narys $i\pi$ yra neatsitiktinai.
\item Pažymėkime $p_C=-uv$. Įrodysime, kad taškas $I$ priklauso pusiaukampinei $P_CC$, analogiškai jis turės priklausyti ir kitoms dviem pusiaukampinėms, todėl $I$ bus jų susikirtimo taškas, kuris ir yra įbrėžtinio apskritimo centras.

Mums reikia įrodyti, kad taškas $i=-(uv+vw+wu)$ priklauso tiesei $P_CC$. Iš tiesių lygiagretumo ir statmenumo teoremos gauname, kad tai yra ekvivalentu $$\frac{c-i}{\overline c - \overline i}=\frac{c-p_C}{\overline {c} - \overline {p_C}}.$$
Įsistatome reikšmes $i=-(uv+vw+wu), p_C=-uv, c=w^2$, nepamirštame, kad $u, v, w$ visi yra ant vienetinio apskritimo ir įrodome, kad lygybė yra teisinga.
\end{enumerate}
\end{sprendimas}






\begin{pavnr}
(LMMO 2005)

Apie trikampį $ABC$ apibrėžtas apskritimas. Taškas $M$ yra lanko $AC$ (kuriam nepriklauso viršūnė $B$) vidurio taškas, o $N$ yra lanko $AB$ (kuriam nepriklauso viršūnė $C$) vidurio taškas. Atkarpos $MN$ ir $AB$ kertasi taške $K$. Įbrėžto į trikampį $ABC$ apskritimo centras yra $O$. Įrodykite, kad $KO$ yra lygiagreti kraštinei $AC$. 
\end{pavnr}
\begin{sprendimas}
Tarsime, kad apie trikampį $ABC$ apibrėžtasis apskritimas yra vienetinis. Šiame uždavinyje mums reikia gudriai apsirašyti taškus $M, N$ ir $O$, todėl mes taikome kvadratinę vienetinio apskritimo teoremą:
$$a = x^2,$$
$$b=y^2,$$
$$c = z^2,$$
$$ o = - (xy+xz+yz),$$
$$m = - xz,$$
$$ n = - xy.$$
Kadangi $K$ yra stygų $MN$ ir $AB$ susikirtimo taškas, tai tašką $K$ randame naudodami pagrindinę vienetinio apskritimo teoremą:
$$k = \frac{ x^2yz(x^2+y^2)+ x^2 y^2 (xy+xz)}{ x^2 yz - x^2 y^2},$$
$\phantom{a}$ 
Galiausiai, naudojant tiesių lygiagretumo ir statmenumo teoremą, įrodome, kad $KO$ yra statmena $AC$:
$$\frac{k-o}{\overline k - \overline o}  = - \frac{ x^2 - z^2}{\overline x^2 - \overline z^2},$$
Suprastinę gauname, kad abi pusės lygios $x^2 z^2$. Taigi, gavome būtent tai, ką reikėjo įrodyti.
\end{sprendimas}







\begin{pavnr}
(IMO 2010)

Tegul $I$ yra trikampio $ABC$ pusiaukampinių susikirtimo taškas, o $\Gamma$ – apie trikampį $ABC$ apibrėžtas apskritimas. Tiesė $AI$ kerta $\Gamma$ taške $D$, kur $D\ne A$. Taškas $E$ priklauso apskritimo $\Gamma$ lankui $\smile BDC$, o taškas $F$ – atkarpai $BC$. Be to, $\angle BAF = \angle CAE < \frac{1}{2}\angle BAC$. Taškas $G$ yra atkarpos $IF$ vidurio taškas. Įrodykite, kad tiesių $DG$ ir $EI$ susikirtimo taškas priklauso apskritimui $\Gamma$.
\end{pavnr}

\begin{sprendimas}

Tegu $\Gamma$ yra vienetinis apskritimas kompleksinėje plokštumoje. Naudojamės kvadratine vienetinio apskritimo teorema, tada $a=u^2, b=v^2, c=w^2$ ir $i = -(uv+vw+wu).$

Tegu $AF$ kerta $\Gamma$ taške $H\ne A$. Tada $\angle BAH =\angle EAC =\alpha$, pažymime $\phi=e^{i\alpha}$. Tuomet $\frac{h-a}{|h-a|}\phi=\frac{b-a}{|b-a|}.$ Pakėlę kvadratu ir pasinaudoję tiesių lygiagretumo ir statmenumo teorema, gauname $h=\frac{v^2}{\phi^2}$. Analogiškai randame $e=\phi^2w^2$.

Tegu $EI$ kerta $\Gamma$ taške $X \ne E$. Iš tiesių lygiagretumo ir statmenumo teoremos gauname $\frac{e-x}{\overline{e}-\overline{x}}=\frac{e-i}{\overline{e}-\overline{i}}$ ir kadangi $x\overline{x}=1$, tai 
$$x=-uvw\frac{w^2\phi ^2 +uv+vw+wu}{uvw +w^2\phi ^2(u+v+w)}.$$

Prisiminę tiesių lygiagretumo ir statmenumo teoremą, randame 
$$f=\frac{v^2w^2(u^2 + \frac{v^2}{\phi ^2}) - \frac{u^2v^2}{\phi ^2}(v^2+w^2)}{v^2w^2 - \frac{u^2v^2}{\phi ^2}}=\frac{w^2(u^2\phi^2 + v^2) - u^2(v^2+w^2)}{w^2\phi^2 - u^2}.$$

Taškas $d=-vw$ pagal kvadratinę vienetinio apskritimo teoremą, nes $AI$ yra trikampio $ABC$ pusiaukampinė.

Galiausiai $g=\frac{1}{2}(f+i)$, tad pagal teoremą tiesėms įrodome, kad $d, g$ ir $x$ priklauso vienai tiesei.
\end{sprendimas}

Nemažai šio uždavinio skaičiavimų yra gana komplikuoti, bet gudriai apsirašydami taškus išvengėme dar baisesnių išraiškų, o šios dar pasirodė įveikiamos. Tokiuose uždaviniuose labai svarbu būti atidžiam ir nedaryti klaidų atliekant reiškinių pertvarkymus.













\begin{thmnr} \textit{(Įbrėžtinio vienetinio apskritimo teorema)} Duotas trikampis $ABC$, kurio įbrėžtinis apskritimas yra vienetinis ir jis liečia trikampio kraštines $BC, CA, AB$ atitinkamai taškuose $P, Q, R,$ tada:
\begin{enumerate}
\item Galioja $a=\frac{2qr}{q+r}$, $b=\frac{2rp}{r+p}$, $c=\frac{2pq}{p+q}$.
\item Jei $H$ yra $\bigtriangleup ABC$ aukštinių susikirtimo taškas, tai
$$h=\frac{2(p^2q^2 + q^2r^2 + r^2p^2 + pqr(p + q+ r))}{(p+q)(q+r)(r+p)}.$$
\item Jei $O$ yra apibrėžtinio apie  $\bigtriangleup ABC$ apskritimo centras, tai
$$o=\frac{2pqr(p+q+r)}{(p+q)(q+r)(r+p)}.$$
\end{enumerate}
\end{thmnr}

\begin{sprendimas}
\begin{enumerate}
\item Ši dalis yra analogiška pagrindinei vienetinio apskritimo teoremos daliai.
\item Pastebėkime, kad $AH$ yra lygiagreti $IP$, nes jos abi yra statmenos $BC$. Vadinasi $\frac{a-h}{\overline a-\overline h}=\frac{i-p}{\overline i-\overline p}.$ Naudojamės $i=0$ ir suprastinę gauname $$\overline {h}= \frac{2}{q+r} + \frac{1}{p^2}(h-\frac{2qr}{q+r}).$$ Analogiškai $BH$ lygiagreti $IQ$, todėl $$\overline {h}= \frac{2}{r+p} + \frac{1}{q^2}(h-\frac{2rp}{r+p}).$$
Sulyginę ir suprastinę gauname $$h=\frac{2(p^2q^2 + q^2r^2 + r^2p^2 + pqr(p + q+ r))}{(p+q)(q+r)(r+p)}.$$
\item Naudojamės formule $h+2o=a+b+c$, kuri ekvivalenti $o=\frac{a+b+c-h}{2},$ įsistatome žinomas $a, b, c, h$ reikšmes ir suprastinę gauname
$$o=\frac{2pqr(p+q+r)}{(p+q)(q+r)(r+p)}.$$
\end{enumerate}
\end{sprendimas}






\begin{pavnr}
(APMO 2005)

Tegu $ABC$ smailusis trikampis, kurio $\angle BAC = \frac{\pi}{3}$ ir $AB>AC$. Tegu $I$ - įbrėžtinio apskritimo centras, o $H$ - aukštinių susikirtimo taškas. Įrodykite, kad $2\angle AHI=3\angle ABC$.
\end{pavnr}

\begin{sprendimas}
Tegu įbrėžtinis į trikampį $ABC$ apskritimas yra vienetinis. Jis liečia kraštines $AB, BC, CA$ taškuose $R, P, Q$. Pažymėkime $\angle AHI = \alpha, \angle ABC = \beta$ ir $\omega = e^{i\alpha}, \phi = e^{i\beta}$.

Tada $i=0$, $a=\frac{2rq}{r+q}$, $b=\frac{2rp}{r+p}$, o $\angle RIQ=\frac{2}{3}\pi$, nes tai randame iš keturkampio $RIQA$ kampų. Gauname, kad $e^{i\frac{2}{3}\pi}=\frac{r}{q}$, todėl $\Big(\frac{r}{q}\Big)^3=1$. Išskaidę $r^3-q^3=0$ ir pasinaudoję $r\ne q$, gauname
\begin{equation}
r^2 + rq + q^2 = 0. \tag{$*$}
\end{equation}
Ši lygybė bus būdas, kuriuo mes išnaudosime sąlygą, kad $\angle BAC = \frac{\pi}{3}$.
Naudodami įbrėžtinio vienetinio apskritimo teoremą ir $(*)$ gauname:
$$h=\frac{2(p^2(q^2 + r^2) + q^2r^2  + pqr(p + q+ r))}{(p+q)(q+r)(r+p)}$$
$$=\frac{2(p^2(-rq) + q^2r^2  + pqr(p + q+ r))}{(p+q)(q+r)(r+p)}=\frac{2qr(pq +qr+rp)}{(p+q)(q+r)(r+p)}.$$
Žinome, kad $\omega$ tenkina:
$$\frac{a-h}{\overline{a}-\overline{h}}\omega^2=\frac{i-h}{\overline{i}-\overline{h}} \Leftrightarrow \omega^2=\frac{qr}{p^3}\frac{(pq +qr+rp)}{(p+q+r)}.$$
Naudodamiesi, kad liestinių ilgiai yra lygūs, gauname, kad $\phi$ tenkina:
$$\frac{p-b}{|p-b|}\phi=\frac{r-b}{|r-b|} \Rightarrow \phi^3=-\frac{r^3}{p^3}.$$
Pasinaudoję $(*)$ įrodome, kad $\phi^3=\omega^2.$ Iš to seka, kad
$$3\beta=2\alpha + 2\pi k \text{, kažkokiam } k\in\mathbb{Z}.$$
Beliko išnagrinėti kelis atvejus. Jeigu $k>0$, tai $\beta>\frac{2}{3}\pi$, o tai nėra įmanoma. Jei $k<0$, tai $\alpha > \pi$ (tai reiškia, kad kampas, kurį matavome, yra išorinis), todėl šiuo atveju taškai $B$ ir $I$ bus skirtingose aukštinės $AH$ pusėse. O būtent šiuo atveju $AB<AC$ (tai galime pamatyti, pvz., pastebėdami, kad $\angle AHI < \frac{\pi}{6}$, todėl $\beta > \frac{\pi}{3}$ ir iš to gauname, kad $\angle ABC > \angle ACB$), bet tai nėra teisinga. Todėl $k=0$ ir $3\beta=2\alpha$.
\end{sprendimas}


















\section{Figūrų plotas}

Bet kurios figūros plotą galima apskaičiuoti suskaidant tą figūrą į trikampius, apskaičiuoti tų trikampių plotą ir viską sudėti. Šiame skyriuje išmoksime trikampio ploto formulę, kuria naudojantis galima apskaičiuoti bet kurios figūros plotą, ir išspręsime kelis su ja susijusius uždavinius.

\begin{thmnr} \textit{(Trikampio ploto teorema)}
Kai viršūnės $A, B, C$ išdėstytos prieš laikrodžio rodyklę, trikampio $ABC$ plotas lygus $S$, kur 
$$S=\frac{i}{4}(a\overline{b} +b\overline{c}+c\overline{a} - b\overline{a} -c\overline{b}-a\overline{c}).$$
Jeigu $A, B, C$ išdėstytos pagal laikrodžio rodyklę, trikampio $ABC$ plotas lygus $-S$.
\end{thmnr}

\begin{sprendimas}
Nagrinėkime atvejį, kai viršūnės $A, B, C$ išdėstytos prieš laikrodžio rodyklę, kitas atvejis įrodomas analogiškai.

Tegu trikampio $ABC$ aukštinė yra $BD$. Tuomet $c-a=|c-a|e^{i\theta}$, be to $\angle BDC = \frac {\pi}{2},$ todėl $(b-d)i=|b-d|e^{i\theta}$. Gauname, kad $\frac{1}{2}\overline{(c-a)} (b-d)i=\frac{1}{2}|c-a||b-d|=S.$

Naudodamiesi tuo, kad taškas $D$ priklauso atkarpai $AC$, bei tuo, kad $BD$ yra statmena $AC$, randame tašką $D$:
$$ d =\frac {1}{2}(a+b + (\bar{b}-\bar{a}) \frac{a-c}{ \bar{a}-\bar{c}}).$$
Įsistatę gautąją $d$ išraišką į $S=\frac{1}{2}\overline{(c-a)} (b-d)i$ bei atlikę prastinimo veiksmus, gauname, kad
$$S= \frac {i}{4}(a\bar{b}+ b \bar{c} + c \bar{a} - b\bar{a} - c \bar{b} - a \bar{c}).$$
\end{sprendimas}



\begin{pavnr}
(LMMO 2008)

Apie smailųjį trikampį $ABC$ apibrėžtas apskritimas. Atkarpa $BD$ yra to
apskritimo skersmuo. Iš viršūnės $A$ nubrėžta aukštinė kerta apskritimą
taške $E$. Įrodykite, kad keturkampio $BECD$ plotas yra lygus trikampio
$ABC$ plotui.
\end{pavnr}
\begin{sprendimas}
Tegu apie $ABC$ apibrėžtas apskritimas bus vienetinis. Tada gauname, kad $d = -b$ bei iš pagrindinės vienetinio apskritimo teoremos gauname
 $$ e= -\frac {bc}{a}= -bc \overline{a}.$$
Liko įrodyti tai, kas yra prašoma, naudojant plotų teoremą.

$$\frac{4}{i}S_{BEC}=b\overline{e}+e\overline{c}+c\overline{b}-\overline{b}e-\overline{e}c-\overline{c}b,$$

$$\frac{4}{i}S_{BCD}=b\overline{c}+c\overline{d}+d\overline{b}-\overline{b}c-\overline{c}d-\overline{d}b.$$
Įsistatome $d$ ir $e$ reikšmes ir gauname, ką ir reikėjo įrodyti:
$$ \frac{4}{i}S_{BECD}=\frac{4}{i}(S_{BEC}+S_{BCD})=   
a\bar b + b \bar c + c \bar a - \bar a b - \bar b c - \bar c a = \frac{4}{i}S_{ABC}.$$
\end{sprendimas}



\begin{pavnr}
(IMO 2007)

Trikampio $ABC$ kampo $BCA$ pusiaukampinė kerta apibrėžtą apie $ABC$ apskritimą kitame taške $R$. Tarkime, kad $K$ yra atkarpos $BC$ vidurio taškas, o $L$ yra atkarpos $AC$ vidurio taškas. Tiesė, kuri eina per tašką $K$ ir yra statmena atkarpai $BC$, kerta tiesę $CR$ taške $P$, o tiesė, kuri eina per tašką $L$ ir yra statmena atkarpai $AC$, kerta tiesę $CR$ taške $Q$. Įrodykite, kad trikampių $RPK$ ir $RQL$ plotai yra lygūs.
\end{pavnr}
\begin{sprendimas}

Naudosimės kvadratine įbrėžtinio apskritimo teorema - tarsime, kad $a=u^2, b=v^2$ ir $c=w^2$. Tada $r=-uv$, o $l=\frac{u^2+w^2}{2}$. Pasinaudoję tuo, kad $Q$ priklauso stygai $RC$, ir tuo, kad $QL$ statmena $AC$, randame $q=\frac{u(w^2 - uv)}{u-v}$. Pažymime trikampio $RQL$ plotą S, o $t=(r\overline{q} +q\overline{l}+l\overline{r})$, tada $S=\frac{i}{4}(t-\overline{t})$. 
Randame
$$t=\frac{w^4(2v-u) +w^2(2u^2v-u^3-3uv^2) +2u^2v^3 - u^3v^2}{2uvw^2(u-v)},$$
$$-\overline{t}=\frac{w^4(2u-v) +w^2(2uv^2-v^3-3u^2v) +2u^3v^2 - u^2v^3}{2uvw^2(u-v)}.$$
Gauname
$$S=\frac{i}{4}(t-\overline{t})=\frac{i}{4}\frac{(w^4(u+v)-w^2(u^3+v^3+u^2v+v^2u)+u^2v^3+v^3u^2)}{2uvw^2(u-v)}.$$

Matome, kad sukeitus ploto išraiškoje $u$ ir $v$ vietomis, gausime $-S$. Be to, žinome, kad sukeitus trikampio viršūnių orientaciją, ploto formulėje turime prirašyti minuso ženklą. Todėl naudojantis trikampių simetrija matome, kad $\triangle RPK$ plotas bus lygus S.
\end{sprendimas}





\section{TODO: liekanos}

Turime trikampį ABC. Tada nesunkiai galime rasti kampą A, kurį pasižymėkime $\theta$. Pastebime, kad $\frac{c-a}{|c-a|}$ yra lygiagreti atkarpai AC ir vienetinio ilgio, analogiškai $\frac{b-a}{|b-a|}$ yra lygiagreti AB ir vienetinio ilgio, todėl 

$$\dfrac{\displaystyle \frac{c-a}{|c-a|}}{\displaystyle \frac{b-a}{|b-a|}}=e^{i\theta}.$$

Dažniausiai sprendžiant uždavinius, paties kampo išsireikšti ir nereikia, nes $e^{i\theta} = e^{i\phi} \Leftrightarrow \theta = \phi + 2\pi k$, kažkokiam sveikajam skaičiui $k$. Todėl kalbant apie mažus kampus (mažesnius už $\pi$) gauname $\theta = \phi $.

\begin{pavnr}
Duota trapecija $ABCD$ su pagrindais $BC$ ir $AD$, o jos įstrižainių susikirtimo taškas yra lygus $P$. Taškas $M$ yra $BC$ vidurio taškas bei yra žinoma, kad $\angle ABD + \angle ACD = \pi.$ Ant kraštinės $AD$ paimtas taškas $X$ toks, kad $(\frac {PD}{PA})^2 = \frac {DX}{XA}$. Įrodykite, kad $MX$ yra statmena trapecijos pagrindams.
\end{pavnr}

\begin{sprendimas}
Šį kartą spręsime kompleksinių koordinačių centru pasirinkę tašką $P$. Iš Euklidinės geometrijos
nesunku pastebėti, kad $\Delta APD$ yra panašus $\Delta CPB$. Šis faktas yra labai naudingas tvarkantis su trapecijos kraštinėmis, nes gauname, kad $c=ka,  b=kd,$ kur $k$ yra realus teigiamas skaičius.

Kadangi yra žinoma, jog $ ( \frac {PD}{PA})^2 = \frac {DX}{XA} $, gauname, kad 
$$ x = \frac { da \bar a + ad \bar d }{ a \bar a + d \bar d},$$
nes $ ( \frac {PD}{PA})^2 =\frac {d \bar d} {a \bar a} $ ir $\frac {DX}{XA} = \frac{d-x}{x-a}.$

Koeficientą $k$ mums padės surasti tai, kad $ \angle ABD + \angle ACD = \pi $, nes (panašiai kaip keturių taškų priklausymo vienam apskritimui teoremos įrodyme) gauname lygybę:
$$\frac {b-a}{\bar b - \bar a} : \frac { b-d}{\bar b - \bar d} =  \frac {d-c}{\bar d - \bar c} : \frac { c-a}{\bar c- \bar a}.$$
Pertvarkius lygybę gauname, kad 
$$ k =  \frac { \bar d a + \bar a d }{a \bar a + d \bar d}. $$
Galiausiai, pagal tiesių lygiagretumo ir statmenumo teoremą, $MX$ yra statmena trapecijos pagrindams tada ir tik tada, kai
 $$  \frac {m-x}{ \bar m - \bar x} = - \frac {a - d}{ \bar a - \bar d},$$
kur žinome, jog $ m =\frac{b+c}{2}= k \frac { a+d} { 2} $, nes $M$ yra atkarpos $BC$ vidurio taškas. Taigi, įsistatę gautas reikšmes, įrodome tai, ką mums ir reikėjo gauti
$$ \frac {m-x}{ \bar m - \bar x}= 
\frac {(a-b)( \bar a b - \bar b a) }{(\bar a- \bar b)( \bar b a - \bar a b) } = - \frac {a - d}{ \bar a - \bar d}.$$
\end{sprendimas}
\end{document}
